\documentclass{swfcthesis}

\begin{document}

\Title{论文标题}        % 论文标题
\Author{作者姓名}       % 作者姓名 
\Advisor{指导教师姓名}      %指导教师姓名
\AdvisorTitle{指导教师职称} %指导教师职称
\AdvisorInfo{指导教师简介(约百余字)}  %指导教师简介(约百余字)
\Month{六}                  %月份(比如 六)
\Year{二〇一二}             %年份(比如 二〇一二)
\Univ{西南林业大学}             %校名
\School{计算机与信息科学学院}           %院系名称
\Subject{计算机科学与技术专业}  %专业名称(比如 电子信息工程专业)
\Docname{本科毕业(设计)论文}  %本科?研究生?
\Abstract{这里写论文摘要(约两百字)}   %论文摘要(约两百字)
\Keywords{这里写关键字,比如 电阻, 电容}        %关键字(比如 电阻, 电容, 电感, 电灯泡)
\Acknowledgments{这里写鸣谢(约百余字)}                %鸣谢 (感谢人民感谢党,约百字)
\enTitle{英文标题}                      %英文标题
\enAuthor{英文姓名}                     %作者英文姓名
\enUniv{Southwest Forestry University}  %英文校名
\enSchool{School of Computer and Information Science}   %英文院系名称
\enAbstract{英文摘要}                    %论文英文摘要
\enKeywords{英文关键字}          %英文关键字

%%% 下面六行不要动!
\makepreliminarypages
\frontmatter
\tableofcontents
\listoffigures
\listoftables
\mainmatter
%%% 上面六行不要动!

\chapter{第一章标题}    %第一章标题
\label{cha:one}


正文部分从此开始。可以参考模版目录中的各章节tex文件来写。

正文部分从此开始。可以参考模版目录中的各章节tex文件来写。

正文部分从此开始。可以参考模版目录中的各章节tex文件来写。

正文部分从此开始。可以参考模版目录中的各章节tex文件来写。

\chapter{步入正题}
\label{cha:two}

可以参考模版目录中的各章节tex文件来写。

可以参考模版目录中的各章节tex文件来写。

可以参考模版目录中的各章节tex文件来写。

可以参考模版目录中的各章节tex文件来写。

可以参考模版目录中的各章节tex文件来写。

可以参考模版目录中的各章节tex文件来写。

%%% 正文部分到此结束。下面是『参考文献』、『指导教师简介』、『鸣谢』、『附录』
\Appendix{} % 不要动这一行!
% 下面是参考文献部分
\begin{thebibliography}{99}
\bibitem{标签}作者,书名,出版社,出版日期
\bibitem{标签}作者,书名,出版社,出版日期
\bibitem{标签}作者,书名,出版社,出版日期
\bibitem{标签}作者,书名,出版社,出版日期
\bibitem{标签}作者,书名,出版社,出版日期
\end{thebibliography}

\advisorinfopage{} % 不要动这一行!
\acknowledgmentspage{} % 不要动这一行!

%%% 下面是附录部分。

\chapter{我也不知道为什么要写附录}              %附录一
\label{cha:app}

可以参考模版目录中的 appendix.tex 文件来写。

可以参考模版目录中的 appendix.tex 文件来写。

可以参考模版目录中的 appendix.tex 文件来写。

\chapter{我居然编程了!}                        %附录二
\label{cha:app}

可以参考模版目录中的 src.tex 文件来插入代码。比如:

\inputminted[fontsize=\small]{c}{hello.c}

或者,

\begin[fontsize=\small]{minted}{c}
  int main()
  {
    printf("Hello, world!\n");
    return 0;
  }
\end{minted}

或者,

\begin{listing}[H]
  \inputminted{c}{hello.c}
  \caption{我居然编程了!}
  \label{lst:hello}
\end{listing}  

或者,

\begin{listing}[H]
  \begin{minted}{c}
    int main()
    {
      printf("Hello, world!\n");
      return 0;
    }
  \end{minted}
  \caption{我居然编程了!}
  \label{lst:helloagain}
\end{listing}  

\end{document} % 此行后面不要有任何文字。

%%% Local Variables:
%%% mode: latex
%%% TeX-master: t
%%% End:
