% Created 2018-09-01 Sat 08:46
% Intended LaTeX compiler: pdflatex
\documentclass[cyan]{elegantnote}
\author{mac}
\date{\today}
\title{}
\hypersetup{
 pdfauthor={mac},
 pdftitle={},
 pdfkeywords={},
 pdfsubject={},
 pdfcreator={Emacs 25.3.1 (Org mode 9.1.2)}, 
 pdflang={English}}
\begin{document}

\tableofcontents

\chapter{test}
\label{sec:orgc76046a}
\section{Wrong}
\label{sec:org264c2ff}
3 2020年,是建成小康的决战期。(错,时间错)
11 实现高等教育外延式发展 (错 内涵发展)
2 辞职申请,一般人员30日,领导干部90日
13 公务员兼职,不可领取报酬。
15 公务员血亲,不能在同一单位直接隶属于同一领导下任职。
16 公务员晋升领导职务,必须进行公示(错,不是视情况)
17 考核不称职,降低一个职务层次
20 公务员只能在内部交流(错,可以在类公务员管理所有岗位)
21 公务员降级、撤职后,原级别、职务自动恢复(错,)
22 公务员被辞退,可以有辞退费及享受失业保险
23 因公致殘,不能辞退。
24 旷工15天,或一年累计30天,辞退。
25 公务员职务与级别对应 国家规定 非省级
31 监察机关管辖有争议,同上级监察机关决定(错,共同上级监察机关决定)
36 通辑应上报有决定权的上级监察机关(错)
37 (错)有决定权的
39 (错)职务违法、犯罪,配合而非共同
45 (错)只要挪用公款,就构成犯罪。
54 (错)警告不影响岗位及合格考核
63 (错)事业单位本单位可以作出开除决定
70 (错)举办展会可以的
71 (错)不能进行评比达标
74 (错)可以组织境外培训
78 (错)只能使用一处办公用房
79 (错)可以租用办公用房
82 (错)非所有的,可以公开的公开
84 (错)不能进行评比达标
85 (错)预算可以追加
86 (错)有些执法车不宜喷涂统一标志
91(错)可以改变办公用房功能
92 (错)适宜公开的公开,非所有
96 (错)党纪政纪处分(分别或语法错误)
99 (错)办公用房项目一律不准
100 (错)没收卖
106 (错)陪餐人数10人以内3人
107 (错)不能承担自行承担的费用
108 (错)可能选乘飞机
110 (错)单位及纪委管
112 (错)非公务活动不接待
114 (错)个人可以出版著作
121 (错)涉及子女回避
124 (错)价值不大的不交出,非受贿
129 (错)机关人员受处分,影响职务、级别晋升
130 (错)开除处分不存在提前解除
132 (错)违法违纪必须处分
135 (错)决定后报
136 (错)按规定处理,而非开除
140 (错)两种处分,分别执行
141 (错)责任追究执行,同时可以申诉
145 (错)处分期满不能自动解除
147 (错)判刑必开
150 (错)组织处理及纪律处理同时进行
154 (错)单位处理
155 (错)
161 (错)个人申报事项,由组织及纪委核实
164 (错)个人申报事项,由组织及纪委负责
166 (错)只要不退
167 (错)主管部门受理
170 (错)管理权限
175 (错)人民行使权力人民代表大会
177 (错)我国现行宪法:序言、总纲、公民权力及义务、国家机关、国家标志
180 (错)最高权力机关是人大
181 (错)部分职权
191 (错)人大人员专职,不能兼任其它
193 (错)本级及上级
197 (错)
200 (错)上级规定
6 (错)非公有制经济非主体
\section{单选}
\label{sec:orgc82cec5}
8 稳中求进的总基调
10 与时俱进的理论品格
\section{多选}
\label{sec:org298a0b8}
2 职务 级别
6 越级晋升 优秀 特殊需要
8 工勤及事业单位不属于公务员
60 解除处分条件 悔改 没有再违纪
65 处分 法定程序 法定事由
\section{一}
\label{sec:orge9be16d}
33 中国精神 中国力量
36 紧紧围绕党同人民群众的血肉联系
10 根本政治前提和制度基础
11 行政体制改革 监察体制改革 司法体制改革
13 上下主体问题 知 存 守 下级
14 统一指挥 全面覆盖 权威高效
15 除科学思维
17 深化改革的总目标是
完善和发展中国特色社会主义制度
推进国家治理体系和治理能力现代化
25 除个人主义(形式主义 官僚主义 享乐主义)
26 三大历史任务 建设 统一 共同发展
29 保护 尊重 顺应自然
31 提倡 简约适度 绿色低碳的生活方式
32 耕地红线早已划定
33 听党指挥 能打胜仗 作风优良
34 新型政商关系 亲 清
37 坚持 德才兼备 以德为先 五湖四海 任人为贤 事业为上 公道正派
38 激励机制 容错纠错机制
39 世界观 价值观 人生观
40 压倒性态势 压倒性胜利
41 党内存在 思想不纯 组织不纯 作用不纯(道德不纯不是)
42 坚持党的中央权威和集中统一领导
44 圈子文化 宗派主义 码头文化
45 四风 特权思想 特权现象
57 坚持 党的领导 人民当家作主 依法治国
59 除金融风险
\section{二}
\label{sec:org2d8132a}
\subsection{53 除公务学习交流}
\label{sec:org9e1894a}
56 退休的,应依法
降级、撤职、开除的 降低岗位等级或取消待遇
58 处分
负有责任的领导人员
直接责任人员单选
1 上下级公务员管理单位在公务员管理上
是指导关系
2 公务员录用范围
副主任科员及以下职位
3 省及以上公务员录用属于
国家公务员局
4 处分不能提前解除
没有发生违纪不作为提前解除处分的条件
5 挂职锻炼
上下级、不同地区、事业及国企
6 重点考核公务员
工作实绩
7 公务员管理单位
不能制定公务员管理规定
国务院制定及发布
8 公务员地域回避
县级主要领导职务
法院主要领导
检察院主要领导
组织部部长
9 不属于辞退公务员条件
当年考核不称职
(旷工 不胜任 不履行义务,不遵守纪律)
10 不影响晋升工作档次
警告(记过 记大过 降级)
11 县级公务员录用
省级公务员管理部门负责
12 夫妻一方担任领导职务
文书(不能从事 人事 财务 审计)
13 公务员职务与级别对应关系
由国务院规定
14 除开除处理
有悔改表现 再无违纪 处分期满 可以解除处分
15 监察法对象
所有行使公权力的公职人员
16 建立 监察体系制
集中统一 权威高效
17 监察机关是行使国家监察职能的专职机关
18 监察机关办理职务违法 应与检察机关、审判机关、执法部门
互相配合 互相制约
19 惩戒与教育相结合 宽严相济
20 不敢腐 不能腐 不想腐的长效机制
21 国家监察委员会
由全国人民代表大会产生
22 国家监察委员会主任 由全国人民代表大会产生 副主任 及委员
由全国人民代表大会常务委员会任免
23 各级监察委员会
由本级人民代表大会产生
24 地方各级监察委员会对
本级人民代表大会、常务委员会、上级监察委员会负责
25 监察委员会履行(监督 调查 处置)职责
26 监察委员会对违法公职人员作出(政务处分)
27 不调查过程中,可以询问证人
违纪及犯罪人员(审问 盘问 讯问)
28 搜查必须出示搜查证
29 查封、扣压财务、文件无关的
三日内归还
30 事业单位处分时长
记过 12个月
警告 6个月
降低岗位等级或撤职 24个月
31 重大贪污贿赂 经严格批准 可以上技术措施
32 事业单位 定为记过处分 不能升级 当年考核不能定为合格以上
33 通辑应上报有决定权的上级监察机关
34 限制出境
报请省及以上监察机关
35 事业单位处分期限 合并最长不超过
48个月
36 证据标准
刑事审判
37 非法方法证据
无效 排除
38 事业单位人员判刑
给予 降低岗位等级或撤职 以上处分
39 职务违法等线索
就移送监察机关 由监察机关依法处理
40 行政机关任命的事业单位人员被判刑的
开除
41 受处分人员申请复核
   30日内
42 职务违法及其它犯罪
以监察机关为主 其它机关予以协助
43 申诉
由同级事业单位人事综合管理部门受理
44 严重职务违法和职务犯罪
通知家属 并向社会公布
45 退休应受处分 的
不再作出处分 降低岗位等级或撤职的 降低岗位等级或取消待遇
46 证据标准
相互印证 完整稳定 的证据链
47 调查人员
依法出示证件 出具书面通知
48 讯问、搜查、查封、扣压
全过程 录音录像 留存备查
49 重要事项
集体研究后按程序请求上报
50 留置 市
要报上级批准 省级报国家级备案
51 留置时间 及延长时间
 三个月 三年月
52 留置措施后 应在
 24小时内通知单位及家属 (解除影响办案)
53 补充调查以一个月内为限,二次为限
应当退回监察机关补充调查,必要时可以自行补充侦查。
54 向 人民法院 提出没收违法所得的申请
只有人民法院有执行权
55 听取和审议本级监察委员会 专项报告 组织执法检查
各级人民代表大会常务委员会
56 监察人员辞职、退休 三年内 不得从事利益相关职业
57 不执行上级决定 负有责任的领导人员和直接责任人员
依法给予处理
58 监察人员造成损害的
依法给予国家赔偿
59 贪污数额巨大,造成巨大损失
处 无期徙刑 或者 死刑 并没收财产
60 回扣、手续费 归个人所有
受贿罪
61 关系人利用影响力索取及收受
利用影响力受贿罪
62 巨额财产不名
 30万元以上立案
63 受贿5万元 150万元财产不明
受贿罪和巨额财产不明罪
64 索贿20万元
以受贿罪从重处罚
65 挪用公款30万,半年后归还
挪用公款罪
66 挪用公款10万 三个月归还
挪用公款罪
67 受贿罪
68 受贿罪
69 招收学生徇私無弊罪
70 玩忽职守暃
71 不属于渎职行为
工作中故意刁难自己的下属
72 执法出问题 猪肉没处理好
玩忽职守暃
73 合同被骗400万
国家机关工作人员签定、履行合同失职被骗罪
74 造成浪费及不良影响 对负有领导责任的主要负责人及相关领导干部
问责
75 厅局级以下可以乘坐
飞机经济舱
76 违反山东节约条例 浪费及影响坏
通报批评或者调离岗位
77 执法执勤用车严格限制在
一线执法执勤用车
78 公车
任何人不能公车私用(身边工作人员 包括司机)
79 不得支付 个人负担的费用
80 不能提供香烟
81 公务接待不能清场闭馆
82 公务接待需警卫的按标准要求
83 合理限定接待费总额
84 接待标准全国统一
不需上报备案
85 不能扩建、改建、超标装修、配备家电
可以推进统一使用
86 向机关有关领导及人员送有价证券
追究 有关领导 责任
87 礼金及有价证券
一个月内全部上交
88 收受礼金及有价证券
根据 数额多少及情节轻重 给予党纪、政纪处分
89 党和国家工作人员收受礼品
必须登记上交
90 礼品
除 价值不大的 必须登记
91 受收礼品之日起
 一个月必须登记上报
92 子女、配偶移居国外
 两个月内必须上报 60日
93 子女、配偶移居国外
 60日内向 组织(人事)部门报告
94 行政机关公务员
受处分期间 不得晋升职务、级别
95 处分决定、解除处分决定
自 作出之日 生效
96 机关负责人的回避
由上一级 行政机关负责人决定
97 处分
坚持 公正、公平 和教育与惩处相结合的原则
98 适用人员上报个人事项发生变化
在30日内向组织(人事)部门报告
99 公务员承担
纪律责任 给予处分
法律判刑了。
100 境内外管理 县副处级以上有关事项
纪检 监察机关和 组织(人事)部门
101 处分各级政府领导人(市长 副职 法院、检察院 监察主任等)
由上一级政府决定
102 担任县处级领导
五年以上工龄 两年以上基层工作经验
103 县处级以上 下级正职提副职
在正职岗位 3年以上
副厅 在正县正职必须三年以上
104 行政机关工作违纪需调查
报任免机关负责人 立案
105 提拔党政领导干部 应具有大专科以上 厅局级应具有大学本科以上
106 主动交代违纪行为,主动挽回或避免损失的
免予以处分
107 引咎辞职、责令智取、 问责免职的干部
一年内不安排职务 两年内不得高于原职务。
108 县处级以上,副职提正职,应在副职岗位
工作两年以上
109 不执行上级决定、命令,无论有无后果
必须处理,破坏了行政机关内部管理秩序
110 情节轻微,批评后改正的
免予处分
111 行政机关公务员,判刑的
给予开除处分
112 处分应与违纪的性质、情节、危害程度 相适应。
113 提拔任职不满 或提拔不满一年 不得破格提拔
114 处分6个月内决定 最长的不超过
 12个月(自立案之日起)
115 公务员处分种类
警告 记过 记大过 降级 撤职 开除
无(降职 调离 停职)处分
116 除法律、法规、规章及(国务院 )规定外,不能设置处分项
117 民主推荐结果 一年内有效
118 处分期未执行相加
119 领导班子换届,按
全额定向
120 公务员处分 由任免机关或者监察机关按管理权限 处分
121 干部任免事项
必须三分之二以上到会
122 境外限期不归
开除
123 干部任免决定 需复议的
需 二分之一同意
124 公务员行政处分
 政府机关工作人员 公安局长
不包括党的机关 权力机关(人大 ) 审判机关 检察机关 政协 群众团体 企事业单位
125 干部任免事项决议
应到会人员超过半数同意 形成决议
126 处分超过一个,可以晋升工资档次
127 任前公示期
不少于五个工作日
128 挪用公款未还事发
撤职或者开除
129 干部任职试用期
一年
130 两种以上行为需处分
分别确定
131 公开选拔面向 社会 竞争上岗 本单位或本系统
132 同一岗位任满
十年 必须交流 任满两届 不再担任同一职务
133 调职处理的 一年内不得提拔
引咎辞职、责令辞职、免职 两年内不得提拔
降职处理的 两年内不得提拔
134 本人、配偶、子女存款不需报告
135 审计机关不能查阅个人报告事项
(组织人事 纪检检察 监察可以)
136 配偶、子女到外国留学不属于移居境外
137 作用问题不属于个人事项调查
138 宪法规定 社会主义制度是我国的根本制度
139 中国共产党的领导 是中国特色社会主义最本质的 特征
140 常务委员会 对全国人大 负责并报告工作
141 人民行使权力的机关
全国人民代表大会 地方各级人民代表大会
142 国家监察委员会主任
全国人民代表大会 任免
143 延长人代会任期
全国人民代表大会常务委员会
144 全国人代会每届任期
 五年
145 常务委员会负责 解释宪法
146 全国人民代表大会 修改宪法
147 总理由 国家主席 提名
148 依法治国 首先坚持 依宪治国
149 宪法经历了
 5次修改
150 一切权力属于 人民
151 国有经济 是国民经济中的主导力量
152 选举权年龄
 18岁
153 全国人民代表大会和全国人民代表大会常务委员会行使 国家立法权
154 全国人民代表大会代表的选举
由全国人民代表大会常务委员会主持 (非主席团)
155 代表届满 两个月前完成选举
156 全国人民代表大会每年举行一次 由常务委员会招集
157 临时全国人民代表大会
常务委员会 或五分之一的代表提议
158 宪法修改
由常务委员会 或五分之一的代表提议
159 宪法修改通过
三分之二代表通过
160 全国人大代表 刑事审判
主席团 或常务委员会
161 主席 副主席 年龄
年满45周岁
162 主席 副主席
由全国人民代表大会选举
163 副主席缺位
由全国人民代表大会补选
164 地方各级人民政府 是各级权力机关的执行机关
165 监察委员会的组织和职权 由法律规定
166 监察委员会是国家的监察机关
1 创立了 习近平新时代中国特色社会主义思想
2 初心和使命
为中国人民谋幸福,为中华民族谋复兴
3 永远把 人民对美好生活的向往 作为奋斗目标
4 奋力夺取 新时代中国特色社会主义 伟大胜利
5 实现中华民族的伟大复兴 是近代以来中华民族的最伟大梦想。
6 中国特色社会主义理论体系
是指导党和人民实现中华民族伟大复兴的正确理论
7 我国发展仍处于重要的 战略机遇期
8 经济新常态,坚持 稳中求进 的工作总基调
9 以 人民为中心 的发展思想
10 马克思政党 与时俱进 的理论品格
11 党正在推进的 党的建设新的伟大工程
12 党的最终理想和最终目标
实现共产主义
13 起决定性作用的是
党的建设新的伟大工程
14 政治建设 处于统领地位
纪律建设 是管党治党的根本之策
15 坚决改变管党治党 宽松软 的状况
16 反腐败,无禁区,全覆盖, 零容忍
17 牢牢坚持 党的基本路线 这个党和国家的生命线、人民的幸福线。
18 发展 不平衡不充分
19 以人民为中心的发展思想
不断 促进人的全面发展
20 我国经济已经由
高速增长 转向 高质量发展 阶段
21 党面临的最大威胁是 腐败
22 全党尊崇 党章
23 反腐败斗争 压倒性态势 已经形成并巩固发展
24 中国特色社会主义文化 是精神力量
25 严明党的 政治纪律 政治规矩
26 发展经济的着力点放在 实体经济上
27 中国特色社会主义道路是实现社会主义现代化、人民幸福生活的必由之路
28 中国特色社会主义是 党的全部理论及实践的主题
29 突出 政治建设 在党的建设中的重要地位
30 发展必须是 科学发展
31 必须坚持 马克思主义
32 文化自信 是一个国家、民族发展更基本、更深沉、更持久的力量
33 发展先进文化,更好构筑 中国精神,中国价值,中国力量
34 建设 生态文明,是千年大计
35 发展积极健康的党内 政治文化,全面净化党内 政治生态
36 加强作风建设,必须紧紧围绕 保持党同人民群众的血肉联系
37 把党的 群众路线 贯彻到治国理政的全部活动中
38 全面依法治国 是中国特色社会主义的本质要求和重要保障
39 提高全民族的 法制素质 和道德素质
40 以 宪法 为核心的法律体系
41 勇于 自我革新 是中国共产党最鲜明的品格
42 以 供给侧结构性改革 为主线
43 转变政府职能,深化 简政放权
44 全面深化改革 推进 国家治理体系及治理能力现代化
45 坚持 精准扶贫 精准脱贫
46 党的 政治建设 为统领
47 共同建设 共同享有的原则
48 中国共产党 把 为人类作出新的更大的贡献 作为自己的使命
49 建立以 国家公园 为主体的自然保护地体系
50 坚持 独立自主 的和平外交政策
51 牢固树立 社会主义生态文明观 人与自然和谐发展
52 增强 港奥的 国家意识 和爱国精神
53 坚持总体安全观,统筹 发展和安全
54 推进 一带一路 建设
55 必须坚持 国家利益 至上
56 构建 人类命运共同体
57 统一 是中华民族的 根本利益 所在
58 一个中国 的原则 是两岸关系的政治基础
59 反腐败斗争的 压倒性态势
60 旗帜鲜明的 讲政治
61 选人用人 突出 政治标准
62 重点强化 政治纪律 和组织纪律
63 实现 中央和省级党委巡视 全覆盖
64 市县党委建立 巡察 制度
65 用 ‘留置’ 取代 ‘两规’措施
66 自我净化能力 强化 党的 自我监督 和群众监督
67 国家、省市县 监察委员会
68 强化自上而下的 组织监督,改进自下而上的 民主监督
69 坚持 问题导向 保持战略定力
70 到本世纪中叶 建成 富强民主文明和谐美丽的社会主义现代化强国 2050年
71 创新 是引领发展的第一动力
72 发展是党的第一要务
73 主要矛盾
 人民日益增长的美好生活需要和不平衡不充分的发展之间的矛盾
74 五大建设
经济建设、政治建设、文化建设、社会建设、生态文明建设
75 总体布局是 五位一体
76 发展理念
创新 协调 绿色 开放 共享
77 建立 市场化 多元化 生态补偿机制
78 坚持正确的 义利观
79 奉行 互利共赢 的开放战略
1 国家层面的价值目标
富强 民主 文明 和谐
2 社会层面的价值目标
自由 平等 公正 法制
3 个人层面的价值目标
爱国 敬业 诚信 友善
4 紧紧围绕的目标是
实现中华民族的伟大复兴梦
5 核心价值观落实到
经济发展实践和社会治理 中
6 充分发挥 党员、干部 的带头作用
7 用 思想艺术性和观赏性相统一的 作品
8 以 文化兴盛 为支撑 以中华文化发展 繁荣为条件
9 13年 必须加强全社会的 思想道德 建设
10 政者,正也。《论语》
11 山东代表团 红色基因就是要传承
12 积极培育和践行 社会主义核心价值观
13 200年县委书记座谈 谷文昌
14 先进事迹作出重要指示 廖俊波
15 科技报国理想 黄大年
16 高校要牢牢抓住 培养社会主义建设者和接班人 这个根本任务
17 培养教师队伍的第一标准是 师德师风
18 要弘扬以(爱国主义)为核心的民族精神,以(改革创新)为中心的时代精神
3 排除 “先天下之忧而忧,后天下之乐而乐” ABC
4 学以致用 知行合一
排除“学而不思则罔,” ABC
5 “善禁者,先紧其身而后人”
排除“奢靡之如,危亡之渐”
BCD
6 落实到( 立法 执法 司法 普法和依法治理)
7 抓住(世界观 人生观 价值观 )这个总开关
8 一切(文化产品 文化服务 文化活动) 排除(文化消费)
14 排除 有选择的加以利用
15 讲道德 守道德 尊道德
16 立政德 就要(明大德 守公德 严私德)
排除 循本心
18 小事小节中(有党性 有原则 有人格)
排除 有人情
20 慎独 (私底下 无人处 细微处)
排除 监督下
22 方向、队伍、体系
6 非公经济 主力军 错
一 多选
10 根本政治前提 和制度基础
11 行政体制改革 权力运行制约 司法体制改革
排除 纪检体制改革
13 让党员干部 (知敬畏 存戒惧 守底线 )
排除 立规矩 组织的事
14 构建党的(统一指挥 全面覆盖 权威高效)的监督体系
15 党的五个思维 坚持战略思维、创新思维、(辩证思维 法治思维 底线思维)
17 深化改革的总目标是
完善和发展中国特色社会主义制度
推进国家治理体系和治理能力现代化
21 坚持走(生产发展 生活富裕 生态良好)的道路
25 八项规定 严厉整治(形式主义 享乐主义 官僚主义)
26 排除 推进世界共同富裕
28 建设(学习型 服务型 创新型)政党
党的封闭的 不是开放型
29 人类必须(尊重自然 顺应自然 保护自然)
排除改造自然
31 倡导(简约适度 绿色低碳)的生活方式
32 要完成(生态保护红线 永久基本农田 城镇开发边界)的划定工作
33 (听党指挥 能打胜仗 作风优良)的军队
34 政商关系 “亲” “清”
37 坚持(德才兼备,以德为先),支持(五湖四海,任人为贤),坚持(事业为上,公道正派)
38 考核 建设(激励机制)(容错纠错机制)
39 “总开关”(世界观 人生观 价值观)
40 反腐败 巩固(压倒性态势),夺取(压倒性胜利)
41 党内存在(思想不纯 组织不纯 作风不纯)问题
42 全党服从中央,坚持(党中央权威)和(集中统一领导)
44 防止和反对(圈子文化 宗派主义 码头文化)
45 四风 坚决反对(圈子文化 特权思想)
48 学十九大(学懂 弄通 做实)
57 坚持(党的领导 人民当家作主 依法治国)的有机统一
59 坚决打好(防范化解重大风险 精准脱贫 污染防治)攻坚战
\subsection{多选}
\label{sec:orgb139bf8}
2 确定公务员工资及其它待遇
职务 和 级别
4 录用公务员 依据(考试成绩 考察情况 体检结果)
没有 群众评议
5 公务员职务任命必备的三个条件
有职数 有编制 有相应的职位空缺
6 公务员越级晋升
表现特别优秀的公务员 工作特殊需要
7 公务员指
依法履行公职
纳入国家行政编制
国家财政负担工资福利
8 不属于《公务员法》管理范围
机关工勤人员
事业单位人员
13 说法正确的
因工作需要可以兼任事业单位的某些职务
兼任机关外职务需经批准,并不得领取兼职报酬
14 机关聘任公务员
可从符合条件的人员中直接选聘
按照公务员录用程序公开招聘
以签定聘用合同的方式确定双方的权利与义务。
17 不得录为公务员
曾判刑的 未满18岁 曾被开除公职的
18 不符合《公务员管理 》法律规定
02 04 考核不称职被辞退(连续两年)
适用期一年,不能提前转正
开除处分 可以申诉,但处分执行(不保留工作)
19 影响工资晋升档次的处分
记过 记大过 降级 撤职
20 公务员可以交流的单位
国有企业及事业单位
人民团体
群众团体
21 公务员交流的方式
调任 转任 挂职锻炼 除(借用)
29 重新作出处分
处分的依据 事实不清、证据不足
违反程序 影响公正
超越职权或滥用职权作出的决定
31 监察职责
除查处违纪(纪委)
32 党员干部违纪(属于纪委)
35 需保密的
国家、个人隐私、商业秘密
社会舆情是公开的
36 调查过程中 可以
调取、查封、扣压 证据
但不能处置
40 监察机关公开信息接受
社会监督 群众监督
60 解除处分条件
有悔改表现
没有再发生违纪违法行为
62 党政机关不得
新建 改建 扩建
63 违纪违法并涉嫌犯罪的
移送司法机关
依法追究刑事责任
64 定点保险 定点维修 定点加油
65 公务员 不受处分
非经法定程序
非经法定事由
67 加重处分
除(同案起次要作用)
69 从轻的条件
除(检举不确认,不能认定)
71 处分条例不是为了
约束政府人员的行为
73 除 正常会议支出
74 与被调查人有上下级领导关系 不需回避
76 除 按照程序作出公正处理
77 除 办公用户
不能统一核定价格
78 除 不完全符合相关规定 外
79 行政执法违法 处理
负有责任的领导人员
和直接责任人员
81 审计
除 公款娱乐(严禁 不能报销)
83 嫖娼 行政处罚
撤职 开除
84 违反 (法律 法规 规章 )给予行政处分
85 从轻的情节
除 承认错误态度较好
86 违反 (法定权限 法定条件 法定程序)给予撤职
(除 法定事由)
87 接待 不能增加陪同人数
89 严禁以任何形式(隐瞒 截留 挤占)坐支或者私分非税收
除(使用)
90 立案调查期间 不得(交流 出境 辞去公职)
96 出境期间 属我国机构 除私人自费宴请外 用公款皆不许
97 收回设备是可以的
98 除 配备必要的公务用车
100 可以购买正常服务

60 解除处分条件
有悔改表现
没有再发生违纪违法行为
62 党政机关不得
新建 改建 扩建
63 违纪违法并涉嫌犯罪的
移送司法机关
依法追究刑事责任
64 定点保险 定点维修 定点加油
65 公务员 不受处分
非经法定程序
非经法定事由
67 加重处分
除(同案起次要作用)
69 从轻的条件
除(检举不确认,不能认定)
71 处分条例不是为了
约束政府人员的行为
73 除 正常会议支出
74 与被调查人有上下级领导关系 不需回避
76 除 按照程序作出公正处理
77 除 办公用户
不能统一核定价格
78 除 不完全符合相关规定 外
79 行政执法违法 处理
负有责任的领导人员
和直接责任人员
81 审计
除 公款娱乐(严禁 不能报销)
83 嫖娼 行政处罚
撤职 开除
84 违反 (法律 法规 规章 )给予行政处分
85 从轻的情节
除 承认错误态度较好
86 违反 (法定权限 法定条件 法定程序)给予撤职
(除 法定事由)
87 接待 不能增加陪同人数
89 严禁以任何形式(隐瞒 截留 挤占)坐支或者私分非税收
除(使用)
90 立案调查期间 不得(交流 出境 辞去公职)
96 出境期间 属我国机构 除私人自费宴请外 用公款皆不许
97 收回设备是可以的
98 除 配备必要的公务用车
100 可以购买正常服务
53 除公务学习交流
56 退休的,应依法
降级、撤职、开除的 降低岗位等级或取消待遇
58 处分
负有责任的领导人员
直接责任人员
60 解除处分条件
有悔改表现
没有再发生违纪违法行为
62 党政机关不得
新建 改建 扩建
63 违纪违法并涉嫌犯罪的
移送司法机关
依法追究刑事责任
64 定点保险 定点维修 定点加油
65 公务员 不受处分
非经法定程序
非经法定事由
67 加重处分
除(同案起次要作用)
69 从轻的条件
除(检举不确认,不能认定)
71 处分条例不是为了
约束政府人员的行为
73 除 正常会议支出
74 与被调查人有上下级领导关系 不需回避
76 除 按照程序作出公正处理
77 除 办公用户
不能统一核定价格
78 除 不完全符合相关规定 外
79 行政执法违法 处理
负有责任的领导人员
和直接责任人员
81 审计
除 公款娱乐(严禁 不能报销)
83 嫖娼 行政处罚
撤职 开除
84 违反 (法律 法规 规章 )给予行政处分
85 从轻的情节
除 承认错误态度较好
86 违反 (法定权限 法定条件 法定程序)给予撤职
(除 法定事由)
87 接待 不能增加陪同人数
89 严禁以任何形式(隐瞒 截留 挤占)坐支或者私分非税收
除(使用)
90 立案调查期间 不得(交流 出境 辞去公职)
96 出境期间 属我国机构 除私人自费宴请外 用公款皆不许
97 收回设备是可以的
98 除 配备必要的公务用车
100 可以购买正常服务
53 除公务学习交流
56 退休的,应依法
降级、撤职、开除的 降低岗位等级或取消待遇
58 处分
负有责任的领导人员
直接责任人员
60 解除处分条件
有悔改表现
没有再发生违纪违法行为
62 党政机关不得
新建 改建 扩建
63 违纪违法并涉嫌犯罪的
移送司法机关
依法追究刑事责任
64 定点保险 定点维修 定点加油
65 公务员 不受处分
非经法定程序
非经法定事由
67 加重处分
除(同案起次要作用)
69 从轻的条件
除(检举不确认,不能认定)
71 处分条例不是为了
约束政府人员的行为
73 除 正常会议支出
74 与被调查人有上下级领导关系 不需回避
76 除 按照程序作出公正处理
77 除 办公用户
不能统一核定价格
78 除 不完全符合相关规定 外
79 行政执法违法 处理
负有责任的领导人员
和直接责任人员
81 审计
除 公款娱乐(严禁 不能报销)
83 嫖娼 行政处罚
撤职 开除
84 违反 (法律 法规 规章 )给予行政处分
85 从轻的情节
除 承认错误态度较好
86 违反 (法定权限 法定条件 法定程序)给予撤职
(除 法定事由)
87 接待 不能增加陪同人数
89 严禁以任何形式(隐瞒 截留 挤占)坐支或者私分非税收
除(使用)
90 立案调查期间 不得(交流 出境 辞去公职)
96 出境期间 属我国机构 除私人自费宴请外 用公款皆不许
97 收回设备是可以的
98 除 配备必要的公务用车
100 可以购买正常服务
53 除公务学习交流
56 退休的,应依法
降级、撤职、开除的 降低岗位等级或取消待遇
58 处分
负有责任的领导人员
直接责任人员
60 解除处分条件
有悔改表现
没有再发生违纪违法行为
62 党政机关不得
新建 改建 扩建
63 违纪违法并涉嫌犯罪的
移送司法机关
依法追究刑事责任
64 定点保险 定点维修 定点加油
65 公务员 不受处分
非经法定程序
非经法定事由
67 加重处分
除(同案起次要作用)
69 从轻的条件
除(检举不确认,不能认定)
71 处分条例不是为了
约束政府人员的行为
73 除 正常会议支出
74 与被调查人有上下级领导关系 不需回避
76 除 按照程序作出公正处理
77 除 办公用户
不能统一核定价格
78 除 不完全符合相关规定 外
79 行政执法违法 处理
负有责任的领导人员
和直接责任人员
81 审计
除 公款娱乐(严禁 不能报销)
83 嫖娼 行政处罚
撤职 开除
84 违反 (法律 法规 规章 )给予行政处分
85 从轻的情节
除 承认错误态度较好
86 违反 (法定权限 法定条件 法定程序)给予撤职
(除 法定事由)
87 接待 不能增加陪同人数
89 严禁以任何形式(隐瞒 截留 挤占)坐支或者私分非税收
除(使用)
90 立案调查期间 不得(交流 出境 辞去公职)
96 出境期间 属我国机构 除私人自费宴请外 用公款皆不许
97 收回设备是可以的
98 除 配备必要的公务用车
100 可以购买正常服务
53 除公务学习交流
56 退休的,应依法
降级、撤职、开除的 降低岗位等级或取消待遇
58 处分
负有责任的领导人员
直接责任人员
60 解除处分条件
有悔改表现
没有再发生违纪违法行为
62 党政机关不得
新建 改建 扩建
63 违纪违法并涉嫌犯罪的
移送司法机关
依法追究刑事责任
64 定点保险 定点维修 定点加油
65 公务员 不受处分
非经法定程序
非经法定事由
67 加重处分
除(同案起次要作用)
69 从轻的条件
除(检举不确认,不能认定)
71 处分条例不是为了
约束政府人员的行为
73 除 正常会议支出
74 与被调查人有上下级领导关系 不需回避
76 除 按照程序作出公正处理
77 除 办公用户
不能统一核定价格
78 除 不完全符合相关规定 外
79 行政执法违法 处理
负有责任的领导人员
和直接责任人员
81 审计
除 公款娱乐(严禁 不能报销)
83 嫖娼 行政处罚
撤职 开除
84 违反 (法律 法规 规章 )给予行政处分
85 从轻的情节
除 承认错误态度较好
86 违反 (法定权限 法定条件 法定程序)给予撤职
(除 法定事由)
87 接待 不能增加陪同人数
89 严禁以任何形式(隐瞒 截留 挤占)坐支或者私分非税收
除(使用)
90 立案调查期间 不得(交流 出境 辞去公职)
96 出境期间 属我国机构 除私人自费宴请外 用公款皆不许
97 收回设备是可以的
98 除 配备必要的公务用车
100 可以购买正常服务
53 除公务学习交流
56 退休的,应依法
降级、撤职、开除的 降低岗位等级或取消待遇
58 处分
负有责任的领导人员
直接责任人员
60 解除处分条件
有悔改表现
没有再发生违纪违法行为
62 党政机关不得
新建 改建 扩建
63 违纪违法并涉嫌犯罪的
移送司法机关
依法追究刑事责任
64 定点保险 定点维修 定点加油
65 公务员 不受处分
非经法定程序
非经法定事由
67 加重处分
除(同案起次要作用)
69 从轻的条件
除(检举不确认,不能认定)
71 处分条例不是为了
约束政府人员的行为
73 除 正常会议支出
74 与被调查人有上下级领导关系 不需回避
76 除 按照程序作出公正处理
77 除 办公用户
不能统一核定价格
78 除 不完全符合相关规定 外
79 行政执法违法 处理
负有责任的领导人员
和直接责任人员
81 审计
除 公款娱乐(严禁 不能报销)
83 嫖娼 行政处罚
撤职 开除
84 违反 (法律 法规 规章 )给予行政处分
85 从轻的情节
除 承认错误态度较好
86 违反 (法定权限 法定条件 法定程序)给予撤职
(除 法定事由)
87 接待 不能增加陪同人数
89 严禁以任何形式(隐瞒 截留 挤占)坐支或者私分非税收
除(使用)
90 立案调查期间 不得(交流 出境 辞去公职)
96 出境期间 属我国机构 除私人自费宴请外 用公款皆不许
97 收回设备是可以的
98 除 配备必要的公务用车
100 可以购买正常服务
53 除公务学习交流
56 退休的,应依法
降级、撤职、开除的 降低岗位等级或取消待遇
58 处分
负有责任的领导人员
直接责任人员
60 解除处分条件
有悔改表现
没有再发生违纪违法行为
62 党政机关不得
新建 改建 扩建
63 违纪违法并涉嫌犯罪的
移送司法机关
依法追究刑事责任
64 定点保险 定点维修 定点加油
65 公务员 不受处分
非经法定程序
非经法定事由
67 加重处分
除(同案起次要作用)
69 从轻的条件
除(检举不确认,不能认定)
71 处分条例不是为了
约束政府人员的行为
73 除 正常会议支出
74 与被调查人有上下级领导关系 不需回避
76 除 按照程序作出公正处理
77 除 办公用户
不能统一核定价格
78 除 不完全符合相关规定 外
79 行政执法违法 处理
负有责任的领导人员
和直接责任人员
81 审计
除 公款娱乐(严禁 不能报销)
83 嫖娼 行政处罚
撤职 开除
84 违反 (法律 法规 规章 )给予行政处分
85 从轻的情节
除 承认错误态度较好
86 违反 (法定权限 法定条件 法定程序)给予撤职
(除 法定事由)
87 接待 不能增加陪同人数
89 严禁以任何形式(隐瞒 截留 挤占)坐支或者私分非税收
除(使用)
90 立案调查期间 不得(交流 出境 辞去公职)
96 出境期间 属我国机构 除私人自费宴请外 用公款皆不许
97 收回设备是可以的
98 除 配备必要的公务用车
100 可以购买正常服务
53 除公务学习交流
56 退休的,应依法
降级、撤职、开除的 降低岗位等级或取消待遇
58 处分
负有责任的领导人员
直接责任人员
60 解除处分条件
有悔改表现
没有再发生违纪违法行为
62 党政机关不得
新建 改建 扩建
63 违纪违法并涉嫌犯罪的
移送司法机关
依法追究刑事责任
64 定点保险 定点维修 定点加油
65 公务员 不受处分
非经法定程序
非经法定事由
67 加重处分
除(同案起次要作用)
69 从轻的条件
除(检举不确认,不能认定)
71 处分条例不是为了
约束政府人员的行为
73 除 正常会议支出
74 与被调查人有上下级领导关系 不需回避
76 除 按照程序作出公正处理
77 除 办公用户
不能统一核定价格
78 除 不完全符合相关规定 外
79 行政执法违法 处理
负有责任的领导人员
和直接责任人员
81 审计
除 公款娱乐(严禁 不能报销)
83 嫖娼 行政处罚
撤职 开除
84 违反 (法律 法规 规章 )给予行政处分
85 从轻的情节
除 承认错误态度较好
86 违反 (法定权限 法定条件 法定程序)给予撤职
(除 法定事由)
87 接待 不能增加陪同人数
89 严禁以任何形式(隐瞒 截留 挤占)坐支或者私分非税收
除(使用)
90 立案调查期间 不得(交流 出境 辞去公职)
96 出境期间 属我国机构 除私人自费宴请外 用公款皆不许
97 收回设备是可以的
98 除 配备必要的公务用车
100 可以购买正常服务
53 除公务学习交流
56 退休的,应依法
降级、撤职、开除的 降低岗位等级或取消待遇
58 处分
负有责任的领导人员
直接责任人员
60 解除处分条件
有悔改表现
没有再发生违纪违法行为
62 党政机关不得
新建 改建 扩建
63 违纪违法并涉嫌犯罪的
移送司法机关
依法追究刑事责任
64 定点保险 定点维修 定点加油
65 公务员 不受处分
非经法定程序
非经法定事由
67 加重处分
除(同案起次要作用)
69 从轻的条件
除(检举不确认,不能认定)
71 处分条例不是为了
约束政府人员的行为
73 除 正常会议支出
74 与被调查人有上下级领导关系 不需回避
76 除 按照程序作出公正处理
77 除 办公用户
不能统一核定价格
78 除 不完全符合相关规定 外
79 行政执法违法 处理
负有责任的领导人员
和直接责任人员
81 审计
除 公款娱乐(严禁 不能报销)
83 嫖娼 行政处罚
撤职 开除
84 违反 (法律 法规 规章 )给予行政处分
85 从轻的情节
除 承认错误态度较好
86 违反 (法定权限 法定条件 法定程序)给予撤职
(除 法定事由)
87 接待 不能增加陪同人数
89 严禁以任何形式(隐瞒 截留 挤占)坐支或者私分非税收
除(使用)
90 立案调查期间 不得(交流 出境 辞去公职)
96 出境期间 属我国机构 除私人自费宴请外 用公款皆不许
97 收回设备是可以的
98 除 配备必要的公务用车
100 可以购买正常服务
53 除公务学习交流
56 退休的,应依法
降级、撤职、开除的 降低岗位等级或取消待遇
58 处分
负有责任的领导人员
直接责任人员
60 解除处分条件
有悔改表现
没有再发生违纪违法行为
62 党政机关不得
新建 改建 扩建
63 违纪违法并涉嫌犯罪的
移送司法机关
依法追究刑事责任
64 定点保险 定点维修 定点加油
65 公务员 不受处分
非经法定程序
非经法定事由
67 加重处分
除(同案起次要作用)
69 从轻的条件
除(检举不确认,不能认定)
71 处分条例不是为了
约束政府人员的行为
73 除 正常会议支出
74 与被调查人有上下级领导关系 不需回避
76 除 按照程序作出公正处理
77 除 办公用户
不能统一核定价格
78 除 不完全符合相关规定 外
79 行政执法违法 处理
负有责任的领导人员
和直接责任人员
81 审计
除 公款娱乐(严禁 不能报销)
83 嫖娼 行政处罚
撤职 开除
84 违反 (法律 法规 规章 )给予行政处分
85 从轻的情节
除 承认错误态度较好
86 违反 (法定权限 法定条件 法定程序)给予撤职
(除 法定事由)
87 接待 不能增加陪同人数
89 严禁以任何形式(隐瞒 截留 挤占)坐支或者私分非税收
除(使用)
90 立案调查期间 不得(交流 出境 辞去公职)
96 出境期间 属我国机构 除私人自费宴请外 用公款皆不许
97 收回设备是可以的
98 除 配备必要的公务用车
100 可以购买正常服务
53 除公务学习交流
56 退休的,应依法
降级、撤职、开除的 降低岗位等级或取消待遇
58 处分
负有责任的领导人员
直接责任人员
60 解除处分条件
有悔改表现
没有再发生违纪违法行为
62 党政机关不得
新建 改建 扩建
63 违纪违法并涉嫌犯罪的
移送司法机关
依法追究刑事责任
64 定点保险 定点维修 定点加油
65 公务员 不受处分
非经法定程序
非经法定事由
67 加重处分
除(同案起次要作用)
69 从轻的条件
除(检举不确认,不能认定)
71 处分条例不是为了
约束政府人员的行为
73 除 正常会议支出
74 与被调查人有上下级领导关系 不需回避
76 除 按照程序作出公正处理
77 除 办公用户
不能统一核定价格
78 除 不完全符合相关规定 外
79 行政执法违法 处理
负有责任的领导人员
和直接责任人员
81 审计
除 公款娱乐(严禁 不能报销)
83 嫖娼 行政处罚
撤职 开除
84 违反 (法律 法规 规章 )给予行政处分
85 从轻的情节
除 承认错误态度较好
86 违反 (法定权限 法定条件 法定程序)给予撤职
(除 法定事由)
87 接待 不能增加陪同人数
89 严禁以任何形式(隐瞒 截留 挤占)坐支或者私分非税收
除(使用)
90 立案调查期间 不得(交流 出境 辞去公职)
96 出境期间 属我国机构 除私人自费宴请外 用公款皆不许
97 收回设备是可以的
98 除 配备必要的公务用车
100 可以购买正常服务
53 除公务学习交流
56 退休的,应依法
降级、撤职、开除的 降低岗位等级或取消待遇
58 处分
负有责任的领导人员
直接责任人员
60 解除处分条件
有悔改表现
没有再发生违纪违法行为
62 党政机关不得
新建 改建 扩建
63 违纪违法并涉嫌犯罪的
移送司法机关
依法追究刑事责任
64 定点保险 定点维修 定点加油
65 公务员 不受处分
非经法定程序
非经法定事由
67 加重处分
除(同案起次要作用)
69 从轻的条件
除(检举不确认,不能认定)
71 处分条例不是为了
约束政府人员的行为
73 除 正常会议支出
74 与被调查人有上下级领导关系 不需回避
76 除 按照程序作出公正处理
77 除 办公用户
不能统一核定价格
78 除 不完全符合相关规定 外
79 行政执法违法 处理
负有责任的领导人员
和直接责任人员
81 审计
除 公款娱乐(严禁 不能报销)
83 嫖娼 行政处罚
撤职 开除
84 违反 (法律 法规 规章 )给予行政处分
85 从轻的情节
除 承认错误态度较好
86 违反 (法定权限 法定条件 法定程序)给予撤职
(除 法定事由)
87 接待 不能增加陪同人数
89 严禁以任何形式(隐瞒 截留 挤占)坐支或者私分非税收
除(使用)
90 立案调查期间 不得(交流 出境 辞去公职)
96 出境期间 属我国机构 除私人自费宴请外 用公款皆不许
97 收回设备是可以的
98 除 配备必要的公务用车
100 可以购买正常服务
53 除公务学习交流
56 退休的,应依法
降级、撤职、开除的 降低岗位等级或取消待遇
58 处分
负有责任的领导人员
直接责任人员
60 解除处分条件
有悔改表现
没有再发生违纪违法行为
62 党政机关不得
新建 改建 扩建
63 违纪违法并涉嫌犯罪的
移送司法机关
依法追究刑事责任
64 定点保险 定点维修 定点加油
65 公务员 不受处分
非经法定程序
非经法定事由
67 加重处分
除(同案起次要作用)
69 从轻的条件
除(检举不确认,不能认定)
71 处分条例不是为了
约束政府人员的行为
73 除 正常会议支出
74 与被调查人有上下级领导关系 不需回避
76 除 按照程序作出公正处理
77 除 办公用户
不能统一核定价格
78 除 不完全符合相关规定 外
79 行政执法违法 处理
负有责任的领导人员
和直接责任人员
81 审计
除 公款娱乐(严禁 不能报销)
83 嫖娼 行政处罚
撤职 开除
84 违反 (法律 法规 规章 )给予行政处分
85 从轻的情节
除 承认错误态度较好
86 违反 (法定权限 法定条件 法定程序)给予撤职
(除 法定事由)
87 接待 不能增加陪同人数
89 严禁以任何形式(隐瞒 截留 挤占)坐支或者私分非税收
除(使用)
90 立案调查期间 不得(交流 出境 辞去公职)
96 出境期间 属我国机构 除私人自费宴请外 用公款皆不许
97 收回设备是可以的
98 除 配备必要的公务用车
100 可以购买正常服务
53 除公务学习交流
56 退休的,应依法
降级、撤职、开除的 降低岗位等级或取消待遇
58 处分
负有责任的领导人员
直接责任人员
60 解除处分条件
有悔改表现
没有再发生违纪违法行为
62 党政机关不得
新建 改建 扩建
63 违纪违法并涉嫌犯罪的
移送司法机关
依法追究刑事责任
64 定点保险 定点维修 定点加油
65 公务员 不受处分
非经法定程序
非经法定事由
67 加重处分
除(同案起次要作用)
69 从轻的条件
除(检举不确认,不能认定)
71 处分条例不是为了
约束政府人员的行为
73 除 正常会议支出
74 与被调查人有上下级领导关系 不需回避
76 除 按照程序作出公正处理
77 除 办公用户
不能统一核定价格
78 除 不完全符合相关规定 外
79 行政执法违法 处理
负有责任的领导人员
和直接责任人员
81 审计
除 公款娱乐(严禁 不能报销)
83 嫖娼 行政处罚
撤职 开除
84 违反 (法律 法规 规章 )给予行政处分
85 从轻的情节
除 承认错误态度较好
86 违反 (法定权限 法定条件 法定程序)给予撤职
(除 法定事由)
87 接待 不能增加陪同人数
89 严禁以任何形式(隐瞒 截留 挤占)坐支或者私分非税收
除(使用)
90 立案调查期间 不得(交流 出境 辞去公职)
96 出境期间 属我国机构 除私人自费宴请外 用公款皆不许
97 收回设备是可以的
98 除 配备必要的公务用车
100 可以购买正常服务
53 除公务学习交流
56 退休的,应依法
降级、撤职、开除的 降低岗位等级或取消待遇
58 处分
负有责任的领导人员
直接责任人员
60 解除处分条件
有悔改表现
没有再发生违纪违法行为
62 党政机关不得
新建 改建 扩建
63 违纪违法并涉嫌犯罪的
移送司法机关
依法追究刑事责任
64 定点保险 定点维修 定点加油
65 公务员 不受处分
非经法定程序
非经法定事由
67 加重处分
除(同案起次要作用)
69 从轻的条件
除(检举不确认,不能认定)
71 处分条例不是为了
约束政府人员的行为
73 除 正常会议支出
74 与被调查人有上下级领导关系 不需回避
76 除 按照程序作出公正处理
77 除 办公用户
不能统一核定价格
78 除 不完全符合相关规定 外
79 行政执法违法 处理
负有责任的领导人员
和直接责任人员
81 审计
除 公款娱乐(严禁 不能报销)
83 嫖娼 行政处罚
撤职 开除
84 违反 (法律 法规 规章 )给予行政处分
85 从轻的情节
除 承认错误态度较好
86 违反 (法定权限 法定条件 法定程序)给予撤职
(除 法定事由)
87 接待 不能增加陪同人数
89 严禁以任何形式(隐瞒 截留 挤占)坐支或者私分非税收
除(使用)
90 立案调查期间 不得(交流 出境 辞去公职)
96 出境期间 属我国机构 除私人自费宴请外 用公款皆不许
97 收回设备是可以的
98 除 配备必要的公务用车
100 可以购买正常服务
53 除公务学习交流
56 退休的,应依法
降级、撤职、开除的 降低岗位等级或取消待遇
58 处分
负有责任的领导人员
直接责任人员
60 解除处分条件
有悔改表现
没有再发生违纪违法行为
62 党政机关不得
新建 改建 扩建
63 违纪违法并涉嫌犯罪的
移送司法机关
依法追究刑事责任
64 定点保险 定点维修 定点加油
65 公务员 不受处分
非经法定程序
非经法定事由
67 加重处分
除(同案起次要作用)
69 从轻的条件
除(检举不确认,不能认定)
71 处分条例不是为了
约束政府人员的行为
73 除 正常会议支出
74 与被调查人有上下级领导关系 不需回避
76 除 按照程序作出公正处理
77 除 办公用户
不能统一核定价格
78 除 不完全符合相关规定 外
79 行政执法违法 处理
负有责任的领导人员
和直接责任人员
81 审计
除 公款娱乐(严禁 不能报销)
83 嫖娼 行政处罚
撤职 开除
84 违反 (法律 法规 规章 )给予行政处分
85 从轻的情节
除 承认错误态度较好
86 违反 (法定权限 法定条件 法定程序)给予撤职
(除 法定事由)
87 接待 不能增加陪同人数
89 严禁以任何形式(隐瞒 截留 挤占)坐支或者私分非税收
除(使用)
90 立案调查期间 不得(交流 出境 辞去公职)
96 出境期间 属我国机构 除私人自费宴请外 用公款皆不许
97 收回设备是可以的
98 除 配备必要的公务用车
100 可以购买正常服务
53 除公务学习交流
56 退休的,应依法
降级、撤职、开除的 降低岗位等级或取消待遇
58 处分
负有责任的领导人员
直接责任人员
60 解除处分条件
有悔改表现
没有再发生违纪违法行为
62 党政机关不得
新建 改建 扩建
63 违纪违法并涉嫌犯罪的
移送司法机关
依法追究刑事责任
64 定点保险 定点维修 定点加油
65 公务员 不受处分
非经法定程序
非经法定事由
67 加重处分
除(同案起次要作用)
69 从轻的条件
除(检举不确认,不能认定)
71 处分条例不是为了
约束政府人员的行为
73 除 正常会议支出
74 与被调查人有上下级领导关系 不需回避
76 除 按照程序作出公正处理
77 除 办公用户
不能统一核定价格
78 除 不完全符合相关规定 外
79 行政执法违法 处理
负有责任的领导人员
和直接责任人员
81 审计
除 公款娱乐(严禁 不能报销)
83 嫖娼 行政处罚
撤职 开除
84 违反 (法律 法规 规章 )给予行政处分
85 从轻的情节
除 承认错误态度较好
86 违反 (法定权限 法定条件 法定程序)给予撤职
(除 法定事由)
87 接待 不能增加陪同人数
89 严禁以任何形式(隐瞒 截留 挤占)坐支或者私分非税收
除(使用)
90 立案调查期间 不得(交流 出境 辞去公职)
96 出境期间 属我国机构 除私人自费宴请外 用公款皆不许
97 收回设备是可以的
98 除 配备必要的公务用车
100 可以购买正常服务
53 除公务学习交流
56 退休的,应依法
降级、撤职、开除的 降低岗位等级或取消待遇
58 处分
负有责任的领导人员
直接责任人员
60 解除处分条件
有悔改表现
没有再发生违纪违法行为
62 党政机关不得
新建 改建 扩建
63 违纪违法并涉嫌犯罪的
移送司法机关
依法追究刑事责任
64 定点保险 定点维修 定点加油
65 公务员 不受处分
非经法定程序
非经法定事由
67 加重处分
除(同案起次要作用)
69 从轻的条件
除(检举不确认,不能认定)
71 处分条例不是为了
约束政府人员的行为
73 除 正常会议支出
74 与被调查人有上下级领导关系 不需回避
76 除 按照程序作出公正处理
77 除 办公用户
不能统一核定价格
78 除 不完全符合相关规定 外
79 行政执法违法 处理
负有责任的领导人员
和直接责任人员
81 审计
除 公款娱乐(严禁 不能报销)
83 嫖娼 行政处罚
撤职 开除
84 违反 (法律 法规 规章 )给予行政处分
85 从轻的情节
除 承认错误态度较好
86 违反 (法定权限 法定条件 法定程序)给予撤职
(除 法定事由)
87 接待 不能增加陪同人数
89 严禁以任何形式(隐瞒 截留 挤占)坐支或者私分非税收
除(使用)
90 立案调查期间 不得(交流 出境 辞去公职)
96 出境期间 属我国机构 除私人自费宴请外 用公款皆不许
97 收回设备是可以的
98 除 配备必要的公务用车
100 可以购买正常服务
53 除公务学习交流
56 退休的,应依法
降级、撤职、开除的 降低岗位等级或取消待遇
58 处分
负有责任的领导人员
直接责任人员
60 解除处分条件
有悔改表现
没有再发生违纪违法行为
62 党政机关不得
新建 改建 扩建
63 违纪违法并涉嫌犯罪的
移送司法机关
依法追究刑事责任
64 定点保险 定点维修 定点加油
65 公务员 不受处分
非经法定程序
非经法定事由
67 加重处分
除(同案起次要作用)
69 从轻的条件
除(检举不确认,不能认定)
71 处分条例不是为了
约束政府人员的行为
73 除 正常会议支出
74 与被调查人有上下级领导关系 不需回避
76 除 按照程序作出公正处理
77 除 办公用户
不能统一核定价格
78 除 不完全符合相关规定 外
79 行政执法违法 处理
负有责任的领导人员
和直接责任人员
81 审计
除 公款娱乐(严禁 不能报销)
83 嫖娼 行政处罚
撤职 开除
84 违反 (法律 法规 规章 )给予行政处分
85 从轻的情节
除 承认错误态度较好
86 违反 (法定权限 法定条件 法定程序)给予撤职
(除 法定事由)
87 接待 不能增加陪同人数
89 严禁以任何形式(隐瞒 截留 挤占)坐支或者私分非税收
除(使用)
90 立案调查期间 不得(交流 出境 辞去公职)
96 出境期间 属我国机构 除私人自费宴请外 用公款皆不许
97 收回设备是可以的
98 除 配备必要的公务用车
100 可以购买正常服务
53 除公务学习交流
56 退休的,应依法
降级、撤职、开除的 降低岗位等级或取消待遇
58 处分
负有责任的领导人员
直接责任人员
60 解除处分条件
有悔改表现
没有再发生违纪违法行为
62 党政机关不得
新建 改建 扩建
63 违纪违法并涉嫌犯罪的
移送司法机关
依法追究刑事责任
64 定点保险 定点维修 定点加油
65 公务员 不受处分
非经法定程序
非经法定事由
67 加重处分
除(同案起次要作用)
69 从轻的条件
除(检举不确认,不能认定)
71 处分条例不是为了
约束政府人员的行为
73 除 正常会议支出
74 与被调查人有上下级领导关系 不需回避
76 除 按照程序作出公正处理
77 除 办公用户
不能统一核定价格
78 除 不完全符合相关规定 外
79 行政执法违法 处理
负有责任的领导人员
和直接责任人员
81 审计
除 公款娱乐(严禁 不能报销)
83 嫖娼 行政处罚
撤职 开除
84 违反 (法律 法规 规章 )给予行政处分
85 从轻的情节
除 承认错误态度较好
86 违反 (法定权限 法定条件 法定程序)给予撤职
(除 法定事由)
87 接待 不能增加陪同人数
89 严禁以任何形式(隐瞒 截留 挤占)坐支或者私分非税收
除(使用)
90 立案调查期间 不得(交流 出境 辞去公职)
96 出境期间 属我国机构 除私人自费宴请外 用公款皆不许
97 收回设备是可以的
98 除 配备必要的公务用车
100 可以购买正常服务
53 除公务学习交流
56 退休的,应依法
降级、撤职、开除的 降低岗位等级或取消待遇
58 处分
负有责任的领导人员
直接责任人员
60 解除处分条件
有悔改表现
没有再发生违纪违法行为
62 党政机关不得
新建 改建 扩建
63 违纪违法并涉嫌犯罪的
移送司法机关
依法追究刑事责任
64 定点保险 定点维修 定点加油
65 公务员 不受处分
非经法定程序
非经法定事由
67 加重处分
除(同案起次要作用)
69 从轻的条件
除(检举不确认,不能认定)
71 处分条例不是为了
约束政府人员的行为
73 除 正常会议支出
74 与被调查人有上下级领导关系 不需回避
76 除 按照程序作出公正处理
77 除 办公用户
不能统一核定价格
78 除 不完全符合相关规定 外
79 行政执法违法 处理
负有责任的领导人员
和直接责任人员
81 审计
除 公款娱乐(严禁 不能报销)
83 嫖娼 行政处罚
撤职 开除
84 违反 (法律 法规 规章 )给予行政处分
85 从轻的情节
除 承认错误态度较好
86 违反 (法定权限 法定条件 法定程序)给予撤职
(除 法定事由)
87 接待 不能增加陪同人数
89 严禁以任何形式(隐瞒 截留 挤占)坐支或者私分非税收
除(使用)
90 立案调查期间 不得(交流 出境 辞去公职)
96 出境期间 属我国机构 除私人自费宴请外 用公款皆不许
97 收回设备是可以的
98 除 配备必要的公务用车
100 可以购买正常服务
53 除公务学习交流
56 退休的,应依法
降级、撤职、开除的 降低岗位等级或取消待遇
58 处分
负有责任的领导人员
直接责任人员
60 解除处分条件
有悔改表现
没有再发生违纪违法行为
62 党政机关不得
新建 改建 扩建
63 违纪违法并涉嫌犯罪的
移送司法机关
依法追究刑事责任
64 定点保险 定点维修 定点加油
65 公务员 不受处分
非经法定程序
非经法定事由
67 加重处分
除(同案起次要作用)
69 从轻的条件
除(检举不确认,不能认定)
71 处分条例不是为了
约束政府人员的行为
73 除 正常会议支出
74 与被调查人有上下级领导关系 不需回避
76 除 按照程序作出公正处理
77 除 办公用户
不能统一核定价格
78 除 不完全符合相关规定 外
79 行政执法违法 处理
负有责任的领导人员
和直接责任人员
81 审计
除 公款娱乐(严禁 不能报销)
83 嫖娼 行政处罚
撤职 开除
84 违反 (法律 法规 规章 )给予行政处分
85 从轻的情节
除 承认错误态度较好
86 违反 (法定权限 法定条件 法定程序)给予撤职
(除 法定事由)
87 接待 不能增加陪同人数
89 严禁以任何形式(隐瞒 截留 挤占)坐支或者私分非税收
除(使用)
90 立案调查期间 不得(交流 出境 辞去公职)
96 出境期间 属我国机构 除私人自费宴请外 用公款皆不许
97 收回设备是可以的
98 除 配备必要的公务用车
100 可以购买正常服务
53 除公务学习交流
56 退休的,应依法
降级、撤职、开除的 降低岗位等级或取消待遇
58 处分
负有责任的领导人员
直接责任人员
60 解除处分条件
有悔改表现
没有再发生违纪违法行为
62 党政机关不得
新建 改建 扩建
63 违纪违法并涉嫌犯罪的
移送司法机关
依法追究刑事责任
64 定点保险 定点维修 定点加油
65 公务员 不受处分
非经法定程序
非经法定事由
67 加重处分
除(同案起次要作用)
69 从轻的条件
除(检举不确认,不能认定)
71 处分条例不是为了
约束政府人员的行为
73 除 正常会议支出
74 与被调查人有上下级领导关系 不需回避
76 除 按照程序作出公正处理
77 除 办公用户
不能统一核定价格
78 除 不完全符合相关规定 外
79 行政执法违法 处理
负有责任的领导人员
和直接责任人员
81 审计
除 公款娱乐(严禁 不能报销)
83 嫖娼 行政处罚
撤职 开除
84 违反 (法律 法规 规章 )给予行政处分
85 从轻的情节
除 承认错误态度较好
86 违反 (法定权限 法定条件 法定程序)给予撤职
(除 法定事由)
87 接待 不能增加陪同人数
89 严禁以任何形式(隐瞒 截留 挤占)坐支或者私分非税收
除(使用)
90 立案调查期间 不得(交流 出境 辞去公职)
96 出境期间 属我国机构 除私人自费宴请外 用公款皆不许
97 收回设备是可以的
98 除 配备必要的公务用车
100 可以购买正常服务
53 除公务学习交流
56 退休的,应依法
降级、撤职、开除的 降低岗位等级或取消待遇
58 处分
负有责任的领导人员
直接责任人员
60 解除处分条件
有悔改表现
没有再发生违纪违法行为
62 党政机关不得
新建 改建 扩建
63 违纪违法并涉嫌犯罪的
移送司法机关
依法追究刑事责任
64 定点保险 定点维修 定点加油
65 公务员 不受处分
非经法定程序
非经法定事由
67 加重处分
除(同案起次要作用)
69 从轻的条件
除(检举不确认,不能认定)
71 处分条例不是为了
约束政府人员的行为
73 除 正常会议支出
74 与被调查人有上下级领导关系 不需回避
76 除 按照程序作出公正处理
77 除 办公用户
不能统一核定价格
78 除 不完全符合相关规定 外
79 行政执法违法 处理
负有责任的领导人员
和直接责任人员
81 审计
除 公款娱乐(严禁 不能报销)
83 嫖娼 行政处罚
撤职 开除
84 违反 (法律 法规 规章 )给予行政处分
85 从轻的情节
除 承认错误态度较好
86 违反 (法定权限 法定条件 法定程序)给予撤职
(除 法定事由)
87 接待 不能增加陪同人数
89 严禁以任何形式(隐瞒 截留 挤占)坐支或者私分非税收
除(使用)
90 立案调查期间 不得(交流 出境 辞去公职)
96 出境期间 属我国机构 除私人自费宴请外 用公款皆不许
97 收回设备是可以的
98 除 配备必要的公务用车
100 可以购买正常服务
53 除公务学习交流
56 退休的,应依法
降级、撤职、开除的 降低岗位等级或取消待遇
58 处分
负有责任的领导人员
直接责任人员
60 解除处分条件
有悔改表现
没有再发生违纪违法行为
62 党政机关不得
新建 改建 扩建
63 违纪违法并涉嫌犯罪的
移送司法机关
依法追究刑事责任
64 定点保险 定点维修 定点加油
65 公务员 不受处分
非经法定程序
非经法定事由
67 加重处分
除(同案起次要作用)
69 从轻的条件
除(检举不确认,不能认定)
71 处分条例不是为了
约束政府人员的行为
73 除 正常会议支出
74 与被调查人有上下级领导关系 不需回避
76 除 按照程序作出公正处理
77 除 办公用户
不能统一核定价格
78 除 不完全符合相关规定 外
79 行政执法违法 处理
负有责任的领导人员
和直接责任人员
81 审计
除 公款娱乐(严禁 不能报销)
83 嫖娼 行政处罚
撤职 开除
84 违反 (法律 法规 规章 )给予行政处分
85 从轻的情节
除 承认错误态度较好
86 违反 (法定权限 法定条件 法定程序)给予撤职
(除 法定事由)
87 接待 不能增加陪同人数
89 严禁以任何形式(隐瞒 截留 挤占)坐支或者私分非税收
除(使用)
90 立案调查期间 不得(交流 出境 辞去公职)
96 出境期间 属我国机构 除私人自费宴请外 用公款皆不许
97 收回设备是可以的
98 除 配备必要的公务用车
100 可以购买正常服务
53 除公务学习交流
56 退休的,应依法
降级、撤职、开除的 降低岗位等级或取消待遇
58 处分
负有责任的领导人员
直接责任人员
60 解除处分条件
有悔改表现
没有再发生违纪违法行为
62 党政机关不得
新建 改建 扩建
63 违纪违法并涉嫌犯罪的
移送司法机关
依法追究刑事责任
64 定点保险 定点维修 定点加油
65 公务员 不受处分
非经法定程序
非经法定事由
67 加重处分
除(同案起次要作用)
69 从轻的条件
除(检举不确认,不能认定)
71 处分条例不是为了
约束政府人员的行为
73 除 正常会议支出
74 与被调查人有上下级领导关系 不需回避
76 除 按照程序作出公正处理
77 除 办公用户
不能统一核定价格
78 除 不完全符合相关规定 外
79 行政执法违法 处理
负有责任的领导人员
和直接责任人员
81 审计
除 公款娱乐(严禁 不能报销)
83 嫖娼 行政处罚
撤职 开除
84 违反 (法律 法规 规章 )给予行政处分
85 从轻的情节
除 承认错误态度较好
86 违反 (法定权限 法定条件 法定程序)给予撤职
(除 法定事由)
87 接待 不能增加陪同人数
89 严禁以任何形式(隐瞒 截留 挤占)坐支或者私分非税收
除(使用)
90 立案调查期间 不得(交流 出境 辞去公职)
96 出境期间 属我国机构 除私人自费宴请外 用公款皆不许
97 收回设备是可以的
98 除 配备必要的公务用车
100 可以购买正常服务
53 除公务学习交流
56 退休的,应依法
降级、撤职、开除的 降低岗位等级或取消待遇
58 处分
负有责任的领导人员
直接责任人员
60 解除处分条件
有悔改表现
没有再发生违纪违法行为
62 党政机关不得
新建 改建 扩建
63 违纪违法并涉嫌犯罪的
移送司法机关
依法追究刑事责任
64 定点保险 定点维修 定点加油
65 公务员 不受处分
非经法定程序
非经法定事由
67 加重处分
除(同案起次要作用)
69 从轻的条件
除(检举不确认,不能认定)
71 处分条例不是为了
约束政府人员的行为
73 除 正常会议支出
74 与被调查人有上下级领导关系 不需回避
76 除 按照程序作出公正处理
77 除 办公用户
不能统一核定价格
78 除 不完全符合相关规定 外
79 行政执法违法 处理
负有责任的领导人员
和直接责任人员
81 审计
除 公款娱乐(严禁 不能报销)
83 嫖娼 行政处罚
撤职 开除
84 违反 (法律 法规 规章 )给予行政处分
85 从轻的情节
除 承认错误态度较好
86 违反 (法定权限 法定条件 法定程序)给予撤职
(除 法定事由)
87 接待 不能增加陪同人数
89 严禁以任何形式(隐瞒 截留 挤占)坐支或者私分非税收
除(使用)
90 立案调查期间 不得(交流 出境 辞去公职)
96 出境期间 属我国机构 除私人自费宴请外 用公款皆不许
97 收回设备是可以的
98 除 配备必要的公务用车
100 可以购买正常服务
53 除公务学习交流
56 退休的,应依法
降级、撤职、开除的 降低岗位等级或取消待遇
58 处分
负有责任的领导人员
直接责任人员
60 解除处分条件
有悔改表现
没有再发生违纪违法行为
62 党政机关不得
新建 改建 扩建
63 违纪违法并涉嫌犯罪的
移送司法机关
依法追究刑事责任
64 定点保险 定点维修 定点加油
65 公务员 不受处分
非经法定程序
非经法定事由
67 加重处分
除(同案起次要作用)
69 从轻的条件
除(检举不确认,不能认定)
71 处分条例不是为了
约束政府人员的行为
73 除 正常会议支出
74 与被调查人有上下级领导关系 不需回避
76 除 按照程序作出公正处理
77 除 办公用户
不能统一核定价格
78 除 不完全符合相关规定 外
79 行政执法违法 处理
负有责任的领导人员
和直接责任人员
81 审计
除 公款娱乐(严禁 不能报销)
83 嫖娼 行政处罚
撤职 开除
84 违反 (法律 法规 规章 )给予行政处分
85 从轻的情节
除 承认错误态度较好
86 违反 (法定权限 法定条件 法定程序)给予撤职
(除 法定事由)
87 接待 不能增加陪同人数
89 严禁以任何形式(隐瞒 截留 挤占)坐支或者私分非税收
除(使用)
90 立案调查期间 不得(交流 出境 辞去公职)
96 出境期间 属我国机构 除私人自费宴请外 用公款皆不许
97 收回设备是可以的
98 除 配备必要的公务用车
100 可以购买正常服务
53 除公务学习交流
56 退休的,应依法
降级、撤职、开除的 降低岗位等级或取消待遇
58 处分
负有责任的领导人员
直接责任人员
60 解除处分条件
有悔改表现
没有再发生违纪违法行为
62 党政机关不得
新建 改建 扩建
63 违纪违法并涉嫌犯罪的
移送司法机关
依法追究刑事责任
64 定点保险 定点维修 定点加油
65 公务员 不受处分
非经法定程序
非经法定事由
67 加重处分
除(同案起次要作用)
69 从轻的条件
除(检举不确认,不能认定)
71 处分条例不是为了
约束政府人员的行为
73 除 正常会议支出
74 与被调查人有上下级领导关系 不需回避
76 除 按照程序作出公正处理
77 除 办公用户
不能统一核定价格
78 除 不完全符合相关规定 外
79 行政执法违法 处理
负有责任的领导人员
和直接责任人员
81 审计
除 公款娱乐(严禁 不能报销)
83 嫖娼 行政处罚
撤职 开除
84 违反 (法律 法规 规章 )给予行政处分
85 从轻的情节
除 承认错误态度较好
86 违反 (法定权限 法定条件 法定程序)给予撤职
(除 法定事由)
87 接待 不能增加陪同人数
89 严禁以任何形式(隐瞒 截留 挤占)坐支或者私分非税收
除(使用)
90 立案调查期间 不得(交流 出境 辞去公职)
96 出境期间 属我国机构 除私人自费宴请外 用公款皆不许
97 收回设备是可以的
98 除 配备必要的公务用车
100 可以购买正常服务
53 除公务学习交流
56 退休的,应依法
降级、撤职、开除的 降低岗位等级或取消待遇
58 处分
负有责任的领导人员
直接责任人员
60 解除处分条件
有悔改表现
没有再发生违纪违法行为
62 党政机关不得
新建 改建 扩建
63 违纪违法并涉嫌犯罪的
移送司法机关
依法追究刑事责任
64 定点保险 定点维修 定点加油
65 公务员 不受处分
非经法定程序
非经法定事由
67 加重处分
除(同案起次要作用)
69 从轻的条件
除(检举不确认,不能认定)
71 处分条例不是为了
约束政府人员的行为
73 除 正常会议支出
74 与被调查人有上下级领导关系 不需回避
76 除 按照程序作出公正处理
77 除 办公用户
不能统一核定价格
78 除 不完全符合相关规定 外
79 行政执法违法 处理
负有责任的领导人员
和直接责任人员
81 审计
除 公款娱乐(严禁 不能报销)
83 嫖娼 行政处罚
撤职 开除
84 违反 (法律 法规 规章 )给予行政处分
85 从轻的情节
除 承认错误态度较好
86 违反 (法定权限 法定条件 法定程序)给予撤职
(除 法定事由)
87 接待 不能增加陪同人数
89 严禁以任何形式(隐瞒 截留 挤占)坐支或者私分非税收
除(使用)
90 立案调查期间 不得(交流 出境 辞去公职)
96 出境期间 属我国机构 除私人自费宴请外 用公款皆不许
97 收回设备是可以的
98 除 配备必要的公务用车
100 可以购买正常服务
53 除公务学习交流
56 退休的,应依法
降级、撤职、开除的 降低岗位等级或取消待遇
58 处分
负有责任的领导人员
直接责任人员
60 解除处分条件
有悔改表现
没有再发生违纪违法行为
62 党政机关不得
新建 改建 扩建
63 违纪违法并涉嫌犯罪的
移送司法机关
依法追究刑事责任
64 定点保险 定点维修 定点加油
65 公务员 不受处分
非经法定程序
非经法定事由
67 加重处分
除(同案起次要作用)
69 从轻的条件
除(检举不确认,不能认定)
71 处分条例不是为了
约束政府人员的行为
73 除 正常会议支出
74 与被调查人有上下级领导关系 不需回避
76 除 按照程序作出公正处理
77 除 办公用户
不能统一核定价格
78 除 不完全符合相关规定 外
79 行政执法违法 处理
负有责任的领导人员
和直接责任人员
81 审计
除 公款娱乐(严禁 不能报销)
83 嫖娼 行政处罚
撤职 开除
84 违反 (法律 法规 规章 )给予行政处分
85 从轻的情节
除 承认错误态度较好
86 违反 (法定权限 法定条件 法定程序)给予撤职
(除 法定事由)
87 接待 不能增加陪同人数
89 严禁以任何形式(隐瞒 截留 挤占)坐支或者私分非税收
除(使用)
90 立案调查期间 不得(交流 出境 辞去公职)
96 出境期间 属我国机构 除私人自费宴请外 用公款皆不许
97 收回设备是可以的
98 除 配备必要的公务用车
100 可以购买正常服务
53 除公务学习交流
56 退休的,应依法
降级、撤职、开除的 降低岗位等级或取消待遇
58 处分
负有责任的领导人员
直接责任人员
60 解除处分条件
有悔改表现
没有再发生违纪违法行为
62 党政机关不得
新建 改建 扩建
63 违纪违法并涉嫌犯罪的
移送司法机关
依法追究刑事责任
64 定点保险 定点维修 定点加油
65 公务员 不受处分
非经法定程序
非经法定事由
67 加重处分
除(同案起次要作用)
69 从轻的条件
除(检举不确认,不能认定)
71 处分条例不是为了
约束政府人员的行为
73 除 正常会议支出
74 与被调查人有上下级领导关系 不需回避
76 除 按照程序作出公正处理
77 除 办公用户
不能统一核定价格
78 除 不完全符合相关规定 外
79 行政执法违法 处理
负有责任的领导人员
和直接责任人员
81 审计
除 公款娱乐(严禁 不能报销)
83 嫖娼 行政处罚
撤职 开除
84 违反 (法律 法规 规章 )给予行政处分
85 从轻的情节
除 承认错误态度较好
86 违反 (法定权限 法定条件 法定程序)给予撤职
(除 法定事由)
87 接待 不能增加陪同人数
89 严禁以任何形式(隐瞒 截留 挤占)坐支或者私分非税收
除(使用)
90 立案调查期间 不得(交流 出境 辞去公职)
96 出境期间 属我国机构 除私人自费宴请外 用公款皆不许
97 收回设备是可以的
98 除 配备必要的公务用车
100 可以购买正常服务
53 除公务学习交流
56 退休的,应依法
降级、撤职、开除的 降低岗位等级或取消待遇
58 处分
负有责任的领导人员
直接责任人员
60 解除处分条件
有悔改表现
没有再发生违纪违法行为
62 党政机关不得
新建 改建 扩建
63 违纪违法并涉嫌犯罪的
移送司法机关
依法追究刑事责任
64 定点保险 定点维修 定点加油
65 公务员 不受处分
非经法定程序
非经法定事由
67 加重处分
除(同案起次要作用)
69 从轻的条件
除(检举不确认,不能认定)
71 处分条例不是为了
约束政府人员的行为
73 除 正常会议支出
74 与被调查人有上下级领导关系 不需回避
76 除 按照程序作出公正处理
77 除 办公用户
不能统一核定价格
78 除 不完全符合相关规定 外
79 行政执法违法 处理
负有责任的领导人员
和直接责任人员
81 审计
除 公款娱乐(严禁 不能报销)
83 嫖娼 行政处罚
撤职 开除
84 违反 (法律 法规 规章 )给予行政处分
85 从轻的情节
除 承认错误态度较好
86 违反 (法定权限 法定条件 法定程序)给予撤职
(除 法定事由)
87 接待 不能增加陪同人数
89 严禁以任何形式(隐瞒 截留 挤占)坐支或者私分非税收
除(使用)
90 立案调查期间 不得(交流 出境 辞去公职)
96 出境期间 属我国机构 除私人自费宴请外 用公款皆不许
97 收回设备是可以的
98 除 配备必要的公务用车
100 可以购买正常服务
53 除公务学习交流
56 退休的,应依法
降级、撤职、开除的 降低岗位等级或取消待遇
58 处分
负有责任的领导人员
直接责任人员
60 解除处分条件
有悔改表现
没有再发生违纪违法行为
62 党政机关不得
新建 改建 扩建
63 违纪违法并涉嫌犯罪的
移送司法机关
依法追究刑事责任
64 定点保险 定点维修 定点加油
65 公务员 不受处分
非经法定程序
非经法定事由
67 加重处分
除(同案起次要作用)
69 从轻的条件
除(检举不确认,不能认定)
71 处分条例不是为了
约束政府人员的行为
73 除 正常会议支出
74 与被调查人有上下级领导关系 不需回避
76 除 按照程序作出公正处理
77 除 办公用户
不能统一核定价格
78 除 不完全符合相关规定 外
79 行政执法违法 处理
负有责任的领导人员
和直接责任人员
81 审计
除 公款娱乐(严禁 不能报销)
83 嫖娼 行政处罚
撤职 开除
84 违反 (法律 法规 规章 )给予行政处分
85 从轻的情节
除 承认错误态度较好
86 违反 (法定权限 法定条件 法定程序)给予撤职
(除 法定事由)
87 接待 不能增加陪同人数
89 严禁以任何形式(隐瞒 截留 挤占)坐支或者私分非税收
除(使用)
90 立案调查期间 不得(交流 出境 辞去公职)
96 出境期间 属我国机构 除私人自费宴请外 用公款皆不许
97 收回设备是可以的
98 除 配备必要的公务用车
100 可以购买正常服务
53 除公务学习交流
56 退休的,应依法
降级、撤职、开除的 降低岗位等级或取消待遇
58 处分
负有责任的领导人员
直接责任人员
60 解除处分条件
有悔改表现
没有再发生违纪违法行为
62 党政机关不得
新建 改建 扩建
63 违纪违法并涉嫌犯罪的
移送司法机关
依法追究刑事责任
64 定点保险 定点维修 定点加油
65 公务员 不受处分
非经法定程序
非经法定事由
67 加重处分
除(同案起次要作用)
69 从轻的条件
除(检举不确认,不能认定)
71 处分条例不是为了
约束政府人员的行为
73 除 正常会议支出
74 与被调查人有上下级领导关系 不需回避
76 除 按照程序作出公正处理
77 除 办公用户
不能统一核定价格
78 除 不完全符合相关规定 外
79 行政执法违法 处理
负有责任的领导人员
和直接责任人员
81 审计
除 公款娱乐(严禁 不能报销)
83 嫖娼 行政处罚
撤职 开除
84 违反 (法律 法规 规章 )给予行政处分
85 从轻的情节
除 承认错误态度较好
86 违反 (法定权限 法定条件 法定程序)给予撤职
(除 法定事由)
87 接待 不能增加陪同人数
89 严禁以任何形式(隐瞒 截留 挤占)坐支或者私分非税收
除(使用)
90 立案调查期间 不得(交流 出境 辞去公职)
96 出境期间 属我国机构 除私人自费宴请外 用公款皆不许
97 收回设备是可以的
98 除 配备必要的公务用车
100 可以购买正常服务
53 除公务学习交流
56 退休的,应依法
降级、撤职、开除的 降低岗位等级或取消待遇
58 处分
负有责任的领导人员
直接责任人员
60 解除处分条件
有悔改表现
没有再发生违纪违法行为
62 党政机关不得
新建 改建 扩建
63 违纪违法并涉嫌犯罪的
移送司法机关
依法追究刑事责任
64 定点保险 定点维修 定点加油
65 公务员 不受处分
非经法定程序
非经法定事由
67 加重处分
除(同案起次要作用)
69 从轻的条件
除(检举不确认,不能认定)
71 处分条例不是为了
约束政府人员的行为
73 除 正常会议支出
74 与被调查人有上下级领导关系 不需回避
76 除 按照程序作出公正处理
77 除 办公用户
不能统一核定价格
78 除 不完全符合相关规定 外
79 行政执法违法 处理
负有责任的领导人员
和直接责任人员
81 审计
除 公款娱乐(严禁 不能报销)
83 嫖娼 行政处罚
撤职 开除
84 违反 (法律 法规 规章 )给予行政处分
85 从轻的情节
除 承认错误态度较好
86 违反 (法定权限 法定条件 法定程序)给予撤职
(除 法定事由)
87 接待 不能增加陪同人数
89 严禁以任何形式(隐瞒 截留 挤占)坐支或者私分非税收
除(使用)
90 立案调查期间 不得(交流 出境 辞去公职)
96 出境期间 属我国机构 除私人自费宴请外 用公款皆不许
97 收回设备是可以的
98 除 配备必要的公务用车
100 可以购买正常服务
53 除公务学习交流
56 退休的,应依法
降级、撤职、开除的 降低岗位等级或取消待遇
58 处分
负有责任的领导人员
直接责任人员
60 解除处分条件
有悔改表现
没有再发生违纪违法行为
62 党政机关不得
新建 改建 扩建
63 违纪违法并涉嫌犯罪的
移送司法机关
依法追究刑事责任
64 定点保险 定点维修 定点加油
65 公务员 不受处分
非经法定程序
非经法定事由
67 加重处分
除(同案起次要作用)
69 从轻的条件
除(检举不确认,不能认定)
71 处分条例不是为了
约束政府人员的行为
73 除 正常会议支出
74 与被调查人有上下级领导关系 不需回避
76 除 按照程序作出公正处理
77 除 办公用户
不能统一核定价格
78 除 不完全符合相关规定 外
79 行政执法违法 处理
负有责任的领导人员
和直接责任人员
81 审计
除 公款娱乐(严禁 不能报销)
83 嫖娼 行政处罚
撤职 开除
84 违反 (法律 法规 规章 )给予行政处分
85 从轻的情节
除 承认错误态度较好
86 违反 (法定权限 法定条件 法定程序)给予撤职
(除 法定事由)
87 接待 不能增加陪同人数
89 严禁以任何形式(隐瞒 截留 挤占)坐支或者私分非税收
除(使用)
90 立案调查期间 不得(交流 出境 辞去公职)
96 出境期间 属我国机构 除私人自费宴请外 用公款皆不许
97 收回设备是可以的
98 除 配备必要的公务用车
100 可以购买正常服务
53 除公务学习交流
56 退休的,应依法
降级、撤职、开除的 降低岗位等级或取消待遇
58 处分
负有责任的领导人员
直接责任人员
60 解除处分条件
有悔改表现
没有再发生违纪违法行为
62 党政机关不得
新建 改建 扩建
63 违纪违法并涉嫌犯罪的
移送司法机关
依法追究刑事责任
64 定点保险 定点维修 定点加油
65 公务员 不受处分
非经法定程序
非经法定事由
67 加重处分
除(同案起次要作用)
69 从轻的条件
除(检举不确认,不能认定)
71 处分条例不是为了
约束政府人员的行为
73 除 正常会议支出
74 与被调查人有上下级领导关系 不需回避
76 除 按照程序作出公正处理
77 除 办公用户
不能统一核定价格
78 除 不完全符合相关规定 外
79 行政执法违法 处理
负有责任的领导人员
和直接责任人员
81 审计
除 公款娱乐(严禁 不能报销)
83 嫖娼 行政处罚
撤职 开除
84 违反 (法律 法规 规章 )给予行政处分
85 从轻的情节
除 承认错误态度较好
86 违反 (法定权限 法定条件 法定程序)给予撤职
(除 法定事由)
87 接待 不能增加陪同人数
89 严禁以任何形式(隐瞒 截留 挤占)坐支或者私分非税收
除(使用)
90 立案调查期间 不得(交流 出境 辞去公职)
96 出境期间 属我国机构 除私人自费宴请外 用公款皆不许
97 收回设备是可以的
98 除 配备必要的公务用车
100 可以购买正常服务
53 除公务学习交流
56 退休的,应依法
降级、撤职、开除的 降低岗位等级或取消待遇
58 处分
负有责任的领导人员
直接责任人员
60 解除处分条件
有悔改表现
没有再发生违纪违法行为
62 党政机关不得
新建 改建 扩建
63 违纪违法并涉嫌犯罪的
移送司法机关
依法追究刑事责任
64 定点保险 定点维修 定点加油
65 公务员 不受处分
非经法定程序
非经法定事由
67 加重处分
除(同案起次要作用)
69 从轻的条件
除(检举不确认,不能认定)
71 处分条例不是为了
约束政府人员的行为
73 除 正常会议支出
74 与被调查人有上下级领导关系 不需回避
76 除 按照程序作出公正处理
77 除 办公用户
不能统一核定价格
78 除 不完全符合相关规定 外
79 行政执法违法 处理
负有责任的领导人员
和直接责任人员
81 审计
除 公款娱乐(严禁 不能报销)
83 嫖娼 行政处罚
撤职 开除
84 违反 (法律 法规 规章 )给予行政处分
85 从轻的情节
除 承认错误态度较好
86 违反 (法定权限 法定条件 法定程序)给予撤职
(除 法定事由)
87 接待 不能增加陪同人数
89 严禁以任何形式(隐瞒 截留 挤占)坐支或者私分非税收
除(使用)
90 立案调查期间 不得(交流 出境 辞去公职)
96 出境期间 属我国机构 除私人自费宴请外 用公款皆不许
97 收回设备是可以的
98 除 配备必要的公务用车
100 可以购买正常服务
53 除公务学习交流
56 退休的,应依法
降级、撤职、开除的 降低岗位等级或取消待遇
58 处分
负有责任的领导人员
直接责任人员
60 解除处分条件
有悔改表现
没有再发生违纪违法行为
62 党政机关不得
新建 改建 扩建
63 违纪违法并涉嫌犯罪的
移送司法机关
依法追究刑事责任
64 定点保险 定点维修 定点加油
65 公务员 不受处分
非经法定程序
非经法定事由
67 加重处分
除(同案起次要作用)
69 从轻的条件
除(检举不确认,不能认定)
71 处分条例不是为了
约束政府人员的行为
73 除 正常会议支出
74 与被调查人有上下级领导关系 不需回避
76 除 按照程序作出公正处理
77 除 办公用户
不能统一核定价格
78 除 不完全符合相关规定 外
79 行政执法违法 处理
负有责任的领导人员
和直接责任人员
81 审计
除 公款娱乐(严禁 不能报销)
83 嫖娼 行政处罚
撤职 开除
84 违反 (法律 法规 规章 )给予行政处分
85 从轻的情节
除 承认错误态度较好
86 违反 (法定权限 法定条件 法定程序)给予撤职
(除 法定事由)
87 接待 不能增加陪同人数
89 严禁以任何形式(隐瞒 截留 挤占)坐支或者私分非税收
除(使用)
90 立案调查期间 不得(交流 出境 辞去公职)
96 出境期间 属我国机构 除私人自费宴请外 用公款皆不许
97 收回设备是可以的
98 除 配备必要的公务用车
100 可以购买正常服务
53 除公务学习交流
56 退休的,应依法
降级、撤职、开除的 降低岗位等级或取消待遇
58 处分
负有责任的领导人员
直接责任人员
60 解除处分条件
有悔改表现
没有再发生违纪违法行为
62 党政机关不得
新建 改建 扩建
63 违纪违法并涉嫌犯罪的
移送司法机关
依法追究刑事责任
64 定点保险 定点维修 定点加油
65 公务员 不受处分
非经法定程序
非经法定事由
67 加重处分
除(同案起次要作用)
69 从轻的条件
除(检举不确认,不能认定)
71 处分条例不是为了
约束政府人员的行为
73 除 正常会议支出
74 与被调查人有上下级领导关系 不需回避
76 除 按照程序作出公正处理
77 除 办公用户
不能统一核定价格
78 除 不完全符合相关规定 外
79 行政执法违法 处理
负有责任的领导人员
和直接责任人员
81 审计
除 公款娱乐(严禁 不能报销)
83 嫖娼 行政处罚
撤职 开除
84 违反 (法律 法规 规章 )给予行政处分
85 从轻的情节
除 承认错误态度较好
86 违反 (法定权限 法定条件 法定程序)给予撤职
(除 法定事由)
87 接待 不能增加陪同人数
89 严禁以任何形式(隐瞒 截留 挤占)坐支或者私分非税收
除(使用)
90 立案调查期间 不得(交流 出境 辞去公职)
96 出境期间 属我国机构 除私人自费宴请外 用公款皆不许
97 收回设备是可以的
98 除 配备必要的公务用车
100 可以购买正常服务
53 除公务学习交流
56 退休的,应依法
降级、撤职、开除的 降低岗位等级或取消待遇
58 处分
负有责任的领导人员
直接责任人员
60 解除处分条件
有悔改表现
没有再发生违纪违法行为
62 党政机关不得
新建 改建 扩建
63 违纪违法并涉嫌犯罪的
移送司法机关
依法追究刑事责任
64 定点保险 定点维修 定点加油
65 公务员 不受处分
非经法定程序
非经法定事由
67 加重处分
除(同案起次要作用)
69 从轻的条件
除(检举不确认,不能认定)
71 处分条例不是为了
约束政府人员的行为
73 除 正常会议支出
74 与被调查人有上下级领导关系 不需回避
76 除 按照程序作出公正处理
77 除 办公用户
不能统一核定价格
78 除 不完全符合相关规定 外
79 行政执法违法 处理
负有责任的领导人员
和直接责任人员
81 审计
除 公款娱乐(严禁 不能报销)
83 嫖娼 行政处罚
撤职 开除
84 违反 (法律 法规 规章 )给予行政处分
85 从轻的情节
除 承认错误态度较好
86 违反 (法定权限 法定条件 法定程序)给予撤职
(除 法定事由)
87 接待 不能增加陪同人数
89 严禁以任何形式(隐瞒 截留 挤占)坐支或者私分非税收
除(使用)
90 立案调查期间 不得(交流 出境 辞去公职)
96 出境期间 属我国机构 除私人自费宴请外 用公款皆不许
97 收回设备是可以的
98 除 配备必要的公务用车
100 可以购买正常服务
53 除公务学习交流
56 退休的,应依法
降级、撤职、开除的 降低岗位等级或取消待遇
58 处分
负有责任的领导人员
直接责任人员
60 解除处分条件
有悔改表现
没有再发生违纪违法行为
62 党政机关不得
新建 改建 扩建
63 违纪违法并涉嫌犯罪的
移送司法机关
依法追究刑事责任
64 定点保险 定点维修 定点加油
65 公务员 不受处分
非经法定程序
非经法定事由
67 加重处分
除(同案起次要作用)
69 从轻的条件
除(检举不确认,不能认定)
71 处分条例不是为了
约束政府人员的行为
73 除 正常会议支出
74 与被调查人有上下级领导关系 不需回避
76 除 按照程序作出公正处理
77 除 办公用户
不能统一核定价格
78 除 不完全符合相关规定 外
79 行政执法违法 处理
负有责任的领导人员
和直接责任人员
81 审计
除 公款娱乐(严禁 不能报销)
83 嫖娼 行政处罚
撤职 开除
84 违反 (法律 法规 规章 )给予行政处分
85 从轻的情节
除 承认错误态度较好
86 违反 (法定权限 法定条件 法定程序)给予撤职
(除 法定事由)
87 接待 不能增加陪同人数
89 严禁以任何形式(隐瞒 截留 挤占)坐支或者私分非税收
除(使用)
90 立案调查期间 不得(交流 出境 辞去公职)
96 出境期间 属我国机构 除私人自费宴请外 用公款皆不许
97 收回设备是可以的
98 除 配备必要的公务用车
100 可以购买正常服务
53 除公务学习交流
56 退休的,应依法
降级、撤职、开除的 降低岗位等级或取消待遇
58 处分
负有责任的领导人员
直接责任人员
60 解除处分条件
有悔改表现
没有再发生违纪违法行为
62 党政机关不得
新建 改建 扩建
63 违纪违法并涉嫌犯罪的
移送司法机关
依法追究刑事责任
64 定点保险 定点维修 定点加油
65 公务员 不受处分
非经法定程序
非经法定事由
67 加重处分
除(同案起次要作用)
69 从轻的条件
除(检举不确认,不能认定)
71 处分条例不是为了
约束政府人员的行为
73 除 正常会议支出
74 与被调查人有上下级领导关系 不需回避
76 除 按照程序作出公正处理
77 除 办公用户
不能统一核定价格
78 除 不完全符合相关规定 外
79 行政执法违法 处理
负有责任的领导人员
和直接责任人员
81 审计
除 公款娱乐(严禁 不能报销)
83 嫖娼 行政处罚
撤职 开除
84 违反 (法律 法规 规章 )给予行政处分
85 从轻的情节
除 承认错误态度较好
86 违反 (法定权限 法定条件 法定程序)给予撤职
(除 法定事由)
87 接待 不能增加陪同人数
89 严禁以任何形式(隐瞒 截留 挤占)坐支或者私分非税收
除(使用)
90 立案调查期间 不得(交流 出境 辞去公职)
96 出境期间 属我国机构 除私人自费宴请外 用公款皆不许
97 收回设备是可以的
98 除 配备必要的公务用车
100 可以购买正常服务
53 除公务学习交流
56 退休的,应依法
降级、撤职、开除的 降低岗位等级或取消待遇
58 处分
负有责任的领导人员
直接责任人员
60 解除处分条件
有悔改表现
没有再发生违纪违法行为
62 党政机关不得
新建 改建 扩建
63 违纪违法并涉嫌犯罪的
移送司法机关
依法追究刑事责任
64 定点保险 定点维修 定点加油
65 公务员 不受处分
非经法定程序
非经法定事由
67 加重处分
除(同案起次要作用)
69 从轻的条件
除(检举不确认,不能认定)
71 处分条例不是为了
约束政府人员的行为
73 除 正常会议支出
74 与被调查人有上下级领导关系 不需回避
76 除 按照程序作出公正处理
77 除 办公用户
不能统一核定价格
78 除 不完全符合相关规定 外
79 行政执法违法 处理
负有责任的领导人员
和直接责任人员
81 审计
除 公款娱乐(严禁 不能报销)
83 嫖娼 行政处罚
撤职 开除
84 违反 (法律 法规 规章 )给予行政处分
85 从轻的情节
除 承认错误态度较好
86 违反 (法定权限 法定条件 法定程序)给予撤职
(除 法定事由)
87 接待 不能增加陪同人数
89 严禁以任何形式(隐瞒 截留 挤占)坐支或者私分非税收
除(使用)
90 立案调查期间 不得(交流 出境 辞去公职)
96 出境期间 属我国机构 除私人自费宴请外 用公款皆不许
97 收回设备是可以的
98 除 配备必要的公务用车
100 可以购买正常服务
53 除公务学习交流
56 退休的,应依法
降级、撤职、开除的 降低岗位等级或取消待遇
58 处分
负有责任的领导人员
直接责任人员
60 解除处分条件
有悔改表现
没有再发生违纪违法行为
62 党政机关不得
新建 改建 扩建
63 违纪违法并涉嫌犯罪的
移送司法机关
依法追究刑事责任
64 定点保险 定点维修 定点加油
65 公务员 不受处分
非经法定程序
非经法定事由
67 加重处分
除(同案起次要作用)
69 从轻的条件
除(检举不确认,不能认定)
71 处分条例不是为了
约束政府人员的行为
73 除 正常会议支出
74 与被调查人有上下级领导关系 不需回避
76 除 按照程序作出公正处理
77 除 办公用户
不能统一核定价格
78 除 不完全符合相关规定 外
79 行政执法违法 处理
负有责任的领导人员
和直接责任人员
81 审计
除 公款娱乐(严禁 不能报销)
83 嫖娼 行政处罚
撤职 开除
84 违反 (法律 法规 规章 )给予行政处分
85 从轻的情节
除 承认错误态度较好
86 违反 (法定权限 法定条件 法定程序)给予撤职
(除 法定事由)
87 接待 不能增加陪同人数
89 严禁以任何形式(隐瞒 截留 挤占)坐支或者私分非税收
除(使用)
90 立案调查期间 不得(交流 出境 辞去公职)
96 出境期间 属我国机构 除私人自费宴请外 用公款皆不许
97 收回设备是可以的
98 除 配备必要的公务用车
100 可以购买正常服务
53 除公务学习交流
56 退休的,应依法
降级、撤职、开除的 降低岗位等级或取消待遇
58 处分
负有责任的领导人员
直接责任人员
60 解除处分条件
有悔改表现
没有再发生违纪违法行为
62 党政机关不得
新建 改建 扩建
63 违纪违法并涉嫌犯罪的
移送司法机关
依法追究刑事责任
64 定点保险 定点维修 定点加油
65 公务员 不受处分
非经法定程序
非经法定事由
67 加重处分
除(同案起次要作用)
69 从轻的条件
除(检举不确认,不能认定)
71 处分条例不是为了
约束政府人员的行为
73 除 正常会议支出
74 与被调查人有上下级领导关系 不需回避
76 除 按照程序作出公正处理
77 除 办公用户
不能统一核定价格
78 除 不完全符合相关规定 外
79 行政执法违法 处理
负有责任的领导人员
和直接责任人员
81 审计
除 公款娱乐(严禁 不能报销)
83 嫖娼 行政处罚
撤职 开除
84 违反 (法律 法规 规章 )给予行政处分
85 从轻的情节
除 承认错误态度较好
86 违反 (法定权限 法定条件 法定程序)给予撤职
(除 法定事由)
87 接待 不能增加陪同人数
89 严禁以任何形式(隐瞒 截留 挤占)坐支或者私分非税收
除(使用)
90 立案调查期间 不得(交流 出境 辞去公职)
96 出境期间 属我国机构 除私人自费宴请外 用公款皆不许
97 收回设备是可以的
98 除 配备必要的公务用车
100 可以购买正常服务
53 除公务学习交流
56 退休的,应依法
降级、撤职、开除的 降低岗位等级或取消待遇
58 处分
负有责任的领导人员
直接责任人员
60 解除处分条件
有悔改表现
没有再发生违纪违法行为
62 党政机关不得
新建 改建 扩建
63 违纪违法并涉嫌犯罪的
移送司法机关
依法追究刑事责任
64 定点保险 定点维修 定点加油
65 公务员 不受处分
非经法定程序
非经法定事由
67 加重处分
除(同案起次要作用)
69 从轻的条件
除(检举不确认,不能认定)
71 处分条例不是为了
约束政府人员的行为
73 除 正常会议支出
74 与被调查人有上下级领导关系 不需回避
76 除 按照程序作出公正处理
77 除 办公用户
不能统一核定价格
78 除 不完全符合相关规定 外
79 行政执法违法 处理
负有责任的领导人员
和直接责任人员
81 审计
除 公款娱乐(严禁 不能报销)
83 嫖娼 行政处罚
撤职 开除
84 违反 (法律 法规 规章 )给予行政处分
85 从轻的情节
除 承认错误态度较好
86 违反 (法定权限 法定条件 法定程序)给予撤职
(除 法定事由)
87 接待 不能增加陪同人数
89 严禁以任何形式(隐瞒 截留 挤占)坐支或者私分非税收
除(使用)
90 立案调查期间 不得(交流 出境 辞去公职)
96 出境期间 属我国机构 除私人自费宴请外 用公款皆不许
97 收回设备是可以的
98 除 配备必要的公务用车
100 可以购买正常服务
53 除公务学习交流
56 退休的,应依法
降级、撤职、开除的 降低岗位等级或取消待遇
58 处分
负有责任的领导人员
直接责任人员
60 解除处分条件
有悔改表现
没有再发生违纪违法行为
62 党政机关不得
新建 改建 扩建
63 违纪违法并涉嫌犯罪的
移送司法机关
依法追究刑事责任
64 定点保险 定点维修 定点加油
65 公务员 不受处分
非经法定程序
非经法定事由
67 加重处分
除(同案起次要作用)
69 从轻的条件
除(检举不确认,不能认定)
71 处分条例不是为了
约束政府人员的行为
73 除 正常会议支出
74 与被调查人有上下级领导关系 不需回避
76 除 按照程序作出公正处理
77 除 办公用户
不能统一核定价格
78 除 不完全符合相关规定 外
79 行政执法违法 处理
负有责任的领导人员
和直接责任人员
81 审计
除 公款娱乐(严禁 不能报销)
83 嫖娼 行政处罚
撤职 开除
84 违反 (法律 法规 规章 )给予行政处分
85 从轻的情节
除 承认错误态度较好
86 违反 (法定权限 法定条件 法定程序)给予撤职
(除 法定事由)
87 接待 不能增加陪同人数
89 严禁以任何形式(隐瞒 截留 挤占)坐支或者私分非税收
除(使用)
90 立案调查期间 不得(交流 出境 辞去公职)
96 出境期间 属我国机构 除私人自费宴请外 用公款皆不许
97 收回设备是可以的
98 除 配备必要的公务用车
100 可以购买正常服务
53 除公务学习交流
56 退休的,应依法
降级、撤职、开除的 降低岗位等级或取消待遇
58 处分
负有责任的领导人员
直接责任人员
60 解除处分条件
有悔改表现
没有再发生违纪违法行为
62 党政机关不得
新建 改建 扩建
63 违纪违法并涉嫌犯罪的
移送司法机关
依法追究刑事责任
64 定点保险 定点维修 定点加油
65 公务员 不受处分
非经法定程序
非经法定事由
67 加重处分
除(同案起次要作用)
69 从轻的条件
除(检举不确认,不能认定)
71 处分条例不是为了
约束政府人员的行为
73 除 正常会议支出
74 与被调查人有上下级领导关系 不需回避
76 除 按照程序作出公正处理
77 除 办公用户
不能统一核定价格
78 除 不完全符合相关规定 外
79 行政执法违法 处理
负有责任的领导人员
和直接责任人员
81 审计
除 公款娱乐(严禁 不能报销)
83 嫖娼 行政处罚
撤职 开除
84 违反 (法律 法规 规章 )给予行政处分
85 从轻的情节
除 承认错误态度较好
86 违反 (法定权限 法定条件 法定程序)给予撤职
(除 法定事由)
87 接待 不能增加陪同人数
89 严禁以任何形式(隐瞒 截留 挤占)坐支或者私分非税收
除(使用)
90 立案调查期间 不得(交流 出境 辞去公职)
96 出境期间 属我国机构 除私人自费宴请外 用公款皆不许
97 收回设备是可以的
98 除 配备必要的公务用车
100 可以购买正常服务
53 除公务学习交流
56 退休的,应依法
降级、撤职、开除的 降低岗位等级或取消待遇
58 处分
负有责任的领导人员
直接责任人员
60 解除处分条件
有悔改表现
没有再发生违纪违法行为
62 党政机关不得
新建 改建 扩建
63 违纪违法并涉嫌犯罪的
移送司法机关
依法追究刑事责任
64 定点保险 定点维修 定点加油
65 公务员 不受处分
非经法定程序
非经法定事由
67 加重处分
除(同案起次要作用)
69 从轻的条件
除(检举不确认,不能认定)
71 处分条例不是为了
约束政府人员的行为
73 除 正常会议支出
74 与被调查人有上下级领导关系 不需回避
76 除 按照程序作出公正处理
77 除 办公用户
不能统一核定价格
78 除 不完全符合相关规定 外
79 行政执法违法 处理
负有责任的领导人员
和直接责任人员
81 审计
除 公款娱乐(严禁 不能报销)
83 嫖娼 行政处罚
撤职 开除
84 违反 (法律 法规 规章 )给予行政处分
85 从轻的情节
除 承认错误态度较好
86 违反 (法定权限 法定条件 法定程序)给予撤职
(除 法定事由)
87 接待 不能增加陪同人数
89 严禁以任何形式(隐瞒 截留 挤占)坐支或者私分非税收
除(使用)
90 立案调查期间 不得(交流 出境 辞去公职)
96 出境期间 属我国机构 除私人自费宴请外 用公款皆不许
97 收回设备是可以的
98 除 配备必要的公务用车
100 可以购买正常服务
53 除公务学习交流
56 退休的,应依法
降级、撤职、开除的 降低岗位等级或取消待遇
58 处分
负有责任的领导人员
直接责任人员 舆论监督
(与人大关系是被领导关系)
43 监察人员
打听、过问、详情干预 都要记录
46 正确的
构成贪污罪
立案调查
47 玩忽职守
除(处理不属于自己的事)
49 不得用公款
宴请 浏览 非工作参观
50 不正确的一项
不能变相补偿租用办公用房
53 除公务学习交流
56 退休的,应依法
降级、撤职、开除的 降低岗位等级或取消待遇
58 处分
负有责任的领导人员
直接责任人员
60 解除处分条件
有悔改表现
没有再发生违纪违法行为
62 党政机关不得
新建 改建 扩建
63 违纪违法并涉嫌犯罪的
移送司法机关
依法追究刑事责任
64 定点保险 定点维修 定点加油
65 公务员 不受处分
非经法定程序
非经法定事由
67 加重处分
除(同案起次要作用)
69 从轻的条件
除(检举不确认,不能认定)
71 处分条例不是为了
约束政府人员的行为
73 除 正常会议支出
74 与被调查人有上下级领导关系 不需回避
76 除 按照程序作出公正处理
77 除 办公用户
不能统一核定价格
78 除 不完全符合相关规定 外
79 行政执法违法 处理
负有责任的领导人员
和直接责任人员
81 审计
除 公款娱乐(严禁 不能报销)
83 嫖娼 行政处罚
撤职 开除
84 违反 (法律 法规 规章 )给予行政处分
85 从轻的情节
除 承认错误态度较好
86 违反 (法定权限 法定条件 法定程序)给予撤职
(除 法定事由)
87 接待 不能增加陪同人数
89 严禁以任何形式(隐瞒 截留 挤占)坐支或者私分非税收
除(使用)
90 立案调查期间 不得(交流 出境 辞去公职)
96 出境期间 属我国机构 除私人自费宴请外 用公款皆不许
97 收回设备是可以的
98 除 配备必要的公务用车
100 可以购买正常服务
101 未经审批的项目
不得安排预算 贷款 捐建 垫资
102 严禁党员干部
到私人会所 变相公款旅游
(到公共场所 到歌舞厅唱歌)
103 选人用人工作追究责任
除(经济指标未完成)文对题
104 八项规定
改进调研 改进会风
改进文风 改进报道
改进警卫 规范出访
文稿发表 严守廉政
111 公务接待情况报
同级党政机关公务接待管理部门
财政部门
纪检监察机关
(组织部)
112 按年度公布
标准 经费支出 接待场所 接待项目
113 除引进的海外人才
114 移居国外
获外国籍 户外居住权 长期居留许可
116 批评教育 组织处理 追究纪律责任和法律责任
(实施问责)不是处理方式
118 在公务活动
礼仪庆典 新闻发布会 经济活动 收受东西及有价卷
119 不登记 不如实登记 不上交 批评或纪律处分
126 除 党委领导 组织负责
127 责令辞职 免职 引咎辞职 一年内不得担任高于原职务
128 个人报告事项 专项治理部门
组织及人事部门 纪检监察机关
130 除 不正当理由不能按时报告的
132 应当报告的事项
本人、配偶、共同生活子女的房产情况
配偶、共同生活子女的企业情况
133 监察委员会组成
主任 副主任 委员
136 军委主席不是任期制
138 行政机关 监察机关 审判机关和检察机关
(公安局属于行政机关)
139 人民代表大会
监督宪法的实施
制订、修改基本法律
140 常务委员会职权
监察宪法的实施
解释法律
141 全国人大常务委员会有权撤销
省 自治区 直辖市 地方性法规和决议
(不包括特别行政区)
142 县级以上地方各级常务委员会组成
主任
副主任
委员
(没有秘书长)
143 群众自治性组织
居民委员会
村民委员会
144 全国监察委员会 对()负责
全国人民代表大会
全国人大常务委员会
145 办理刑事案件
人民法院 检察院 公安机关
146 国家标示
国旗 国徽 首都 国歌
147 监察机关应与
审判机关
检察机关
执法部门
(权力机关)
148 法院不受
行政机关
社会团体
个人
(受检察院监督与制约)
\end{document}