% Created 2018-09-19 Wed 14:34
% Intended LaTeX compiler: pdflatex
\documentclass[11pt]{ctexart}
                                        \usepackage[utf8]{inputenc}
                                        \usepackage[T1]{fontenc}
                                        \usepackage{fixltx2e}
                                        \usepackage{graphicx}
                                        \usepackage{longtable}
                                        \usepackage{float}
                                        \usepackage{wrapfig}
                                        \usepackage{rotating}
                                        \usepackage[normalem]{ulem}
                                        \usepackage{amsmath}
                                        \usepackage{textcomp}
                                        \usepackage{marvosym}
                                        \usepackage{wasysym}
                                        \usepackage{amssymb}
                                        \usepackage{booktabs}
                                        \usepackage[colorlinks,linkcolor=black,anchorcolor=black,citecolor=black]{hyperref}
                                        \tolerance=1000
                                        \usepackage{listings}
                                        \usepackage{xcolor}
                                        \lstset{
                                        %行号
                                        numbers=left,
                                        %背景框
                                        framexleftmargin=10mm,
                                        frame=none,
                                        %背景色
                                        %backgroundcolor=\color[rgb]{1,1,0.76},
                                        backgroundcolor=\color[RGB]{245,245,244},
                                        %样式
                                        keywordstyle=\bf\color{blue},
                                        identifierstyle=\bf,
                                        numberstyle=\color[RGB]{0,192,192},
                                        commentstyle=\it\color[RGB]{0,96,96},
                                        stringstyle=\rmfamily\slshape\color[RGB]{128,0,0},
                                        %显示空格
                                        showstringspaces=false
                                        }
\author{mac}
\date{\today}
\title{}
\hypersetup{
 pdfauthor={mac},
 pdftitle={},
 pdfkeywords={},
 pdfsubject={},
 pdfcreator={Emacs 26.1 (Org mode 9.1.14)}, 
 pdflang={English}}
\begin{document}

\tableofcontents

\#+STARTUP indent
\section{工作记录}
\label{sec:orgaafcede}
\section{Quick notes}
\label{sec:orgb5bca01}

\subsection{learn a new command}
\label{sec:orgfe60dca}
spc x w d dictionary

\textit{[2017-09-11 Mon 16:54]}

\subsection{{\bfseries\sffamily DONE} 给张清民书记汇报}
\label{sec:org1cfeb70}
校企合作意向,请学校研究决定。

\textit{[2017-09-13 Wed 09:34]}

\subsection{today afternoon 4:00}
\label{sec:org138a121}
A pain in my chest.
\textit{[2017-09-17 Sun 20:19]}

\subsection{Today zeng's mother's died day}
\label{sec:org4995e85}

\textit{[2017-09-24 Sun 11:32]}

\subsection{Today I'm goining to intensive read a book}
\label{sec:org7ff6d2c}
Today I want to read a book finely.
\textit{[2017-09-26 Tue 08:17]}

\subsection{书写工作心得一篇}
\label{sec:org058d8bb}
\begin{center}
\includegraphics[width=.9\linewidth]{/Users/mac/Pictures/IMG_2140.png}
\end{center}
\textit{[2017-09-26 Tue 09:04]}

\textit{<2017-09-26 Tue>}
\begin{center}
\includegraphics[width=.9\linewidth]{/Users/mac/Pictures/IMG_2139.png}
\end{center}

\subsection{There's a sport's meet today}
\label{sec:orgabd79fb}
The information departerment is third formation.
I stays on stage untill eight o'clock and twenty.
then back to the office to work.

\textit{[2017-09-28 Thu 08:26]}

\subsection{{\bfseries\sffamily DONE} 同中兴谈校企合作}
\label{sec:org0c13c79}
\begin{itemize}
\item 规模:30-60人
\item 企业投入:学费收入1/3
\item 合作期:三届学生五年。
\end{itemize}
\textit{[2017-09-28 Thu 11:55]}

\subsection{{\bfseries\sffamily DONE} Review One,Two,Three}
\label{sec:org4cad785}
99.8\%,99.8\%,99.6\%
lasts thirty minutes
Four,Five
99.9\%,99.6\%
\textit{[2017-09-29 Fri 10:45]}
99.4\% lesson six
\subsection{{\bfseries\sffamily DONE} Review}
\label{sec:org916387d}
eight 98.4\%
\subsection{今天下午市政府评方案,至六点结束。}
\label{sec:orge65626d}

\textit{[2017-09-29 Fri 21:43]}

\subsection{断网第二天}
\label{sec:org330563b}
重大事故
管理水平不如中专。
一群猪。


\textit{[2017-09-30 Sat 09:07]}

\subsection{断网不启动,以后要牢记。}
\label{sec:orgc4bcf00}

\textit{[2017-10-02 Mon 17:10]}

\subsection{relax in nature}
\label{sec:org3a22ecf}
到东御道游玩,上到水库。有卖野山货的。吃巧食的;你买了东西,他要骂你。
后到金山路18号,到故居看了一眼。买了两种馒头。归途精力不集中,险追尾。


\textit{[2017-10-03 Tue 12:27]}

\subsection{Today is 中秋节}
\label{sec:org852965e}
中午六个菜。小酒一罐。
\textit{[2017-10-04 Wed 14:06]}

\subsection{{\bfseries\sffamily DONE} bring them to the center of hospital}
\label{sec:orgb7f3a32}
\textit{[2017-10-07 Sat 09:10]}

\subsection{{\bfseries\sffamily DONE} Today is going to xian and taking meeting}
\label{sec:org90e5015}
\textit{[2017-10-12 Thu 19:52]}

\subsection{Today 21:00upata 13.1beta}
\label{sec:orgeaae56d}

\textit{[2017-10-24 Tue 21:24]}

\subsection{Today there is a checking so clean}
\label{sec:org1159140}

\textit{[2017-10-30 Mon 08:16]}

\subsection{与张天瑞谈话}
\label{sec:orga6ce682}

\textit{[2017-10-30 Mon 21:18]}

\subsection{read gnu Emacs Lisp 21pages}
\label{sec:orgba7d5de}

\textit{[2017-11-07 Tue 10:04]}

\subsection{{\bfseries\sffamily TODO} }
\label{sec:org2bc1a76}

\href{file:///Users/mac/org-notes/notes.org}{你只需将你的文件放入上面所指定的文件夹中就可以开始使用 Agenda 模式了。}
\textit{[2017-11-07 Tue 14:17]}

\subsection{眼睛难受,不宜看出了,休息。}
\label{sec:org0342473}

\textit{[2017-12-05 Tue 09:39]}

\subsection{校园招聘会}
\label{sec:org2b7725a}

\textit{[2018-06-14 Thu 09:55]}

\section{区块链}
\label{sec:orgd690561}
\subsection{\href{http://www.sohu.com/a/228894640\_182338}{别炒作区块链了,它其实是一个算法复杂一般人很费解的数据库}}
\label{sec:org86bcf7c}
别在炒作区块链了,不要被所谓的区块链神秘概念所迷惑。区块链其实就是一个由全体联网
节点共同维护并持有同一账本的分布式数据库,它通过算法来达成共识,在无需信任的各节
点中构建一个无单一故障点或控制点的去中心化可信系统。

用人话说就是:区块链像一本网络上多人同步记录的公共账薄,记录了每一个人的交易记录。
这项技术最为核心的创新就在于,它是所有网络上交易的信任证明机制问题。历史上互联网
对于世界上的信息流通产生了根本性的变革。但是互联网对于资产交易和价值交易的影响却
受到了限制。主要是互联网解决不了信任的问题。传统的解决信用都需要靠法律、道德、垄
断组织第三方背书。区块链技术的产生,回归到了互联网试图用技术来解决这一重大问题的
一脉以承的发展路线。从这个角度看,区块链技术可能会演变成建基于协议之上的一种协议,
基于这种协议区块链可以构建出:能够在全世界范围内进行资产交换和价值交易的价值互享
网。该技术凭借"无需任何可信的第三方"的特征,以点对点的体系重塑传统互联网体系。



区块链的应用,除了数字货币了,金融还是主战场。以往行业通常做法就是凭借用户体验优
化、市场定位细分和监管套利,却远远没有触及金融行业的底层架构。但区块链有望改造金
融基础设施,释放存量市场的活力。当前开始商用测试的应用领域包括跨境支付、证券交易
结算和证券发行等,2018或将看到更大规模的商业化。一切中心结算机构:清算所、交易所、
投资银行、商业银行、等或将被革命。对于监管者而言,区块链有利于实现原本互不相容的
两个目标:降低系统风险和减少监管负担。

每一次文明的进化,都是悄无声息,却势不可挡的。在区块链重造帝国之前,来看看其怎样
攻城略地!

\subsection{1、区块链不等于任何虚拟货币,它可以塑造无数场景}
\label{sec:org009f7fe}

我们再从技术的层面来看的它的意义。先洗洗脑子:区块链不等于比特币区块链不等于以太
坊区块链不等于任何虚拟货币。虽然区块链脱胎于比特币,但区块链无论作为一个系统还是
作为一项技术,它的应用领域及发展潜力,将远不止货币。比特币(任何一种虚拟代币)它
们只能算作区块链运行场景中的一种。如果说区块链这种全新的底层架构类似于互联网,比
特币只能算基于其的一款伟大的APP,而以太坊以及各种代币则是基于比特币改版或者抄袭
的app。因为它们全部都基于货币结算与流通这个场景!而基于区块链这个底层架构,有无
数的场景,有无数的APP可供开发!



区块链发展分为三个阶段,区块链1.0时代主要是以比特币为代表的在数字领域的运用;区
块链2.0阶段主要运用于商业领域金融领域:像数字货币的清算和结算,数字资产管理,众
筹,智能合同,法律文件验证等;在区块链3.0时代才会触及社会和政府的公共职能,比如
说身份认证,仲裁,审计,域名,物流,医疗,邮件,签证,投票等社会领域。所以,区块
链革掉现有秩序的命,离我们还有一段距离。

\subsection{2、区块链高度灵活,可以利用它的某一特性与现有商业场景结合}
\label{sec:orge8df7d3}

技术要与现实相结合,就必须改良自身。前文我们知道区块链有四大特性,利用其中某一二
特性,就可以做出伟大的产品,但是其中也有些特性对这款产品来说就是累赘。所幸,区块
链自身非常灵活,可以利用它的某一特性与现有商业场景结合。就拿去中心化这一特性来说。
有些人言必称区块链的去中心化。然而,但现实环境是不是所有商业场景都愿意被查账本,
愿意去中心去。所以就延伸出了联盟链和私链(部分人可以查账本)充分利用区块链信息无
法造假的优势!这些链在信息公开程度和中心控制力度方面有所限制,但能够充分解决具体
商业场景中沟通和信任成本的问题。同时可以利用区块链的不可篡改性,来解决商品的假货
掉包等问题,利用区块链的同时按统一规则进行,可以解决物流问题\ldots{}很多巨头都已布局。



\subsection{3、更大的概率是与AI,物联网和大数据联手}
\label{sec:orgdf9a940}

区块链技术的发展价值已经无需赘述。可以预见的是代表着"数据共享"的区块链技术将与最
新的技术结合,与最前沿的技术一起落地。就拿物联网来说,虽然概念已经深入人心了,但
实际应用依然很少,普及依然很低。究其原因,一方面是硬件设备同时数据联网成本太高。
虽说物联网的价值就在于智能设备共享数据时,实时操作。但目前不同厂商设备之间的连接
性存在问题,而且目前的中心化的网络虽然可以适应十亿级别的移动互联网设备,但是万物
互联时代,数百亿级别物联网设备,对中心网络来说,速度和成本都难以维持。另一方面,
人们担心智能家居和车联网等带来的安全性和隐私性问题。区块链对于物联网的意义来说,
就是为数以亿计的设备之间建立低成本了的P2P的数据互联模式,通过其去中心化的特性,
所有的数据不用回到中心服务器溜一圈,再发射到其他设备,而是直接点对点的发射。让数
据摆脱对中央控制系统的依赖,用户也不用担心中心化架构中政府,工程师,黑客对你的数
据进行窃听。这便是巨头IBM在干的事儿!每一轮的大的技术革命,都会开启一个新的时代。
而作为参与其中的人的奖赏附带的是一次大规模的财富转移。新时代与旧时代的转角,参与
者才有可能获奖。

现在区块链技术所驱动的价值互联网时代,从各个方面的指标来看,都已经站在了转角的十
字路口,新旧时代的交接,伴随着的是巨大的财富转移。
\subsection{\href{http://www.sohu.com/a/228937573\_204078}{区块链技术:在线教育共享生态的基石}}
\label{sec:org8783aac}
\subsection{\href{http://www.sohu.com/a/228894330\_182338}{鸭梨来了!区块链革命思潮下这些传统商业思维将被颠覆}}
\label{sec:org9479982}
\subsection{\href{http://www.sohu.com/a/228903451\_251558}{李笑来:“去中心化”的稳定货币不高效,未来主权货币将区块链化}}
\label{sec:orgc3c55b6}
4月18日晚,雄岸基金合伙人、INBlockchain合伙人李笑来和快校CEO、搜狐前高级副总裁方
刚,做客区块链探长和31区主办的《区块链50人》第五期,双方就比特币的应用可行性、发
展趋势以及对国家竞争力建设的战略意义等问题进行了探讨,对话中李笑来明确表达了“未
来地球上所有的主权货币,都会区块链化”的观点。

针对方刚抛出的关于“比特币用于支付日常生活中的商品和服务可行性”问题,李笑来称比
特币应用于支付这是一种误解,一个通胀的系统是用来支付的,而比特币这样基于一个通缩
的系统用来支付会有很多的不方便,甚至有扭曲的地方,认为并非称之为“币”就要充当货
币职能用于支付,并将其与游戏币类比。

“比特币从一开始就并未被设计成一个支付系统,而是清算系统,要解决的是其系统中的价
值不可双花问题。”李笑来认为,比特币不大可能成为一种主要的支付系统,即便有可能成
为支付工具,也需要“巨大用户基数”的前提。当前市面上盛行的“去中心化的稳定货币方
案”肯定不高效,李笑来称,稳定、高效、去中心化,对于货币系统而言,三者最多只能占
有其中两个。

除了比特币,其他token也有支付的潜力,譬如日本的银行很快就会推出锚定日元的区块链
货币,其稳定的价格也决定着其用来支付的潜力也是非常之大。但是,这种支付潜力是需要
官方认可的,就像USDT事实上并不被货币发行方(美国)所支持,而夫多银行一直再尝试
USDT。现在乃至将来,法币依然是最主要、最安全的支付手段,李笑来认为,未来地球上所
有的主权货币,都会区块链化。而区块链化的货币能够终结假钞的现状,这将极大提高社会
运行效率。“假钞也许不是核心,但也是痛点。假钞有很多种形式,除了我们日常生活中遇
到的,银行账上也可能有不真实的记录—— 那也是假钞的一种”,而这核心的问题依然是信
任问题。

而对于区块链的发展,李笑来充满了信心。“我们是没有办法回到没有比特币的世界,就像
我们没有办法回到没有计算机没有互联网的时代一样”,而对于中国区块链的发展环境,李
笑来称,“中国是未来区块链最大的市场,是发展最快的地区,你让我走,我也不一定想
走。”

\section{spacemacs}
\label{sec:org7022325}
\subsection{\href{https://github.com/zilongshanren}{zilongshanren-github}}
\label{sec:orgeffcadd}
\subsection{\href{https://github.com/fidding/spacemacs.d/blob/master/fidding/packages.el}{fidding/spacemacs.d}}
\label{sec:org4aefed8}
\subsection{GUN Emacs lisp}
\label{sec:org113aa54}
\subsection{{\bfseries\sffamily DONE} 第九章 列表是如何实现的}
\label{sec:org669df8f}
\subsection{{\bfseries\sffamily DONE} 第十章 找回文本}
\label{sec:orge74f5fc}
\subsection{{\bfseries\sffamily DONE} 第11章 循环和递归}
\label{sec:org81c2cdb}
\subsection{{\bfseries\sffamily DONE} Spacemacs 使用总结}
\label{sec:org098a01e}
\url{https://segmentfault.com/a/1190000004351857}
简介
Spacemacs 是一份 emacs 的配置文件,想要使用它,你先要有 emacs。

安装 \& 使用
\$ git clone \url{https://github.com/syl20bnr/spacemacs} \textasciitilde{}/.emacs.d
\$ emacs
常用快捷键
配置文件管理

SPC f e d 快速打开配置文件 .spacemacs
SPC f e R 同步配置文件

文件管理

SPC f f 打开文件(夹),相当于 \$ open xxx 或 \$ cd /path/to/project
SPC p f 搜索文件名,相当于 ST / Atom 中的 Ctrl + p
SPC s a p 搜索内容,相当于 \$ ag xxx 或 ST / Atom 中的 Ctrl + Shift + f

SPC b k 关闭当前 buffer (SPC b d)
SPC SPC 搜索当前文件 (spc s s )

窗口管理

SPC f t 打开/关闭侧边栏,相当于 ST / Atom 中的 Ctrl(cmd) + k + b

SPC 0 光标跳转到侧边栏(NeoTree)中
SPC n(数字) 光标跳转到第 n 个 buffer 中

SPC w s | SPC w - 水平分割窗口
SPC w v | SPC w / 垂直分割窗口
SPC w c 关闭当前窗口

对齐

SPC j = 自动对齐,相当于 beautify

Emacs 服务器
Spacemacs 会在启动时启动服务器,这个服务器会在 Spacemacs 关闭的时候被杀掉。

使用 Emacs 服务器
当 Emacs 服务器启动的时候,我们可以在命令行中使用 emacsclient 命令:

\$ emacsclient -c 用 Emacs GUI 来打开文件
\$ emacsclient -t 用命令行中 Emacs 来打开文件
杀掉 Emacs 服务器
除了关闭 Spacemacs 之外,我们还可以用下面的命令来杀掉 Emacs 服务器:

\$ emacsclient -e '(kill-emacs)'
持久化 Emacs 服务器
我们可以持久化 Emacs 服务器,在 Emacs 关闭的时候,服务器不被杀掉。只要设置
\textasciitilde{}/.spacemacs 中 dotspacemacs-persistent-server 为 t 即可。

但这种情况下,我们只可以通过以下方式来杀掉服务器了:

SPC q q 退出 Emacs 并杀掉服务器,会对已修改的 Buffer 给出保存的提示。
SPC q Q 同上,但会丢失所有未保存的修改。

\subsection{插入图片}
\label{sec:org3ea4032}
[[]]
中间输入图片的绝对路径
\emph{Users/mac/Pictures}\ldots{}
中文为图片,英文为Pictures
\textit{<2017-09-26 Tue>}
\subsection{Org-mode 简明手册}
\label{sec:org78e1ce1}
\url{http://www.cnblogs.com/Open\_Source/archive/2011/07/17/2108747.html\#sec-2-7}
\subsection{{\bfseries\sffamily DONE} bindings key learn}
\label{sec:org19636ba}
\subsection{Spacemacs uses a leader key to bind almost all its key bindings.}
\label{sec:org6b69f1e}

This leader key is comronly set to ​,​ by Vim users, in Spacemacs the leader key is set on SPC (the space bar, hence the name spacemacs). This key is the most accessible key on a keyboard and it is pressed with the thumb which is a good choice to lower the risk of RSI. It can be customized to any other key using the variable dotspacemacs-leader-key and dotspacemacs-emacs-leader-key.

With Spacemacs there is no need to remap your keyboard modifiers to attempt to reduce the risk of RSI, every command can be executed very easily while you are in normal mode by pressing the SPC leader key, here are a few examples:

\begin{itemize}
\item Save a buffer: SPC f s
\item Save all opened buffers: SPC f S
\item Open (switch) to a buffer with helm: SPC b b
\end{itemize}
\subsection{It is possible to define your default themes in your \textasciitilde{}/.spacemacs with the variable dotspacemacs-themes. For instance, to specify spacemacs-light, leuven and zenburn:}
\label{sec:org2bbbe50}

(setq-default dotspacemacs-themes '(spacemacs-light leuven zenburn))
Key Binding	Description
\begin{itemize}
\item SPC T n	switch to next theme listed in dotspacemacs-themes.
\item SPC T s	select a theme using a helm buffer.
\end{itemize}
You can see samples of all included themes in this theme gallery from Rob Merrell.

Note:

You don't need to explicitly list in a layer the theme packages you are defining in dotspacemacs-themes, Spacemacs is smart enough to remove those packages from the list of orphans.
Due to the inner working of themes in Emacs, switching theme during the same session may have some weird side effects. Although these side effects should be pretty rare.
In the terminal version of Emacs, color themes will not render correctly as colors are rendered by the terminal and not by emacs. You will probably have to change your terminal color palette. More explanations can be found on emacs-color-theme-solarized webpage.
Hint: If you are an Org user, leuven-theme is amazing ;-)
\subsection{12.3 GUI Toggles}
\label{sec:org87768e5}

Some graphical UI indicators can be toggled on and off (toggles start with t and T):

Key Binding	Description
SPC t 8	highlight any character past the 80th column
SPC t f	display the fill column (by default the fill column is set to 80)
SPC t h h	toggle highlight of the current line
SPC t h i	toggle highlight indentation levels
SPC t h c	toggle highlight indentation current column
SPC t i	toggle indentation guide at point
SPC t l	toggle truncate lines
SPC t L	toggle visual lines
SPC t n	toggle line numbers
SPC t v	toggle smooth scrolling
Key Binding	Description
SPC T \textasciitilde{}	display \textasciitilde{} in the fringe on empty lines
SPC T F	toggle frame fullscreen
SPC T f	toggle display of the fringe
SPC T m	toggle menu bar
SPC T M	toggle frame maximize
SPC T t	toggle tool bar
SPC T T	toggle frame transparency and enter transparency transient state
Note: These toggles are all available via the helm-spacemacs-help interface (press SPC h SPC to display the helm-spacemacs-help buffer).

\subsection{Some elements can be dynamically toggled:}
\label{sec:org98d9930}

Key Binding	Description
SPC t m b	toggle the battery status
SPC t m c	toggle the org task clock (available in org layer)
SPC t m m	toggle the minor mode lighters
SPC t m M	toggle the major mode
SPC t m n	toggle the cat! (if colors layer is declared in your dotfile)
SPC t m p	toggle the point character position
SPC t m t	toggle the mode line itself
SPC t m v	toggle the version control info
SPC t m V	toggle the new version lighter

\subsection{14.1.2 Executing Vim and Emacs ex/M-x commands}
\label{sec:org6518538}

Command	Key Binding
Vim (ex-command)	:
Emacs (M-x)	SPC SPC
The emacs command key SPC (executed after the leader key) can be changed with the variable dotspacemacs-emacs-command-key of your \textasciitilde{}/.spacemacs.

\subsection{14.1.4 Additional text objects}
\label{sec:org0448b54}

Additional text objects are defined in Spacemacs:

Object	Description
a	an argument
g	the entire buffer
\$	text between \$
\begin{itemize}
\item text between *
\end{itemize}
8	text between \emph{* and *}
\%	text between \%
\(\vert{}\) text between \(\vert{}\)

\subsection{14.4.2 Getting help}
\label{sec:orgc6df558}

Describe functions are powerful Emacs introspection commands to get information about functions, variables, modes etc. These commands are bound thusly:

Key Binding	Description
SPC h d b	describe bindings in a helm buffer
SPC h d c	describe current character under point
SPC h d d	describe current expression under point
SPC h d f	describe a function
SPC h d F	describe a face
SPC h d k	describe a key
SPC h d K	describe a keymap
SPC h d l	copy last pressed keys that you can paste in gitter chat
SPC h d m	describe current modes
SPC h d p	describe a package (Emacs built-in function)
SPC h d P	describe a package (Spacemacs layer information)
SPC h d s	copy system information that you can paste in giA strong country controls week countries.
tter chat
SPC h d t	describe a theme
SPC h d v	describe a variable
Other help key bindings:

Key Binding	Description
SPC h SPC	discover Spacemacs documentation, layers and packages using helm
SPC h i	search in info pages with the symbol at point
SPC h k	show top-level bindings with which-key
SPC h m	search available man pages
SPC h n	browse emacs news
Navigation key bindings in help-mode:

Key Binding	Description
g b or [	go back (same as clicking on [back] button)
g f or ]	go forward (same as clicking on [forward] button)
g h	go to help for symbol under point
Reporting an issue:

Key Binding	Description
SPC h I	Open Spacemacs GitHub issue page with pre-filled information
SPC u SPC h I	Open Spacemacs GitHub issue page with pre-filled information - include last pressed keys
Note: If these two bindings are used with the \textbf{Backtrace} buffer open, the backtrace is automatically included

\subsection{Org Mode - Organize Your Life In Plain Text!}
\label{sec:orge09725d}
\url{http://doc.norang.ca/org-mode.html\#OrgMimeMail}
\subsection{Spacemacs documentation}
\label{sec:org8ac22ac}
\url{http://spacemacs.org/doc/DOCUMENTATION.html\#vim}
\subsection{The Org Manualhttp://orgmode.org/org.html}
\label{sec:org7714515}
\url{http://orgmode.org/org.html}
\subsection{{\bfseries\sffamily DONE} Vim教程}
\label{sec:orgc9c82c3}
\url{http://wenku.baidu.com/view/8463bd6102020740be1e9ba7.html?re=view\&pn=51}

\subsection{\href{http://orgmode.org/org.html}{The Org Manual}}
\label{sec:org6285013}

\subsection{\href{http://www.cnblogs.com/holbrook/archive/2012/04/12/2444992.html\#sec-3-1}{org-mode: 最好的文档编辑利器,没有之一}}
\label{sec:orgcf5f3bc}
\subsection{\href{http://www.worldhello.net/gotgithub/appendix/markups.html}{轻量级标记语言}}
\label{sec:orgc6987a5}
;* 一级标题
;** 二级标题
;*** 三级标题

行尾两个反斜线\\
保持段落内换行符。

Org-mode式段落缩进:

\begin{quote}
段落缩进。
\end{quote}

下面是代码块。

\lstset{language=Ruby,label= ,caption= ,captionpos=b,numbers=none}
\begin{lstlisting}
require 'redcarpet'
md = Redcarpet.new("Hello, world.")
puts md.to_html
\end{lstlisting}

列表语法和Markdown、reST类似。
星号和列表语法冲突需缩进,不建议使用。

\begin{itemize}
\item 减号、加号开始列表。

\begin{itemize}
\item 列表层级和缩进有关。

\begin{itemize}
\item 和具体符号无关。
\end{itemize}
\end{itemize}

\item 返回一级列表。

\item 数字和点或右括号开始有序列表。

\begin{enumerate}
\item 缩进即为子列表。

\begin{enumerate}
\item 三级列表。
\item 编号会自动更正。
\end{enumerate}

\item 二级列表。
\end{enumerate}

\item 返回一级列表。

\item 列表项可以折行,
对齐则自动续行。

\item 列表项可包含多个段落。

空行开始的新段落,
新段落和列表项内容对齐即可。

\item 列表下的代码段注意缩进。

\lstset{language=bash,label= ,caption= ,captionpos=b,numbers=none}
\begin{lstlisting}
$ printf "Hello, world.\n"
\end{lstlisting}

\item[{git}] Simple and beautiful.
\item[{hg}] Another DVCS.
\item[{subversion}] VCS with many constrains.

Why not Git?
\end{itemize}


\textbf{加粗} \emph{斜体} \sout{删除线} 效果 \uline{下划线} 效果

\begin{itemize}
\item Water: H\(_{\text{2}}\) O
\item E = mc\(^{\text{2}}\)
\end{itemize}

行内用字符=或\textasciitilde{}嵌入代码,如:
\texttt{git status} 和 \texttt{git st} 。



\begin{itemize}
\item 网址 \url{http://github.com/}
\item 邮件 me@foo.bar

\item 访问 \href{http://google.com}{Google}

\item GitHub Logo: \begin{center}
\includegraphics[width=.9\linewidth]{/images/github.png}
\end{center}
\begin{itemize}
\item 带链接的图片:
\href{https://github.com/}{file:/images/\ldots{}}
\end{itemize}
\end{itemize}

\begin{table}[htbp]
\caption{\label{tab:org58bb1e4}
示例表格}
\centering
\begin{tabular}{lll}
head1 & head2 & head3\\
\hline
cell & cell & cell\\
cell & cell & cell\\
\end{tabular}
\end{table}

\#+ 缩进行首用 ``\#+`` 标记为注释。

\subsection{{\bfseries\sffamily DONE} \href{https://github.com/emacs-china/Spacemacs-rocks}{spacemacs-rocks}}
\label{sec:org0d0f6b2}
\subsection{常见符号所代表的意义如下}
\label{sec:orgc9097eb}

M(eta),在 Mac 下为 Option 键
s(uper),在 Mac 环境下为左 Command 键
S(Shift)
C(trl)
\subsection{光标的移动是编辑器中最常用的操作所以必须熟知。}
\label{sec:org05540bc}

C-f 为前移一个字符, f 代表 forward。
C-b 为后移一个字符, b 代表 backward。
C-p 为上移至前一行, p 代表 previous。
C-n 为上移至下一行, n 代表 next。
C-a 为移至行首, a 代表 ahead。
C-e 为移至行尾, e 代表 end。
\subsection{常用的文件操作快捷键组合也必须熟记。}
\label{sec:org1a0ec5e}

C-x C-f 为打开目标文件, f 代表 find/file
C-x C-s 为保存当前缓冲区(Buffer), s 代表 save
C-x 是 Emacs 的快捷键中常用的前缀命令。这些前缀命令常常代表了一系列有关联的指 令,
十分重要,请特别牢记。其它常见的还有 C-c, C-h 。打断组合键为 C-g ,它 用于终端取
消之前的指令。快捷键就是用预先绑定好的方式告诉 Emacs 去执行指定的命令。 (之后会
介绍到更多有关绑定的内容)
\subsection{内置功能}
\label{sec:orgdd3dbe4}

Emacs 功能强大,但是部分功能默认情况下并未开启。下面就有几个例子,

编辑器内显示行号可使用 M-x linum-mode 来开启。

\subsection{获取帮助}
\label{sec:orgf3c5dff}

Emacs 是一个富文档编辑器(Self document, extensible editor)而下面的三种方法在学 习 Emacs 的过程中也非常重要。他们分别是,

C-h k 寻找快捷键的帮助信息
C-h v 寻找变量的帮助信息
C-h f 寻找函数的帮助信息

在每次编辑配置文件后,刚刚做的修改并不会立刻生效。这时你需要重启编辑器或者重新加
载配置文件。重新加载配置文件你需要在当前配置文件中使用 M-x load-file 双击两次 回
车确认默认文件名,或者使用 M-x eval-buffer 去执行当前缓冲区的所有 Lisp 命令。 你
也可以使用 C-x C-e 来执行某一行的 Lisp 代码。这些可使刚刚修改的配置文件生效。 当
然你也可以将这些函数绑定为快捷键。
\subsection{学习基础 Elisp}
\label{sec:org71b6104}

请务必完成这篇教程(Learn X in Y Minutes)来了解 Elisp 的使用(阅读时间大约 15 分钟),你也可以在这里找到它的中文版。Emacs Lisp 为一个函数式的语言,所以它全部 功能都是由函数来实现的。

下面为一些简单的例子
\lstset{language=Lisp,label= ,caption= ,captionpos=b,numbers=none}
\begin{lstlisting}
 ;; 2 + 2
(+ 2 2)

;; 2 + 3 * 4
(+ 2 (* 3 4))

;; 定义变量
(setq name "username")
(message name) ; -> "username"

;; 定义函数
(defun func ()
  (message "Hello, %s" name))

;; 执行函数
(func) ; -> Hello, username

;; 设置快捷键
(global-set-key (kbd "<f1>") 'func)

;; 使函数可直接被调用可添加 (interactive)
(defun func ()
  (interactive)
  (message "Hello, %s" name))
\end{lstlisting}
\subsection{关于分屏的使用,如果你已经读过 Emacs 自带的教程,现在你应该已经掌握了基本的分屏 操作方法了。关于分屏的更多内容你可以在这里找到。}
\label{sec:org6152fdf}


C-x 1 仅保留当前窗口
C-x 2 将当前窗口分到上边
C-x 3 将当前窗口分到右边

\subsection{下面的这些函数可以让你找到不同函数,变量以及快捷键所定义的文件位置。因为非常常用}
\label{sec:org497319d}
所以我们建议将其设置为与查找文档类似的快捷键(如下所示),


find-function ( C-h C-f )
find-variable ( C-h C-v )
find-function-on-key ( C-h C-k )
\subsection{{\bfseries\sffamily DONE} 你只需将你的文件放入上面所指定的文件夹中就可以开始使用 Agenda 模式了。}
\label{sec:org5d81437}
C-c C-s 选择想要开始的时间
C-c C-d 选择想要结束的时间
C-c a 可以打开 Agenda 模式菜单并选择不同的可视方式( r )

\subsection{Dired Mode}
\label{sec:org8314d52}

Dired Mode 是一个强大的模式它能让我们完成和文件管理相关的所有操作。

使用 C-x d 就可以进入 Dired Mode,这个模式类似于图形界面系统中的资源管理器。你 可以在其中查看文件和目录的详细信息,对他们进行各种操作,甚至复制粘贴缓冲区中的内 容。下面是一些常用的操作(下面的所有键均需在 Dired Mode 下使用),

\begin{itemize}
\item 创建目录
\end{itemize}
g 刷新目录
C 拷贝
D 删除
R 重命名
d 标记删除
u 取消标记
x 执行所有的标记
这里有几点可以优化的地方。第一是删除目录的时候 Emacs 会询问是否递归删除或拷贝, 这也有些麻烦我们可以用下面的配置将其设定为默认递归删除目录(出于安全原因的考虑, 也许你需要保持此行为。所有文中的配置请务必按需配置)。

(setq dired-recursive-deletes 'always)
(setq dired-recursive-copies 'always)
第二是,每一次你进入一个回车进入一个新的目录中是,一个新的缓冲区就会被建立。这使 得我们的缓冲区列表中充满了大量没有实际意义的记录。我们可以使用下面的代码,让 Emacs 重用唯一的一个缓冲区作为 Dired Mode 显示专用缓冲区。

(put 'dired-find-alternate-file 'disabled nil)

;; 主动加载 Dired Mode
;; (require 'dired)
;; (defined-key dired-mode-map (kbd "RET") 'dired-find-alternate-file)

;; 延迟加载
(with-eval-after-load 'dired
(define-key dired-mode-map (kbd "RET") 'dired-find-alternate-file))
使用延迟加载可以使编辑器加载速度有所提升。

启用 dired-x 可以让每一次进入 Dired 模式时,使用新的快捷键 C-x C-j 就可以进 入当前文件夹的所在的路径。

(require 'dired-x)
使用 (setq dired-dwin-target 1) 则可以使当一个窗口(frame)中存在两个分屏 (window)时,将另一个分屏自动设置成拷贝地址的目标。

最后如果你是 Mac OS X 的用户,可以安装 reveal-in-osx-finder 这个插件(你可以在
这里找到它),它可以将任意文件直接在 Finder 中打开。你想安装这个插件,将其添加至
第二天的插件列表中即可,下次启动 Emacs 时,它就会自动帮你完成下载。
\subsection{expand-region}
\label{sec:org5d3161e}

使用同样的方法将 expand-region 添加至我们的插件列表中,重启 Emacs 安装插件。

再为其绑定一个快捷键,

(global-set-key (kbd "C-=") 'er/expand-region)
使用这个插件可以使我们更方便的选中一个区域。(更多使用方法和文档可以在这里找到)

\subsection{iedit}
\label{sec:org74cf410}

iedit 是一个可以同时编辑多个区域的插件,它类似 Sublime Text 中的多光标编辑。它的 GitHub 仓库在这里。

我们将其绑定快捷键以便更快捷的使用这个模式( C-; 为默认快捷键),

(global-set-key (kbd "M-s e") 'iedit-mode)
我们可以使用 Customized-group 来更改其高亮的背景色,将 highlight 改为 region 。
\subsection{\href{http://blog.csdn.net/pfanaya/article/details/6676307}{Emacs Org-Mode Note}}
\label{sec:org4587609}

\subsection{\href{http://blog.csdn.net/jiluben/article/details/39505203}{emacs之org-mode的转接(Refiling)}}
\label{sec:org9ac898b}
 据说emacs的org-mode是最好的任务管理器,心向往之,于是最近学了一段时间。作为小白,我不敢说它是最好的,只能说
它是适合我的。
        看了很多org-mode的教程,思想大体相似,就算创建几个.org的文件,如task.org、finished.org、project.org等,分别存放
要完成的任务、已完成的任务以及要做的项目等信息。比如我把一些公司的工作等任务写到了task.org中,等到有的任务完成了,
就把改任务转移到finished.org中。当初我以为是自己手动的把task.org中的那个任务剪切了,然后在打开finished.org,并将内容
粘贴到finished.org中,实现任务的转换。现在想起来,觉得自己的想法它可笑了,太低效了。
    其实,org-mode中提供了不同任务的转接功能。我们通过快捷键可以轻松的完成同一个文件在不同标题下的转接以及在不同文件
间的转接。本功能的英文原称是Refiling,即转接。
    关于转接功能,网上的资料很少,大都是一代而过。经过不断的寻找与尝试,我终于找到了配置的源代码,我们只需要将其拷贝
到配置文件中即可使用。如果你用的emacs中有.emacs文件,将源地貌直接拷贝到.emacs上即可。我用的是Steve Purcell的配置,
并没有.emacs文件,如果添加.emacs文件为使Steve Purcell的配置失效,但我们可以在custom.el中配置,将配置代码加入custom.el
即可。
     配置代码如下:

代码1:
(setq org-agenda-files (list "\textasciitilde{}/doc/org/inbox.org"
                             "\textasciitilde{}/doc/org/task.org"
                             "\textasciitilde{}/doc/org/finished.org"
                             "\textasciitilde{}/doc/org/project.org"))



代码2:
; Targets include this file and any file contributing to the agenda - up to 9 levels deep
(setq org-refile-targets (quote ((nil :maxlevel . 9)
                                 (org-agenda-files :maxlevel . 9))))

; Use full outline paths for refile targets - we file directly with IDO
(setq org-refile-use-outline-path t)

; Targets complete directly with IDO
(setq org-outline-path-complete-in-steps nil)

; Allow refile to create parent tasks with confirmation
(setq org-refile-allow-creating-parent-nodes (quote confirm))

; Use IDO for both buffer and file completion and ido-everywhere to t
(setq org-completion-use-ido t)
(setq ido-everywhere t)
(setq ido-max-directory-size 100000)
(ido-mode (quote both))
; Use the current window when visiting files and buffers with ido
(setq ido-default-file-method 'selected-window)
(setq ido-default-buffer-method 'selected-window)
; Use the current window for indirect buffer display
(setq org-indirect-buffer-display 'current-window)

;;;; Refile settings
; Exclude DONE state tasks from refile targets
(defun bh/verify-refile-target ()
  "Exclude todo keywords with a done state from refile targets"
  (not (member (nth 2 (org-heading-components)) org-done-keywords)))

(setq org-refile-target-verify-function 'bh/verify-refile-target)

      上述代码分为两部分,第一部分可以让你能够查看任务。加入第一部分后,按下C-c a t,会列从所有的任务,再按下C-c a t的基础上,再按下1-r,会
显示所有TODO的事务,按下2-r会显示所以STARTED的事务,等等,大家可以自己探索。在源代码中,大家要将那些.org文件的地址换成自己的文件
地址。
     上述第二部分,大家可以直接拷贝,无需修改。再加入代码后,大家可以加鼠标放到一个事务上,按C-c C-w,屏幕的下方就会显示出所有.org文件名
与其事务名的组合,大家可以用左右键选择要转移到的目的地。如,我的鼠标在“** 买书”上,按下C-c C-w,用左右箭头选择“购物/(inbox.org)”,按下
回车键,就可以将当前文件中的这个“买书”项转移到inbox.org文件的“购物”项下,达到了不同文件键内容转接的目的,当然同一个文件下也可以进行转接。

参考文章:\url{http://doc.norang.ca/org-mode.html\#Refiling}

\subsection{\href{http://www.fidding.me/article/16}{在emacs使用magit管理git版本控制}}
\label{sec:orga06b434}
\subsection{\hyperref[sec:org230d937]{emacs下的git工具 magit 简介}}
\label{sec:org8fc07a3}
\subsection{\href{https://www.liaoxuefeng.com/wiki/0013739516305929606dd18361248578c67b8067c8c017b000/00137583770360579bc4b458f044ce7afed3df579123eca000}{搭建Git服务器}}
\label{sec:orgf39b787}
\subsection{\href{http://blog.csdn.net/gcc\_sky/article/details/14047687}{让你的代码有个归宿:Git学习及Emacs之magit}}
\label{sec:orgbe26421}

转载:请注明\url{http://blog.csdn.net/gcc\_sky}
本文主要为Git新手介绍Git的命令操作,以及emacs编辑代码过程中使用magit管理代码。
也许你正在为以下问题而烦恼:
每个项目工程代码以单独目录存在于本地,队友们想共享代码,必须到对方目录下进行拷贝,变化的路径名难免带来不便;
半个钟前,哪些代码被修改了?你记性没那么好;
每次为了bug修复或新功能开发,不得不另拷贝原工程;修改完还要进行手动diff并合并;
。。。。。。。。。。。可能还有更多烦人问题!
Git是一款分布式版本控制/软件配置管理软件。与CVS、Subversion一类的集中式版本控制工具不同,它采用了分布式版本库的作法,不需要服务器端软件,就可以运作版本控制,使得源代码的发布和交流极其方便。Git的速度很快,这对于诸如Linux kernel这样的大项目来说自然很重要。Git最为出色的是它的合并追踪(merge tracing)能力。
实际上内核开发团队决定开始开发和使用Git来作为内核开发的版本控制系统的时候,世界开源社群的反对声音不少,最大的理由是Git太艰涩难懂,从Git的内部工作机制来说,的确是这样。但是随着开发的深入,Git的正常使用都由一些友善的命令稿来执行,使Git变得非常好用。
Git工作流程
代码仓库的管理,即管理四棵“Tree”,远程仓库(remote repository)、本地仓库(HEAD/local repository)、缓冲区(INDEX/stage)、本地目录(Local directory)。
严格来说,管理的三棵树:仓库、缓冲区、本地目录;使用频繁的基本操作:
检出仓库:git clone <server:/repository>
跟踪文件:git add <filename>,添加至缓存区
取消跟踪文件:git rm --cached <filename>
提交:git commit -m "代码提交信息",提交到本地仓库(Local repository)
删除文件:git rm <filename>,取消HEAD的提交,并将本地文件置为“Stage:delete”状态
检出文件:git checkout <filename>,使用最新的提交覆盖目标文件
rollback:git reset --soft <commit>,保持当前文件修改的状态,版本rollback至commit版本
                git reset --hard <commit>, 版本rollback至commit版本
新建/切换分支:git branch <branch\(_{\text{name}}\)>
删除分支:git branch -d <branch\(_{\text{name}}\)>
检出(远程)分支:git checkout -b <(origin/)branch\(_{\text{name}}\)>
推送分支:git push origin <branch\(_{\text{name}}\)>,推送至远程仓库
设置推送的目标远程仓库:git remote add origin <server>,在推送分支前,需先配置
同步代码:git pull,本地仓库更新到远程仓库的最新提交
合并分支至当前分支:git merge <branch\(_{\text{name}}\)>
丢弃本地改动及提交,同步最新远程仓库版本:git reset --hard <origin/branch\(_{\text{name}}\)> 或 git fetch orgin
阅读下图,有助于记忆和理解工作流程:




个人习惯使用emacs阅读编辑代码,使用其magit-mode来进行Git管理。这里简单介绍一些操作:
emacs编辑状态时,输入M-x: magit-status,进入magit管理界面:

下面是快捷键列表:
TAB             magit-toggle-section
RET             magit-visit-item
C-w             magit-copy-item-as-kill
C-x             Prefix Command
ESC             Prefix Command
SPC             magit-show-item-or-scroll-up
!               magit-key-mode-popup-running
\$               magit-display-process
\begin{itemize}
\item magit-diff-larger-hunks
\item magit-diff-smaller-hunks
\end{itemize}
.               magit-mark-item
0               magit-diff-default-hunks
1               magit-show-level-1
2               magit-show-level-2
3               magit-show-level-3
4               magit-show-level-4
\begin{verbatim}
magit-git-command
\end{verbatim}
=               magit-diff-with-mark
?               magit-describe-item
A               magit-cherry-pick-item
B               magit-key-mode-popup-bisecting
C               magit-add-log
D               magit-diff
E               magit-interactive-rebase
F               magit-key-mode-popup-pulling
G               magit-refresh-all
I               magit-ignore-item-locally
L               magit-add-change-log-entry-no-option
M               magit-key-mode-popup-remoting
P               magit-key-mode-popup-pushing
R               magit-rebase-step
S               magit-stage-all
U               magit-unstage-all
X               magit-reset-working-tree
\^{}               magit-goto-parent-section
a               magit-apply-item
b               magit-key-mode-popup-branching
c               magit-log-edit
d               magit-diff-working-tree
e               magit-ediff
f               magit-key-mode-popup-fetching
g               magit-refresh
h               magit-toggle-diff-refine-hunk
i               magit-ignore-item
k               magit-discard-item
l               magit-key-mode-popup-logging
m               magit-key-mode-popup-merging
n               magit-goto-next-section
o               magit-key-mode-popup-submodule
p               magit-goto-previous-section
q               magit-quit-session
r               magit-key-mode-popup-rewriting
s               magit-stage-item
t               magit-key-mode-popup-tagging
u               magit-unstage-item
v               magit-revert-item
w               magit-wazzup
x               magit-reset-head
z               magit-key-mode-popup-stashing
DEL             magit-show-item-or-scroll-down
本文介绍到这里,如有纰漏,还请各大侠指出\^{}\^{}
\section{{\bfseries\sffamily DONE} \latex}
\label{sec:org4b05dae}
\subsection{The Tex Book}
\label{sec:orgd2b7dd5}
\url{http://math.ecnu.edu.cn/\~latex/docs/Eng\_doc/The\_TeXBook.pdf}
free learn
\subsection{基本结构}
\label{sec:orgffa90b7}
\lstset{language=org,label= ,caption= ,captionpos=b,numbers=none}
\begin{lstlisting}
\documentclass[12pt,a4paper]{ctexart}
\title{泰山职业技术学院}
\author{信息技术工程系}
\date{\today}
\begin{document}
\maketitle
\part{学院简介}
\section{一、信息技术工程系简介}
\subsection{云计算机应用专业}
\subsubsection{培养方案}
\pargraph{}
本专业是新开设的前沿专业。Suspendisse vel felis. Ut lorem lorem, interdum eu, tincidunt sit amet, laoreet vitae, arcu. Aenean faucibus pede eu ante. Praesent enim elit, rutrum at, molestie non, nonummy vel, nisl. Ut lectus eros, malesuada sit amet, fer- mentum eu, sodales cursus, magna. Donec eu purus. Quisque vehicula, urna sed ultricies auctor, pede lorem egestas dui, et convallis elit erat sed nulla. Donec luctus. Curabitur et nunc. Aliquam dolor odio, commodo pretium, ultricies non, pharetra in, velit. Integer arcu est, nonummy in, fermentum faucibus, egestas vel, odio.

%\end{document}
\end{lstlisting}

\subsection{Huang Zhenghua's home page}
\label{sec:org96ec49a}
\url{http://aff.whu.edu.cn/huangzh/}
\subsection{LATEX 排版学习笔记}
\label{sec:org4a9151a}
\url{http://www.latexstudio.net/wp-content/uploads/2014/02/latexlog-1310.pdf}
\subsection{\LaTeX{} 科技排版}
\label{sec:org3567c52}
\url{http://math.ecnu.edu.cn/\~latex/doc.html}
\subsection{CTEX 宏集手册}
\label{sec:org406641b}
\url{http://mirrors.ustc.edu.cn/CTAN/language/chinese/ctex/ctex.pdf}
\subsection{简单粗暴 LATEX}
\label{sec:org72d413e}
\url{http://www.latexstudio.net/wp-content/uploads/2017/08/Note-by-LaTeX-cn.pdf}
**
\section{lisp}
\label{sec:org92d7a9b}
\subsection{{\bfseries\sffamily DONE} Common lisp}
\label{sec:org3e03901}
\subsection{计算机程序的构造与解释(structure and interpretation of computer programs)}
\label{sec:org7a7b30d}
\subsection{{\bfseries\sffamily DONE} 序及第一章 构造过程抽象}
\label{sec:org1ef5f35}
  “每一位严肃的计算机科学家都应该阅读此书。本书清晰、简洁和富于才智,适合所有希望深刻理解计算机科学的人们。
            ------Mitchell Wand
             《美国科学家》杂志
1984年出版,成型于麻省理工学院(MIT)多年使用的教材。1996年修订第2版。在过去的二十多年里,对于计算机科学的教育计划产生了深刻的影响。
源代码及教辅资料:\ref{sec:orgf6e2c86}

计算机科学专业入门教材之一,从理论上讲解计算机程序的创建、执行知研究。主要内容包
括:构造过程抽象,构造数据抽象,模块化、对象和状态,元语言抽象,寄存器机器里的计
算等。

文艺复兴以降,源远流长的科学精神和逐步形成的学术规范,使西方国家在自然科学的各个
领域取得了垄断性的优势;也正是这样的传统,使美国的信息技术发展的六十多年间名家辈
出、独领风骚。在商业化的进程中,美国的产业界与教育界越来越紧密的结合,计算机学科
中的许多泰斗同时身处科研和教学的最前线,由此而产生的经典科学著作,不仅擘划了研究
的范畴,还揭橥了学术的源变,既遵循学术规范,又自有学者个性,其价值并不会因年月的
流逝而减退。

引进国外优秀教材,与世界接轨、建设真正的世界一流大学。
序
教育家、将军、减肥专家、心理学家和父母做规划(program),而军人、学生和另一些社
会阶层则被人规划(are programmed)。解决大规模问题需要经过一系列规划,其中的大部
分东西只有在工作进程中才能做出来,这些规划中充满着与手头问题的特殊性相关的情况。
如果想要把做规划这件事情本身作为一种智力活动来欣赏,你就必须转到计算机和程序设计
(programming),你需要读或者写计算机程序————而且要大量地做。有关这些程序具体是
关于什么的、服务于哪类应用等等的情况常常并不重要,重要的是它们的性能如何,在用于
构造更大的程序时能否与其他程序平滑衔接。程序员们必须同时追求具体部分的完美和汇合
的适宜性。在此书里使用“程序设计”一词时,所关注的是程序的创建、执行和研究,这些
程序是用一种lisp方言写的,为了在计算机上执行。采用Lisp并没有对我们可以编程的范围
施以任何约束或者限制,而只不过确定了程序描述的记法形式。

本书中要讨论的各种问题都牵涉到三类需要关注的对象:人的大脑、计算机程序的集合以及
计算机本身。每一个计算机程序都是现实中的或者精神中的某个过程的一个模型,通过人的
头脑孵化出来。这些过程出现在人们的经验或者思维之中,数量上数不胜数,详情琐碎繁杂,
任何时候人们都只能部分地理解它们。我们很少能通过自己的程序将这种过程模拟到永远令
人满意的程度。正因为如此,即使我们写出的程序是一集经过仔细雕琢的离散符号,是交织
在一起的一组函数,它们也需要不断的演化:当我们对于模型的认识更深入、更扩大、更广
泛时,就需要我们去为之奋斗的模型。计算机程序设计领域之令人兴奋的源泉,就在于它所
引起连绵不绝的发现,在我们的头脑之中,在由程序所表达的计算机制之中,以及在由此所
导致的认识爆炸之中。如果说艺术解释了我们的梦想,那么计算机就是以程序的名义执行着
它们。
\subsection{GNU Emacs Lisp编程入门读书笔记}
\label{sec:org49066de}
\subsection{本书作者罗伯特・卡塞尔是自由软件基金合创人之一,也是理查德・斯托曼博士青年结交的挚友,他精通GNU Emacs Lisp的每一方面。}
\label{sec:org922f184}
\subsection{致中国读者}
\label{sec:orgde38ccb}
Calendars, email, writing in general, programming, and debugging programs: GNU
Emacs gives you tools for all these actions, and gives you even more. GNU Emacs
is truly an integrated environment. Emacs Lisp is the language in which most of
Emacs is written. It is a simple yet powerful language that is easily understood
and learned.

FSF-CHINA led by Hong Feng is doing us all a great service by translating this
   document from Englis into Chinese. The work will bring the joy, the
   efficiency, and the power of GNU Emacs to amy people. I hope you will gain as
   much from reading this book as I did from writing it.

My best wishes to you. Robert J. Chassell
\subsection{译者序}
\label{sec:org7902158}
GNU Emacs长期以来一直是自由软件基金会的旗舰产品。它由(Richard Stallman)博士为
   GUN 工程开发的第一个自由软件。 在所有目前已开发的GNU软件中,GNU Emacs的作用和
   地位是非常突出的,因为几乎所有其他的自由软件基金会的工具都是用GNU Emacs编写出
   来的。
GNU Emacs有许多特点。最为突出是在创造 Emacs 编辑器是非常巧妙地揉合了用 Lisp 语言
   和 C 语言编写的代码。
Lisp发明于20世纪50年代,广泛地应用于人工智能研究领域,Stallman早年曾经在麻省理工
   学院人工智能实验室工作过很长时间,他非常熟悉 Lisp 语言的优点。Lisp 是解释性语
   言,开发的程序有良好的可读性,用于处理文本编辑是非常合适的。与硬件直接作用时
   效率与编译性比不高,由C语言代码来完成。
天才构想:利用C语言编写与硬件直接作用的GNU Emacs模块(如显示模块),而绝大多数文
   本编辑模块则统统利用 Lisp来写。Lisp 语言是一种功能全语言,其解释器被嵌入了
   Emacs后,用户自行对其寶。这几乎无限的灵活性是其他编辑器很难做到的。在Emacs 中
   Lisp代码模块和 C 代码模块组织良好,取长补短,相得益彰。
为保持源代码的可读性与一致性,Emacs中的 C 语言代码模块的函数名写得很像 Lisp函数
   名。难将两者区分。GUN Emacs成了“高级的、自带文档的、可定制的可扩充的实时显示
   的编辑器”
自Emacs 问世以来,这一天才的泛对称设计思想极富艺术性,具有方法论的研究的永久价值。
\subsection{{\bfseries\sffamily DONE} 第十二章 正则表达式查询 ** \href{http://www.hao123.com/mid/14816051782311469487?key=\&from=tuijian\&pn=1}{为何说Lisp是有史以来最牛的编程语言?}}
\label{sec:orge704847}
面向对象之父Alan Kay对“Lisp是有史以来最牛的编程语言”进行了解答。原回答如下:
Alan Kay:首先对我以前的一些答案进行澄清。有些人要尝试着用Lisp做操作系统,这看上
去好像很难。事实上,我曾经做过最好的操作系统之一就是利用的“The Lisp Machines”,
它是以“Parc Machines”和Smalltalk为首的硬件和软件——而我们也受到了编程以及实现
Lisp模型的影响(这些操作系统在Smalltalk和Lisp中都比今天的标准版本要更容易编写)。

另一个有趣的答案是:“时间的考验”在某种程度上是宇宙的优化。但是正如每个科学家知
道的一样,达尔文进化论发现了环境适应的重要性,一旦环境缺失,那么适应性也会随着缺
失。同样,如果大多数计算机科学家缺乏必要的理解和知识,那么他们的选择也可能会是错
误的。如今也有大量的证据表明这个结论是正确的。
但是这两个答案都不足以表达我对Lisp的赞美(另外我在“ Smalltalk的早期历史 ”这个问
题中更详细的解释了我的意思)。
我们很容易联想到历史上最伟大的天才——牛顿。他在很多领域都能流利的应用微积分。而在
牛顿之后的科学家在质量研究方面比以前有更大的突破。所以我认为“观点价值80分”——一
个知识贫乏的人很有可能减去80 IQ值,而一些更强的人会完成以前人们认为的很困难的创
新。

人类众多的思想问题之一是“认知负担”:事件的数量会立即引起我们的注意,一般来讲为
7±2,但对于许多事情来讲这还是少的。我们通过解决这些琐碎的问题来使自己成长。
这就是数学家们喜欢符号的原因之一——而缺点是科学家需要额外学习抽象层和符号所隐藏的
含义——实际上这正是小提琴演奏的实践部分。但你一旦做到这一点,你头脑中立即思维就会
被放大。以原始形式存在的20个麦克斯韦方程(以偏微分和卡迪尔坐标表示),今天我们可以
一眼就想到四个方程式,主要是由于这些方程式被Heaviside重新设计、强调了重点(有可能
这个重点是存在问题的,例如电场和磁场在运动方面应该对称等问题)。

现代科学是基于体验现象和设计模型关系的,这些关系可以进行必要的“negotiated”,因
为在我们脑海中的系统与“外界是什么”无必然联系。

从这个角度我们可以联想到“桥梁科学”和“桥梁科学家”——工程师建立桥梁,为科学家提
供可用的桥梁模型。同样,从“桥梁科学”可以衍生出来“计算机科学”和“计算机科学
家”,开发人员构建硬件和软件为科学家提供可用的计算机模型。实际上这是60年代初期
“计算机科学”的主导思想,但只是一个期望而已并没有完全实现。
Lisp背后的故事很有趣(你可以在第一编程语言史中查阅John McCarthy的文章)。它被构造
的目的之一是建造“数学物理”,也就是“数学的计算机理论”。另一个原因是John
McCarthy在50年代后期考虑过使用一种最普通的编程语言来构造一个用户界面的AI(称为
“警告”)。
Lisp可以进行编程,大多数的程序都是机器代码;Fortran表处理语言存在,语言也有链表。
John开发的 Lisp 可以编译任何编程语言都能做到的程序,而且相对简单,这也体现了它的本
质。(这让我们联想到数学部分或现代麦克斯韦方程式),而这样的方式也会比图灵机器更简
洁。
我们知道从最简单的机器结构到最高级语言的发展斜率都是最陡峭的——这就意味着可识别的
硬件到宇宙表达式会呈现火箭式飞跃的趋势。

通常情况下,特别是在工程中,一个伟大的科学模型往往都优于现有的模型,这就会导致棒
的工作。史蒂夫·拉塞尔(程序员,也是“太空站的”的发明者)看了约翰所做的工作后说:
“这个程序如果我来编码,我们现在已经有一个可运行的版本”。正如约翰所说的:“他做
到了,我们也做到了!”

而最终的结果就是“unlimited programming in an eyeful”(在Lisp 1.5手册中第13页的
下半部分)。其实问题的关键并不在于“Lisp”而是在于这些代表性方法是否对多种编程语
言方案开放。
这件事情一旦完成可以立即考虑比Lisp更优秀的编程语言,你也会立即想出比John更好的方
法来编写meta描述。这就是所谓的“POV = 80 IQ points”的部分。但这听起来就像是一旦
看到牛顿就会电动力学相对论一样。所以说科学上最大的壮举还是牛顿!
这就是为什么Lisp是最棒的原因。
\subsection{{\bfseries\sffamily DONE} \href{http://smacs.github.io/elisp/}{Lisp 简明教程}}
\label{sec:orgeda52fd}
这是叶文彬(水木ID: happierbee)写的一份Emacs Lisp的教程, 深入浅出, 非常适合初学者. 文档的\TeX{}代码及PDF文档可在此处下载.
emacs 的高手不能不会 elisp。但是对于很多人来说 elisp 学习是一个痛苦的历程,至少我是有这样一段经历。因此,我写了这一系列文章,希望能为后来者提供一点捷径。
一个 Hello World 例子
基础知识
基本数据类型之一 ── 数字
基本数据类型之二 ── 字符和字符串
基本数据类型之三 ── cons cell 和列表
基本数据类型之四 ── 数组和序列
基本数据类型之五 ── 符号
求值规则
变量
函数和命令
正则表达式
操作对象之一 ── 缓冲区
操作对象之二 ── 窗口
操作对象之三 ── 文件
操作对象之四 ── 文本
后记
\subsection{{\bfseries\sffamily DONE} 基础知识}
\label{sec:org5e0640a}
这一节介绍一下 elisp 编程中一些最基本的概念,比如如何定义函数,程序的控制结构,变量的使用和作用域等等。
SPA O (进入elisp 交互模式
\begin{enumerate}
\item 函数和变量
\label{sec:org1567417}
\begin{enumerate}
\item elisp 中定义一个函数是用这样的形式:
\label{sec:org92f780e}
(defun function-name (arguments-list) ;;defun define function
"document string"
body)
比如:
(defun hello-world (name)
"Say hello to user whose name is NAME."
(message "Hello, \%s" name))
其中函数的文档字符串是可以省略的。但是建议为你的函数(除了最简单,不作为接口的)都加上文档字符串。这样将来别人使用你的扩展或者别人阅读你的代码或者自己进行维护都提供很大的方便。
在 emacs 里,当光标处于一个函数名上时,可以用 C-h f 查看这个函数的文档。比如前面这个函数,在 \textbf{Help} 缓冲区里的文档是:
hello-world is a Lisp function.
(hello-world name)

Say hello to user whose name is name.
\textbf{如果你的函数是在文件中定义的。这个文档里还会给出一个链接能跳到定义的地方。}

\item 要运行一个函数,最一般的方式是:
\label{sec:org98d4d4e}
(function-name arguments-list)
比如前面这个函数:
(hello-world "Emacser")                 ; => "Hello, Emacser"
每个函数都有一个返回值。这个返回值一般是函数定义里的最后一个表达式的值。

\item elisp 里的变量使用无需象 C 语言那样需要声明,你可以用 setq 直接对一个变量赋值。
\label{sec:orgb2d5a49}
(setq foo "I'm foo")                    ; => "I'm foo"
(message foo)                           ; => "I'm foo"
和函数一样,你可以用 C-h v 查看一个变量的文档。比如当光标在 foo 上时用 C-h v 时,文档是这样的:
foo's value is "I'm foo"

Documentation:
Not documented as a variable.
有一个特殊表达式(special form)defvar,它可以声明一个变量,一般的形式是:
(defvar variable-name value
"document string")
它与 setq 所不同的是,如果变量在声明之前,这个变量已经有一个值的话,用 defvar 声
明的变量值不会改变成声明的那个值。另一个区别是 defvar 可以为变量提供文档字符串,
当变量是在文件中定义的话,C-h v 后能给出变量定义的位置。比如:

(defvar foo "Did I have a value?"
"A demo variable")                    ; => foo
foo                                     ; => "I'm foo"
(defvar bar "I'm bar"
"A demo variable named $\backslash$"bar$\backslash$"")      ; => bar
bar                                     ; => "I'm bar"
用 C-h v 查看 foo 的文档,可以看到它已经变成:
foo's value is "I'm foo"

Documentation:
A demo variable
由于 elisp 中函数是全局的,变量也很容易成为全局变量(因为全局变量和局部变量的赋
值都是使用 setq 函数),名字不互相冲突是很关键的。所以除了为你的函数和变量选择一
个合适的前缀之外,用 C-h f 和 C-h v 查看一下函数名和变量名有没有已经被使用过是很
关键的。

局部作用域的变量

如果没有局部作用域的变量,都使用全局变量,函数会相当难写。elisp 里可以用 let 和
let* 进行局部变量的绑定。let 使用的形式是:

(let (bindings)
body)
bingdings 可以是 (var value) 这样对 var 赋初始值的形式,或者用 var 声明一个初始值为 nil 的变量。比如:
(defun circle-area (radix)
(let ((pi 3.1415926)
area)
(setq area (* pi radix radix))
(message "直径为 \%.2f 的圆面积是 \%.2f" radix area)))
(circle-area 3)
C-h v 查看 area 和 pi 应该没有这两个变量。
let* 和 let 的使用形式完全相同,唯一的区别是在 let* 声明中就能使用前面声明的变量,比如:
(defun circle-area (radix)
(let* ((pi 3.1415926)
(area (* pi radix radix)))
(message "直径为 \%.2f 的圆面积是 \%.2f" radix area)))
\end{enumerate}

\item {\bfseries\sffamily DONE} lambda 表达式
\label{sec:org76c1985}
可能你久闻 lambda 表达式的大名了。其实依我的理解,lambda 表达式相当于其它语言中
的匿名函数。比如 perl 里的匿名函数。它的形式和 defun 是完全一样的:
(lambda (arguments-list)
"documentation string"
body)
调用 lambda 方法如下:
(funcall (lambda (name)
(message "Hello, \%s!" name)) "Emacser")
你也可以把 lambda 表达式赋值给一个变量,然后用 funcall 调用:
(setq foo (lambda (name)
(message "Hello, \%s!" name)))
(funcall foo "Emacser")                   ; => "Hello, Emacser!"
lambda 表达式最常用的是作为参数传递给其它函数,比如 mapc。
\end{enumerate}
\subsection{{\bfseries\sffamily DONE} 控制结构}
\label{sec:orga0af27e}
\begin{enumerate}
\item 顺序执行
\label{sec:orgdc439df}
一般来说程序都是按表达式顺序依次执行的。这在 defun 等特殊环境中是自动进行的。但是一般情况下都不是这样的。比如你无法用 eval-last-sexp 同时执行两个表达式,在 if 表达式中的条件为真时执行的部分也只能运行一个表达式。这时就需要用 progn 这个特殊表达式。它的使用形式如下:
(progn A B C \ldots{})
它的作用就是让表达式 A, B, C 顺序执行。比如:
(progn
(setq foo 3)
(message "Square of \%d is \%d" foo (* foo foo)))
\item {\bfseries\sffamily DONE} 条件判断
\label{sec:orgcb02d26}
elisp 有两个最基本的条件判断表达式 if 和 cond。使用形式分别如下:
(if condition
then
else)

(cond (case1 do-when-case1)
(case2 do-when-case2)
\ldots{}
(t do-when-none-meet))
使用的例子如下:
(defun my-max (a b)
(if (> a b)
a b))
(my-max 3 4)                            ; => 4

(defun fib (n)
(cond ((= n 0) 0)
((= n 1) 1)
(t (+ (fib (- n 1))
(fib (- n 2))))))
(fib 10)                                ; => 55
还有两个宏 when 和 unless,从它们的名字也就能知道它们是作什么用的。使用这两个宏的好处是使代码可读性提高,when 能省去 if 里的 progn 结构,unless 省去条件为真子句需要的的 nil 表达式。
\item {\bfseries\sffamily DONE} 循环
\label{sec:orge78f8fa}
循环使用的是 while 表达式。它的形式是:
(while condition
body)
比如:
(defun factorial (n)
(let ((res 1))
(while (> n 1)
(setq res (* res n)
n (- n 1)))
res))
(factorial 10)                          ; => 3628800
1=(1*1)
2=(2*1)
6=3*1*(3-1)
24=4*1*(4-1)*(3-1)(2-1)
120=(5*1)*(5-1)*(4-1)*(3-1)*(2-1)

\begin{enumerate}
\item {\bfseries\sffamily DONE} 逻辑运算
\label{sec:org3be0e24}
条件的逻辑运算和其它语言都是很类似的,使用 and、or、not。and 和 or 也同样具有短路性质。很多人喜欢在表达式短时,用 and 代替 when,or 代替 unless。当然这时一般不关心它们的返回值,而是在于表达式其它子句的副作用。比如 or 经常用于设置函数的缺省值,而 and 常用于参数检查:
(defun hello-world (\&optional name)
(or name (setq name "Emacser"))
(message "Hello, \%s" name))           ; => hello-world
(hello-world)                           ; => "Hello, Emacser"
(hello-world "Ye")                      ; => "Hello, Ye"

(defun square-number-p (n)
(and (>= n 0)
(= (/ n (sqrt n)) (sqrt n))))
(square-number-p -1)                    ; => nil
(square-number-p 25)                    ; => t
\end{enumerate}
\end{enumerate}
\subsection{函数列表}
\label{sec:orgc6a59a0}
(defun NAME ARGLIST [DOCSTRING] BODY\ldots{})
(defvar SYMBOL \&optional INITVALUE DOCSTRING)
(setq SYM VAL SYM VAL \ldots{})
(let VARLIST BODY\ldots{})
(let* VARLIST BODY\ldots{})
(lambda ARGS [DOCSTRING] [INTERACTIVE] BODY)
(progn BODY \ldots{})
(if COND THEN ELSE\ldots{})
(cond CLAUSES\ldots{})
(when COND BODY \ldots{})
(unless COND BODY \ldots{})
(when COND BODY \ldots{})
(or CONDITIONS \ldots{})
(and CONDITIONS \ldots{})
(not OBJECT)

\subsection{基本数据类型之一 ── 数字}
\label{sec:org9eaa1b0}

elisp 里的对象都是有类型的,而且每一个对象它们知道自己是什么类型。你得到一个变量名之后可以用一系列检测方法来测试这个变量是什么类型(好像没有什么方法来让它说出自己是什么类型的)。内建的 emacs 数据类型称为 primitive types,包括整数、浮点数、cons、符号(symbol)、字符串、向量(vector)、散列表(hash-table)、subr(内建函数,比如 cons, if, and 之类)、byte-code function,和其它特殊类型,例如缓冲区(buffer)。
在开始前有必要先了解一下读入语法和输出形式。所谓读入语法是让 elisp 解释器明白输入字符所代表的对象,你不可能让 elisp 读入 .\#@!? 这样奇怪的东西还能好好工作吧(perl好像经常要受这样的折磨:))。简单的来说,一种数据类型有(也可能没有,比如散列表)对应的规则来让解释器产生这种数据类型,比如 123 产生整数 123,(a . b) 产生一个 cons。所谓输出形式是解释器用产生一个字符串来表示一个数据对象。比如整数 123 的输出形式就是 123,cons cell (a . b) 的输出形式是 (a . b)。与读入语法不同的是,数据对象都有输出形式。比如散列表的输出可能是这样的:
\#<hash-table 'eql nil 0/65 0xa7344c8>
通常一个对象的数据对象的输出形式和它的读入形式都是相同的。现在就先从简单的数据类型──数字开始吧。
emacs 的数字分为整数和浮点数(和 C 比没有双精度数 double)。1, 1.,+1, -1, 536870913, 0, -0 这些都是整数。整数的范围是和机器是有关的,一般来最小范围是在 -268435456 to 268435455(29位,-2**28 \textasciitilde{} 2**28-1)。可以从 most-positive-fixnum 和 most-negative-fixnum 两个变量得到整数的范围。
你可以用多种进制来输入一个整数。比如:
\#b101100 => 44      ; 二进制
\#o54 => 44          ; 八进制
\#x2c => 44          ; 十进制
最神奇的是你可以用 2 到 36 之间任意一个数作为基数,比如:
\#24r1k => 44        ; 二十四进制
之所以最大是 36,是因为只有 0-9 和 a-z 36 个字符来表示数字。但是我想基本上不会有人会用到 emacs 的这个特性。
1500.0, 15e2, 15.0e2, 1.5e3, 和 .15e4 都可以用来表示一个浮点数 1500.。遵循 IEEE 标准,elisp 也有一个特殊类型的值称为 NaN (not-a-number)。你可以用 (/ 0.0 0.0) 产生这个数。
\subsubsection{测试函数}
\label{sec:orgae4a1a1}
整数类型测试函数是 integerp,浮点数类型测试函数是 floatp。数字类型测试用 numberp。你可以分别运行这几个例子来试验一下:
(integerp 1.)                           ; => t
(integerp 1.0)                          ; => nil
(floatp 1.)                             ; => nil
(floatp -0.0e+NaN)                      ; => t
(numberp 1)                             ; => t
还提供一些特殊测试,比如测试是否是零的 zerop,还有非负整数测试的 wholenump。
注:elisp 测试函数一般都是用 p 来结尾,p 是 predicate 的第一个字母。如果函数名是一个单词,通常只是在这个单词后加一个 p,如果是多个单词,一般是加 -p。
\subsubsection{数的比较}
\label{sec:orgcdeca4d}
常用的比较操作符号是我们在其它言中都很熟悉的,比如 <, >, >=, <=,不一样的是,由于赋值是使用 set 函数,所以 = 不再是一个赋值运算符了,而是测试数字相等符号。和其它语言类似,对于浮点数的相等测试都是不可靠的。比如:
(setq foo (- (+ 1.0 1.0e-3) 1.0))       ; => 0.0009999999999998899
(setq bar 1.0e-3)                       ; \texttt{> 0.001
      (} foo bar)                             ; => nil
所以一定要确定两个浮点数是否相同,是要在一定误差内进行比较。这里给出一个函数:
(defvar fuzz-factor 1.0e-6)
(defun approx-equal (x y)
(or (and (= x 0) (= y 0))
(< (/ (abs (- x y))
(max (abs x) (abs y)))
fuzz-factor)))
(approx-equal foo bar)                  ; => t
还有一个测试数字是否相等的函数 eql,这是函数不仅测试数字的值是否相等,还测试数字类型是否一致,比如:
(= 1.0 1)                               ; => t
(eql 1.0 1)                             ; \texttt{> nil
      elisp 没有 +}, -\texttt{, /}, *= 这样的命令式语言里常见符号,如果你想实现类似功能的语句,只能用赋值函数 setq 来实现了。 /= 符号被用来作为不等于的测试了。
\subsubsection{数的转换}
\label{sec:org5dd4f45}
整数向浮点数转换是通过 float 函数进行的。而浮点数转换成整数有这样几个函数:
truncate 转换成靠近 0 的整数
floor 转换成最接近的不比本身大的整数
ceiling 转换成最接近的不比本身小的整数
round 四舍五入后的整数,换句话说和它的差绝对值最小的整数
很晕是吧。自己用 1.2, 1.7, -1.2, -1.7 对这四个函数操作一遍就知道区别了(可以直接看 info。按键顺序是 C-h i m elisp RET m Numeric Conversions RET。以后简写成 info elisp - Numeric Conversions)。
这里提一个问题,浮点数的范围是无穷大的,而整数是有范围的,如果用前面的函数转换 1e20 成一个整数会出现什么情况呢?试试就知道了。
数的运算
四则运算没有什么好说的,就是 + - * \emph{。值得注意的是,和 C 语言类似,如果参数都是整数,作除法时要记住 (} 5 6) 是会等于 0 的。如果参数中有浮点数,整数会自动转换成浮点数进行运算,所以 (/ 5 6.0) 的值才会是 5/6。
没有 \sout{+ 和 -- 操作了,类似的两个函数是 1} 和 1-。可以用 setq 赋值来代替 ++ 和 --:
(setq foo 10)                           ; => 10
(setq foo (1+ foo))                     ; => 11
(setq foo (1- foo))                     ; => 10
注:可能有人看过有 incf 和 decf 两个实现 ++ 和 -- 操作。这两个宏是可以用的。这两个宏是 Common Lisp 里的,emacs 有模拟的 Common Lisp 的库 cl。但是 RMS 认为最好不要使用这个库。但是你可以在你的 elisp 包中使用这两个宏,只要在文件头写上:
(eval-when-compile
(require 'cl))
由于 incf 和 decf 是两个宏,所以这样写不会在运行里导入 cl 库。有点离题是,总之一句话,教主说不好的东西,我们最好不要用它。其它无所谓,只可惜了两个我最常用的函数 remove-if 和 remove-if-not。不过如果你也用 emms 的话,可以在 emms-compat 里找到这两个函数的替代品。
abs 取数的绝对值。
有两个取整的函数,一个是符号 \%,一个是函数 mod。这两个函数有什么差别呢?一是 \% 的第个参数必须是整数,而 mod 的第一个参数可以是整数也可以是浮点数。二是即使对相同的参数,两个函数也不一定有相同的返回值:
(+ (\% DIVIDEND DIVISOR)
(* (/ DIVIDEND DIVISOR) DIVISOR))
和 DIVIDEND 是相同的。而:
(+ (mod DIVIDEND DIVISOR)
(* (floor DIVIDEND DIVISOR) DIVISOR))
和 DIVIDEND 是相同的。
三角运算有函数: sin, cos, tan, asin, acos, atan。开方函数是 sqrt。
exp 是以 e 为底的指数运算,expt 可以指定底数的指数运算。log 默认底数是 e,但是也可以指定底数。log10 就是 (log x 10)。logb 是以 2 为底数运算,但是返回的是一个整数。这个函数是用来计算数的位。
random 可以产生随机数。可以用 (random t) 来产生一个新种子。虽然 emacs 每次启动后调用 random 总是产生相同的随机数,但是运行过程中,你不知道调用了多少次,所以使用时还是不需要再调用一次 (random t) 来产生新的种子。
位运算这样高级的操作我就不说了,自己看 info elisp - Bitwise Operations on Integers 吧。
\subsection{函数列表}
\label{sec:org5c351c1}
;; 测试函数
(integerp OBJECT)
(floatp OBJECT)
(numberp OBJECT)
(zerop NUMBER)
(wholenump OBJECT)
;; 比较函数
(> NUM1 NUM2)
(< NUM1 NUM2)
(>= NUM1 NUM2)
(<= NUM1 NUM2)
(= NUM1 NUM2)
(eql OBJ1 OBJ2)
(/= NUM1 NUM2)
;; 转换函数
(float ARG)
(truncate ARG \&optional DIVISOR)
(floor ARG \&optional DIVISOR)
(ceiling ARG \&optional DIVISOR)
(round ARG \&optional DIVISOR)
;; 运算
(+ \&rest NUMBERS-OR-MARKERS)
(- \&optional NUMBER-OR-MARKER \&rest MORE-NUMBERS-OR-MARKERS)
(* \&rest NUMBERS-OR-MARKERS)
(/ DIVIDEND DIVISOR \&rest DIVISORS)
(1+ NUMBER)
(1- NUMBER)
(abs ARG)
(\% X Y)
(mod X Y)
(sin ARG)
(cos ARG)
(tan ARG)
(asin ARG)
(acos ARG)
(atan Y \&optional X)
(sqrt ARG)
(exp ARG)
(expt ARG1 ARG2)
(log ARG \&optional BASE)
(log10 ARG)
(logb ARG)
;; 随机数
(random \&optional N)
变量列表
most-positive-fixnum
most-negative-fixnum
\subsection{{\bfseries\sffamily DONE} 基本数据类型之一 ── 数字}
\label{sec:orgb95fbb6}
elisp 里的对象都是有类型的,而且每一个对象它们知道自己是什么类型。你得到一个变量
名之后可以用一系列检测方法来测试这个变量是什么类型(好像没有什么方法来让它说出自
己是什么类型的)。内建的 emacs 数据类型称为 primitive types,包括整数、浮点数、
cons、符号(symbol)、字符串、向量(vector)、散列表(hash-table)、subr(内建函数,比
如 cons, if, and 之类)、byte-code function,和其它特殊类型,例如缓冲区(buffer)。

在开始前有必要先了解一下读入语法和输出形式。所谓读入语法是让 elisp 解释器明白输
入字符所代表的对象,你不可能让 elisp 读入 .\#@!? 这样奇怪的东西还能好好工作吧
(perl好像经常要受这样的折磨:))。简单的来说,一种数据类型有(也可能没有,比如散
列表)对应的规则来让解释器产生这种数据类型,比如 123 产生整数 123,(a . b) 产生
一个 cons。所谓输出形式是解释器用产生一个字符串来表示一个数据对象。比如整数 123
的输出形式就是 123,cons cell (a . b) 的输出形式是 (a . b)。与读入语法不同的是,
数据对象都有输出形式。比如散列表的输出可能是这样的:

\#<hash-table 'eql nil 0/65 0xa7344c8>

通常一个对象的数据对象的输出形式和它的读入形式都是相同的。现在就先从简单的数据类
型──数字开始吧。

emacs 的数字分为整数和浮点数(和 C 比没有双精度数 double)。1, 1.,+1, -1,
536870913, 0, -0 这些都是整数。整数的范围是和机器是有关的,一般来最小范围是在
-268435456 to 268435455(29位,-2**28 \textasciitilde{} 2**28-1)。可以从 most-positive-fixnum
和 most-negative-fixnum 两个变量得到整数的范围。

你可以用多种进制来输入一个整数。比如:
\#b101100 => 44      ; 二进制
\#o54 => 44          ; 八进制
\#x2c => 44          ; 十进制

最神奇的是你可以用 2 到 36 之间任意一个数作为基数,比如:
\#24r1k => 44        ; 二十四进制
之所以最大是 36,是因为只有 0-9 和 a-z 36 个字符来表示数字。但是我想基本上不会有
人会用到 emacs 的这个特性。

1500.0, 15e2, 15.0e2, 1.5e3, 和 .15e4 都可以用来表示一个浮点数 1500.。遵循 IEEE
标准,elisp 也有一个特殊类型的值称为 NaN (not-a-number)。你可以用 (/ 0.0 0.0) 产
生这个数。

\subsubsection{测试函数}
\label{sec:org9097ace}

\textbf{\textbf{**}} 整数类型测试函数是 integerp,浮点数类型测试函数是 floatp。数字类型测试用 numberp。你可以分别运行这几个例子来试验一下:
(integerp 1.)                           ; => t
(integerp 1.0)                          ; => nil
(floatp 1.)                             ; => nil
(floatp -0.0e+NaN)                      ; => t
(numberp 1)                             ; => t
\begin{enumerate}
\item 还提供一些特殊测试,比如测试是否是零的 zerop,还有非负整数测试的 wholenump。
\label{sec:orgd2d60f2}
注:elisp 测试函数一般都是用 p 来结尾,p 是 predicate 的第一个字母。如果函数名是一个单词,通常只是在这个单词后加一个 p,如果是多个单词,一般是加 -p。
\end{enumerate}
\subsection{{\bfseries\sffamily DONE} 数的比较}
\label{sec:orgb92db8c}
常用的比较操作符号是我们在其它言中都很熟悉的,比如 <, >, >=, <=,不一样的是,由
于赋值是使用 set 函数,所以 = 不再是一个赋值运算符了,而是测试数字相等符号。和其
它语言类似,对于浮点数的相等测试都是不可靠的。比如:

setq foo (- (+ 1.0 1.0e-3) 1.0))       ; => 0.0009999999999998899
(setq bar 1.0e-3)                       ; \texttt{> 0.001
    (} foo bar)                             ; => nil
所以一定要确定两个浮点数是否相同,是要在一定误差内进行比较。这里给出一个函数:
\begin{SCR}
(defvar fuzz-factor 1.0e-6)
(defun approx-equal (x y)
(or (and (= x 0) (= y 0))
(< (/ (abs (- x y))
(max (abs x) (abs y)))
fuzz-factor)))
(approx-equal foo bar)                  ; => t
\end{SCR}
还有一个测试数字是否相等的函数 eql,这是函数不仅测试数字的值是否相等,还测试数字
类型是否一致,比如:

(= 1.0 1)                               ; => t
(eql 1.0 1)                             ; \texttt{> nil
    elisp 没有 +}, -\texttt{, /}, *= 这样的命令式语言里常见符号,如果你想实现类似功能的语句,
只能用赋值函数 setq 来实现了。 /= 符号被用来作为不等于的测试了。

\subsection{数的转换}
\label{sec:org3078360}
整数向浮点数转换是通过 float 函数进行的。而浮点数转换成整数有这样几个函数:
truncate 转换成靠近 0 的整数
floor 转换成最接近的不比本身大的整数
ceiling 转换成最接近的不比本身小的整数
round 四舍五入后的整数,换句话说和它的差绝对值最小的整数
很晕是吧。自己用 1.2, 1.7, -1.2, -1.7 对这四个函数操作一遍就知道区别了(可以直接
看 info。按键顺序是 C-h i m elisp RET m Numeric Conversions RET。以后简写成 info
elisp - Numeric Conversions)。

这里提一个问题,浮点数的范围是无穷大的,而整数是有范围的,如果用前面的函数转换
1e20 成一个整数会出现什么情况呢?试试就知道了。

\subsubsection{数的运算}
\label{sec:org624879b}
四则运算没有什么好说的,就是 + - * \emph{。值得注意的是,和 C 语言类似,如果参数都是
整数,作除法时要记住 (} 5 6) 是会等于 0 的。如果参数中有浮点数,整数会自动转换成
浮点数进行运算,所以 (/ 5 6.0) 的值才会是 5/6。

没有 \sout{+ 和 -- 操作了,类似的两个函数是 1} 和 1-。可以用 setq 赋值来代替 ++ 和 --:
(setq foo 10)                           ; => 10
(setq foo (1+ foo))                     ; => 11
(setq foo (1- foo))                     ; => 10
注:可能有人看过有 incf 和 decf 两个实现 ++ 和 -- 操作。这两个宏是可以用的。这两
个宏是 Common Lisp 里的,emacs 有模拟的 Common Lisp 的库 cl。但是 RMS 认为最好不
要使用这个库。但是你可以在你的 elisp 包中使用这两个宏,只要在文件头写上:

(eval-when-compile
(require 'cl))
由于 incf 和 decf 是两个宏,所以这样写不会在运行里导入 cl 库。有点离题是,总之一
句话,教主说不好的东西,我们最好不要用它。其它无所谓,只可惜了两个我最常用的函
数 remove-if 和 remove-if-not。不过如果你也用 emms 的话,可以在 emms-compat 里
找到这两个函数的替代品。

\subsubsection{abs 取数的绝对值。}
\label{sec:orgd31e5c6}
有两个取整的函数,一个是符号 \%,一个是函数 mod。这两个函数有什么差别呢?一是 \%
的第个参数必须是整数,而 mod 的第一个参数可以是整数也可以是浮点数。二是即使对相
同的参数,两个函数也不一定有相同的返回值:

(+ (\% DIVIDEND DIVISOR)
(* (/ DIVIDEND DIVISOR) DIVISOR))
和 DIVIDEND 是相同的。而:
(+ (mod DIVIDEND DIVISOR)
(* (floor DIVIDEND DIVISOR) DIVISOR))
和 DIVIDEND 是相同的。
\subsubsection{三角运算有函数: sin, cos, tan, asin, acos, atan。开方函数是 sqrt。}
\label{sec:org3dc3aaf}
exp 是以 e 为底的指数运算,expt 可以指定底数的指数运算。log 默认底数是 e,但是也
可以指定底数。log10 就是 (log x 10)。logb 是以 2 为底数运算,但是返回的是一个整
数。这个函数是用来计算数的位。

random 可以产生随机数。可以用 (random t) 来产生一个新种子。虽然 emacs 每次启动后
调用 random 总是产生相同的随机数,但是运行过程中,你不知道调用了多少次,所以使用
时还是不需要再调用一次 (random t) 来产生新的种子。

位运算这样高级的操作我就不说了,自己看 info elisp - Bitwise Operations on
Integers 吧。
\subsubsection{函数列表}
\label{sec:orgd19cc7a}
;; 测试函数
(integerp OBJECT)
(floatp OBJECT)
(numberp OBJECT)
(zerop NUMBER)
(wholenump OBJECT)
;; 比较函数
(> NUM1 NUM2)
(< NUM1 NUM2)
(>= NUM1 NUM2)
(<= NUM1 NUM2)
(= NUM1 NUM2)
(eql OBJ1 OBJ2)
(/= NUM1 NUM2)
;; 转换函数
(float ARG)
(truncate ARG \&optional DIVISOR)
(floor ARG \&optional DIVISOR)
(ceiling ARG \&optional DIVISOR)
(round ARG \&optional DIVISOR)
;; 运算
(+ \&rest NUMBERS-OR-MARKERS)
(- \&optional NUMBER-OR-MARKER \&rest MORE-NUMBERS-OR-MARKERS)
(* \&rest NUMBERS-OR-MARKERS)
(/ DIVIDEND DIVISOR \&rest DIVISORS)
(1+ NUMBER)
(1- NUMBER)
(abs ARG)
(\% X Y)
(mod X Y)
(sin ARG)
(cos ARG)
(tan ARG)
(asin ARG)
(acos ARG)
(atan Y \&optional X)
(sqrt ARG)
(exp ARG)
(expt ARG1 ARG2)
(log ARG \&optional BASE)
(log10 ARG)
(logb ARG)
;; 随机数
(random \&optional N)
变量列表
most-positive-fixnum
most-negative-fixnum


\subsection{{\bfseries\sffamily DONE} 基本数据类型之二 ── 字符和字符串}
\label{sec:org2ae4550}
在 emacs 里字符串是有序的字符数组。和 c 语言的字符串数组不同,emacs 的字符串可以容纳任何字符,包括 $\backslash$0:
(setq foo "abc$\backslash$000abc")                 ; => "abc\^{}@abc"
关于字符串有很多高级的属性,例如字符串的表示有单字节和多字节类型,字符串可以有文本属性(text property)等等。但是对于刚接触字符串,还是先学一些基本操作吧。
首先构成字符串的字符其实就是一个整数。一个字符 'A' 就是一个整数 65。但是目前字符串中的字符被限制在 0-524287 之间。字符的读入语法是在字符前加上一个问号,比如 ?A 代表字符 'A'。
?A                                      ; => 65
?a                                      ; => 97
对于标点来说,也可以用同样的语法,但是最好在前面加上转义字符 $\backslash$,因为有些标点会有岐义,比如 ?$\backslash$(。$\backslash$ 必须用 ?$\backslash$\ 表示。控制字符,退格、制表符,换行符,垂直制表符,换页符,空格,回车,删除和 escape 表示为 ?\a, ?\b, ?\t, ?\n, ?\v, ?\f, ?\s, ?\r, ?\d, 和 ?\e。对于没有特殊意义的字符,加上转义字符 $\backslash$ 是没有副作用的,比如 ?$\backslash$+ 和 ?+ 是完全一样的。所以标点还是都用转义字符来表示吧。
?\a => 7                 ; control-g, `C-g'
?\b => 8                 ; backspace, <BS>, `C-h'
?\t => 9                 ; tab, <TAB>, `C-i'
?\n => 10                ; newline, `C-j'
?\v => 11                ; vertical tab, `C-k'
?\f => 12                ; formfeed character, `C-l'
?\r => 13                ; carriage return, <RET>, `C-m'
?\e => 27                ; escape character, <ESC>, `C-['
?\s => 32                ; space character, <SPC>
?$\backslash$\ => 92                ; backslash character, `$\backslash$'
?\d => 127               ; delete character, <DEL>
控制字符可以有多种表示方式,比如 C-i,这些都是对的:
?$\backslash$\(^{\text{I}}\)  ?$\backslash$\(^{\text{i}}\)  ?\C-I  ?\C-i
它们都对应数字 9。
meta 字符是用 修饰键(通常就是 Alt 键)输入的字符。之所以称为修饰键,是因为这样输入的字符就是在其修饰字符的第 27 位由 0 变成 1 而成,也就是如下操作:
(logior (lsh 1 27) ?A)                  ; => 134217793
?\M-A                                   ; => 134217793
你可以用 \M- 代表 meta 键,加上修饰的字符就是新生成的字符。比如:?\M-A, ?\M-\C-b. 后面这个也可以写成 ?\C-\M-b。
如果你还记得前面说过字符串里的字符不能超过 524287 的话,这就可以看出字符串是不能放下一个 meta 字符的。所以按键序列在这时只能用 vector 来储存。
其它的修饰键也是类似的。emacs 用 2**25 位来表示 shift 键,2**24 对应 hyper,2**23 对应 super,2**22 对应 alt。
\subsubsection{测试函数}
\label{sec:orgaac448b}
字符串测试使用 stringp,没有 charp,因为字符就是整数。 string-or-null-p 当对象是
一个字符或 nil 时返回 t。char-or-string-p 测试是否是字符串或者字符类型。比较头疼
的是 emacs 没有测试字符串是否为空的函数。
这是我用的这个测试函数,使用前要测试字符串是否为 nil:
(defun string-emptyp (str)
(not (string< "" str)))
\subsubsection{构造函数}
\label{sec:orgfbe0d51}
产生一个字符串可以用 make-string。这样生成的字符串包含的字符都是一样的。要生成不同的字符串可以用 string 函数。
(make-string 5 ?x)                      ; => "xxxxx"
(string ?a ?b ?c)                       ; => "abc"
在已有的字符串生成新的字符串的方法有 substring, concat。substring 的后两个参数是起点和终点的位置。如果终点越界或者终点比起点小都会产生一个错误。这个在使用 substring 时要特别小心。
(substring "0123456789" 3)              ; => "3456789"
(substring "0123456789" 3 5)            ; => "34"
(substring "0123456789" -3 -1)          ; => "78"
concat 函数相对简单,就是把几个字符串连接起来。
\subsubsection{字符串比较}
\label{sec:orgae25169}
char-equal 可以比较两个字符是否相等。与整数比较不同,这个函数还考虑了大小写。如
果 case-fold-search 变量是 t 时,这个函数的字符比较是忽略大小写的。编程时要小心,
因为通常 case-fold-search 都是 t,这样如果要考虑字符的大小写时就不能用
char-equal 函数了。

字符串比较使用 string=,string-equal 是一个别名。

string< 是按字典序比较两个字符串,string-less 是它的别名。空字符串小于所有字符串,除了空字符串。前面 string-emptyp 就是用这个特性。当然直接用 length 检测字符串长度应该也可以,还可以省去检测字符串是否为空。没有 string> 函数。
\subsubsection{转换函数}
\label{sec:orgfcd561d}
字符转换成字符串可以用 char-to-string 函数,字符串转换成字符可以用 string-to-char。当然只是返回字符串的第一个字符。
数字和字符串之间的转换可以用 number-to-string 和 string-to-number。其中 string-to-number 可以设置字符串的进制,可以从 2 到 16。number-to-string 只能转换成 10 进制的数字。如果要输出八进制或者十六进制,可以用 format 函数:
(string-to-number "256")                ; => 256
(number-to-string 256)                  ; => "256"
(format "\%\#o" 256)                      ; => "0400"
(format "\%\#x" 256)                      ; => "0x100"
如果要输出成二进制,好像没有现成的函数了。calculator 库倒是可以,这是我写的函数:
(defun number-to-bin-string (number)
(require 'calculator)
(let ((calculator-output-radix 'bin)
(calculator-radix-grouping-mode nil))
(calculator-number-to-string number)))
(number-to-bin-string 256)              ; => "100000000"
其它数据类型现在还没有学到,不过可以先了解一下吧。concat 可以把一个字符构成的列表或者向量转换成字符串,vconcat 可以把一个字符串转换成一个向量,append 可以把一个字符串转换成一个列表。
(concat '(?a ?b ?c ?d ?e))              ; => "abcde"
(concat [?a ?b ?c ?d ?e])               ; => "abcde"
(vconcat "abdef")                       ; => [97 98 100 101 102]
(append "abcdef" nil)                   ; => (97 98 99 100 101 102)
大小写转换使用的是 downcase 和 upcase 两个函数。这两个函数的参数既可以字符串,也可以是字符。capitalize 可以使字符串中单词的第一个字符大写,其它字符小写。upcase-initials 只使第一个单词的第一个字符大写,其它字符小写。这两个函数的参数如果是一个字符,那么只让这个字符大写。比如:
(downcase "The cat in the hat")         ; => "the cat in the hat"
(downcase ?X)                           ; => 120
(upcase "The cat in the hat")           ; => "THE CAT IN THE HAT"
(upcase ?x)                             ; => 88
(capitalize "The CAT in tHe hat")       ; => "The Cat In The Hat"
(upcase-initials "The CAT in the hAt")  ; => "The CAT In The HAt"
\subsubsection{格式化字符串}
\label{sec:org147af0c}
format 类似于 C 语言里的 printf 可以实现对象的字符串化。数字的格式化和 printf 的参数差不多,值得一提的是 "\%S" 这个格式化形式,它可以把对象的输出形式转换成字符串,这在调试时是很有用的。
\subsubsection{查找和替换}
\label{sec:org866af29}
字符串查找的核心函数是 string-match。这个函数可以从指定的位置对字符串进行正则表达式匹配,如果匹配成功,则返回匹配的起点,如:
(string-match "34" "01234567890123456789")    ; => 3
(string-match "34" "01234567890123456789" 10) ; => 13
注意 string-match 的参数是一个 regexp。emacs 好象没有内建的查找子串的函数。如果你想把 string-match 作为一个查找子串的函数,可以先用 regexp-quote 函数先处理一下子串。比如:
(string-match "2*" "232*3=696")                ; => 0
(string-match (regexp-quote "2*") "232*3=696") ; => 2
事实上,string-match 不只是查找字符串,它更重要的功能是捕捉匹配的字符串。如果你对正则表达式不了解,可能需要先找一本书,先了解一下什么是正则表达式。string-match 在查找的同时,还会记录下每个要捕捉的字符串的位置。这个位置可以在匹配后用 match-data、match-beginning 和 match-end 等函数来获得。先看一下例子:
(progn
(string-match "3$\backslash$\(4\\)" "01234567890123456789")
(match-data))                         ; => (3 5 4 5)
最后返回这个数字是什么意思呢?正则表达式捕捉的字符串按括号的顺序对应一个序号,整个模式对应序号 0,第一个括号对应序号 1,第二个括号对应序号 2,以此类推。所以 "3\(4\)" 这个正则表达式中有序号 0 和 1,最后 match-data 返回的一系列数字对应的分别是要捕捉字符串的起点和终点位置,也就是说子串 "34" 起点从位置 3 开始,到位置 5 结束,而捕捉的字符串 "4" 的起点是从 4 开始,到 5 结束。这些位置可以用 match-beginning 和 match-end 函数用对应的序号得到。要注意的是,起点位置是捕捉字符串的第一个字符的位置,而终点位置不是捕捉的字符串最后一个字符的位置,而是下一个字符的位置。这个性质对于循环是很方便的。比如要查找上面这个字符串中所有 34 出现的位置:
(let ((start 0))
(while (string-match "34" "01234567890123456789" start)
(princ (format "find at \%d\n" (match-beginning 0)))
(setq start (match-end 0))))
查找会了,就要学习替换了。替换使用的函数是 replace-match。这个函数既可以用于字符串的替换,也可以用于缓冲区的文本替换。对于字符串的替换,replace-match 只是按给定的序号把字符串中的那一部分用提供的字符串替换了而已:
(let ((str "01234567890123456789"))
(string-match "34" str)
(princ (replace-match "x" nil nil str 0))
(princ "\n")
(princ str))
可以看出 replace-match 返回的字符串是替换后的新字符串,原字符串被没有改变。
如果你想挑战一下,想想怎样把上面这个字符串中所有的 34 都替换掉?如果想就使用同一个字符串来存储,可能对于固定的字符串,这个还容易一些,如果不是的话,就要花一些脑筋了,因为替换之后,新的字符串下一个搜索起点的位置就不能用 (match-end 0) 给出来的位置了,而是要扣除替换的字符串和被替换的字符串长度的差值。
emacs 对字符串的替换有一个函数 replace-regexp-in-string。这个函数的实现方法是把每次匹配部分之前的子串收集起来,最后再把所有字符串连接起来。
单字符的替换有 subst-char-in-string 函数。但是 emacs 没有类似 perl函数或者程序 tr 那样进行字符替换的函数。只能自己建表进行循环操作了。
\subsubsection{函数列表}
\label{sec:org59546f0}
;; 测试函数
(stringp OBJECT)
(string-or-null-p OBJECT)
(char-or-string-p OBJECT)
;; 构建函数
(make-string LENGTH INIT)
(string \&rest CHARACTERS)
(substring STRING FROM \&optional TO)
(concat \&rest SEQUENCES)
;; 比较函数
(char-equal C1 C2)
(string= S1 S2)
(string-equal S1 S2)
(string< S1 S2)
;; 转换函数
(char-to-string CHAR)
(string-to-char STRING)
(number-to-string NUMBER)
(string-to-number STRING \&optional BASE)
(downcase OBJ)
(upcase OBJ)
(capitalize OBJ)
(upcase-initials OBJ)
(format STRING \&rest OBJECTS)
;; 查找与替换
(string-match REGEXP STRING \&optional START)
(replace-match NEWTEXT \&optional FIXEDCASE LITERAL STRING SUBEXP)
(replace-regexp-in-string REGEXP REP STRING \&optional FIXEDCASE LITERAL SUBEXP START)
(subst-char-in-string FROMCHAR TOCHAR STRING \&optional INPLACE)
\subsection{基本数据类型之三 ── cons cell 和列表}
\label{sec:orgb187278}
如果从概念上来说,cons cell 其实非常简单的,就是两个有顺序的元素。第一个叫 CAR,
第二个就 CDR。CAR 和 CDR 名字来自于 Lisp。它最初在IBM 704机器上的实现。在这种机
器有一种取址模式,使人可以访问一个存储地址中的“地址(address)”部分和“减量
(decrement)”部分。CAR 指令用于取出地址部分,表示(Contents of Address part of
Register),CDR 指令用于取出地址的减量部分(Contents of the Decrement part of
Register)。cons cell 也就是 construction of cells。car 函数用于取得 cons cell 的
CAR 部分,cdr 取得cons cell 的 CDR 部分。cons cell 如此简单,但是它却能衍生出许
多高级的数据结构,比如链表,树,关联表等等。

cons cell 的读入语法是用 . 分开两个部分,比如:

'(1 . 2)                                ; => (1 . 2)
'(?a . 1)                               ; => (97 . 1)
'(1 . "a")                              ; => (1 . "a")
'(1 . nil)                              ; => (1)
'(nil . nil)                            ; => (nil)

注意到前面的表达式中都有一个 ' 号,这是什么意思呢?其实理解了 eval-last-sexp 的
作用就能明白了。eval-last-sexp 其实包含了两个步骤,一是读入前一个 S-表达式,二是
对读入的 S-表达式求值。这样如果读入的 S-表达式是一个 cons cell 的话,求值时会把
这个 cons cell 的第一个元素作为一个函数来调用。而事实上,前面这些例子的第一个元
素都不是一个函数,这样就会产生一个错误 invalid-function。之所以前面没有遇到这个
问题,那是因为前面数字和字符串是一类特殊的 S-表达式,它们求值后和求值前是不变,
称为自求值表达式(self-evaluating form)。' 号其实是一个特殊的函数 quote,它的作
用是将它的参数返回而不作求值。'(1 . 2) 等价于 (quote (1 . 2))。为了证明 cons
cell 的读入语法确实就是它的输出形式,可以看下面这个语句:

(read "(1 . 2)")                        ; => (1 . 2)

列表包括了 cons cell。但是列表中有一个特殊的元素──空表 nil。

nil                                     ; => nil
'()                                     ; => nil
空表不是一个 cons cell,因为它没有 CAR 和 CDR 两个部分,事实上空表里没有任何内容。
但是为了编程的方便,可以认为 nil 的 CAR 和 CDR 都是 nil:
(car nil)                               ; => nil
(cdr nil)                               ; => nil
按列表最后一个 cons cell 的 CDR 部分的类型分,可以把列表分为三类。如果它是 nil
的话,这个列表也称为“真列表”(true list)。如果既不是 nil 也不是一个 cons cell,
则这个列表称为“点列表”(dotted list)。还有一种可能,它指向列表中之前的一个 cons
cell,则称为环形列表(circular list)。这里分别给出一个例子:
'(1 2 3)                                  ; => (1 2 3)
'(1 2 . 3)                                ; => (1 2 . 3)
'(1 . \#1=(2 3 . \#1\#))                     ; => (1 2 3 . \#1)
从这个例子可以看出前两种列表的读入语法和输出形式都是相同的,而环形列表的读入语法
是很古怪的,输出形式不能作为环形列表的读入形式。
如果把真列表最后一个 cons cell 的 nil 省略不写,也就是 (1 . nil) 简写成 (1),把
( obj1 . ( obj2 . list)) 简写成 (obj1 obj2 . list),那么列表最后可以写成一个用括
号括起的元素列表:
'(1 . (2 . (3 . nil)))                  ; => (1 2 3)
尽管这样写是清爽多了,但是,我觉得看一个列表时还是在脑子里反映的前面的形式,这样
在和复杂的 cons cell 打交道时就不会搞不清楚这个 cons cell 的 CDR 是一个列表呢,
还是一个元素或者是嵌套的列表。
\subsubsection{测试函数}
\label{sec:orgf37d198}
测试一个对象是否是 cons cell 用 consp,是否是列表用 listp。
(consp '(1 . 2))                        ; => t
(consp '(1 . (2 . nil)))                ; => t
(consp nil)                             ; => nil
(listp '(1 . 2))                        ; => t
(listp '(1 . (2 . nil)))                ; => t
(listp nil)                             ; => t
没有内建的方法测试一个列表是不是一个真列表。通常如果一个函数需要一个真列表作为参数,都是在运行时发出错误,而不是进行参数检查,因为检查一个列表是真列表的代价比较高。
测试一个对象是否是 nil 用 null 函数。只有当对象是空表时,null 才返回空值。

\subsubsection{构造函数}
\label{sec:org38f1d69}

生成一个 cons cell 可以用 cons 函数。比如:
(cons 1 2)                              ; => (1 . 2)
(cons 1 '())                            ; => (1)
也是在列表前面增加元素的方法。比如:
(setq foo '(a b))                       ; => (a b)
(cons 'x foo)                           ; => (x a b)
值得注意的是前面这个例子的 foo 值并没有改变。事实上有一个宏 push 可以加入元素的同时改变列表的值:
(push 'x foo)                           ; => (x a b)
foo                                     ; => (x a b)
生成一个列表的函数是 list。比如:
(list 1 2 3)                            ; => (1 2 3)
可能这时你有一个疑惑,前面产生一个列表,我常用 quote(也就是 ' 符号)这个函数,它和这个 cons 和 list 函数有什么区别呢?其实区别是很明显的,quote 是把参数直接返回不进行求值,而 list 和 cons 是对参数求值后再生成一个列表或者 cons cell。看下面这个例子:
'((+ 1 2) 3)                            ; => ((+ 1 2) 3)
(list (+ 1 2) 3)                        ; => (3 3)
前一个生成的列表的 CAR 部分是 (+ 1 2) 这个列表,而后一个是先对 (+ 1 2) 求值得到
3 后再生成列表。

\subsubsection{思考题}
\label{sec:org06f9986}

如果你觉得你有点明白的话,我提一个问题考考你:怎样用 list 函数构造一个 (a b c)
这样的列表呢?
前面提到在列表前端增加元素的方法是用 cons,在列表后端增加元素的函数是用 append。比如:
(append '(a b) '(c))                    ; => (a b c)
append 的功能可以认为它是把第一个参数最后一个列表的 nil 换成第二个参数,比如前面这个例子,第一个参数写成 cons cell 表示方式是(a . (b . nil)),把这个 nil 替换成 (c) 就成了 (a . (b . (c)))。对于多个参数的情况也是一样的,依次把下一个参数替换新列表最后一个 nil 就是最后的结果了。
(append '(a b) '(c) '(d))               ; => (a b c d)
一般来说 append 的参数都要是列表,但是最后一个参数可以不是一个列表,这也不违背前面说的,因为 cons cell 的 CDR 部分本来就可以是任何对象:
(append '(a b) 'c)                      ; => (a b . c)
这样得到的结果就不再是一个真列表了,如果再进行 append 操作就会产生一个错误。
如果你写过 c 的链表类型,可能就知道如果链表只保留一个指针,那么链表只能在一端增加元素。elisp 的列表类型也是类似的,用 cons 在列表前增加元素比用 append 要快得多。
append 的参数不限于列表,还可以是字符串或者向量。前面字符串里已经提到可以把一个字符串转换成一个字符列表,同样可能把向量转换成一个列表:
(append [a b] "cd" nil)                 ; => (a b 99 100)
注意前面最后一个参数 nil 是必要的,不然你可以想象得到的结果是什么。
\subsubsection{把列表当数组用}
\label{sec:org976af5c}

要得到列表或者 cons cell 里元素,唯一的方法是用 car 和 cdr 函数。很容易明白,car
就是取得 cons cell 的 CAR 部分,cdr 函数就是取得 cons cell 的 CDR 部分。通过这两
个函数,我们就能访问 cons cell 和列表中的任何元素。

思考题

你如果知道 elisp 的函数如果定义,并知道 if 的使用方法,不妨自己写一个函数来取得
一个列表的第 n 个 CDR。
通过使用 elisp 提供的函数,我们事实上是可以把列表当数组来用。依惯例,我们用 car
来访问列表的第一个元素,cadr 来访问第二个元素,再往后就没有这样的函数了,可以用
nth 函数来访问:

(nth 3 '(0 1 2 3 4 5))                  ; => 3
获得列表一个区间的函数有 nthcdr、last 和 butlast。nthcdr 和 last 比较类似,它们都是返回列表后端的列表。nthcdr 函数返回第 n 个元素后的列表:
(nthcdr 2 '(0 1 2 3 4 5))               ; => (2 3 4 5)
last 函数返回倒数 n 个长度的列表:
(last '(0 1 2 3 4 5) 2)                 ; => (4 5)
butlast 和前两个函数不同,返回的除了倒数 n 个元素的列表。
(butlast '(0 1 2 3 4 5) 2)              ; => (0 1 2 3)
思考题
如何得到某个区间(比如从3到5之间)的列表(提示列表长度可以用 length函数得到):
(my-subseq '(0 1 2 3 4 5) 2 5)          ; => (2 3 4)
使用前面这几个函数访问列表是没有问题了。但是你也可以想象,链表这种数据结构是不适合随机访问的,代价比较高,如果你的代码中频繁使用这样的函数或者对一个很长的列表使用这样的函数,就应该考虑是不是应该用数组来实现。
直到现在为止,我们用到的函数都不会修改一个已有的变量。这是函数式编程的一个特点。只用这些函数编写的代码是很容易调试的,因为你不用去考虑一个变量在执行一个代码后就改变了,不用考虑变量的引用情况等等。下面就要结束这样轻松的学习了。
首先学习怎样修改一个 cons cell 的内容。首先 setcar 和 setcdr 可以修改一个 cons cell 的 CAR 部分和 CDR 部分。比如:
(setq foo '(a b c))                     ; => (a b c)
(setcar foo 'x)                         ; => x
foo                                     ; => (x b c)
(setcdr foo '(y z))                     ; => (y z)
foo                                     ; => (x y z)
思考题
好像很简单是吧。我出一个比较 bt 的一个问题,下面代码运行后 foo 是什么东西呢?
(setq foo '(a b c))                     ; => (a b c)
(setcdr foo foo)
现在来考虑一下,怎样像数组那样直接修改列表。使用 setcar 和 nthcdr 的组合就可以实现了:
(setq foo '(1 2 3))                     ; => (1 2 3)
(setcar foo 'a)                         ; => a
(setcar (cdr foo) 'b)                   ; => b
(setcar (nthcdr 2 foo) 'c)              ; => c
foo                                     ; => (a b c)
\subsubsection{把列表当堆栈用}
\label{sec:orge3cc11c}

前面已经提到过可以用 push 向列表头端增加元素,在结合 pop 函数,列表就可以做为一
个堆栈了。
(setq foo nil)                          ; => nil
(push 'a foo)                           ; => (a)
(push 'b foo)                           ; => (b a)
(pop foo)                               ; => b
foo                                     ; => (a)
\subsubsection{重排列表}
\label{sec:orgd09be3c}
如果一直用 push 往列表里添加元素有一个问题是这样得到的列表和加入的顺序是相反的。
通常我们需要得到一个反向的列表。reverse 函数可以做到这一点:
(setq foo '(a b c))                     ; => (a b c)
(reverse foo)                           ; => (c b a)
需要注意的是使用 reverse 后 foo 值并没有改变。不要怪我太啰唆,如果你看到一个函数
nreverse,而且确实它能返回逆序的列表,不明所以就到处乱用,迟早会写出一个错误的函
数。这个 nreverse 和前面的 reverse 差别就在于它是一个有破坏性的函数,也就是说它
会修改它的参数。
(nreverse foo)                          ; => (c b a)
foo                                     ; => (a)
为什么现在 foo 指向的是列表的末端呢?如果你实现过链表就知道,逆序操作是可以在原
链表上进行的,这样原来头部指针会变成链表的尾端。列表也是(应该是,我也没有看过实
现)这个原理。使用 nreverse 的唯一的好处是速度快,省资源。所以如果你只是想得到逆
序后的列表就放心用 nreverse,否则还是用 reverse 的好。
elisp 还有一些是具有破坏性的函数。最常用的就是 sort 函数:
(setq foo '(3 2 4 1 5))                 ; => (3 2 4 1 5)
(sort foo '<)                           ; => (1 2 3 4 5)
foo                                     ; => (3 4 5)
这一点请一定要记住,我就曾经在 sort 函数上犯了好几次错误。那如果我既要保留原列表,
又要进行 sort 操作怎么办呢?可以用 copy-sequence 函数。这个函数只对列表进行复制,
返回的列表的元素还是原列表里的元素,不会拷贝列表的元素。
nconc 和 append 功能相似,但是它会修改除最后一个参数以外的所有的参数,nbutlast 和 butlast 功能相似,也会修改参数。这些函数都是在效率优先时才使用。总而言之,以 n 开头的函数都要慎用。
\subsubsection{把列表当集合用}
\label{sec:org90c18e4}
列表可以作为无序的集合。合并集合用 append 函数。去除重复的 equal 元素用
delete-dups。查找一个元素是否在列表中,如果测试函数是用 eq,就用 memq,如果测试
用 equal,可以用 member。删除列表中的指定的元素,测试函数为 eq 对应 delq 函数,
equal 对应 delete。还有两个函数 remq 和 remove 也是删除指定元素。它们的差别是
delq 和 delete 可能会修改参数,而 remq 和 remove 总是返回删除后列表的拷贝。注意
前面这是说的是可能会修改参数的值,也就是说可能不会,所以保险起见,用 delq 和
delete 函数要么只用返回值,要么用 setq 设置参数的值为返回值。
(setq foo '(a b c))                     ; => (a b c)
(remq 'b foo)                           ; => (a c)
foo                                     ; => (a b c)
(delq 'b foo)                           ; => (a c)
foo                                     ; => (a c)
(delq 'a foo)                           ; => (c)
foo                                     ; => (a c)
\subsubsection{把列表当关联表}
\label{sec:org27c1896}
用在 elisp 编程中,列表最常用的形式应该是作为一个关联表了。所谓关联表,就是可以
用一个字符串(通常叫关键字,key)来查找对应值的数据结构。由列表实现的关联表有一
个专门的名字叫 association list。尽管 elisp里也有 hash table,但是 hash table 相
比于 association list 至少这样几个缺点:
hash table 里的关键字(key)是无序的,而 association list 的关键字 可以按想要的顺序排列
hash table 没有列表那样丰富的函数,只有一个 maphash 函数可以遍历列 表。而
association list 就是一个列表,所有列表函数都能适用
hash table 没有读入语法和输入形式,这对于调试和使用都带来很多不便
所以 elisp的hash table 不是一个首要的数据结构,只要不对效率要求很高,通常直接用
association list。数组可以作为关联表,但是数组不适合作为与人交互使用数据结构(毕
竟一个有意义的名字比纯数字的下标更适合人脑)。所以关联表的地位在 elisp 中就非比
寻常了,emacs 为关联表专门用 c 程序实现了查找的相关函数以提高程序的效率。在
association list 中关键字是放在元素的 CAR 部分,与它对应的数据放在这个元素的 CDR
部分。根据比较方法的不同,有 assq 和assoc 两个函数,它们分别对应查找使用 eq 和
equal 两种方法。例如:
(assoc "a" '(("a" 97) ("b" 98)))        ; => ("a" 97)
(assq 'a '((a . 97) (b . 98)))          ; => (a . 97)
通常我们只需要查找对应的数据,所以一般来说都要用 cdr 来得到对应的数据:
(cdr (assoc "a" '(("a" 97) ("b" 98))))  ; => (97)
(cdr (assq 'a '((a . 97) (b . 98))))    ; => 97
assoc-default 可以一步完成这样的操作:
(assoc-default "a" '(("a" 97) ("b" 98)))          ; => (97)
如果查找用的键值(key)对应的数据也可以作为一个键值的话,还可以用 rassoc 和 rassq 来根据数据查找键值:
(rassoc '(97) '(("a" 97) ("b" 98)))     ; => ("a" 97)
(rassq '97 '((a . 97) (b . 98)))        ; => (a . 97)
如果要修改关键字对应的值,最省事的作法就是用 cons 把新的键值对加到列表的头端。但是这会让列表越来越长,浪费空间。如果要替换已经存在的值,一个想法就是用 setcdr 来更改键值对应的数据。但是在更改之前要先确定这个键值在对应的列表里,否则会产生一个错误。另一个想法是用 assoc 查找到对应的元素,再用 delq 删除这个数据,然后用 cons 加到列表里:
(setq foo '(("a" . 97) ("b" . 98)))     ; => (("a" . 97) ("b" . 98))

;; update value by setcdr
(if (setq bar (assoc "a" foo))
(setcdr bar "this is a")
(setq foo (cons '("a" . "this is a") foo))) ; => "this is a"
foo                         ; => (("a" . "this is a") ("b" . 98))
;; update value by delq and cons
(setq foo (cons '("a" . 97)
(delq (assoc "a" foo) foo))) ; => (("a" . 97) ("b" . 98))
如果不对顺序有要求的话,推荐用后一种方法吧。这样代码简洁,而且让最近更新的元素放到列表前端,查找更快。
\subsubsection{把列表当树用}
\label{sec:org8d644cc}
列表的第一个元素如果作为结点的数据,其它元素看作是子节点,就是一个树了。由于树的操作都涉及递归,现在还没有说到函数,我就不介绍了。(其实是我不太熟,就不班门弄斧了)。
\subsubsection{遍历列表}
\label{sec:org5983396}
遍历列表最常用的函数就是 mapc 和 mapcar 了。它们的第一个参数都是一个函数,这个函数只接受一个参数,每次处理一个列表里的元素。这两个函数唯一的差别是前者返回的还是输入的列表,而 mapcar 返回的函数返回值构成的列表:
(mapc '1+ '(1 2 3))                     ; => (1 2 3)
(mapcar '1+ '(1 2 3))                   ; => (2 3 4)
另一个比较常用的遍历列表的方法是用 dolist。它的形式是:
(dolist (var list [result]) body\ldots{})
其中 var 是一个临时变量,在 body 里可以用来得到列表中元素的值。使用 dolist 的好处是不用写lambda 函数。一般情况下它的返回值是 nil,但是你也可以指定一个值作为返回值(我觉得这个特性没有什么用,只省了一步而已):
(dolist (foo '(1 2 3))
(incf foo))                           ; => nil
(setq bar nil)
(dolist (foo '(1 2 3) bar)
(push (incf foo) bar))                ; => (4 3 2)
\subsubsection{其它常用函数}
\label{sec:org9de67da}
如果看过一些函数式语言教程的话,一定对 fold(或叫 accumulate、reduce)和 filter 这些函数记忆深刻。不过 elisp 里好像没有提供这样的函数。remove-if 和 remove-if-not 可以作 filter 函数,但是它们是 cl 里的,自己用用没有关系,不能强迫别人也跟着用,所以不能写到 elisp 里。如果不用这两个函数,也不用别人的函数的话,自己实现不妨用这样的方法:
(defun my-remove-if (predicate list)
(delq nil (mapcar (lambda (n)
(and (not (funcall predicate n)) n))
list)))
(defun evenp (n)
(= (\% n 2) 0))
(my-remove-if 'evenp '(0 1 2 3 4 5))    ; => (1 3 5)
fold 的操作只能用变量加循环或 mapc 操作来代替了:
(defun my-fold-left (op initial list)
(dolist (var list initial)
(setq initial (funcall op initial var))))
(my-fold-left '+ 0 '(1 2 3 4))          ; => 10
这里只是举个例子,事实上你不必写这样的函数,直接用函数里的遍历操作更好一些。
产生数列常用的方法是用 number-sequence(这里不禁用说一次,不要再用 loop 产生 tab-stop-list 了,你们 too old 了)。不过这个函数好像 在emacs21 时好像还没有。
解析文本时一个很常用的操作是把字符串按分隔符分解,可以用 split-string 函数:
(split-string "key = val" "$\backslash$\s-\textbf{=$\backslash$\s-}")  ; => ("key" "val")
与 split-string 对应是把几个字符串用一个分隔符连接起来,这可以用 mapconcat 完成。比如:
(mapconcat 'identity '("a" "b" "c") "\t") ; => "a   b   c"
identity 是一个特殊的函数,它会直接返回参数。mapconcat 第一个参数是一个函数,可以很灵活的使用。
\subsubsection{函数列表}
\label{sec:org0796677}
;; 列表测试
(consp OBJECT)
(listp OBJECT)
(null OBJECT)
;; 列表构造
(cons CAR CDR)
(list \&rest OBJECTS)
(append \&rest SEQUENCES)
;; 访问列表元素
(car LIST)
(cdr LIST)
(cadr X)
(caar X)
(cddr X)
(cdar X)
(nth N LIST)
(nthcdr N LIST)
(last LIST \&optional N)
(butlast LIST \&optional N)
;; 修改 cons cell
(setcar CELL NEWCAR)
(setcdr CELL NEWCDR)
;; 列表操作
(push NEWELT LISTNAME)
(pop LISTNAME)
(reverse LIST)
(nreverse LIST)
(sort LIST PREDICATE)
(copy-sequence ARG)
(nconc \&rest LISTS)
(nbutlast LIST \&optional N)
;; 集合函数
(delete-dups LIST)
(memq ELT LIST)
(member ELT LIST)
(delq ELT LIST)
(delete ELT SEQ)
(remq ELT LIST)
(remove ELT SEQ)
;; 关联列表
(assoc KEY LIST)
(assq KEY LIST)
(assoc-default KEY ALIST \&optional TEST DEFAULT)
(rassoc KEY LIST)
(rassq KEY LIST)
;; 遍历函数
(mapc FUNCTION SEQUENCE)
(mapcar FUNCTION SEQUENCE)
(dolist (VAR LIST [RESULT]) BODY\ldots{})
;; 其它
(number-sequence FROM \&optional TO INC)
(split-string STRING \&optional SEPARATORS OMIT-NULLS)
(mapconcat FUNCTION SEQUENCE SEPARATOR)
(identity ARG)
问题解答
用 list 生成 (a b c)
答案是 (list 'a 'b 'c)。很简单的一个问题。从这个例子可以看出为什么要想出 用 ' 来输入列表。这就是程序员“懒”的美德呀!
nthcdr 的一个实现
(defun my-nthcdr (n list)
(if (or (null list) (= n 0))
(car list)
(my-nthcdr (1- n) (cdr list))))
这样的实现看上去很简洁,但是一个最大的问题的 elisp 的递归是有限的,所以如果想这个函数没有问题,还是用循环还实现比较好。
my-subseq 函数的定义
(defun my-subseq (list from \&optional to)
(if (null to) (nthcdr from list)
(butlast (nthcdr from list) (- (length list) to))))
(setcdr foo foo) 是什么怪东西?
可能你已经想到了,这就是传说中的环呀。这在 info elisp - Circular Objects 里有介
绍。elisp 里用到这样的环状列表并不多见,但是也不是没有,org 和 session 那个 bug
就是由于一个环状列表造成的。

\subsection{{\bfseries\sffamily DONE} 基本数据类型之五 ── 符号}
\label{sec:org3f479b2}
符号是有名字的对象。可能这么说有点抽象。作个不恰当的比方,符号可以看作是 C
语言里的指针。

过符号你可以得到和这个符号相关联的信息,比如值,函数,属性列表等等。

首先必须知道的是符号的命名规则。符号名字可以含有任何字符。大多数的符号名字只含有
字母、数字和标点“-+=*/”。这样的名字需要其它标点。名字前缀要足够把符号名和数字
区分开来,如果需要的话,可以在前面用 $\backslash$ 表示为符号,比如:

(symbolp '+1)                           ; => nil
(symbolp '$\backslash$+1)                          ; => t
(symbol-name '$\backslash$+1)                      ; => "+1"
其它字符 \_\textasciitilde{}!@\$\%\^{}\&amp;:<>\{\}? 用的比较少。但是也可以直接作为符号的名字。任何其它字
符都可以用 $\backslash$ 转义后用在符号名字里。是和字符串里字符表示不同,$\backslash$ 转义后只是表示其
后的字符,比如 \t 代表的字符 t,而不是制表符。如果要在符号名里使用制表符,必须在
$\backslash$ 后加上制表符本身。

符号名是区分大小写的。这里有一些符号名的例子:
foo                 ; 名为 `foo' 的符号
FOO                 ; 名为 `FOO' 的符号,和 `foo' 不同
char-to-string      ; 名为 `char-to-string' 的符号
1+                  ; 名为 `1+' 的符号 (不是整数 `+1')
$\backslash$+1                 ; 名为 `+1' 的符号 (可读性很差的名字)
\(*\ 1\ 2\)         ; 名为 `(* 1 2)' 的符号 (更差劲的名字).
\sout{-*/\_\textasciitilde{}!@\$\%\^{}\&=:<>\{\}  ; 名为 `}-*/\_\textasciitilde{}!@\$\%\^{}\&=:<>\{\}' 的符号.
;   这些字符无须转义
\subsubsection{创建符号}
\label{sec:orgd98c0d1}

一个名字如何与数据对应上呢?这就要了解一下符号是如何创建的了。符号名要有唯一性,
所以一定会有一个表与名字关联,这个表在 elisp 里称为 obarray。从这个名字可以看出
这个表是用数组类型,事实上是一个向量。当 emacs 创建一个符号时,首先会对这个名字
求 hash 值以得到一个在 obarray 这个向量中查找值所用的下标。hash 是查找字符串的很
有效的方法。这里强调的是obarray 不是一个特殊的数据结构,就是一个一般的向量。全局
变量 obarray里 emacs 所有变量、函数和其它符号所使用的 obarray(注意不同语境中
obarray 的含义不同。前一个 obarray 是变量名,后一个 obarray 是数据类型名)。也可
以自己建立向量,把这个向量作为 obarray 来使用。这是一种代替散列的一种方法。它比
直接使用散列有这样一些好处:

符号不仅可以有一个值,还可以用属性列表,后者又可以相当于一个关联列表。这样有很高
的扩展性,而且可以表达更高级的数据结构。

emacs 里有一些函数可以接受 obarray 作为参数,比如补全相关的函数。

当 lisp 读入一个符号时,通常会先查找这个符号是否在 obarray 里出现过,如果没有则
会把这个符号加入到 obarray 里。这样查找并加入一个符号的过程称为是 intern。intern
函数可以查找或加入一个名字到 obarray 里,返回对应的符号。默认是全局的obarray,也
可以指定一个 obarray。intern-soft 与 intern 不同的是,当名字不在 obarray 里时,
intern-soft 会返回 nil,而 intern 会加入到 obarray里。为了不污染 obarray,我下面
的例子中尽量在 foo 这个 obarray 里进行。一般来说,去了 foo 参数,则会在 obarray
里进行。其结果应该是相同的:

(setq foo (make-vector 10 0))           ; => [0 0 0 0 0 0 0 0 0 0]
(intern-soft "abc" foo)                 ; => nil
(intern "abc" foo)                      ; => abc
(intern-soft "abc" foo)                 ; => abc
lisp 每读入一个符号都会 intern 到 obarray 里,如果想避免,可以用在符号名前加上 \#::
(intern-soft "abc")                     ; => nil
'abc                                    ; => abc
(intern-soft "abc")                     ; => abc
(intern-soft "abcd")                    ; => nil
'\#:abcd                                 ; => abcd
(intern-soft "abcd")                    ; => nil
如果想除去 obarray 里的符号,可以用 unintern 函数。unintern 可以用符号名或符号作
参数在指定的 obarray 里去除符号,成功去除则返回 t,如果没有查找到对应的符号则返
回 nil:

(intern-soft "abc" foo)                 ; => abc
(unintern "abc" foo)                    ; => t
(intern-soft "abc" foo)                 ; => nil
和 hash-table 一样,obarray 也提供一个 mapatoms 函数来遍历整个 obarray。比如要计
算 obarray 里所有的符号数量:

(setq count 0)                          ; => 0
(defun count-syms (s)
(setq count (1+ count)))              ; => count-syms
(mapatoms 'count-syms)                  ; => nil
count                                   ; => 28371
(length obarray)                        ; => 1511
\subsubsection{思考题}
\label{sec:org6954b94}
由前面的例子可以看出elisp 中的向量长度都是有限的,而 obarray 里的符号有成千上万
个。那这些符号是怎样放到 obarray 里的呢?

\subsubsection{符号的组成}
\label{sec:org126a957}
每个符号可以对应四个组成部分,一是符号的名字,可以用 symbol-name 访问。二是符号
的值。符号的值可以通过 set 函数来设置,用 symbol-value 来访问。

(set (intern "abc" foo) "I'm abc")      ; => "I'm abc"
(symbol-value (intern "abc" foo))       ; => "I'm abc"
可能大家最常见到 setq 函数,而 set 函数确很少见到。setq 可以看成是一个宏,它可以
让你用 (setq sym val) 代替 (set (quote sym) val)。事实上这也是它名字的来源 (q 代
表 quoted)。但是 setq 只能设置 obarray 里的变量,前面这个例子中就只能用 set 函数。

\subsubsection{思考题}
\label{sec:org93b918c}
参考 assoc-default 的代码,写一个函数从一个关联列表中除去一个关键字对应的元素。
这个函数可以直接修改关联列表符号的值。要求可以传递一个参数作为测试关键字是否相同
的函数。比如:

(setq foo '((?a . a) (?A . c) (?B . d)))
(remove-from-list 'foo ?b 'char-equal)  ; => ((97 . a) (65 . c))
foo                                     ; => ((97 . a) (65 . c))
如果一个符号的值已经有设置过的话,则 boundp 测试返回 t,否则为 nil。对于 boundp
测试返回 nil 的符号,使用符号的值会引起一个变量值为 void 的错误。

符号的第三个组成部分是函数。它可以用 symbol-function 来访问,用 fset 来设置
(fset (intern "abc" foo) (symbol-function 'car)) ; => \#<subr car>
(funcall (intern "abc" foo) '(a . b))            ; => a
类似的,可以用 fboundp 测试一个符号的函数部分是否有设置。
符号的第四个组成部分是属性列表(property list)。通常属性列表用于存储和符号相关的
信息,比如变量和函数的文档,定义的文件名和位置,语法类型。属性名和值可以是任意的
lisp 对象,但是通常名字是符号,可以用 get 和 put 来访问和修改属性值,用
symbol-plist 得到所有的属性列表:

(put (intern "abc" foo) 'doc "this is abc")      ; => "this is abc"
(get (intern "abc" foo) 'doc)                    ; => "this is abc"
(symbol-plist (intern "abc" foo))                ; => (doc "this is abc")
关联列表和属性列表很相似。符号的属性列表在内部表示上是用(prop1 value1 prop2
value2 \ldots{}) 的形式,和关联列表也是很相似的。属性列表在查找和这个符号相关的信息时,
要比直接用关联列表要简单快捷的多。所以变量的文档等信息都是放在符号的属性列表里。
但是关联表在头端加入元素是很快的,而且它可以删除表里的元素。而属性列表则不能删除
一个属性。

如果已经把属性列表取出,那么还可以用 plist-get 和 plist-put 的方法来访问和设置属
性列表

(plist-get '(foo 4) 'foo)               ; => 4
(plist-get '(foo 4 bad) 'bar)           ; => nil
(setq my-plist '(bar t foo 4))          ; => (bar t foo 4)
(setq my-plist (plist-put my-plist 'foo 69)) ; => (bar t foo 69)
(setq my-plist (plist-put my-plist 'quux '(a))) ; => (bar t foo 69 quux (a))
\subsubsection{思考题}
\label{sec:orgfb7ba87}
你能不能用已经学过的函数来实现 plist-get 和 plist-put?
\subsubsection{函数列表}
\label{sec:org74b4eb0}
(symbolp OBJECT)
(intern-soft NAME \&optional OBARRAY)
(intern STRING \&optional OBARRAY)
(unintern NAME \&optional OBARRAY)
(mapatoms FUNCTION \&optional OBARRAY)
(symbol-name SYMBOL)
(symbol-value SYMBOL)
(boundp SYMBOL)
(set SYMBOL NEWVAL)
(setq SYM VAL SYM VAL \ldots{})
(symbol-function SYMBOL)
(fset SYMBOL DEFINITION)
(fboundp SYMBOL)
(symbol-plist SYMBOL)
(get SYMBOL PROPNAME)
(put SYMBOL PROPNAME VALUE)
\subsubsection{问题解答}
\label{sec:org276bc18}
obarray 里符号数为什么大于向量长度
其实这和散列的的实现是一样的。obarray 里的每一个元素通常称为 bucket。 一个
bucket 是可以容纳多个相同 hash 值的字符串和它们的数据。我们可以用 这样的方法来模
拟一下:
\begin{SCR}
(defun hash-string (str)
(let ((hash 0) c)
(dotimes (i (length str))
(setq c (aref str i))
(if (> c \#o140)
(setq c (- c 40)))
(setq hash (+ (setq hash (lsh hash 3))
(lsh hash -28)
c)))
hash))

(let ((len 10) str hash)
(setq foo (make-vector len 0))
(dotimes (i (1+ len))
(setq str (char-to-string (+ ?a i))
hash (\% (hash-string str) len))
(message "I put \%s in slot \%d"
str hash)
(if (eq (aref foo hash) 0)
(intern str foo)
(message "I found \%S is already taking the slot: \%S"
(aref foo hash) foo)
(intern str foo)
(message "Now I'am in the slot too: \%S" foo))))
\end{SCR}

在我这里的输出是
I put a in slot 7
I put b in slot 8
I put c in slot 9
I put d in slot 0
I put e in slot 1
I put f in slot 2
I put g in slot 3
I put h in slot 4
I put i in slot 5
I put j in slot 6
I put k in slot 7
I found a is already taking the slot: [d e f g h i j a b c]
Now I'am in the slot too: [d e f g h i j k b c]
当然,这个 hash-string 和实际 obarray 里用的 hash-string 只是算法上是 相同的,但
是由于数据类型和 c 不是完全相同,所以对于长一点的字符串结果 可能不一样,我只好
用单个字符来演示一下。

根据关键字删除关联列表中的元素
\begin{SCR}
(defun remove-from-list (list-var key \&optional test)
(let ((prev (symbol-value list-var))
tail found value elt)
(or test (setq test 'equal))
(if (funcall test (caar prev) key)
(set list-var (cdr prev))
(setq tail (cdr prev))
(while (and tail (not found))
(setq elt (car tail))
(if (funcall test (car elt) key)
(progn
(setq found t)
(setcdr prev (cdr tail)))
(setq tail (cdr tail)
prev (cdr prev)))))
(symbol-value list-var)))
\end{SCR}
注意这个函数的参数 list-var 是一个符号,所以这个函数不能直接传递一个列表。这和
add-to-list 的参数是一样的。

plist-get 和 plist-put 的实现
(defun my-plist-get (plist prop)
(cadr (memq plist prop)))
(defun my-plist-put (plist prop val)
(let ((tail (memq prop plist)))
(if tail
(setcar (cdr tail) val)
(setcdr (last plist) (list prop val))))
plist)
my-plist-put 函数没有 plist-put 那样 robust,如果属性列表是 '(bar t foo) 这样的
话,这个函数就会出错。而且加入一个属性的时间复杂度比 plist 更高(memq 和 last 都
是 O(n)),不过可以用循环来达到相同的时间复杂度。

\subsection{求值规则}
\label{sec:orgd8b3cf7}
至此,elisp 中最常见的数据类型已经介绍完了。我们可以真正开始学习怎样写一个 elisp
程序。如果想深入了解一下 lisp 是如何工作的,不妨先花些时间看看 lisp 的求值过程。
当然忽略这一部分也是可以的,因为我觉得这个求值规则是那么自然,以至于你会认为它就
是应该这样的。

求值是 lisp 解释器的核心,理解了求值过程也就学会了 lisp 编程的一半。正因为这样,
我有点担心自己说得不清楚或者理解错误,会误导了你。所以如果真想深入了解的话,还是
自己看 \emph{info elisp - Evaluation} 这一章吧。

一个要求值的 lisp 对象被称为表达式(form)。所有的表达式可以分为三种:符号、列表
和其它类型(废话)。下面一一说明各种表达式的求值规则。

第一种表达式是最简单的,自求值表达式。前面说过数字、字符串、向量都是自求值表达式。
还有两个特殊的符号 t 和 nil 也可以看成是自求值表达式。

第二种表达式是符号。符号的求值结果就是符号的值。如果它没有值,就会出现
void-variable 的错误。

第三种表达式是列表表达式。而列表表达式又可以根据第一个元素分为函数调用、宏调用和
特殊表达式(special form)三种。列表的第一个表达式如果是一个符号,解释器会查找这
个表达式的函数值。如果函数值是另一个符号,则会继续查找这个符号的函数值。这称为
“symbol function indirection”。最后直到某个符号的函数值是一个 lisp 函数
(lambda 表达式)、byte-code 函数、原子函数(primitive function)、宏、特殊表达
式或 autoload 对象。如果不是这些类型,比如某个符号的函数值是前面出现的某个符号导
致无限循环,或者某个符号函数值为空,都会导致一个错误 invalid-function。

这个函数显示 indirection function
(symbol-function 'car)                  ; => \#<subr car>
(fset 'first 'car)                      ; => car
(fset 'erste 'first)                    ; => first
(erste '(1 2 3))                        ; => 1
对于第一个元素是 lisp 函数对象、byte-code 对象和原子函数时,这个列表也称为函数调
用(funtion call)。对这样的列表求值时,先对列表中其它元素先求值,求值的结果作为
函数调用的真正参数。然后使用 apply 函数用这些参数调用函数。如果函数是用 lisp 写
的,可以理解为把参数和变量绑定到函数后,对函数体顺序求值,返回最后一个 form 的值。

如果第一个元素是一个宏对象,列表里的其它元素不会立即求值,而是根据宏定义进行扩展。
如果扩展后还是一个宏调用,则会继续扩展下去,直到扩展的结果不再是一个宏调用为止。
例如

(defmacro cadr (x)
(list 'car (list 'cdr x)))
这样 (cadr (assq 'handler list)) 扩展后成为 (car (cdr (assq 'handler list)))。
第一个元素如果是一个特殊表达式时,它的参数可能并不会全求值。这些特殊表达式通常是
用于控制结构或者变量绑定。每个特殊表达式都有对应的求值规则。这在下面会提到。

最后用这个伪代码来说明一下 elisp 中的求值规则:
(defun (eval exp)
(cond
((numberp exp) exp)
((stringp exp) exp)
((arrayp exp) exp)
((symbolp exp) (symbol-value exp))
((special-form-p (car exp))
(eval-special-form exp))
((fboundp (car exp))
(apply (car exp) (cdr exp)))
(t
(error "Unknown expression type -- EVAL \%S" exp))))
\subsection{变量}
\label{sec:orgbc0c2d0}
\begin{SCR}
在此之前,我们已经见过 elisp 中的两种变量,全局变量和 let 绑定的局部变量。它们相当于其它语言中的全局变量和局部变量。
关于 let 绑定的变量,有两点需要补充的。当同一个变量名既是全局变量也是局部变量,或者用 let 多层绑定,只有最里层的那个变量是有效的,用 setq 改变的也只是最里层的变量,而不影响外层的变量。比如
(progn
(setq foo "I'm global variable!")
(let ((foo 5))
(message "foo value is: \%S" foo)
(let (foo)
(setq foo "I'm local variable!")
(message foo))
(message "foo value is still: \%S" foo))
(message foo))
另外需要注意一点的是局部变量的绑定不能超过一定的层数,也就是说,你不能把 foo 用 let 绑定 10000 层。当然普通的函数是不可能写成这样的,但是递归函数就不一定了。限制层数的变量在 max-specpdl-size 中定义。如果你写的递归函数有这个需要的话,可以先设置这个变量的值。
emacs 有一种特殊的局部变量 ── buffer-local 变量。
buffer-local 变量
emacs 能有如此丰富的模式,各个缓冲区之间能不相互冲突,很大程度上要归功于 buffer-local 变量。
声明一个 buffer-local 的变量可以用 make-variable-buffer-local 或用 make-local-variable。这两个函数的区别在于前者是相当于在所有变量中都产生一个 buffer-local 的变量。而后者只在声明时所在的缓冲区内产生一个局部变量,而其它缓冲区仍然使用的是全局变量。一般来说推荐使用 make-local-variable。
为了方便演示,下面的代码我假定你是在 \textbf{scratch} 缓冲区里运行。我使用另一个一般都会有的缓冲区 \textbf{Messages} 作为测试。先介绍两个用到的函数( with-current-buffer 其实是一个宏)。
with-current-buffer 的使用形式是
(with-current-buffer buffer
body)
其中 buffer 可以是一个缓冲区对象,也可以是缓冲区的名字。它的作用是使其中的 body 表达式在指定的缓冲区里执行。
get-buffer 可以用缓冲区的名字得到对应的缓冲区对象。如果没有这样名字的缓冲区会返回 nil。
下面是使用 buffer-local 变量的例子:
(setq foo "I'm global variable!")       ; => "I'm global variable!"
(make-local-variable 'foo)              ; => foo
foo                                     ; => "I'm global variable!"
(setq foo "I'm buffer-local variable!") ; => "I'm buffer-local variable!"
foo                                  ; => "I'm buffer-local variable!"
(with-current-buffer "\textbf{Messages}" foo)  ; => "I'm global variable!"
从这个例子中可以看出,当一个符号作为全局变量时有一个值的话,用 make-local-variable 声明为 buffer-local 变量时,这个变量的值还是全局变量的值。这时候全局的值也称为缺省值。你可以用 default -value 来访问这个符号的全局变量的值
(default-value 'foo)                    ; => "I'm global variable!"
如果一个变量是 buffer-local,那么在这个缓冲区内使用用 setq 就只能用改变当前缓冲区里这个变量的值。setq-default 可以修改符号作为全局变量的值。通常在 .emacs 里经常使用 setq-default,这样可以防止修改的是导入 .emacs 文件对应的缓冲区里的 buffer-local 变量,而不是设置全局的值。
测试一个变量是不是 buffer-local 可以用 local-variable-p
(local-variable-p 'foo)                           ; => t
(local-variable-p 'foo (get-buffer "\textbf{Messages}")) ; => nil
如果要在当前缓冲区里得到其它缓冲区的 buffer-local 变量可以用 buffer-local-value
(with-current-buffer "\textbf{Messages}"
(buffer-local-value 'foo (get-buffer "\textbf{scratch}")))
; => "I'm buffer local variable!"
变量的作用域
我们现在已经学习这样几种变量:
全局变量
buffer-local 变量
let 绑定局部变量
如果还要考虑函数的参数列表声明的变量,也就是 4 种类型的变量。那这种变量的作用范围(scope)和生存期(extent)分别是怎样的呢?
作用域(scope)是指变量在代码中能够访问的位置。emacs lisp 这种绑定称为 indefinite scope。indefinite scope 也就是说可以在任何位置都可能访问一个变量名。而 lexical scope(词法作用域)指局部变量只能作用在函数中和一个块里(block)。
比如 let 绑定和函数参数列表的变量在整个表达式内都是可见的,这有别于其它语言词法作用域的变量。先看下面这个例子:
(defun binder (x)                      ; `x' is bound in `binder'.
(foo 5))                             ; `foo' is some other function.
(defun user ()                         ; `x' is used "free" in `user'.
(list x))
(defun foo (ignore)
(user))
(binder 10)                            ; => (10)
对于词法作用域的语言,在 user 函数里无论如何是不能访问 binder 函数中绑定的 x。但是在 elisp 中可以。
生存期是指程序运行过程中,变量什么时候是有效的。全局变量和 buffer-local 变量都是始终存在的,前者只能当关闭emacs 或者用 unintern 从 obarray 里除去时才能消除。而 buffer-local 的变量也只能关闭缓冲区或者用 kill-local-variable 才会消失。而对于局部变量,emacs lisp 使用的方式称为动态生存期:只有当绑定了这个变量的表达式运行时才是有效的。这和 C 和 Pascal 里的 Local 和automatic 变量是一样的。与此相对的是 indefinite extent,变量即使离开绑定它的表达式还能有效。比如:
(defun make-add (n)
(function (lambda (m) (+ n m))))      ; Return a function.
(fset 'add2 (make-add 2))               ; Define function `add2'
;   with `(make-add 2)'.
(add2 4)                                ; Try to add 2 to 4.
其它 Lisp 方言中有闭包,但是 emacs lisp 中没有。
说完这些概念,可能你还是一点雾水。我给一个判断变量是否有效的方法吧:
看看包含这个变量的 form 中是否有 let 绑定这个局部变量。如果这个 form 不是在定义一个函数,则跳到第 3 步。
如果是在定义函数,则不仅要看这个函数的参数中是否有这个变量,而且还要看所有直接或间接调用这个函数的函数中是否有用 let 绑定或者参数列表里有这个变量名。这就没有办法确定了,所以你永远无法判断一个函数中出现的没有用 let 绑定,也不在参数列表中的变量是否是没有定义过的。但是一般来说这不是一个好习惯。
看这个变量是否是一个全局变量或者是 buffer-local 变量。
对于在一个函数中绑定一个变量,而在另一个函数中还在使用,manual 里认为这两个种情况下是比较好的:
这个变量只有相关的几个函数中使用,在一个文件中放在一起。这个变量起程序里通信的作用。而且需要写好注释告诉其它程序员怎样使用它。
如果这个变量是定义明确、有很好文档作用的,可能让所有函数使用它,但是不要设置它。比如 case-fold-search。(我怎么觉得这里是用全局变量呢。)
思考题
先在 scratch 缓冲区里运行了 (kill-local-variable 'foo) 后,运行几次下面的表达式,你能预测它们结果吗?
(progn
(setq foo "I'm local variable!")
(let ((foo "I'm local variable!"))
(set (make-local-variable 'foo) "I'm buffer-local variable!")
(setq foo "This is a variable!")
(message foo))
(message foo))
其它函数
一个符号如果值为空,直接使用可能会产生一个错误。可以用 boundp 来测试一个变量是否有定义。这通常用于 elisp 扩展的移植(用于不同版本或 XEmacs)。对于一个 buffer-local 变量,它的缺省值可能是没有定义的,这时用 default-value 函数可能会出错。这时就先用 default-boundp 先进行测试。
使一个变量的值重新为空,可以用 makunbound。要消除一个 buffer-local 变量用函数 kill-local-variable。可以用 kill-all-local-variables 消除所有的 buffer-local 变量。但是有属性 permanent-local 的不会消除,带有这些标记的变量一般都是和缓冲区模式无关的,比如输入法。
foo                                     ; => "I'm local variable!"
(boundp 'foo)                           ; => t
(default-boundp 'foo)                   ; => t
(makunbound 'foo)                       ; => foo
foo                                     ; This will signal an error
(default-boundp 'foo)                   ; => t
(kill-local-variable 'foo)              ; => foo
变量名习惯
对于变量的命名,有一些习惯,这样可以从变量名就能看出变量的用途:
hook 一个在特定情况下调用的函数列表,比如关闭缓冲区时,进入某个模式时。
function 值为一个函数
functions 值为一个函数列表
flag 值为 nil 或 non-nil
predicate 值是一个作判断的函数,返回 nil 或 non-nil
program 或 -command 一个程序或 shell 命令名
form 一个表达式
forms 一个表达式列表。
map 一个按键映射(keymap)
函数列表
(make-local-variable VARIABLE)
(make-variable-buffer-local VARIABLE)
(with-current-buffer BUFFER \&rest BODY)
(get-buffer NAME)
(default-value SYMBOL)
(local-variable-p VARIABLE \&optional BUFFER)
(buffer-local-value VARIABLE BUFFER)
(boundp SYMBOL)
(default-boundp SYMBOL)
(makunbound SYMBOL)
(kill-local-variable VARIABLE)
(kill-all-local-variables)
变量列表
max-specpdl-size
问题解答
同一个表达式运行再次结果不同?
运行第一次时,foo 缺省值为 "I'm local variable!",而 buffer-local 值为 "This is a variable!"。第一个和第二个 message 都会显示 "This is a variable!"。运行第二次时,foo 缺省值和 buffer-local 值都成了 "I'm local variable!",而第一次 message 显示 "This is a variable!",第二次 显示 "I'm local variable!"。这是由于 make-local-variable 在这个符号是 否已经是 buffer-local 变量时有不同表现造成的。如果已经是一个 buffer-local 变量,则它什么也不做,而如果不是,则会生成一个 buffer-local 变量,这时在这个表达式内的所有 foo 也被重新绑定了。希望你 写的函数能想到一点。
\end{SCR}
\subsection{{\bfseries\sffamily DONE} 函数和命令}
\label{sec:orge0fd629}
在 elisp 里类似函数的对象很多,比如:
函数。这里的函数特指用 lisp 写的函数。
原子函数(primitive)。用 C 写的函数,比如 car、append。
lambda 表达式
特殊表达式
宏(macro)。宏是用 lisp 写的一种结构,它可以把一种 lisp 表达式转换成等价的另一个
表达式。

命令。命令能用 command-execute 调用。函数也可以是命令。
以上这些用 functionp 来测试都会返回 t。
我们已经学过如何定义一个函数。但是这些函数的参数个数都是确定。但是你可以看到
emacs 里有很多函数是接受可选参数,比如 random 函数。还有一些函数可以接受不确定的
参数,比如加减乘除。这样的函数在 elisp 中是如何定义的呢?

\subsubsection{参数列表的语法}
\label{sec:org7ad3995}

这是参数列表的方法形式:
(REQUIRED-VARS\ldots{}
[\&optional OPTIONAL-VARS\ldots{}]
[\&rest REST-VAR])
它的意思是说,你必须把必须提供的参数写在前面,可选的参数写在后面,最后用一个符号
表示剩余的所有参数。比如

(defun foo (var1 var2 \&optional opt1 opt2 \&rest rest)
(list var1 var2 opt1 opt2 rest))

(foo 1 2)                               ; => (1 2 nil nil nil)
(foo 1 2 3)                             ; => (1 2 3 nil nil)
(foo 1 2 3 4 5 6)                       ; => (1 2 3 4 (5 6))
从这个例子可以看出,当可选参数没有提供时,在函数体里,对应的参数值都是 nil。同样
调用函数时没有提供剩余参数时,其值也为 nil,但是一旦提供了剩余参数,则所有参数是
以列表的形式放在对应变量里。

\subsubsection{思考题}
\label{sec:org9e02320}
写一个函数测试两个浮点数是否相等,设置一个可选参数,如果提供这个参数,则用这个参
数作为测试误差,否则用 1.0e-6 作为误差。

\subsubsection{关于文档字符串}
\label{sec:orgc07fa80}

最好为你的函数都提供一个文档字符串。关于文档字符串有一些规范,最好遵守这些约定。
字符串的第一行最好是独立的。因为 apropos 命令只能显示第一行的文档。所以最好用一
行(一两个完整的句子)总结这个函数的目的。

文档的缩进最好要根据最后的显示的效果来调用。因为引号之类字符会多占用一个字符,所
以在源文件里缩进最好看,不一定显示的最好。

如果你想要让你的函数参数显示的与函数定义的不同(比如提示用户如何调用这个函数),
可以在文档最后一行,加上一行:

$\backslash$(fn ARGLIST)
注意这一行前面要有一个空行,这一行后不能再有空行。比如
(defun foo (var1 var2 \&optional opt1 opt2 \&rest rest)
"You should call the function like:

$\backslash$(fn v1 v2)"
(list var1 var2 opt1 opt2 rest))
还有一些有特殊标记功能的符号,比如 `' 引起的符号名可以生成一个链接,这样可以在
\textbf{Help} 中更方便的查看相关变量或函数的文档。$\backslash$\\{major-mode-map\} 可以显示扩展成这
个模式按键的说明,例如:

(defun foo ()
"A simple document string to show how to use `' and $\backslash$\=$\backslash$\\{\}.
You can press this button `help' to see the document of
function $\backslash$"help$\backslash$".

This is keybind of text-mode(substitute from $\backslash$\=$\backslash$\\{text-mode-map\}):
$\backslash$\\{text-mode-map\}

See also `substitute-command-keys' and `documentation'"
)

\subsubsection{调用函数}
\label{sec:orgc82bb1b}

通常函数的调用都是用 eval 进行的,但是有时需要在运行时才决定使用什么函数,这时就
需要用 funcall 和 apply 两个函数了。这两个函数都是把其余的参数作为函数的参数进行
调用。那这两个函数有什么参数呢?唯一的区别就在于 funcall 是直接把参数传递给函数,
而 apply 的最后一个参数是一个列表,传入函数的参数把列表进行一次平铺后再传给函数,
看下面这个例子就明白了

(funcall 'list 'x '(y) '(z))               ; => (x (y) (z))
(apply 'list 'x '(y ) '(z))                ; => (x (y) z)
\subsubsection{思考题}
\label{sec:org64d55c0}
如果一个 list 作为一个树的结构,任何是 cons cell 的元素都是一个内部节点(不允许
有 dotted list 出现),任何不是 cons cell 的元素都是树的叶子。请写一个函数,调用
的一个类似 mapcar 的函数,调用一个函数遍历树的叶子,并收集所有的结果,返回一个结
构相同的树,比如:

(tree-mapcar '1+ '(1 (2 (3 4)) (5)))    ; => (2 (3 (4 5)) (6))

\subsubsection{宏}
\label{sec:org722f383}

前面在已经简单介绍过宏。宏的调用和函数是很类似的,它的求值和函数差不多,但是有一
个重要的区别是,宏的参数是出现在最后扩展后的表达式中,而函数参数是求值后才传递给
这个函数:

(defmacro foo (arg)
(list 'message "\%d \%d" arg arg))

(defun bar (arg)
(message "\%d \%d" arg arg))

(let ((i 1))
(bar (incf i)))                       ; => "2 2"

(let ((i 1))
(foo (incf i)))                       ; => "2 3"
也许你对前面这个例子 foo 里为什么要用 list 函数很不解。其实宏可以这样看,如果把
宏定义作一个表达式来运行,最后把参数用调用时的参数替换,这样就得到了宏调用最后
用于求值的表达式。这个过程称为扩展。可以用 macroexpand 函数进行模拟

(macroexpand '(foo (incf i))) ; => (message "\%d \%d" (incf i) (incf i))
上面用 macroexpand 得到的结果就是用于求值的表达式。
使用 macroexpand 可以使宏的编写变得容易一些。但是如果不能进行 debug 是很不方便的。
在宏定义里可以引入 declare 表达式,它可以增加一些信息。目前只支持两类声明:debug
和 indent。debug 可选择的类型很多,具体参考 info elisp - Edebug 一章,一般情况下
用 t 就足够了。indent 的类型比较简单,它可以使用这样几种类型:

nil 也就是一般的方式缩进
defun 类似 def 的结构,把第二行作为主体,对主体里的表达式使用同样的缩进
整数 表示从第 n 个表达式后作为主体。比如 if 设置为 2,而 when 设置为 1
符号 这个是最坏情况,你要写一个函数自己处理缩进。
看 when 的定义就能知道 declare 如何使用了
(defmacro when (cond \&rest body)
(declare (indent 1) (debug t))
(list 'if cond (cons 'progn body)))
实际上,declare 声明只是设置这个符号的属性列表
(symbol-plist 'when)    ; => (lisp-indent-function 1 edebug-form-spec t)
\subsubsection{思考题}
\label{sec:org5d081c9}
一个比较常用的结构是当 buffer 是可读情况下,绑定 inhibit-read-only 值为 t 来强制
插入字符串。请写一个这样的宏,处理好缩进和调用。

从前面宏 when 的定义可以看出直接使用 list,cons,append 构造宏是很麻烦的。为了使
记号简洁,lisp 中有一个特殊的宏 "`",称为 backquote。在这个宏里,所有的表达式都
是引起(quote)的,如果要让一个表达式不引起(也就是列表中使用的是表达式的值),
需要在前面加 “,”,如果要让一个列表作为整个列表的一部分(slice),可以用 ",@"。

`(a list of ,(+ 2 3) elements)          ; => (a list of 5 elements)
(setq some-list '(2 3))                 ; => (2 3)
`(1 ,some-list 4 ,@some-list)           ; => (1 (2 3) 4 2 3)
有了这些标记,前面 when 这个宏可以写成
(defmacro when (cond \&rest body)
`(if ,cond
(progn ,@body)))
值得注意的是这个 backquote 本身就是一个宏,从这里可以看出宏除了减少重复代码这个
作用之外的另一个用途:定义新的控制结构,甚至增加新的语法特性。

\subsubsection{命令}
\label{sec:org655de27}

emacs 运行时就是处于一个命令循环中,不断从用户那得到按键序列,然后调用对应命令来
执行。lisp 编写的命令都含有一个 interactive 表达式。这个表达式指明了这个命令的参
数。比如下面这个命令
\begin{SCR}
(defun hello-world (name)
(interactive "sWhat you name? ")
(message "Hello, \%s" name))
\end{SCR}
现在你可以用 M-x 来调用这个命令。让我们来看看 interactive 的参数是什么意思。这个
字符串的第一个字符(也称为代码字符)代表参数的类型,比如这里 s 代表参数的类型
是一个字符串,而其后的字符串是用来提示的字符串。如果这个命令有多个参数,可以在
这个提示字符串后使用换行符分开,比如:

(defun hello-world (name time)
(interactive "sWhat you name? \nnWhat the time? ")
(message "Good \%s, \%s"
(cond ((< time 13) "morning")
((< time 19) "afternoon")
(t "evening"))
name))
interactive 可以使用的代码字符很多,虽然有一定的规则,比如字符串用 s,数字用 n,
文件用 f,区域用 r,但是还是很容易忘记,用的时候看 interactive 函数的文档还是
很有必要的。但是不是所有时候都参数类型都能使用代码字符,而且一个好的命令,应该
尽可能的让提供默认参数以让用户少花时间在输入参数上,这时,就有可能要自己定制参
数。

首先学习和代码字符等价的几个函数。s 对应的函数是 read-string。比如
(read-string "What your name? " user-full-name)
n 对应的函数是 read-number,文件对应 read-file-name。很容易记对吧。其实大部分代
码字符都是有这样对应的函数或替换的方法(见下表)。

代码字符	代替的表达式
a	(completing-read prompt obarray 'fboundp t)
b	(read-buffer prompt nil t)
B	(read-buffer prompt)
c	(read-char prompt)
C	(read-command prompt)
d	(point)
D	(read-directory-name prompt)
e	(read-event)
f	(read-file-name prompt nil nil t)
F	(read-file-name prompt)
G	暂时不知道和 f 的差别
k	(read-key-sequence prompt)
K	(read-key-sequence prompt nil t)
m	(mark)
n	(read-number prompt)
N	(if current-prefix-arg (prefix-numeric-value current-prefix-arg) (read-number prompt))
p	(prefix-numeric-value current-prefix-arg)
P	current-prefix-arg
r	(region-beginning) (region-end)
s	(read-string prompt)
S	(completing-read prompt obarray nil t)
v	(read-variable prompt)
x	(read-from-minibuffer prompt nil nil t)
X	(eval (read-from-minibuffer prompt nil nil t))
z	(read-coding-system prompt)
Z	(and current-prefix-arg (read-coding-system prompt))
知道这些表达式如何用于 interactive 表达式里呢?简而言之,如果 interactive 的参数
是一个表达式,则这个表达式求值后的列表元素对应于这个命令的参数。请看这个例子:

(defun read-hiden-file (file arg)
(interactive
(list (read-file-name "Choose a hiden file: " "\textasciitilde{}/" nil nil nil
(lambda (name)
(string-match "\^{}$\backslash$\." (file-name-nondirectory name))))
current-prefix-arg))
(message "\%s, \%S" file arg))
第一个参数是读入一个以 "." 开头的文件名,第二个参数为当前的前缀参数(prefix
argument),它可以用 C-u 或 C-u 加数字提供。list 把这两个参数构成一个列表。这
就是命令一般的自定义设定参数的方法。

需要注意的是 current-prefix-arg 这个变量。这个变量当一个命令被调用,它就被赋与一
个值,你可以用 C-u 就能改变它的值。在命令运行过程中,它的值始终都存在。即使你
的命令不用参数,你也可以访问它

(defun foo ()
(interactive)
(message "\%S" current-prefix-arg))
用 C-u foo 调用它,你可以发现它的值是 (4)。那为什么大多数命令还单独为它设置一个
参数呢?这是因为命令不仅是用户可以调用,很可能其它函数也可以调用,单独设置一个
参数可以方便的用参数传递的方法调用这个命令。事实上所有的命令都可以不带参数,而
使用前面介绍的方法在命令定义的部分读入需要的参数,但是为了提高命令的可重用性和
代码的可读性,还是把参数分离到 interactive 表达式里好。

从现在开始可能会遇到很多函数,它们的用法有的简单,有的却复杂的要用大段篇幅来解释。
我可能就会根据需要来解释一两个函数,就不一一介绍了。自己看 info elisp,用 i 来查
找对应的函数。

\subsubsection{思考题}
\label{sec:orgd5675f6}
写一个命令用来切换 major-mode。要求用户输入一个 major-mode 的名字,就切换到这个
major-mode,而且要提供一种补全的办法,去除所有不是 major-mode 的符号,这样用户需
要输入少量词就能找到对应的 major-mode。

\subsubsection{函数列表}
\label{sec:orgfba98fe}
(functionp OBJECT)
(apply FUNCTION \&rest ARGUMENTS)
(funcall FUNCTION \&rest ARGUMENTS)
(defmacro NAME ARGLIST [DOCSTRING] [DECL] BODY\ldots{})
(macroexpand FORM \&optional ENVIRONMENT)
(declare \&rest SPECS)
(` ARG)
(interactive ARGS)
(read-string PROMPT \&optional INITIAL-INPUT HISTORY DEFAULT-VALUE
INHERIT-INPUT-METHOD)
(read-file-name PROMPT \&optional DIR DEFAULT-FILENAME MUSTMATCH
INITIAL PREDICATE)
(completing-read PROMPT COLLECTION \&optional PREDICATE
REQUIRE-MATCH INITIAL-INPUT HIST DEF
INHERIT-INPUT-METHOD)
(read-buffer PROMPT \&optional DEF REQUIRE-MATCH)
(read-char \&optional PROMPT INHERIT-INPUT-METHOD SECONDS)
(read-command PROMPT \&optional DEFAULT-VALUE)
(read-directory-name PROMPT \&optional DIR DEFAULT-DIRNAME
MUSTMATCH INITIAL)
(read-event \&optional PROMPT INHERIT-INPUT-METHOD SECONDS)
(read-key-sequence PROMPT \&optional CONTINUE-ECHO
DONT-DOWNCASE-LAST CAN-RETURN-SWITCH-FRAME
COMMAND-LOOP)
(read-number PROMPT \&optional DEFAULT)
(prefix-numeric-value RAW)
(read-from-minibuffer PROMPT \&optional INITIAL-CONTENTS KEYMAP
READ HIST DEFAULT-VALUE INHERIT-INPUT-METHOD)
(read-coding-system PROMPT \&optional DEFAULT-CODING-SYSTEM)
变量列表
current-prefix-arg
\subsubsection{问题解答}
\label{sec:org1aad77b}
\subsubsection{可选误差的浮点数比较}
\label{sec:orge56996f}
(defun approx-equal (x y \&optional err)
(if err
(setq err (abs err))
(setq err 1.0e-6))
(or (and (= x 0) (= y 0))
(< (/ (abs (- x y))
(max (abs x) (abs y)))
err)))
这个应该是很简单的一个问题。
\subsubsection{遍历树的函数}
\label{sec:orgb752973}
(defun tree-mapcar (func tree)
(if (consp tree)
(mapcar (lambda (child)
(tree-mapcar func child))
tree)
(funcall func tree)))
这个函数可能对于树算法比较熟悉的人一点都不难,就当练手吧。
宏 with-inhibit-read-only-t
(defmacro with-inhibit-read-only-t (\&rest body)
(declare (indent 0) (debug t))
(cons 'let (cons '((inhibit-read-only t))
body)))
如果用 backquote 来改写一个就会发现这个宏会很容易写,而且更容易读了。
\subsubsection{切换 major-mode 的命令}
\label{sec:org2e8c4d3}
(defvar switch-major-mode-history nil)
(defun switch-major-mode (mode)
(interactive
(list
(intern
(completing-read "Switch to mode: "
obarray (lambda (s)
(and (fboundp s)
(string-match "-mode\$" (symbol-name s))))
t nil 'switch-major-mode-history))))
(setq switch-major-mode-history
(cons (symbol-name major-mode) switch-major-mode-history))
(funcall mode))
这是我常用的一个命令之一。这个实现也是一个使用 minibuffer 历史的例子。
\subsection{{\bfseries\sffamily DONE} 正则表达式}
\label{sec:orgd74e4cc}
如果你不懂正则表达式,而你还想进一步学习编程的话,那你应该停下手边的事情,先学学
正则表达式。即使你不想学习编程,也不喜欢编程,学习一点正则表达式,也可能让你的文
本编辑效率提高很多。

在这里,我不想详细介绍正则表达式,因为我觉得这类的文档已经很多了,比我写得好的文
章多的是。如果你找不到一个好的入门教程,我建议你不妨看看 perlretut。我想说的是和
emacs 有关的正则表达式的内容,比如,和 Perl 正则表达式的差异、语法表格(syntax
table)和字符分类(category)等。
\subsubsection{与 Perl 正则表达式比较}
\label{sec:orgb834faa}
Perl是文本处理的首选语言。它内置强大而简洁的正则表达式,许多程序也都兼容 Perl
的正则表达式。说实话,就简洁而言,我对 emacs 的正则表达式是非常反感的,那么多的
反斜线经常让我抓狂。首先,emacs 里的反斜线构成的特殊结构(backslash construct)
出现是相当频繁的。在 Perl 正则表达式里,()[]\{\}| 都是特殊字符,而 emacs 它们不是
这样。所以它们匹配字符时是不用反斜线,而作为特殊结构时就要用反斜线。而事实上()|
作为字符来匹配的情形远远小于作为捕捉字符串和作或运算的情形概率小。而 emacs 的正
则表达式又没有 Perl 那种简洁的记号,完全用字符串来表示,这样使得一个正则表达式常
常一眼看去全是 $\backslash$\。到底要用多少个$\backslash$? 经常会记不住在 emacs 的正则表达式中应该用几个 $\backslash$。有一个比较好
的方法,首先想想没有引号引起时的正则表达式是怎样。比如对于特殊字符 \$ 要用 $\backslash$$,对
于反斜线结构是 $\backslash$(, $\backslash${,$\backslash$| 等等。知道这个怎样写之后,再把所有 $\backslash$ 替换成 $\backslash$\,这就是
最后写到双引号里形式。所以要匹配一个 $\backslash$,应该用 $\backslash$\,而写在引号里就应该用 $\backslash$\$\backslash$\ 来
匹配。

emacs 里匹配的对象不仅包括字符串,还有 buffer,所以有一些对字符串和 buffer 有区
分的结构。比如 \$ 对于字符串是匹配字符串的末尾,而在 buffer 里是行尾。而 $\backslash$' 匹配
的是字符串和 buffe 的末尾。对应 \^{} 和 $\backslash$` 也是这样。

emacs 对字符有很多种分类方法,在正则表达式里也可以使用。比如按语法类型分类,可以
用 "\s" 结构匹配一类语法分类的字符,最常用的比如匹配空格的 \s- 和匹配词的
\sw(等价于 \w)。这在 Perl 里是没有的。另外 emacs 里字符还对应一个或多个分类
(category),比如所有汉字在分类 c 里。这样可以用 \cc 来匹配一个汉字。这在 Perl
里也有类似的分类。除此之外,还有一些预定义的字符分类,可以用 [:class:] 的形式,
比如 [:digit:] 匹配 0-9 之间的数,[:ascii:] 匹配所有 ASCII 字符等等。在 Perl 里
只定义几类最常用的字符集,比如 \d, \s, \w,但是我觉得非常实用。比 emacs 这样长的
标记好用的多。

另外在用 [] 表示一个字符集时,emacs 里不能用 $\backslash$ 进行转义,事实上 $\backslash$ 在这里不是一个
特殊字符。所以 emacs 里的处理方法是,把特殊字符提前或放在后面,比如如果要在字符
集里包括 ] 的话,要把 ] 放在第一位。而如果要包括 -,只能放在最后一位,如果要包括
\^{} 不能放在第一位。如果只想匹配一个 \^{},就只能用 $\backslash$^ 的形式。比较拗口,希望下面这个
例子能帮你理解
(let ((str "abc]-\^{}]123"))
(string-match "[]\(^{\text{0}}\)-9-]+" str)
(match-string 0 str))                 ; => "]-\^{}]123"
最后提示一下,emacs 提供一个很好的正则表达式调试工具:M-x re-builder。它能显示
buffer 匹配正则表达式的字符串,并用不同颜色显示捕捉的字符串。

\subsubsection{语法表格和分类表格简介}
\label{sec:org74cb1de}

语法表格指的是 emacs 为每个字符都指定了语法功能,这为解析函数,复杂的移动命令等
等提供了各种语法结构的起点和终点。语法表使用的数据结构是一种称为字符表
(char-table)的数组,它能以字符作为下标(还记得 emacs 里的字符就是整数吗)来得
到对应的值。语法表里一个字符对应一个语法分类的列表。每一个分类都有一个助记字符
(mnemonic character)。一共有哪几类分类呢?
名称	助记符	说明
空白(whitespace)	- 或 ' '
词(word)	w
符号(symbol)	\_	这是除 word 之外其它用于变量和命令名的字符。
标点(punctuation)	.
open 和 close	( 和 )	一般是括号 ()[]\{\}
字符串引号(string quote)	"
转义符(escape-syntax)	$\backslash$	用于转义序列,比如 C 和 Lisp 字符串中的 $\backslash$。
字符引号(character quote)	/
paired delimiter	\$	只有 \TeX{} 模式中使用
expression prefix	'
注释开始和注释结束	< 和 >
inherit standard syntax	@
generic comment delimiter	!
语法表格可以继承,所以基本上所有语法表格都是从 standard-syntax-table 继承而来,
作少量修改,加上每个模式特有的语法构成就行了。一般来说记住几类重要的分类就行了,
比如,空白包括空格,制表符,换行符,换页符。词包括所有的大小写字母,数字。符号一
般按使用的模式而定,比如C中包含 \_,而Lisp 中是 \$\&*+-\_<>。可以用 M-x
describe-syntax 来查看所有字符的语法分类。
字符分类(category)是另一种分类方法,每个分类都有一个名字,对应一个从 到 \textasciitilde{} 的
ASCII 字符。可以用 M-x describe-categories 查看所有字符的分类。每一种分类都有说
明,我就不详细解释了。

\subsubsection{几个常用的函数}
\label{sec:org242e52c}

如果你要匹配的字符串中含有很多特殊字符,而你又想用正则表达式进行匹配,可以使用
regexp-quote 函数,它可以让字符串中的特殊字符自动转义。

一般多个可选词的匹配可以用或运算连接起来,但是这有两个不好的地方,一是要写很长的
正则表达式,还含有很多反斜线,不好看,容易出错,也不好修改,二是效率很低。这时可
以使用 regexp-opt 还产生一个更好的正则表达式
(regexp-opt '("foobar" "foobaz" "foo")) ; => "foo$\backslash$\(?:ba[rz]\\)?"
\subsubsection{函数列表}
\label{sec:org90a724f}
(regexp-quote STRING)
(regexp-opt STRINGS \&optional PAREN)
\subsubsection{命令列表}
\label{sec:orgf79a502}
describe-syntax
describe-categories
\subsection{{\bfseries\sffamily DONE} 操作对象之一 ── 缓冲区}
\label{sec:orgd9e5b0e}
缓冲区(buffer)是用来保存要编辑文本的对象。通常缓冲区都是和文件相关联的,但是也
有很多缓冲区没有对应的文件。emacs 可以同时打开多个文件,也就是说能同时有很多个缓
冲区存在,但是在任何时候都只有一个缓冲区称为当前缓冲区(current buffer)。即使在
lisp 编程中也是如此。许多编辑和移动的命令只能针对当前缓冲区。

\subsubsection{缓冲区的名字}
\label{sec:org5b09ea9}

emacs 里的所有缓冲区都有一个不重复的名字。所以和缓冲区相关的函数通常都是可以接受
一个缓冲区对象或一个字符串作为缓冲区名查找对应的缓冲区。下面的函数列表中如果参数
是 BUFFER-OR-NAME 则是能同时接受缓冲区对象和缓冲区名的函数,否则只能接受一种参数。
有一个习惯是名字以空格开头的缓冲区是临时的,用户不需要关心的缓冲区。所以现在一般
显示缓冲区列表的命令都不会显示这样的变量,除非这个缓冲区关联一个文件。
要得到缓冲区的名字,可以用 buffer-name 函数,它的参数是可选的,如果不指定参数,
则返回当前缓冲区的名字,否则返回指定缓冲区的名字。更改一个缓冲区的名字用
rename-buffer,这是一个命令,所以你可以用 M-x 调用来修改当前缓冲区的名字。如果你
指定的名字与现有的缓冲区冲突,则会产生一个错误,除非你使用第二个可选参数以产生一
个不相同的名字,通常是在名字后加上 <序号> 的方式使名字变得不同。你也可以用
generate-new-buffer-name 来产生一个唯一的缓冲区名。

\subsubsection{当前缓冲区}
\label{sec:org1112020}

当前缓冲区可以用 current-buffer 函数得到。当前缓冲区不一定是显示在屏幕上的那个缓
冲区,你可以用 set-buffer 来指定当前缓冲区。但是需要注意的是,当命令返回到命令循
环时,光标所在的缓冲区 会自动成为当前缓冲区。这也是单独在 \textbf{scratch} 中执行
set-buffer 后并不能改变当前缓冲区,而必须使用 progn 语句同时执行多个语句才能改变
当前缓冲区的原因
(set-buffer "\textbf{Messages}")   ; => \#<buffer \textbf{Messages*>
(message (buffer-name))                  ; => "*scratch}"
(progn
(set-buffer "\textbf{Messages}")
(message (buffer-name)))               ; "\textbf{Messages}"
但是你不能依赖命令循环来把当前缓冲区设置成使用 set-buffer 之前的。因为这个命令很
可以会被另一个程序员来调用。你也不能直接用 set-buffer 设置成原来的缓冲区,比如

(let (buffer-read-only
(obuf (current-buffer)))
(set-buffer \ldots{})
\ldots{}
(set-buffer obuf))
因为 set-buffer 不能处理错误或退出情况。正确的作法是使用 save-current-buffer、
with-current-buffer 和 save-excursion 等方法。save-current-buffer 能保存当
前缓冲区,执行其中的表达式,最后恢复为原来的缓冲区。如果原来的缓冲区被关闭
了,则使用最后使用的那个当前缓冲区作为语句返回后的当前缓冲区。lisp 中很多
以 with 开头的宏,这些宏通常是在不改变当前状态下,临时用另一个变量代替现有
变量执行语句。比如 with-current-buffer 使用另一个缓冲区作为当前缓冲区,语
句执行结束后恢复成执行之前的那个缓冲区
(with-current-buffer BUFFER-OR-NAME
body)
相当于
(save-current-buffer
(set-buffer BUFFER-OR-NAME)
body)
save-excursion 与 save-current-buffer 不同之处在于,它不仅保存当前缓冲区,还保存
了当前的位置和 mark。在 \textbf{scratch} 缓冲区中运行下面两个语句就能看出它们的差
别了
(save-current-buffer
(set-buffer "\textbf{scratch}")
(goto-char (point-min))
(set-buffer "\textbf{Messages}"))

(save-excursion
(set-buffer "\textbf{scratch}")
(goto-char (point-min))
(set-buffer "\textbf{Messages}"))
\subsubsection{创建和关闭缓冲区}
\label{sec:org0475247}
产生一个缓冲区必须用给这个缓冲区一个名字,所以两个能产生新缓冲区的函数都是以一个
字符串为参数:get-buffer-create 和 generate-new-buffer。这两个函数的差别在于前者
如果给定名字的缓冲区已经存在,则返回这个缓冲区对象,否则新建一个缓冲区,名字为参
数字符串,而后者在给定名字的缓冲区存在时,会使用加上后缀 <N>(N 是一个整数,从2
开始) 的名字创建新的缓冲区。
关闭一个缓冲区可以用 kill-buffer。当关闭缓冲区时,如果要用户确认是否要关闭缓冲区,
可以加到 kill-buffer-query-functions 里。如果要做一些善后处理,可以用
kill-buffer-hook。
通常一个接受缓冲区作为参数的函数都需要参数所指定的缓冲区是存在的。如果要确认一个
缓冲区是否依然还存在可以使用 buffer-live-p。
要对所有缓冲区进行某个操作,可以用 buffer-list 获得所有缓冲区的列表。

如果你只是想使用一个临时的缓冲区,而不想先建一个缓冲区,使用结束后又需要关闭这个
缓冲区,可以用 with-temp-buffer 这个宏。从这个宏的名字可以看出,它所做的事情是先
新建一个临时缓冲区,并把这个缓冲区作为当前缓冲区,使用结束后,关闭这个缓冲区,并
恢复之前的缓冲区为当前缓冲区。

\subsubsection{在缓冲区内移动}
\label{sec:orgf932df8}

在学会移动函数之前,先要理解两个概念:位置(position)和标记(mark)。位置是指某
个字符在缓冲区内的下标,它从1开始。更准确的说位置是在两个字符之间,所以有在位置
之前的字符和在位置之后的字符之说。但是通常我们说在某个位置的字符都是指在这个位置
之后的字符。
标记和位置的区别在于位置会随文本插入和删除而改变位置。一个标记包含了缓冲区和位置
两个信息。在插入和删除缓冲区里的文本时,所有的标记都会检查一遍,并重新设置位置。
这对于含有大量标记的缓冲区处理是很花时间的,所以当你确认某个标记不用的话应该释放
这个标记。
创建一个标记使用函数 make-marker。这样产生的标记不会指向任何地方。你需要用
set-marker 命令来设置标记的位置和缓冲区
(setq foo (make-marker))             ; => \#<marker in no buffer>
(set-marker foo (point))             ; => \#<marker at 3594 in *scratch*>
也可以用 point-marker 直接得到 point 处的标记。或者用 copy-marker 复制一个标记或
者直接用位置生成一个标记
(point-marker)                       ; => \#<marker at 3516 in *scratch*>
(copy-marker 20)                     ; => \#<marker at 20 in *scratch*>
(copy-marker foo)                    ; => \#<marker at 3502 in *scratch*>
如果要得一个标记的内容,可以用 marker-position,marker-buffer
(marker-position foo)                ; => 3502
(marker-buffer foo)                  ; => \#<buffer *scratch*>
位置就是一个整数,而标记在一般情况下都是以整数的形式使用,所以很多接受整数运算的
函数也可以接受标记为参数。比如加减乘。
和缓冲区相关的变量,有的可以用变量得到,比如缓冲区关联的文件名,有的只能用函数来
得到,比如 point。point 是一个特殊的缓冲区位置,许多命令在这个位置进行文本插入。
每个缓冲区都有一个 point 值,它总是比函数point-min 大,比另一个函数 point-max 返
回值小。注意,point-min 的返回值不一定是 1,point-max 的返回值也不定是比缓冲区大
小函数 buffer-size 的返回值大 1 的数,因为 emacs 可以把一个缓冲区缩小(narrow)
到一个区域,这时 point-min 和 point-max 返回值就是这个区域的起点和终点位置。所以
要得到 point 的范围,只能用这两个函数,而不能用 1 和 buffer-size 函数。
和 point 类似,有一个特殊的标记称为 "the mark"。它指定了一个区域的文本用于某些命
令,比如 kill-region,indent-region。可以用 mark 函数返回当前 mark 的值。如果使
用 transient-mark-mode,而且 mark-even-if-inactive值是 nil 的话,在 mark 没有激
活时(也就是 mark-active 的值为 nil),调用 mark 函数会产生一个错误。如果传递一
个参数 force 才能返回当前缓冲区 mark 的位置。mark-marker 能返回当前缓冲区的 mark,
这不是 mark 的拷贝,所以设置它的值会改变当前 mark 的值。set-mark 可以设置 mark
的值,并激活 mark。每个缓冲区还维护一个mark-ring,这个列表里保存了 mark 的前一个
值。当一个命令修改了 mark 的值时,通常要把旧的值放到 mark-ring 里。可以用
push-mark 和 pop-mark 加入或删除 mark-ring 里的元素。当缓冲区里 mark 存在且指向
某个位置时,可以用 region-beginning 和 region-end 得到 point 和 mark 中较小的和
较大的值。当然如果使用 transient-mark-mode 时,需要激活 mark,否则会产生一个错误。
\subsubsection{思考题}
\label{sec:orgad0afc4}
写一个命令,对于使用 transient-mark-mode 时,当选中一个区域时显示区域 的起点和终
点,否则显示 point-min 和 point-max 的位置。如果不使用 transient-mark-mode,则显
示 point 和 mark 的位置。
按单个字符位置来移动的函数主要使用 goto-char 和 forward-char、backward-char。前
者是按缓冲区的绝对位置移动,而后者是按 point 的偏移位置移动比如
(goto-char (point-min))                   ; 跳到缓冲区开始位置
(forward-char 10)                         ; 向前移动 10 个字符
(forward-char -10)                        ; 向后移动 10 个字符
可能有一些写 elisp 的人没有读文档或者贪图省事,就在写的 elisp 里直接用
beginning-of-buffer 和 end-of-buffer 来跳到缓冲区的开头和末尾,这其实是不对的。
因为这两个命令还做了其它事情,比如设置标记等等。同样,还有一些函数都是不推荐在
elisp 中使用的,如果你准备写一个要发布 elisp,还是要注意一下。
按词移动使用 forward-word 和 backward-word。至于什么是词,这就要看语法表格的定义了。
按行移动使用 forward-line。没有 backward-line。forward-line 每次移动都是移动到行
首的。所以,如果要移动到当前行的行首,使用 (forward-line 0)。如果不想移动就得到
行首和行尾的位置,可以用 line-beginning-position 和 line-end-position。得到当前
行的行号可以用 line-number-at-pos。需要注意的是这个行号是从当前状态下的行号,如
果使用 narrow-to-region 或者用 widen 之后都有可能改变行号。
由于 point 只能在 point-min 和 point-max 之间,所以 point 位置测试有时是很重要的,
特别是在循环条件测试里。常用的测试函数是 bobp(beginning of buffer predicate)和
eobp(end of buffer predicate)。对于行位置测试使用 bolp(beginning of line
predicate)和 eolp(end of line predicate)。
\subsubsection{缓冲区的内容}
\label{sec:orgc9834b0}
要得到整个缓冲区的文本,可以用 buffer-string 函数。如果只要一个区间的文本,使用
buffer-substring 函数。point 附近的字符可以用 char-after 和 char-before 得到。
point 处的词可以用 current-word 得到,其它类型的文本,比如符号,数字,S 表达式等
等,可以用 thing-at-point 函数得到。
\subsubsection{思考题}
\label{sec:org75900f1}
参考 thing-at-point 写一个命令标记光标处的 S 表达式。这个命令和 mark-sexp 不同的
是,它能从整个 S 表达式的开始标记。
\subsubsection{修改缓冲区的内容}
\label{sec:org234b488}
要修改缓冲区的内容,最常见的就是插入、删除、查找、替换了。下面就分别介绍这几种操作。
插入文本最常用的命令是 insert。它可以插入一个或者多个字符串到当前缓冲区的 point
后。也可以用 insert-char 插入单个字符。插入另一个缓冲区的一个区域使用
insert-buffer-substring。
删除一个或多个字符使用 delete-char 或 delete-backward-char。删除一个区间使用
delete-region。如果既要删除一个区间又要得到这部分的内容使用
delete-and-extract-region,它返回包含被删除部分的字符串。
最常用的查找函数是 re-search-forward 和 re-search-backward。这两个函数参数如下
(re-search-forward REGEXP \&optional BOUND NOERROR COUNT)
(re-search-backward REGEXP \&optional BOUND NOERROR COUNT)
其中 BOUND 指定查找的范围,默认是 point-max(对于 re-search-forward)或
point-min(对于 re-search-backward),NOERROR 是当查找失败后是否要产生一个错误,
一般来说在 elisp 里都是自己进行错误处理,所以这个一般设置为 t,这样在查找成功
后返回区配的位置,失败后会返回 nil。COUNT 是指定查找匹配的次数。
替换一般都是在查找之后进行,也是使用 replace-match 函数。和字符串的替换不同的是
不需要指定替换的对象了。

\subsubsection{思考题}
\label{sec:org83085a5}

从 OpenOffice 字处理程序里拷贝到 emacs 里的表格通常都是每一个单元格就是一行的。
写一个命令,让用户输入表格的列数,把选中区域转换成用制表符分隔的表格。

\subsubsection{函数列表}
\label{sec:orgfed08fd}

(buffer-name \&optional BUFFER)
(rename-buffer NEWNAME \&optional UNIQUE)
(generate-new-buffer-name NAME \&optional IGNORE)
(current-buffer)
(set-buffer BUFFER-OR-NAME))
(save-current-buffer \&rest BODY)
(with-current-buffer BUFFER-OR-NAME \&rest BODY)
(save-excursion \&rest BODY)
(get-buffer-create NAME)
(generate-new-buffer NAME)
(kill-buffer BUFFER-OR-NAME)
(buffer-live-p OBJECT)
(buffer-list \&optional FRAME)
(with-temp-buffer \&rest BODY)
(make-marker)
(set-marker MARKER POSITION \&optional BUFFER)
(point-marker)
(copy-marker MARKER \&optional TYPE)
(marker-position MARKER)
(marker-buffer MARKER)
(point)
(point-min)
(point-max)
(buffer-size \&optional BUFFER)
(mark \&optional FORCE)
(mark-marker)
(set-mark POS)
(push-mark \&optional LOCATION NOMSG ACTIVATE)
(pop-mark)
(region-beginning)
(region-end)
(goto-char POSITION)
(forward-char \&optional N)
(backward-char \&optional N)
(beginning-of-buffer \&optional ARG)
(end-of-buffer \&optional ARG)
(forward-word \&optional ARG)
(backward-word \&optional ARG)
(forward-line \&optional N)
(line-beginning-position \&optional N)
(line-end-position \&optional N)
(line-number-at-pos \&optional POS)
(narrow-to-region START END)
(widen)
(bobp)
(eobp)
(bolp)
(eolp)
(buffer-string)
(buffer-substring START END)
(char-after \&optional POS)
(char-before \&optional POS)
(current-word \&optional STRICT REALLY-WORD)
(thing-at-point THING)
(insert \&rest ARGS)
(insert-char CHARACTER COUNT \&optional INHERIT)
(insert-buffer-substring BUFFER \&optional START END)
(delete-char N \&optional KILLFLAG)
(delete-backward-char N \&optional KILLFLAG)
(delete-region START END)
(delete-and-extract-region START END)
(re-search-forward REGEXP \&optional BOUND NOERROR COUNT)
(re-search-backward REGEXP \&optional BOUND NOERROR COUNT)
\subsubsection{问题解答}
\label{sec:orgac977ed}
可选择区域也可不选择区域的命令
(defun show-region (beg end)
(interactive
(if (or (null transient-mark-mode)
mark-active)
(list (region-beginning) (region-end))
(list (point-min) (point-max))))
(message "Region start from \%d to \%d" beg end))
这是通常那种如果选择区域则对这个区域应用命令,否则对整个缓冲区应用命令的方法。我喜欢用 transient-mark-mode,因为它让这种作用于区域的命令更灵活。当然也有人反对,无所谓了,emacs 本身就是很个性化的东西。
标记整个S表达式
(defun mark-whole-sexp ()
(interactive)
(let ((bound (bounds-of-thing-at-point 'sexp)))
(if bound
(progn
(goto-char (car bound))
(set-mark (point))
(goto-char (cdr bound)))
(message "No sexp found at point!"))))
学习过程中应该可以看看其它一些函数是怎样实现的,从这些源代码中常常能学到很多有用的技巧和方法。比如要标记整个 S 表达式,联想到 thing-at-point 能得到整个 S 表达式,那自然能得到整个S表达式的起点和终点了。所以看看 thing-at-point 的实现,一个很简单的函数,一眼就能发现其中最关键的函数是 bounds-of-thing-at-point,它能返回某个语法实体(syntactic entity)的起点和终点。这样这个命令就很容易就能写出来了。从这个命令中还应该注意到的是对于错误应该很好的处理,让用户明白发生什么错了。比如这里,如果当前光标下没有 S 表达式时,bound 变量为 nil,如果不进行判断,会出现错误:
Wrong type argument: integer-or-marker-p, nil
加上这个判断,用户就明白发生什么事了。
oowriter 表格转换
实现这个目的有多种方法:
一行一行移动,删除回车,替换成制表符
(defun oowrite-table-convert (col beg end)
(interactive "nColumns of table: \nr")
(setq col (1- col))
(save-excursion
(save-restriction
(narrow-to-region beg end)
(goto-char (point-min))
(while (not (eobp))
(dotimes (i col)
(forward-line 1)
(backward-delete-char 1)
(insert-char ?\t 1))
(forward-line 1)))))
用 subst-char-in-region 函数直接替换
(defun oowrite-table-convert (col beg end)
(interactive "nColumns of table: \nr")
(save-excursion
(save-restriction
(narrow-to-region beg end)
(goto-char (point-min))
(while (not (eobp))
(subst-char-in-region
(point) (progn (forward-line col) (1- (point)))
?\n ?\t)))))
用 re-search-forward 和 replace-match 查找替
(defun oowrite-table-convert (col beg end)
(interactive "nColumns of table: \nr")
(let (start bound)
(save-excursion
(save-restriction
(narrow-to-region beg end)
(goto-char (point-min))
(while (not (eobp))
(setq start (point))
(forward-line col)
(setq bound (copy-marker (1- (point))))
(goto-char start)
(while (re-search-forward "\n" bound t)
(replace-match "\t"))
(goto-char (1+ bound)))))))
之所以要给出这三种方法,是想借此说明 elisp 编程其实要实现一个目的通常有 很多种方
法,选择一种适合的方法。比如这个问题较好的方法是使用第二种方法, 前提是你要知
道有 subst-char-in-region 这个函数,这就要求你对 emacs提供 的内置的函数比较熟
悉了,没有别的办法,只有自己多读 elisp manual,如果你 真想学习 elisp 的话,读
manual 还是值得的,我每读一遍都会有一些新的发 现。如果你不知道这个函数,只知道
常用的函数,那么相比较而言,第一种方法 是比较容易想到,也比较容易实现的。但是
事实上第三种方法才是最重要的方法, 因为这个方法是适用范围最广的。试想一下你如
果不是替换两个字符,而是字符 串的话,前面两种方法都没有办法使用了,而这个方法
只要稍微修改就能用了。
另外,需要特别说明的是这个命令中 bound 使用的是一个标记而不是一个位置, 如果替换
的字符串和删除的字符串是相同长度的,当前用什么都可以,否则就要 注意了,因为在
替换之后,边界就有可能改变。这也是写查找替换的函数中很容 易出现的一个错误。解
决的办法,一是像我这样用一个标记来记录边界位置。另 一种就是用 narrow-to-region
的方法,先把缓冲区缩小到查找替换的区域,结 束后用 widen 展开。当然为了省事,可
以直接用 save-restriction。

\subsection{操作对象之二 ── 窗口}
\label{sec:org22ba30e}

首先还是要定义一下什么是窗口(window)。窗口是屏幕上用于显示一个缓冲区 的部分。和它要区分开来的一个概念是 frame。frame 是 Emacs 能够使用屏幕的 部分。可以用窗口的观点来看 frame 和窗口,一个 frame 里可以容纳多个(至 少一个)窗口,而 Emacs 可以有多个 frame。(可能需要和通常所说的窗口的概 念要区分开来,一般来说,我们所说的其它程序的窗口更类似于 Emacs 的一个 frame,所以也有人认为这里 window 译为窗格更好一些。但是窗格这个词是一个 生造出来的词,我还是用窗口比较顺一些,大家自己注意就行了。)在任何时候, 都有一个被选中的 frame,而在这个 frame 里有一个被选中的窗口,称为选择的 窗口(selected window)。
\subsubsection{分割窗口}
\label{sec:org0144310}
刚启动时,emacs 都是只有一个 frame 一个窗口。多个窗口都是用分割窗口的函 数生成的。分割窗口的内建函数是split-window。这个函数的参数如下:
(split-window \&optional window size horizontal)
这个函数的功能是把当前或者指定窗口进行分割,默认分割方式是水平分割,可 以将参数中的 horizontal 设置为 non-nil 的值,变成垂直分割。如果不指定 大小,则分割后两个窗口的大小是一样的。分割后的两个窗口里的缓冲区是同 一个缓冲区。使用这个函数后,光标仍然在原窗口,而返回的新窗口对象:
(selected-window)                       ; => \#<window 136 on *scratch*>
(split-window)                          ; => \#<window 138 on *scratch*>
需要注意的是,窗口的分割也需要用树的结构来看分割后的窗口,比如这样一个过程:
\sout{---------------}         \sout{---------------}
\begin{center}
\begin{tabular}{llll}
 &  &  & \\
win1 &  & win1 & win2\\
 & --> &  & \\
 &  &  & \\
 &  &  & \\
\end{tabular}
\end{center}
\sout{---------------}         \sout{---------------}
\begin{center}
\begin{tabular}{}
\\
\end{tabular}
\end{center}
v
\sout{---------------}         \sout{---------------}
\begin{center}
\begin{tabular}{rrlll}
win1 &  &  &  & \\
 & win2 &  & win1 & win2\\
\hline
3 & 4 &  &  & win3 & \\
 &  &  &  &  & \\
\end{tabular}
\end{center}
\sout{---------------}         \sout{---------------}
可以看成是这样一种结构:
(win1) ->  (win1 win2) -> ((win1 win3) win2) -> ((win1 (win3 win4)) win2)
事实上可以用 window-tree 函数得到当前窗口的结构,如果忽略 minibuffer 对应的窗口,得到的应该类似这样的一个结果:
(nil (0 0 170 42)
(t (0 0 85 42)
\#<win 3>
(nil (0 21 85 42) \#<win 8> \#<win 10>))
\#<win 6>)
window-tree 返回值的第一个元素代表子窗口的分割方式,nil 表示水平分割, t 表示垂直分割。第二个元素代表整个结构的大小,这四个数字可以看作是左上 和右下两个顶点的坐标。其余元素是子窗口。每个子窗口也是同样的结构。所以 把前面这个列表还原成窗口排列应该是这样:
(0,0) \sout{-------------------}
\begin{center}
\begin{tabular}{ll}
 & \\
win 3 & win6\\
 & \\
\end{tabular}
\end{center}
(0,21) |---------|         |
\begin{center}
\begin{tabular}{rrl}
 &  & \\
8 & 10 & \\
 &  & \\
\end{tabular}
\end{center}
\sout{-------------------} (170, 42)
(85, 42)
由上面的图可以注意到由 window-tree 返回的结果一些窗口的大小不能确定, 比较上面的 win 8 和 win 10 只能知道它们合并起来的大小,不能确定它们分 别的宽度是多少。
\subsubsection{删除窗口}
\label{sec:org190a76f}
如果要让一个窗口不显示在屏幕上,要使用 delete-window 函数。如果没有指定 参数,删除的窗口是当前选中的窗口,如果指定了参数,删除的是这个参数对应 的窗口。删除的窗口多出来的空间会自动加到它的邻接的窗口中。如果要删除除 了当前窗口之外的窗口,可以用 delete-other-windows 函数。
当一个窗口不可见之后,这个窗口对象也就消失了
(setq foo (selected-window))            ; => \#<window 90 on *scratch*>
(delete-window)
(windowp foo)                           ; => t
(window-live-p foo)                     ; => nil
\subsubsection{窗口配置}
\label{sec:orge1084a3}
窗口配置(window configuration) 包含了 frame 中所有窗口的位置信息:窗口 大小,显示的缓冲区,缓冲区中光标的位置和 mark,还有 fringe,滚动条等等。 用 current-window-configuration 得到当前窗口配置,用 set-window-configuration 来还原
(setq foo (current-window-configuration))
;; do sth to make some changes on windows
(set-window-configuration foo)
\subsubsection{选择窗口}
\label{sec:org635044d}
可以用 selected-window 得到当前光标所在的窗口
(selected-window)                       ; => \#<window 104 on *scratch*>
可以用 select-window 函数使某个窗口变成选中的窗口
(progn
(setq foo (selected-window))
(message "Original window: \%S" foo)
(other-window 1)
(message "Current window: \%S" (selected-window))
(select-window foo)
(message "Back to original window: \%S" foo))
两个特殊的宏可以保存窗口位置执行语句:save-selected-window 和 with-selected-window。它们的作用是在执行语句结束后选择的窗口仍留在执行 语句之前的窗口。with-selected-window 和 save-selected-window 几乎相同, 只不过 save-selected-window 选择了其它窗口。这两个宏不会保存窗口的位置 信息,如果执行语句结束后,保存的窗口已经消失,则会选择最后一个选择的窗 口。
;; 让另一个窗口滚动到缓冲区开始
(save-selected-window
(select-window (next-window))
(goto-char (point-min)))
当前 frame 里所有的窗口可以用 window-list 函数得到。可以用 next-window 来得到在 window-list 里排在某个 window 之后的窗口。对应的用 previous-window 得到排在某个 window 之前的窗口。
(selected-window)                       ; => \#<window 245 on *scratch*>
(window-list)
;; => (\#<window 245 on *scratch*> \#<window 253 on *scratch*> \#<window 251 on *info*>)
(next-window)                           ; => \#<window 253 on *scratch*>
(next-window (next-window))             ; => \#<window 251 on *info*>
(next-window (next-window (next-window))) ; => \#<window 245 on *scratch*>
walk-windows 可以遍历窗口,相当于 (mapc proc (window-list))。 get-window-with-predicate 用于查找符合某个条件的窗口。
\subsubsection{窗口大小信息}
\label{sec:orgffdd3e1}
窗口是一个长方形区域,所以窗口的大小信息包括它的高度和宽度。用来度量窗 口大小的单位都是以字符数来表示,所以窗口高度为 45 指的是这个窗口可以容 纳 45 行字符,宽度为 140 是指窗口一行可以显示 140 个字符。
mode line 和 header line 都包含在窗口的高度里,所以有 window-height 和 window-body-height 两个函数,后者返回把 mode-line 和 header line 排除后 的高度。
(window-height)                         ; => 45
(window-body-height)                    ; => 44
滚动条和 fringe 不包括在窗口的亮度里,window-width 返回窗口的宽度
(window-width)                          ; => 72
也可以用 window-edges 返回各个顶点的坐标信息
(window-edges)                          ; => (0 0 73 45)
window-edges 返回的位置信息包含了滚动条、fringe、mode line、header line 在内,window-inside-edges 返回的就是窗口的文本区域的位置
(window-inside-edges)                   ; => (1 0 73 44)
如果需要的话也可以得到用像素表示的窗口位置信息
(window-pixel-edges)                    ; => (0 0 511 675)
(window-inside-pixel-edges)             ; => (7 0 511 660)
\subsubsection{思考题}
\label{sec:orgd285715}
current-window-configuration 可以将当前窗口的位置信 息保存到一个变量中以便将来恢
复窗口。但是这个对象没有读入形式,所以不 能保存到文件中。请写
一个函数可以把当前窗口的位置信息生成一个列表,然 后用一个函数
就能从这个列表恢复窗口。提示:这个列表结构用窗口的分割顺 序表
示。比如用这样一个列表表示对应的窗口:
;; \sout{---------------}
;; |   |   |       |
;; |   |   |       |
;; |-------|       |
;; |       |       |
;; |       |       |
;; \sout{---------------}
;; =>
(horizontal 73
(vertical 22
(horizontal 36 win win)
win)
win)
\subsubsection{窗口对应的缓冲区}
\label{sec:orge472e7b}
窗口对应的缓冲区可以用 window-buffer 函数得到:
(window-buffer)                         ; => \#<buffer *scratch*>
(window-buffer (next-window))           ; => \#<buffer *info*>
缓冲区对应的窗口也可以用 get-buffer-window 得到。如果有多个窗口显示同一 个缓冲区,那这个函数只能返回其中的一个,由window-list 决定。如果要得到 所有的窗口,可以用 get-buffer-window-list
(get-buffer-window (get-buffer "\textbf{scratch}"))
;; => \#<window 268 on \textbf{scratch*>
(get-buffer-window-list (get-buffer "*scratch}"))
;; => (\#<window 268 on *scratch*> \#<window 270 on *scratch*>)
让某个窗口显示某个缓冲区可以用 set-window-buffer 函数。 让选中窗口显示某个缓冲区也可以用 switch-to-buffer,但是一般不要在 elisp 编程中用这个命令,如果需要让某个缓冲区成为当前缓冲区使用 set-buffer 函数,如果要让当前窗口显示某个缓冲区,使用 set-window-buffer 函数。
让一个缓冲区可见可以用 display-buffer。默认的行为是当缓冲区已经显示在某 个窗口中时,如果不是当前选中窗口,则返回那个窗口,如果是当前选中窗口, 且如果传递的 not-this-window 参数为 non-nil 时,会新建一个窗口,显示缓 冲区。如果没有任何窗口显示这个缓冲区,则新建一个窗口显示缓冲区,并返回 这个窗口。display-buffer 是一个比较高级的命令,用户可以通过一些变量来改 变这个命令的行为。比如控制显示的 pop-up-windows, display-buffer-reuse-frames,pop-up-frames,控制新建窗口高度的 split-height-threshold,even-window-heights,控制显示的 frame 的 special-display-buffer-names,special-display-regexps, special-display-function,控制是否应该显示在当前选中窗口 same-window-buffer-names,same-window-regexps 等等。如果这些还不能满 足你的要求(事实上我觉得已经太复杂了),你还可以自己写一个函数,将 display-buffer-function 设置成这个函数。
\subsubsection{思考题}
\label{sec:org23b32ea}
前一个思考题只能还原窗口,不能还原缓冲区。请修改一下使它能保存缓冲区信息,还原时
让对应的窗口显示对应的缓冲区。

\subsubsection{改变窗口显示区域}
\label{sec:org74a15f7}

每个窗口会保存一个显示缓冲区的起点位置,这个位置对应于窗口左上角光标在 缓冲区里
的位置。可以用 window-start 函数得到某个窗口的起点位置。可以
通 过 set-window-start 来改变显示起点位置。可以通过
pos-visible-in-window-p 来检测缓冲区中某个位置是否是可见的。
但是直接通过 set-window-start 来控制显示比较容易出现错误,因
为 set-window-start 并不会改变 point 所在的位置,在窗口调用
redisplay 函 数之后 point 会跳到相应的位置。如果你确实有这个
需要,我建议还是用: (with-selected-window window (goto-char
pos)) 来代替。

\subsubsection{函数列表}
\label{sec:org1e7ef6c}
(windowp OBJECT)
(split-window \&optional WINDOW SIZE HORFLAG)
(selected-window)
(window-tree \&optional FRAME)
(delete-window \&optional WINDOW)
(delete-other-windows \&optional WINDOW)
(current-window-configuration \&optional FRAME)
(set-window-configuration CONFIGURATION)
(other-window ARG \&optional ALL-FRAMES)
(save-selected-window \&rest BODY)
(with-selected-window WINDOW \&rest BODY)
(window-list \&optional FRAME MINIBUF WINDOW)
(next-window \&optional WINDOW MINIBUF ALL-FRAMES)
(previous-window \&optional WINDOW MINIBUF ALL-FRAMES)
(walk-windows PROC \&optional MINIBUF ALL-FRAMES)
(get-window-with-predicate PREDICATE \&optional MINIBUF ALL-FRAMES DEFAULT)
(window-height \&optional WINDOW)
(window-body-height \&optional WINDOW)
(window-width \&optional WINDOW)
(window-edges \&optional WINDOW)
(window-inside-edges \&optional WINDOW)
(window-pixel-edges \&optional WINDOW)
(window-inside-pixel-edges \&optional WINDOW)
(window-buffer \&optional WINDOW)
(get-buffer-window BUFFER-OR-NAME \&optional FRAME)
(get-buffer-window-list BUFFER-OR-NAME \&optional MINIBUF FRAME)
(set-window-buffer WINDOW BUFFER-OR-NAME \&optional KEEP-MARGINS)
(switch-to-buffer BUFFER-OR-NAME \&optional NORECORD)
(display-buffer BUFFER-OR-NAME \&optional NOT-THIS-WINDOW FRAME)
(window-start \&optional WINDOW)
(set-window-start WINDOW POS \&optional NOFORCE)
\subsubsection{问题解答}
\label{sec:orge514fce}
\subsubsection{保存窗口位置信息}
\label{sec:org9deaea3}
这是我的答案。欢迎提出改进意见
(defun my-window-tree-to-list (tree)
(if (windowp tree)
'win
(let ((dir (car tree))
(children (cddr tree)))
(list (if dir 'vertical 'horizontal)
(if dir
(my-window-height (car children))
(my-window-width (car children)))
(my-window-tree-to-list (car children))
(if (> (length children) 2)
(my-window-tree-to-list (cons dir (cons nil (cdr children))))
(my-window-tree-to-list (cadr children)))))))

(defun my-window-width (win)
(if (windowp win)
(window-width win)
(let ((edge (cadr win)))
(- (nth 2 edge) (car edge)))))

(defun my-window-height (win)
(if (windowp win)
(window-height win)
(let ((edge (cadr win)))
(- (nth 3 edge) (cadr edge)))))

(defun my-list-to-window-tree (conf)
(when (listp conf)
(let (newwin)
(setq newwin (split-window nil (cadr conf)
(eq (car conf) 'horizontal)))
(my-list-to-window-tree (nth 2 conf))
(select-window newwin)
(my-list-to-window-tree (nth 3 conf)))))

(defun my-restore-window-configuration (winconf)
(delete-other-windows)
(my-list-to-window-tree winconf))

(defun my-current-window-configuration ()
(my-window-tree-to-list (car (window-tree))))

;; test code here
(setq foo (my-current-window-configuration))
;; do sth to change windows
(my-restore-window-configuration foo)
\subsubsection{改进的保存窗口信息的函数}
\label{sec:org1a4ab2a}
由于缓冲区对象也是没有读入形式的,所以返回的列表里只能用缓冲区名来代表 缓冲区,只要没有修改过缓冲区的名字,就能正确的还原缓冲区。如果对于访问 文件的缓冲区,使用文件名可能是更好的想法。保存信息只要对 my-window-tree-to-list 函数做很小的修改就能用了。而恢复窗口则要做较大 改动。my-list-to-window-tree 加了一个函数参数,这样这个函数的可定制性 更高一些。
(defun my-window-tree-to-list (tree)
(if (windowp tree)
(buffer-name (window-buffer tree))
(let ((dir (car tree))
(children (cddr tree)))
(list (if dir 'vertical 'horizontal)
(if dir
(my-window-height (car children))
(my-window-width (car children)))
(my-window-tree-to-list (car children))
(if (> (length children) 2)
(my-window-tree-to-list (cons dir (cons nil (cdr children))))
(my-window-tree-to-list (cadr children)))))))

(defun my-list-to-window-tree (conf set-winbuf)
(let ((newwin (split-window nil (cadr conf)
(eq (car conf) 'horizontal))))
(if (eq (car conf) 'horizontal)
(progn
(funcall set-winbuf (selected-window) (nth 2 conf))
(select-window newwin)
(if (listp (nth 3 conf))
(my-list-to-window-tree (nth 3 conf))
(funcall set-winbuf newwin (nth 3 conf))))
(if (listp (nth 2 conf))
(my-list-to-window-tree (nth 2 conf))
(funcall set-winbuf (selected-window) (nth 2 conf)))
(select-window newwin)
(funcall set-winbuf newwin (nth 3 conf)))))

(defun my-restore-window-configuration (winconf)
(let ((buf (current-buffer)))
(delete-other-windows)
(my-list-to-window-tree winconf
(lambda (win name)
(set-window-buffer win (or (get-buffer name)
buf))))))
\subsection{操作对象之三 ── 文件}
\label{sec:orgbd27f65}

作为一个编辑器,自然文件是最重要的操作对象之一。这一节要介绍有关文件的一系列命令,比如查找文件,读写文件,文件信息、读取目录、文件名操作等。
\subsubsection{打开文件的过程}
\label{sec:orgae74bf3}
当你打开一个文件时,实际上 emacs 做了很多事情:
把文件名展开成为完整的文件名
\subsubsection{判断文件是否存在}
\label{sec:org3214539}
判断文件是否可读或者文件大小是否太大
查看文件是否已经打开,是否被锁定
向缓冲区插入文件内容
\subsubsection{设置缓冲区的模式}
\label{sec:org23765d9}
这还只是简单的一个步骤,实际情况比这要复杂的多,许多异常需要考虑。而且 为了所有函数的可扩展性,许多变量、handler 和 hook 加入到文件操作的函数 中,使得每一个环节都可以让用户或者 elisp 开发者可以定制,甚至完全接管 所有的文件操作。
这里需要区分两个概念:文件和缓冲区。它们是两个不同的对象,文件是在计算 机上可持久保存的信息,而缓冲区是 Emacs 中包含文件内容信息的对象,在 emacs 退出后就会消失,只有当保存缓冲区之后缓冲区里的内容才写到文件中去。
\subsubsection{文件读写}
\label{sec:orgab65a6e}
打开一个文件的命令是 find-file。这命令使一个缓冲区访问某个文件,并让这 个缓冲区成为当前缓冲区。在打开文件过程中会调用 find-file-hook。 find-file-noselect 是所有访问文件的核心函数。与 find-file 不同,它只返 回访问文件的缓冲区。这两个函数都有一个特点,如果 emacs 里已经有一个缓冲 区访问这个文件的话,emacs 不会创建另一个缓冲区来访问文件,而只是简单返 回或者转到这个缓冲区。怎样检查有没有缓冲区是否访问某个文件呢?所有和文 件关联的缓冲区里都有一个 buffer-local 变量buffer-file-name。但是不要直 接设置这个变量来改变缓冲区关联的文件。而是使用 set-visited-file-name 来 修改。同样不要直接从 buffer-list 里搜索buffer-file-name 来查找和某个文 件关联的缓冲区。应该使用get-file-buffer 或者 find-buffer-visiting。
(find-file "\textasciitilde{}/temp/test.txt")
(with-current-buffer
(find-file-noselect "\textasciitilde{}/temp/test.txt")
buffer-file-name)                     ; => "/home/ywb/temp/test.txt"
(find-buffer-visiting "\textasciitilde{}/temp/test.txt") ; => \#<buffer test.txt>
(get-file-buffer "\textasciitilde{}/temp/test.txt")      ; => \#<buffer test.txt>
保存一个文件的过程相对简单一些。首先创建备份文件,处理文件的位模式,将 缓冲区写入文件。保存文件的命令是 save-buffer。相当于其它编辑器里另存为 的命令是 write-file。在这个过程中会调用一些函数或者 hook。 write-file-functions 和 write-contents-functions 几乎功能完全相同。它们 都是在写入文件之前运行的函数,如果这些函数中有一个返回了 non-nil 的值, 则会认为文件已经写入了,后面的函数都不会运行,而且也不会使用再调用其它 写入文件的函数。这两个变量有一个重要的区别是write-contents-functions 在 改变主模式之后会被修改,因为它没有permanent-local 属性,而 write-file-functions 则会仍然保留。before-save-hook 和 write-file-functions 功能也比较类似,但是这个变量里的函数会逐个执行,不 论返回什么值也不会影响后面文件的写入。after-save-hook 是在文件已经写入 之后才调用的 hook,它是 save-buffer 最后一个动作。
但是实际上在 elisp 编程过程中经常遇到的一个问题是读取一个文件中的内容, 读取完之后并不希望这个缓冲区还留下来,如果直接用 kill-buffer 可能会把 用户打开的文件关闭。而且 find-file-noselect 做的事情实在超出我们的需要 的。这时你可能需要的是更底层的文件读写函数,它们是 insert-file-contents 和 write-region,调用形式分别是
(insert-file-contents filename \&optional visit beg end replace)
(write-region start end filename \&optional append visit lockname mustbenew)
insert-file-contents 可以插入文件中指定部分到当前缓冲区中。如果指定 visit 则会标记缓冲区的修改状态并关联缓冲区到文件,一般是不用的。 replace 是指是否要删除缓冲区里其它内容,这比先删除缓冲区其它内容后插入文 件内容要快一些,但是一般也用不上。insert-file-contents 会处理文件的编 码,如果不需要解码文件的话,可以用 insert-file-contents-literally。
write-region 可以把缓冲区中的一部分写入到指定文件中。如果指定 append 则是添加到文件末尾。和 insert-file-contents 相似,visit 参数也会把缓冲 区和文件关联,lockname 则是文件锁定的名字,mustbenew 确保文件存在时会 要求用户确认操作。
\subsubsection{思考题}
\label{sec:orgb3c7fd6}
写一个函数提取出某个 c 头文件中的函数声明中的函数名和声明位置。
\subsubsection{文件信息}
\label{sec:org617c709}
文件是否存在可以使用 file-exists-p 来判断。对于目录和一般文件都可以用 这个函数进行判断,但是符号链接只有当目标文件存在时才返回 t。
如何判断文件是否可读或者可写呢?file-readable-p、file-writable-p, file-executable-p 分用来测试用户对文件的权限。文件的位模式还可以用 file-modes 函数得到。
(file-exists-p "\textasciitilde{}/temp/test.txt")              ; => t
(file-readable-p "\textasciitilde{}/temp/test.txt")            ; => t
(file-writable-p "\textasciitilde{}/temp/test.txt")            ; => t
(file-executable-p "\textasciitilde{}/temp/test.txt")          ; => nil
(format "\%o" (file-modes "\textasciitilde{}/temp/test.txt"))   ; => "644"
文件类型判断可以使用 file-regular-p、file-directory-p、file-symlink-p, 分别判断一个文件名是否是一个普通文件(不是目录,命名管道、终端或者其它 IO 设备)、文件名是否一个存在的目录、文件名是否是一个符号链接。其中 file-symlink-p 当文件名是一个符号链接时会返回目标文件名。文件的真实名 字也就是除去相对链接和符号链接后得到的文件名可以用 file-truename 得到。 事实上每个和文件关联的 buffer 里也有一个缓冲区局部变量 buffer-file-truename 来记录这个文件名。
\$ ls -l t.txt
lrwxrwxrwx 1 ywb ywb 8 2007-07-15 15:51 t.txt -> test.txt
(file-regular-p "\textasciitilde{}/temp/t.txt")         ; => t
(file-directory-p "\textasciitilde{}/temp/t.txt")       ; => nil
(file-symlink-p "\textasciitilde{}/temp/t.txt")         ; => "test.txt"
(file-truename "\textasciitilde{}/temp/t.txt")          ; => "/home/ywb/temp/test.txt"
文件更详细的信息可以用 file-attributes 函数得到。这个函数类似系统的 stat 命令,返回文件几乎所有的信息,包括文件类型,用户和组用户,访问日 期、修改日期、status change 日期、文件大小、文件位模式、inode number、 system number。这是我写的方便使用的帮助函数:
(defun file-stat-type (file \&optional id-format)
(car (file-attributes file id-format)))
(defun file-stat-name-number (file \&optional id-format)
(cadr (file-attributes file id-format)))
(defun file-stat-uid (file \&optional id-format)
(nth 2 (file-attributes file id-format)))
(defun file-stat-gid (file \&optional id-format)
(nth 3 (file-attributes file id-format)))
(defun file-stat-atime (file \&optional id-format)
(nth 4 (file-attributes file id-format)))
(defun file-stat-mtime (file \&optional id-format)
(nth 5 (file-attributes file id-format)))
(defun file-stat-ctime (file \&optional id-format)
(nth 6 (file-attributes file id-format)))
(defun file-stat-size (file \&optional id-format)
(nth 7 (file-attributes file id-format)))
(defun file-stat-modes (file \&optional id-format)
(nth 8 (file-attributes file id-format)))
(defun file-stat-guid-changep (file \&optional id-format)
(nth 9 (file-attributes file id-format)))
(defun file-stat-inode-number (file \&optional id-format)
(nth 10 (file-attributes file id-format)))
(defun file-stat-system-number (file \&optional id-format)
(nth 11 (file-attributes file id-format)))
(defun file-attr-type (attr)
(car attr))
(defun file-attr-name-number (attr)
(cadr attr))
(defun file-attr-uid (attr)
(nth 2 attr))
(defun file-attr-gid (attr)
(nth 3 attr))
(defun file-attr-atime (attr)
(nth 4 attr))
(defun file-attr-mtime (attr)
(nth 5 attr))
(defun file-attr-ctime (attr)
(nth 6 attr))
(defun file-attr-size (attr)
(nth 7 attr))
(defun file-attr-modes (attr)
(nth 8 attr))
(defun file-attr-guid-changep (attr)
(nth 9 attr))
(defun file-attr-inode-number (attr)
(nth 10 attr))
(defun file-attr-system-number (attr)
(nth 11 attr))
前一组函数是直接由文件名访问文件信息,而后一组函数是由 file-attributes 的返回值来得到文件信息。
修改文件信息
重命名和复制文件可以用 rename-file 和 copy-file。删除文件使用 delete-file。创建目录使用 make-directory 函数。不能用 delete-file 删除 目录,只能用 delete-directory 删除目录。当目录不为空时会产生一个错误。
设置文件修改时间使用 set-file-times。设置文件位模式可以用 set-file-modes 函数。set-file-modes函数的参数必须是一个整数。你可以用位 函数 logand、logior 和 logxor 函数来进行位操作。
\subsubsection{思考题}
\label{sec:org73cf5e0}
写一个函数模拟 chmod 命令的行为。
\subsubsection{文件名操作}
\label{sec:orgc01c289}
虽然 MSWin 的文件名使用的路径分隔符不同,但是这里介绍的函数都能用于 MSWin 形式的文件名,只是返回的文件名都是 Unix 形式了。路径一般由目录和 文件名,而文件名一般由主文件名(basename)、文件名后缀和版本号构成。 Emacs 有一系列函数来得到路径中的不同部分
(file-name-directory "\textasciitilde{}/temp/test.txt")      ; => "\textasciitilde{}/temp/"
(file-name-nondirectory "\textasciitilde{}/temp/test.txt")   ; => "test.txt"
(file-name-sans-extension "\textasciitilde{}/temp/test.txt") ; => "\textasciitilde{}/temp/test"
(file-name-extension "\texttt{/temp/test.txt")      ; => "txt"
     (file-name-sans-versions "\textasciitilde{}/temp/test.txt}") ; => "\texttt{/temp/test.txt"
     (file-name-sans-versions "\textasciitilde{}/temp/test.txt.\textasciitilde{}1}") ; => "\textasciitilde{}/temp/test.txt"
路径如果是从根目录开始的称为是绝对路径。测试一个路径是否是绝对路径使用 file-name-absolute-p。如果在 Unix 或 GNU/Linux 系统,以 \textasciitilde{} 开头的路径也是绝对路径。在 MSWin 上,以 "/" 、 "$\backslash$"、"X:" 开头的路径都是绝对路径。如果不是绝对路径,可以使用 expand-file-name 来得到绝对路径。把一个绝对路径转换成相对某个路径的相 对路径的可以用 file-relative-name 函数。
(file-name-absolute-p "\textasciitilde{}rms/foo")       ; => t
(file-name-absolute-p "/user/rms/foo")  ; => t
(expand-file-name "foo")                ; => "\emph{home/ywb/foo"
(expand-file-name "foo" "/usr/spool}")  ; => "\emph{usr/spool/foo"
(file-relative-name "/foo/bar" "/foo}") ; => "bar"
(file-relative-name "\emph{foo/bar" "/hack}") ; => "../foo/bar"
对于目录,如果要将其作为目录,也就是确保它是以路径分隔符结束,可以用 file-name-as-directory。不要用 (concat dir "\emph{") 来转换,这会有移植问题。 和它相对应的函数是 directory-file-name
(file-name-as-directory "\textasciitilde{}rms/lewis")   ; => "\textasciitilde{}rms/lewis}"
(directory-file-name "\textasciitilde{}lewis/")         ; => "\textasciitilde{}lewis"
如果要得到所在系统使用的文件名,可以用 convert-standard-filename。比如 在 MSWin 系统上,可以用这个函数返回用 "$\backslash$" 分隔的文件名
(convert-standard-filename "c:/windows")  ;=> "c:$\backslash$\windows"
\subsubsection{临时文件}
\label{sec:org20eae2f}
如果需要产生一个临时文件,可以使用 make-temp-file。这个函数按给定前缀产 生一个不和现有文件冲突的文件,并返回它的文件名。如果给定的名字是一个相 对文件名,则产生的文件名会用 temporary-file-directory 进行扩展。也可以 用这个函数产生一个临时文件夹。如果只想产生一个不存在的文件名,可以用 make-temp-name 函数
(make-temp-file "foo")                  ; => "/tmp/foo5611dxf"
(make-temp-name "foo")                  ; => "foo5611q7l"
\subsubsection{读取目录内容}
\label{sec:org56d0a73}
可以用 directory-files 来得到某个目录中的全部或者符合某个正则表达式的 文件名。
(directory-files "\textasciitilde{}/temp/dir/")
;; =>
;; ("\#foo.el\#" "." ".\#foo.el" ".." "foo.el" "t.pl" "t2.pl")
(directory-files "\textasciitilde{}/temp/dir/" t)
;; =>
;; ("\emph{home/ywb/temp/dir/\#foo.el\#"
;;  "/home/ywb/temp/dir}."
;;  "\emph{home/ywb/temp/dir}.\#foo.el"
;;  "\emph{home/ywb/temp/dir}.."
;;  "/home/ywb/temp/dir/foo.el"
;;  "/home/ywb/temp/dir/t.pl"
;;  "\emph{home/ywb/temp/dir/t2.pl")
(directory-files "\textasciitilde{}/temp/dir}" nil "$\backslash$\.pl\$") ; => ("t.pl" "t2.pl")
directory-files-and-attributes 和 directory-files 相似,但是返回的列表 中包含了 file-attributes 得到的信息。file-name-all-versions 用于得到某 个文件在目录中的所有版本,file-expand-wildcards 可以用通配符来得到目录 中的文件列表。
\subsubsection{思考题}
\label{sec:org3f018e6}
写一个函数返回当前目录包括子目录中所有文件名。
\subsubsection{神奇的 Handle}
\label{sec:org932b14d}
如果不把文件局限在存储在本地机器上的信息,如果有一套基本的文件操作,比 如判断文件是否存在、打开文件、保存文件、得到目录内容之类,那远程的文件 和本地文件的差别也仅在于文件名表示方法不同而已。在 Emacs 里,底层的文件 操作函数都可以托管给 elisp 中的函数,这样只要用 elisp 实现了某种类型文 件的基本操作,就能像编辑本地文件一样编辑其它类型文件了。
决定何种类型的文件名使用什么方式来操作是在 file-name-handler-alist 变 量定义的。它是由形如 (REGEXP . HANDLER) 的列表。如果文件名匹配这个 REGEXP 则使用 HANDLER 来进行相应的文件操作。这里所说的文件操作,具体的 来说有这些函数:
`access-file', `add-name-to-file', `byte-compiler-base-file-name',
`copy-file', `delete-directory', `delete-file',
`diff-latest-backup-file', `directory-file-name', `directory-files',
`directory-files-and-attributes', `dired-call-process',
`dired-compress-file', `dired-uncache',
`expand-file-name', `file-accessible-directory-p', `file-attributes',
`file-directory-p', `file-executable-p', `file-exists-p',
`file-local-copy', `file-remote-p', `file-modes',
`file-name-all-completions', `file-name-as-directory',
`file-name-completion', `file-name-directory', `file-name-nondirectory',
`file-name-sans-versions', `file-newer-than-file-p',
`file-ownership-preserved-p', `file-readable-p', `file-regular-p',
`file-symlink-p', `file-truename', `file-writable-p',
`find-backup-file-name', `find-file-noselect',
`get-file-buffer', `insert-directory', `insert-file-contents',
`load', `make-auto-save-file-name', `make-directory',
`make-directory-internal', `make-symbolic-link',
`rename-file', `set-file-modes', `set-file-times',
`set-visited-file-modtime', `shell-command', `substitute-in-file-name',
`unhandled-file-name-directory', `vc-registered',
`verify-visited-file-modtime',
`write-region'.
在 HANDLE 里,可以只接管部分的文件操作,其它仍交给 emacs 原来的函数来完 成。举一个简单的例子。比如最新版本的 emacs 把 \textbf{scratch} 的 auto-save-mode 打开了。如果你不想这个缓 冲区的自动保存的文件名散布得到处都是,可以想办法让这个缓冲区的自动保存 文件放到指定的目录中。刚好 make-auto-save-file-name 是在上面这个列表里 的,但是不幸的是在函数定义里 make-auto-save-file-name 里不对不关联文件 的缓冲区使用 handler(我觉得是一个 bug 呀),继续往下看,发现生成保存文 件名是使用了 expand-file-name 函数,所以解决办法就产生了:
(defun my-scratch-auto-save-file-name (operation \&rest args)
(if (and (eq operation 'expand-file-name)
(string= (car args) "\#*scratch*\#"))
(expand-file-name (concat "\textasciitilde{}/.emacs.d/backup/" (car args)))
(let ((inhibit-file-name-handlers
(cons 'my-scratch-auto-save-file-name
(and (eq inhibit-file-name-operation operation)
inhibit-file-name-handlers)))
(inhibit-file-name-operation operation))
(apply operation args))))
\subsubsection{函数列表}
\label{sec:org481c66f}
(find-file FILENAME \&optional WILDCARDS)
(find-file-noselect FILENAME \&optional NOWARN RAWFILE WILDCARDS)
(set-visited-file-name FILENAME \&optional NO-QUERY ALONG-WITH-FILE)
(get-file-buffer FILENAME)
(find-buffer-visiting FILENAME \&optional PREDICATE)
(save-buffer \&optional ARGS)
(insert-file-contents FILENAME \&optional VISIT BEG END REPLACE)
(insert-file-contents-literally FILENAME \&optional VISIT BEG END REPLACE)
(write-region START END FILENAME \&optional APPEND VISIT LOCKNAME MUSTBENEW)
(file-exists-p FILENAME)
(file-readable-p FILENAME)
(file-writable-p FILENAME)
(file-executable-p FILENAME)
(file-modes FILENAME)
(file-regular-p FILENAME)
(file-directory-p FILENAME)
(file-symlink-p FILENAME)
(file-truename FILENAME)
(file-attributes FILENAME \&optional ID-FORMAT)
(rename-file FILE NEWNAME \&optional OK-IF-ALREADY-EXISTS)
(copy-file FILE NEWNAME \&optional OK-IF-ALREADY-EXISTS KEEP-TIME PRESERVE-UID-GID)
(delete-file FILENAME)
(make-directory DIR \&optional PARENTS)
(delete-directory DIRECTORY)
(set-file-modes FILENAME MODE)
(file-name-directory FILENAME)
(file-name-nondirectory FILENAME)
(file-name-sans-extension FILENAME)
(file-name-sans-versions NAME \&optional KEEP-BACKUP-VERSION)
(file-name-absolute-p FILENAME)
(expand-file-name NAME \&optional DEFAULT-DIRECTORY)
(file-relative-name FILENAME \&optional DIRECTORY)
(file-name-as-directory FILE)
(directory-file-name DIRECTORY)
(convert-standard-filename FILENAME)
(make-temp-file PREFIX \&optional DIR-FLAG SUFFIX)
(make-temp-name PREFIX)
(directory-files DIRECTORY \&optional FULL MATCH NOSORT)
(dired-files-attributes DIR)
\subsubsection{问题解答}
\label{sec:orgac86497}
\subsubsection{提取头文件中函数名}
\label{sec:org2705c96}
这是我写的一个版本,主要是函数声明的正则表达式不好写,函数是很简单的。 从这个例子也可以看出它错误的把那个 typedef void 当成函数声明了。如果你 知道更好的正则表达式,请告诉我一下。
(defvar header-regexp-list
'(("\^{}$\backslash$\(?:\\(?:G_CONST_RETURN\\|extern\\|const\\)$\backslash$\s-\sout{$\backslash$\)?[a-zA-Z][\(_{\text{a}}\)-zA-Z0-9]*$\backslash$
$\backslash$\(?:$\backslash$\s-\textbf{[*]}[ \t\n]}$\backslash$\|$\backslash$\s-\sout{[*]*$\backslash$\)$\backslash$\([a-zA-Z][_a-zA-Z0-9]*\\)$\backslash$\s-*(" . 1)
("\^{}$\backslash$\s-*\#$\backslash$\s-*define$\backslash$\s-}$\backslash$\([a-zA-Z][_a-zA-Z0-9]*\\)" . 1)))
(defun parse-c-header (file)
"Extract function name and declaration position using
`header-regexp-list'."
(interactive "fHeader file: \nP")
(let (info)
(with-temp-buffer
(insert-file-contents file)
(dolist (re header-regexp-list)
(goto-char (point-min))
(while (re-search-forward (car re) nil t)
(push (cons (match-string (cdr re)) (line-beginning-position)) info))))
info))
(parse-c-header "/usr/include/glib-2.0/gmodule.h")
;; =>
;; (("g\(_{\text{module}}\)\(_{\text{name}}\)" . 1788)
;;  ("g\(_{\text{module}}\)\(_{\text{open}}\)" . 1747)
;;  ("G\(_{\text{MODULE}}\)\(_{\text{EXPORT}}\)" . 1396)
;;  ("G\(_{\text{MODULE}}\)\(_{\text{EXPORT}}\)" . 1317)
;;  ("G\(_{\text{MODULE}}\)\(_{\text{IMPORT}}\)" . 1261)
;;  ("g\(_{\text{module}}\)\(_{\text{build}}\)\(_{\text{path}}\)" . 3462)
;;  ("g\(_{\text{module}}\)\(_{\text{name}}\)" . 2764)
;;  ("g\(_{\text{module}}\)\(_{\text{symbol}}\)" . 2570)
;;  ("g\(_{\text{module}}\)\(_{\text{error}}\)" . 2445)
;;  ("g\(_{\text{module}}\)\(_{\text{make}}\)\(_{\text{resident}}\)" . 2329)
;;  ("g\(_{\text{module}}\)\(_{\text{close}}\)" . 2190)
;;  ("g\(_{\text{module}}\)\(_{\text{open}}\)" . 2021)
;;  ("g\(_{\text{module}}\)\(_{\text{supported}}\)" . 1894)
;;  ("void" . 1673))
\subsubsection{模拟 chmod 的函数}
\label{sec:org5947bfe}
这是一个改变单个文件模式的 chmod 版本。递归版本的就自己作一个练习吧。最 好不要直接调用这个函数,因为每次调用都要解析一次 mode 参数,想一个只解 析一次的方法吧。
(defun chmod (mode file)
"A elisp function to simulate command chmod.
Note that the command chmod can accept MODE match
`[ugoa]*([-\sout{=]([rwxXst]*|[ugo]))}', but this version only can process
MODE match `[ugoa]*[-+=]([rwx]*|[ugo])'.
"
(cond ((integerp mode)
(if (> mode \#o777)
(error "Unknown mode option: \%d" mode)))
((string-match "\^{}[0-7]$\backslash$\\{3$\backslash$\\}\(" mode)
      (setq mode (string-to-number mode 8)))
      ((string-match "^\\([ugoa]*\\)\\([-+=]\\)\\([rwx]*\\|[ugo]\\)\)" mode)
(let ((users (append (match-string 1 mode) nil))
(mask-func (string-to-char (match-string 2 mode)))
(bits (append (match-string 3 mode) nil))
(oldmode (file-modes file))
(user-list '((?a . \#o777)
(?u . \#o700)
(?g . \#o070)
(?o . \#o007)))
mask)
(when bits
(setq bits (* (cond ((= (car bits) ?u)
(lsh (logand oldmode \#o700) -6))
((= (car bits) ?g)
(lsh (logand oldmode \#o070) -3))
((= (car bits) ?o)
(logand oldmode \#o007))
(t
(+ (if (member ?r bits) 4 0)
(if (member ?w bits) 2 0)
(if (member ?x bits) 1 0))))
\#o111))
(if users
(setq mask (apply 'logior
(delq nil (mapcar
(lambda (u)
(assoc-default u user-list))
users))))
(setq mask \#o777))
(setq mode
(cond ((= mask-func ?$\backslash$=)
(logior (logand bits mask)
(logand oldmode (logxor mask \#o777))))
((= mask-func ?$\backslash$+)
(logior oldmode (logand bits mask)))
(t
(logand oldmode
(logxor (logand bits mask) \#o777))))))))
(t (error "Unknow mode option: \%S" mode)))
(set-file-modes file mode))
\subsubsection{列出目录中所有文件}
\label{sec:org226591f}
为了让这个函数更类似 directory-files 函数,我把参数设置为和它一样的:
(defun my-directory-all-files (dir \&optional full match nosort)
(apply 'append
(delq nil
(mapcar
(lambda (file)
(if (and (not (string-match "\^{}[.]+\$" (file-name-nondirectory file)))
(file-directory-p (expand-file-name file dir)))
(if full
(my-directory-all-files file full match nosort)
(mapcar (lambda (f)
(concat (file-name-as-directory file) f))
(my-directory-all-files (expand-file-name file dir)
full match nosort)))
(if (string-match match file)
(list file))))
(directory-files dir full nil nosort)))))
\subsection{操作对象之四 ── 文本}
\label{sec:orgb379b31}

文本的插入删除,查找替换操作已经在缓冲区一节中讲过了。这一节主要介绍文 本属性。
如果使用过其它图形界面的文本组件进行编程,它们对于文本的高亮一般都是采 用给对应
文本贴上相应标签的方法。Emacs 的处理方法也是类似的,但是相比之 下,要强大的多。
在 Emacs 里,在不同位置上的每个字符都可以有一个属性列表。 这个属性列表和符号的属
性列表很相似,都是由一个名字和值构成的对组成。名 字和值都可以是一个 lisp 对象,
但是通常名字都是一个符号,这样可以用这个 符号来查找相应的属性值。复制文本通常都
会复制相应的字符的文本属性,但是 也可以用相应的函数只复制文本字符串,比如
substring-no-properties、 insert-buffer-substring-no-properties、
buffer-substring-no-properties。

产生一个带属性的字符串可以用 propertize 函数
(propertize "abc" 'face 'bold)          ; => \#("abc" 0 3 (face bold))
如果你在一个 text-mode 的缓冲区内用 M-x eval-expression 用 insert 函数 插入前面
这个字符串,就会发现插入的文本已经是粗体字了。之所以不能在 \textbf{scratch} 产生这种效
果,是因为通常我们是开启了 font-lock-mode,在 font-lock-mode 里,文本的 face 属
性是实时计算出来的。 在插入文本之后,它的 face 属性已经很快地被改变了。你可以在
关闭 font-lock-mode 后再测试一次应该是可以看到 \textbf{scratch} 里也是可以用这种方法插
入带 face 属性的文本的。
虽然文本属性的名字可以是任意的,但是一些名字是有特殊含义的。

\subsubsection{属性名	含义}
\label{sec:org0d46de0}

category	值必须是一个符号,这个符号的属性将作为这个字符的属性
face	控制文本的字体和颜色
font-lock-face	和 face 相似,可以作为 font-lock-mode 中静态文本的 face
mouse-face	当鼠标停在文本上时的文本 face
fontified	记录是否使用 font lock 标记了 face
display	改变文本的显示方式,比如高、低、长短、宽窄,或者用图片代替
help-echo	鼠标停在文本上时显示的文字
keymap	光标或者鼠标在文本上时使用的按键映射
local-map	和 keymap 类似,通常只使用 keymap
syntax-table	字符的语法表
read-only	不能修改文本,通过 stickness 来选择可插入的位置
invisible	不显示在屏幕上
intangible	把文本作为一个整体,光标不能进入
field	一个特殊标记,有相应的函数可以操作带这个标记的文本
cursor	(不知道具体用途)
pointer	修改鼠标停在文本上时的图像
line-spacing	新的一行的距离
line-height	本行的高度
modification-hooks	修改这个字符时调用的函数
insert-in-front-hooks	与 modification-hooks 相似,在字符前插入调用的函数
insert-behind-hooks	与 modification-hooks 相似,在字符后插入调用的函数
point-entered	当光标进入时调用的函数
point-left	当光标离开时调用的函数
composition	将多个字符显示为一个字形
正是由于 emacs 的文本有如此丰富的属性,使得 emacs 里的文字才变得多彩, 变得人性
化。

\subsubsection{查看文本属性}
\label{sec:org85e2b86}

由于字符串和缓冲区都可以有文本属性,所以下面的函数通常不提供特定参数就是检 查当
前缓冲区的文本属性,如果提供文本对象,则是操作对应的文本属性。
查看文本对象在某处的文本属性可以用 get-text-property 函数。
(setq foo (concat "abc"
(propertize "cde" 'face 'bold))) ; => \#("abccde" 3 6 (face bold))
(get-text-property 3 'face foo)                    ; => bold
(save-excursion
(goto-char (point-min))
(insert foo))
(get-text-property 4 'face)                        ; => bold
get-char-property 和 get-text-property 相似,但是它是先查找 overlay 的 文本属性。
overlay 是缓冲区文字在屏幕上的显示方式,它属于某个缓冲区,具 有起点和终点,也具
有文本属性,可以修改缓冲区对应区域上文本的显示方式。
get-text-property 是查找某个属性的值,用 text-properties-at 可以得到某 个位置上
文本的所有属性。

\subsubsection{修改文本属性}
\label{sec:org4ded085}

put-text-property 可以给文本对象添加一个属性。比如
(let ((str "abc"))
(put-text-property 0 3 'face 'bold str)
str)                                  ; \texttt{> \#("abc" 0 3 (face bold))
     和 put-text-property 类似,add-text-properties 可以给文本对象添加一系 列的属性。和 add-text-properties 不同,可以用 set-text-properties 直接 设置文本属性列表。你可以用 =(set-text-properties start end nil)} 来除去 某个区间上的文本属性。也可以用 remove-text-properties 和 remove-list-of-text-properties 来除去某个区域的指定文本属性。这两个函 数的属性列表参数只有名字起作用,值是被忽略的。
(setq foo (propertize "abcdef" 'face 'bold
'pointer 'hand))
;; => \#("abcdef" 0 6 (pointer hand face bold))
(set-text-properties 0 2 nil foo)       ; => t
foo   ; => \#("abcdef" 2 6 (pointer hand face bold))
(remove-text-properties 2 4 '(face nil) foo) ; => t
foo   ; => \#("abcdef" 2 4 (pointer hand) 4 6 (pointer hand face bold))
(remove-list-of-text-properties 4 6 '(face nil pointer nil) foo) ; => t
foo   ; => \#("abcdef" 2 4 (pointer hand))

\subsubsection{查找文本属性}
\label{sec:orgcd60c2e}

文本属性通常都是连成一个区域的,所以查找文本属性的函数是查找属性变化的 位置。这
些函数一般都不作移动,只是返回查找到的位置。使用这些函数时最好 使用 LIMIT 参数,
这样可以提高效率,因为有时一个属性直到缓冲区末尾也没 有变化,在这些文本中可能就
是多余的。
next-property-change 查找从当前位置起任意一个文本属性发生改变的位置。
next-single-property-change 查找指定的一个文本属性改变的位置。
next-char-property-change 把 overlay 的文本属性考虑在内查找属性发生改 变的位置。
next-single-property-change 类似的查找指定的一个考虑 overlay 后文本属性改变的位
置。这四个函数都对应有 previous- 开头的函数,用于查 找当前位置之前文本属性改变的
位置
(setq foo (concat "ab"
(propertize "cd" 'face 'bold)
(propertize "ef" 'pointer 'hand)))
;; => \#("abcdef" 2 4 (face bold) 4 6 (pointer hand))
(next-property-change 1 foo)                  ; => 2
(next-single-property-change 1 'pointer foo)  ; => 4
(previous-property-change 6 foo)              ; => 4
(previous-single-property-change 6 'face foo) ; => 4
text-property-any 查找区域内第一个指定属性值为给定值的字符位置。
text-property-not-all 和它相反,查找区域内第一个指定属性值不是给定值的 字符位置。
(text-property-any 0 6 'face 'bold foo)          ; => 2
(text-property-any 0 6 'face 'underline foo)     ; => nil
(text-property-not-all 2 6 'face 'bold foo)      ; => 4
(text-property-not-all 2 6 'face 'underline foo) ; => 2
\subsubsection{思考题}
\label{sec:org8316fd7}
写一个命令,可在 text-mode 里用指定模式给选中的文本添加高亮。
\subsubsection{函数列表}
\label{sec:org935447b}
(propertize STRING \&rest PROPERTIES)
(get-text-property POSITION PROP \&optional OBJECT)
(get-char-property POSITION PROP \&optional OBJECT)
(text-properties-at POSITION \&optional OBJECT)
(put-text-property START END PROPERTY VALUE \&optional OBJECT)
(add-text-properties START END PROPERTIES \&optional OBJECT)
(set-text-properties START END PROPERTIES \&optional OBJECT)
(remove-text-properties START END PROPERTIES \&optional OBJECT)
(remove-list-of-text-properties START END LIST-OF-PROPERTIES \&optional OBJECT)
(next-property-change POSITION \&optional OBJECT LIMIT)
(next-single-property-change POSITION PROP \&optional OBJECT LIMIT)
(next-char-property-change POSITION \&optional LIMIT)
(next-single-char-property-change POSITION PROP \&optional OBJECT LIMIT)
(previous-property-change POSITION \&optional OBJECT LIMIT)
(previous-single-property-change POSITION PROP \&optional OBJECT LIMIT)
(previous-char-property-change POSITION \&optional LIMIT)
(previous-single-char-property-change POSITION PROP \&optional OBJECT LIMIT)
(text-property-any START END PROPERTY VALUE \&optional OBJECT)
(text-property-not-all START END PROPERTY VALUE \&optional OBJECT)
\subsubsection{问题解答}
\label{sec:orgbb15e8a}
\subsubsection{手工高亮代码}
\label{sec:orge31dc6f}
(defun my-fontify-region (beg end mode)
(interactive
(list (region-beginning)
(region-end)
(intern
(completing-read "Which mode to use: "
obarray (lambda (s)
(and (fboundp s)
(string-match "-mode\$" (symbol-name s))))
t))))
(let ((buf (current-buffer))
(font-lock-verbose nil)
(start 1) face face-list)
(set-text-properties beg end '(face nil))
(with-temp-buffer
(goto-char (point-min))
(insert-buffer-substring buf beg end)
(funcall mode)
(font-lock-fontify-buffer)
(or (get-text-property start 'face)
(setq start (next-single-property-change start 'face)))
(while (and start (< start (point-max)))
(setq end (or (next-single-property-change start 'face)
(point-max))
face (get-text-property start 'face))
(and face end (setq face-list (cons (list (1- start) (1- end) face) face-list)))
(setq start end)))
(when face-list
(dolist (f (nreverse face-list))
(put-text-property (+ beg (car f)) (+ beg (cadr f))
'face (nth 2 f))))))
但是直接从那个临时缓冲区里把整个代码拷贝出来也可以了,但是可能某些情况 下,不好
修改当前缓冲区,或者不想把那个模式里其它文本属性拷贝出来,这个 函数还是有用的。
当然最主要的用途是演示使用查找和添加文本属性的方法。事 实上这个函数也是我用来
高亮 muse 模式里 src 标签内源代码所用的方法。但是 不幸的是 muse 模式里这个函数
并不能产生很好的效果,不知道为什么。

\subsection{后记}
\label{sec:org267fde0}

到现在为止,我计划写的 elisp 入门内容已经写完了。如果你都看完看懂这些内 容,我想
写一些简单的 elisp 应用应该是没有什么问题了。还有一些比较重要的 内容没有涉及到,
我在这列一下,如果你对此有兴趣,可以自己看 elisp manual 里相关章节:
按键映射(keymap) 和菜单
Minibuffer 和补全
进程
调试
主模式(major mode) 和从属模式(minor mode)
定制声明
修正函数 (advising function)
非 ASCII 字符
其实看一遍 elisp manual 也是很好的选择。我在写这些文字时就是一边参考 elisp
manual 一边写的。写的时候我一直有种不安的感觉,这近 3M 的文字被 我压缩到这么一点
点是不是太过份了。在 elisp manual 里一些很重要的说明经 常被我一两句话就带过了,
有时根本就没有提到,这会不会让刚学 elisp 的人 误入歧途呢?每每想到这一点,我就想
就此停住。但是半途而废总是不好的。所 以我还是决定写完应该写的就好了。其它的再说
吧。
如果你是一个新手,我很想知道你看完这个入门教程的感受。当然如果实在没有 兴趣看,
也可以告诉我究竟哪里写的不好。我希望在这份文档上花的时间和精力 没有白费。
\subsection{{\bfseries\sffamily DONE} Emacs Lisp 简明教程}
\label{sec:org36752d8}
\subsection{一个 Hello World 例子}
\label{sec:org621d694}

自从 K\&R 以来,hello world 程序历来都是程序语言教程的第一个例子。我也用一个
hello world 的例子来演示 emacs 里执行 elisp 的环境。下面就是这个语句:

(message "hello world")
前面我没有说这个一个程序,这是因为,elisp 不好作为可执行方式来运行(当然也不是不
可能),所有的 elisp 都是运行在 emacs 这个环境下。

首先切换到 \textbf{scratch} 缓冲区里,如果当前模式不是 lisp-interaction-mode,用 M-x
lisp-interaction-mode 先转换到 lisp-interaction-mode。然后输入前面这一行语句。在
行尾右括号后,按 C-j 键。如果 Minibuffer 里显示 hello world,光标前一行也显示
"hello world",那说明你的操作没有问题。我们就可以开始 elisp 学习之旅了。

注:elisp 里的一个完整表达式,除了简单数据类型(如数字,向量),都是用括号括起来,
称为一个 S-表达式。让 elisp 解释器执行一个 S-表达式除了前一种方法之外,还可以用
C-x C-e。它们的区别是,C-x C-e 是一个全局按键绑定,几乎可以在所有地方都能用。它
会将运行返回值显示在 Minibuffer 里。这里需要强调一个概念是返回值和作用是不同的。
比如前面 message 函数它的作用是在 Minibuffer 里显示一个字符串,但是它的返回值是
"hello world" 字符串。

\subsection{基础知识}
\label{sec:orgd88a577}

这一节介绍一下 elisp 编程中一些最基本的概念,比如如何定义函数,程序的控制结构,
变量的使用和作用域等等。

\subsubsection{函数和变量}
\label{sec:orge466368}
\subsubsection{elisp 中定义一个函数是用这样的形式:}
\label{sec:org02b73fe}
(defun function-name (arguments-list)
"document string"
body)
比如:
\begin{SCR}
(defun hello-world (name)
"Say hello to user whose name is NAME."
(message "Hello, \%s" name))
\end{SCR}

其中函数的文档字符串是可以省略的。但是建议为你的函数(除了最简单,不作为接口的)都加上文档字符串。这样将来别人使用你的扩展或者别人阅读你的代码或者自己进行维护都提供很大的方便。
在 emacs 里,当光标处于一个函数名上时,可以用 C-h f 查看这个函数的文档。比如前面这个函数,在 \textbf{Help} 缓冲区里的文档是:
hello-world is a Lisp function.
\begin{SCR}
(hello-world name)
\end{SCR}
Say hello to user whose name is name.
如果你的函数是在文件中定义的。这个文档里还会给出一个链接能跳到定义的地方。
\subsubsection{要运行一个函数,最一般的方式是:}
\label{sec:org7cfccfa}
(function-name arguments-list)
比如前面这个函数:
(hello-world "Emacser")                 ; => "Hello, Emacser"
每个函数都有一个返回值。这个返回值一般是函数定义里的最后一个表达式的值。
\subsubsection{elisp 里的变量使用无需象 C 语言那样需要声明,你可以用 setq 直接对一个变量赋值。}
\label{sec:org1541b9a}
(setq foo "I'm foo")                    ; => "I'm foo"
(message foo)                           ; => "I'm foo"
和函数一样,你可以用 C-h v 查看一个变量的文档。比如当光标在 foo 上时用 C-h v 时,文档是这样的:
foo's value is "I'm foo"

Documentation:
Not documented as a variable.
\subsubsection{有一个特殊表达式(special form)defvar,它可以声明一个变量,一般的形式是:}
\label{sec:org65d24af}
(defvar variable-name value
"document string")
它与 setq 所不同的是,如果变量在声明之前,这个变量已经有一个值的话,用 defvar 声明的变量值不会改变成声明的那个值。另一个区别是 defvar 可以为变量提供文档字符串,当变量是在文件中定义的话,C-h v 后能给出变量定义的位置。比如:
(defvar foo "Did I have a value?"
"A demo variable")                    ; => foo
foo                                     ; => "I'm foo"
(defvar bar "I'm bar"
"A demo variable named $\backslash$"bar$\backslash$"")      ; => bar
bar                                     ; => "I'm bar"
用 C-h v 查看 foo 的文档,可以看到它已经变成:
foo's value is "I'm foo"

Documentation:
A demo variable
由于 elisp 中函数是全局的,变量也很容易成为全局变量(因为全局变量和局部变量的赋值都是使用 setq 函数),名字不互相冲突是很关键的。所以除了为你的函数和变量选择一个合适的前缀之外,用 C-h f 和 C-h v 查看一下函数名和变量名有没有已经被使用过是很关键的。
\subsubsection{局部作用域的变量}
\label{sec:org3248962}
如果没有局部作用域的变量,都使用全局变量,函数会相当难写。elisp 里可以用 let 和 let* 进行局部变量的绑定。let 使用的形式是:
(let (bindings)
body)
bingdings 可以是 (var value) 这样对 var 赋初始值的形式,或者用 var 声明一个初始值为 nil 的变量。比如:
(defun circle-area (radix)
(let ((pi 3.1415926)
area)
(setq area (* pi radix radix))
(message "直径为 \%.2f 的圆面积是 \%.2f" radix area)))
(circle-area 3)
C-h v 查看 area 和 pi 应该没有这两个变量。
let* 和 let 的使用形式完全相同,唯一的区别是在 let* 声明中就能使用前面声明的变量,比如:
(defun circle-area (radix)
(let* ((pi 3.1415926)
(area (* pi radix radix)))
(message "直径为 \%.2f 的圆面积是 \%.2f" radix area)))
\subsubsection{lambda 表达式}
\label{sec:org3147457}
可能你久闻 lambda 表达式的大名了。其实依我的理解,lambda 表达式相当于其它语言中的匿名函数。比如 perl 里的匿名函数。它的形式和 defun 是完全一样的:
(lambda (arguments-list)
"documentation string"
body)
调用 lambda 方法如下:
(funcall (lambda (name)
(message "Hello, \%s!" name)) "Emacser")
你也可以把 lambda 表达式赋值给一个变量,然后用 funcall 调用:
(setq foo (lambda (name)
(message "Hello, \%s!" name)))
(funcall foo "Emacser")                   ; => "Hello, Emacser!"
lambda 表达式最常用的是作为参数传递给其它函数,比如 mapc。
\subsubsection{控制结构}
\label{sec:orgded0ba4}
\subsubsection{顺序执行}
\label{sec:org9819b64}
一般来说程序都是按表达式顺序依次执行的。这在 defun 等特殊环境中是自动进行的。但是一般情况下都不是这样的。比如你无法用 eval-last-sexp 同时执行两个表达式,在 if 表达式中的条件为真时执行的部分也只能运行一个表达式。这时就需要用 progn 这个特殊表达式。它的使用形式如下:
(progn A B C \ldots{})
它的作用就是让表达式 A, B, C 顺序执行。比如:
(progn
(setq foo 3)
(message "Square of \%d is \%d" foo (* foo foo)))
\subsubsection{条件判断}
\label{sec:org3edb155}
elisp 有两个最基本的条件判断表达式 if 和 cond。使用形式分别如下:
(if condition
then
else)

(cond (case1 do-when-case1)
(case2 do-when-case2)
\ldots{}
(t do-when-none-meet))
使用的例子如下:
(defun my-max (a b)
(if (> a b)
a b))
(my-max 3 4)                            ; => 4

(defun fib (n)
(cond ((= n 0) 0)
((= n 1) 1)
(t (+ (fib (- n 1))
(fib (- n 2))))))
(fib 10)                                ; => 55
还有两个宏 when 和 unless,从它们的名字也就能知道它们是作什么用的。使用这两个宏的好处是使代码可读性提高,when 能省去 if 里的 progn 结构,unless 省去条件为真子句需要的的 nil 表达式。
\subsubsection{循环}
\label{sec:org49a4fb9}
循环使用的是 while 表达式。它的形式是:
(while condition
body)
比如:
(defun factorial (n)
(let ((res 1))
(while (> n 1)
(setq res (* res n)
n (- n 1)))
res))
(factorial 10)                          ; => 3628800
\subsubsection{逻辑运算}
\label{sec:orgdbb320c}
条件的逻辑运算和其它语言都是很类似的,使用 and、or、not。and 和 or 也同样具有短路性质。很多人喜欢在表达式短时,用 and 代替 when,or 代替 unless。当然这时一般不关心它们的返回值,而是在于表达式其它子句的副作用。比如 or 经常用于设置函数的缺省值,而 and 常用于参数检查:
(defun hello-world (\&optional name)
(or name (setq name "Emacser"))
(message "Hello, \%s" name))           ; => hello-world
(hello-world)                           ; => "Hello, Emacser"
(hello-world "Ye")                      ; => "Hello, Ye"

(defun square-number-p (n)
(and (>= n 0)
(= (/ n (sqrt n)) (sqrt n))))
(square-number-p -1)                    ; => nil
(square-number-p 25)                    ; => t
\subsubsection{函数列表}
\label{sec:org0c61e60}
(defun NAME ARGLIST [DOCSTRING] BODY\ldots{})
(defvar SYMBOL \&optional INITVALUE DOCSTRING)
(setq SYM VAL SYM VAL \ldots{})
(let VARLIST BODY\ldots{})
(let* VARLIST BODY\ldots{})
(lambda ARGS [DOCSTRING] [INTERACTIVE] BODY)
(progn BODY \ldots{})
(if COND THEN ELSE\ldots{})
(cond CLAUSES\ldots{})
(when COND BODY \ldots{})
(unless COND BODY \ldots{})
(when COND BODY \ldots{})
(or CONDITIONS \ldots{})
(and CONDITIONS \ldots{})
(not OBJECT)** DONE \href{http://ergoemacs.org/emacs/elisp.html}{Practical Emacs Lisp}
CLOSED: \textit{[2017-10-22 Sun 21:06]}
\textit{<2017-10-13 Fri>}
\subsection{{\bfseries\sffamily DONE} Emacs Lisp Basics}
\label{sec:orgeb07682}
By Xah Lee. Date: 2005-10-30. Last updated: 2017-01-04.
This page is a short, practical, tutorial of Emacs Lisp the language.
To evaluate elisp code, for example, type (+ 3 4), then move your cursor to
after the closing parenthesis, then call eval-last-sexp 【Ctrl+x Ctrl+e】. Emacs
will evaluate the lisp expression to the left of the cursor.

Alternatively, you can select the lisp code, then Alt+x eval-region.
Alternatively, you can Alt+x ielm. It will start a interactive elisp command line interface.
To find the doc string of a function, Alt+x describe-function 【Ctrl+h f】. (If
the word under cursor is a function, emacs will lookup that by default, saves
typing.)


eval emacs lisp code basics. [see How to Evaluate Emacs Lisp Code]
\subsubsection{Printing}
\label{sec:orgfe0cb76}
\subsubsection{; printing}
\label{sec:org984a536}
(message "hi")

\subsubsection{; printing variable values}
\label{sec:org96aca1c}
(message "Her age is: \%d" 16)        ; \%d is for number
(message "Her name is: \%s" "Vicky")  ; \%s is for string
(message "My list is: \%S" (list 8 2 3))  ; \%S is for any lisp expression
You can see the output in the buffer named “*Messages*”. You can switch to it by Alt+x view-echo-area-messages 【Ctrl+h e】.
More detail: Emacs Lisp's print, princ, prin1, format, message.
\subsubsection{Arithmetic Functions}
\label{sec:org68afb49}
(+ 4 5 1)     ;    ⇒ 10
(- 9 2)       ;    ⇒  7
(- 9 2 3)     ;    ⇒  4
(* 2 3)       ;    ⇒  6
(* 2 3 2)     ;    ⇒ 12
(/ 7 2)       ;    ⇒  3 (integer part of quotient)
(/ 7 2.0)     ;    ⇒  3.5
(\% 7 4)       ;    ⇒  3 (mod, remainder)
(expt 2 3)    ;    ⇒ 8 (power; exponential)
\textbf{WARNING: single digit decimal number such as 2. needs a zero after the dot, like this: 2.0. For example, (/ 7 2.) returns 3, not 3.5.}
;; 3. is a integer, 3.0 is a float
(integerp 3.) ; returns t
(floatp 3.) ; returns nil
(floatp 3.0) ; returns t
Function names that end with a “p” often means it return either true or false. (The “p” stands for “predicate”) t means true; nil means false.
\subsubsection{Convert String and Numbers}
\label{sec:org9a754fe}
(string-to-number "3")
(number-to-string 3)
You can also use format to convert number to string. [see Elisp: print, princ, prin1, format, message]
\subsubsection{(info "(elisp) Numbers")}
\label{sec:org49fc04f}
\subsubsection{True, False}
\label{sec:org7d31993}
In elisp, the symbol nil is false, anything else is considered true. Also, nil
is equivalent to the empty list (), so () is also false.

;; all the following are false. They all evaluate to “nil”
(if nil "yes" "no") ; ⇒ "no"
(if () "yes" "no") ; ⇒ "no"
(if '() "yes" "no") ; ⇒ "no"
(if (list) "yes" "no") ; ⇒ "no", because (list) eval to a empty list, same as ()
By convention, the symbol t is used for true.
(if t "yes" "no") ; ⇒ "yes"
(if 0 "yes" "no") ; ⇒ "yes"
(if "" "yes" "no") ; ⇒ "yes"
(if [] "yes" "no") ; ⇒ "yes". The [] is vector of 0 elements
There is no “boolean datatype” in elisp. Just remember that nil and empty list
() are false, anything else is true.

\subsubsection{Boolean Functions}
\label{sec:org4c55eb3}
Here's and and or.
(and t nil) ; ⇒ nil
(or t nil) ; ⇒ t

;; can take multiple args
(and t nil t t t t) ; ⇒ nil
Comparing numbers:
(< 3 4) ; less than
(> 3 4) ; greater than

(<= 3 4) ; less or equal to
(>= 3 4) ; greater or equal to

(= 3 3)   ; ⇒ t
(= 3 3.00000000000000001) ; ⇒ t

(/= 3 4) ; not equal. ⇒ t
\subsubsection{Comparing strings:}
\label{sec:org02bbcff}
;; compare string
(equal "abc" "abc") ; ⇒ t

\subsubsection{;; dedicated function for comparing string}
\label{sec:orgfd55d12}
(string-equal "abc" "abc") ; ⇒ t

(string-equal "abc" "Abc") ; ⇒ nil. Case matters

;; can be used to compare string and symbol
(string-equal "abc" 'abc) ; ⇒ t
For generic equality test, use equal. It tests if two values have the same datatype and value.
;; test if two values have the same datatype and value.

(equal 3 3) ; ⇒ t
(equal 3.0 3.0) ; ⇒ t

(equal 3 3.0) ; ⇒ nil. Because datatype doesn't match.

\subsubsection{;; test equality of lists}
\label{sec:org8535f41}
(equal '(3 4 5) '(3 4 5))  ; ⇒ t
(equal '(3 4 5) '(3 4 "5")) ; ⇒ nil

\subsubsection{;; test equality of strings}
\label{sec:org3614bb8}
(equal "e" "e") ; ⇒ t

\subsubsection{;; test equality of symbols}
\label{sec:orgad6188e}
(equal 'abc 'abc) ; ⇒ t
There's also the function eq, it returns t if the two args are the same Lisp
object. This is usually not what you want. (eq "e" "e") returns nil.

To test for inequality, the /= is for numbers only, and doesn't work for strings
and other lisp data. Use not to negate your equality test, like this:

(not (= 3 4)) ; ⇒ t
(/= 3 4) ; ⇒ t. “/=” is for comparing numbers only

(not (equal 3 4)) ; ⇒ t. General way to test inequality.
•	(info "(elisp) Comparison of Numbers")
•	(info "(elisp) Equality Predicates")
even, odd
(= (\% n 2) 0) ; test even

(= (\% n 2) 1) ; test odd
Variables
Global Variables
setq is used to set variables. Variables need not be declared, and is global.
(setq x 1) ; assign 1 to x
(setq a 3 b 2 c 7) ; assign 3 to a, 2 to b, 7 to c
Local Variables
To define local variables, use let. The form is: (let (var1 var2 …) body) where body is (one or more) lisp expressions. The body's last expression's value is returned.
(let (a b)
(setq a 3)
(setq b 4)
(+ a b)
) ; returns 7
Another form of let is this: (let ((var1 val1) (var2 val2) …) body). Example:
(let ((a 3) (b 4))
(+ a b)
) ; returns 7
This form lets you set values to variable without using many setq in the body. This form is convenient if you just have a few simple local vars with known values.
\subsubsection{(info "(elisp) Variables")}
\label{sec:org789550e}
If Then Else
The form for “if” expression is: (if test body).
If you want a “else” part, the form is (if test true\(_{\text{body}}\) false\(_{\text{body}}\)).
Examples:
(if (< 3 2) (message "yes") ) ; does nothing. returns nil

(if (< 3 2) 7 8 ) ; returns 8
\subsubsection{(info "(elisp) Control Structures")}
\label{sec:orga5dca27}
If you do not need a “else” part, you should use the function when instead, because it is more clear. The form is this: (when test expr1 expr2 …). Its meaning is the same as (if test (progn expr1 expr2 …)).
Block of Expressions
Sometimes you need to group several expressions together as one single expression. This can be done with progn.
(progn (message "a") (message "b"))
;; is equivalent to
(message "a") (message "b")
The purpose of (progn …) is similar to a block of code \{…\} in C-like languages. It is used to group together a bunch of expressions into one single parenthesized expression. Most of the time it's used inside “if”. For example:
(if something
(progn ; true
…
)
(progn ; else
…
)
)
progn returns the last expression in its body.
(progn 3 4 ) ; ⇒ 4
(info "(elisp) Sequencing")
Loop
Most basic loop in elisp is with while.
(while test body)
, where body is one or more lisp expressions.
(setq x 0)

(while (< x 4)
(print (format "number is \%d" x))
(setq x (1+ x)))
;; inserts Unicode chars 32 to 126
(let ((x 32))
(while (< x 127)
(insert-char x)
(setq x (+ x 1))))
Usually it's better to use dolist or dotimes.
[see Elisp: Map / Loop Thru List / Vector]
(info "(elisp) Iteration")
Break/Exit a Loop
Elisp: Exit Loop/Function, catch/throw
Sequence, List, Vector
Elisp: Sequence: List, Array
Elisp: Vector
Elisp: List
Define a Function
Basic function definition is of the form:
(defun function\(_{\text{name}}\) (param1 param2 …) "doc\(_{\text{string}}\)" body)
Example:
(defun myFunction ()
"testing"
(message "Yay!"))
When a function is called, the last expression in the function's definition body is returned. (there's no “return statement”.)
(info "(elisp) Defining Functions")
Define a Command
A command is a function that emacs user can call by execute-extended-command 【Alt+x】.
When a function is also a command, we say that the function is available for interactive use.
To make a function available for interactive use, add (interactive) right after the doc string.
Evaluate the following code. Then, you can call it by execute-extended-command 【Alt+x】
(defun yay ()
"Insert “Yay!” at cursor position."
(interactive)
(insert "Yay!"))
(info "(elisp) Defining Commands")
Here is a function definition template that majority of elisp commands follow:
(defun myCommand ()
"One sentence summary of what this command do.

More detailed documentation here."
(interactive)
(let (localVar1 localVar2 …)
; do something here …
; …
; last expression is returned
)
)
See also:
•	Elisp: Function Optional Parameters
•	Elisp: Doc String Markup
Emacs Lisp Basics Topic
1	Emacs Lisp Basics
2	Overview of Text-Processing in Emacs Lisp
3	Emacs Lisp Examples, page 1
4	How to Evaluate Emacs Lisp Code
5	Elisp: Documentation Lookup
6	Elisp: Search Documentation
7	How to Edit Lisp Code with Emacs

\subsection{{\bfseries\sffamily DONE} \href{http://www.yiibai.com/lisp/lisp\_basic\_syntax.html}{LISP - 基本语法}}
\label{sec:org5e54425}
\subsection{{\bfseries\sffamily DONE} 概述介绍}
\label{sec:org121ba81}
约翰·麦卡锡发明LISP于1958年,FORTRAN语言的发展后不久。首次由史蒂夫·拉塞尔实施在
IBM704计算机上。

它特别适合用于人工智能方案,因为它有效地处理的符号信息。

Common Lisp的起源,20世纪80年代和90年代,分别接班人Maclisp像ZetaLisp和NIL(Lisp语
言的新实施)等开发。

它作为一种通用语言,它可以很容易地扩展为具体实施。

编写Common Lisp程序不依赖于机器的具体特点,如字长等。

\subsubsection{Common Lisp的特点}
\label{sec:org790c6ca}

\subsubsection{这是机器无关}
\label{sec:orgf284ec4}

\subsubsection{它采用迭代设计方法,且易于扩展。}
\label{sec:org1a05ec2}

\subsubsection{它允许动态更新的程序。}
\label{sec:orgca9e93f}

\subsubsection{它提供了高层次的调试。}
\label{sec:org3d39c23}

\subsubsection{它提供了先进的面向对象编程。}
\label{sec:orge897e00}

\subsubsection{它提供了方便的宏系统。}
\label{sec:orgb089cca}

\subsubsection{它提供了对象,结构,列表,向量,可调数组,哈希表和符号广泛的数据类型。}
\label{sec:org7a69108}

\subsubsection{它是以表达为主。}
\label{sec:org280dc43}

\subsubsection{它提供了一个面向对象的系统条件。}
\label{sec:orgd9dcf63}

\subsubsection{它提供完整的I/ O库。}
\label{sec:org06a51ad}

\subsubsection{它提供了广泛的控制结构。}
\label{sec:org9135423}

\subsubsection{LISP的内置应用程序}
\label{sec:org06eb10d}

大量成功的应用建立在Lisp语言。

\subsubsection{Emacs}
\label{sec:org6cc55eb}

\subsubsection{G20}
\label{sec:orgb8bcf16}

\subsubsection{AutoCad}
\label{sec:org0acdf4c}

\subsubsection{Igor Engraver}
\label{sec:orgd8e5250}

\subsubsection{Yahoo Store}
\label{sec:orgb7321de}

\subsection{程序结构}
\label{sec:org3bf93ac}
LISP表达式称为符号表达式或S-表达式。s表达式是由三个有效对象,原子,列表和字符串。

任意的s-表达式是一个有效的程序。

Lisp程序在解释器或编译的代码运行。

解释器会检查重复的循环,这也被称为读 - 计算 - 打印循环(REPL)源代码。它读取程序代码,计算,并打印由程序返回值。

一个简单的程序

让我们写一个s-表达式找到的三个数字7,9和11的总和。要做到这一点,我们就可以输入在提示符的解释器 ->:

(+7911)
LISP返回结果:

27
如果想运行同一程序的编译代码,那么创建一个名为myprog的一个LISP源代码文件。并在其中输入如下代码:

(write(\sout{7911))
单击Execute按钮,或按下Ctrl} E,LISP立即执行它,返回的结果是:

27
Lisp使用前缀表示法

可能已经注意到,使用LISP前缀符号。

在上面的程序中的+符号可以作为对数的求和过程中的函数名。

在前缀表示法,运算符在自己操作数前写。例如,表达式,

a * ( b + c ) / d
将被写为:

(/ (* a (+ b c) ) d)
让我们再举一个例子,让我们写的代码转换为60o F华氏温度到摄氏刻度:

此转换的数学表达式为:

(60 * 9 / 5) + 32
创建一个名为main.lisp一个源代码文件,并在其中输入如下代码:

(write(+ (* (/ 9 5) 60) 32))
当单击Execute按钮,或按下Ctrl+ E,MATLAB立即执行它,返回的结果是:

140
计算Lisp程序

计算LISP程序有两部分:

程序文本由一个读取器程序转换成Lisp对象

语言的语义在这些对象中的条款执行求值程序

计算过程采用下面的步骤:

读取器转换字符到LISP对象或S-表达式的字符串。

求值器定义为那些从s-表达式内置的Lisp语法形式。计算第二个级别定义的语法决定了S-表达式是LISP语言形式。

求值器可以作为一个函数,它接受一个有效的LISP语言的形式作为参数并返回一个值。这就是为什么我们把括号中的LISP语言表达,因为我们要发送的整个表达式/形式向求值作为参数的原因。

'Hello World' 程序

(write-line "Hello World")
(write-line "I am at 'Tutorials Yiibai'! Learning LISP")
当单击Execute按钮,或按下Ctrl+ E,LISP立即执行它,返回的结果是:

Hello World
I am at 'Tutorials Yiibai'! Learning LISP
\subsection{基本语法}
\label{sec:orgf03b19a}
\subsubsection{LISP基本构建块}
\label{sec:org4baa76f}

Lisp程序是由三个基本构建块:

\begin{itemize}
\item atom

\item list

\item string
\end{itemize}

\subsubsection{一个原子是一个数字连续字符或字符串。它包括数字和特殊字符。}
\label{sec:orga959142}

以下是一些有效的原子的例子:

hello-from-tutorials-yiibai
name
123008907
\textbf{hello}
Block\#221
abc123
\subsubsection{列表是包含在括号中的原子和/或其他列表的序列。以下是一些有效的列表的示例:}
\label{sec:org9db8570}

( i am a list)
(a ( a b c) d e fgh)
(father tom ( susan bill joe))
(sun mon tue wed thur fri sat)
( )
\subsubsection{字符串是一组括在双引号字符。以下是一些有效的字符串的例子:}
\label{sec:org08523b2}

" I am a string"
"a ba c d efg \#\$\%\^{}\&!"
"Please enter the following details :"
"Hello from 'Tutorials Yiibai'! "
添加注释

\subsubsection{分号符号(;)是用于表示一个注释行。}
\label{sec:org504fae5}

例如,

(write-line "Hello World") ; greet the world
; tell them your whereabouts
(write-line "I am at 'Tutorials Yiibai'! Learning LISP")
当单击Execute按钮,或按下Ctrl+ E,LISP立即执行它,返回的结果是:

Hello World
I am at 'Tutorials Yiibai'! Learning LISP
移动到下一节之前的一些值得注意的要点

\subsubsection{以下是一些要点需要注意:}
\label{sec:orgf38a114}

在LISP语言的基本数学运算是 +, -, *, 和 /

Lisp实际上是一个函数调用f(x)为 (f x),例如 cos(45)被写入为 cos 45

LISP表达式是不区分大小写的,cos 45 或COS 45是相同的。

LISP尝试计算一切,包括函数的参数。只有三种类型的元素是常数,总是返回自己的值:

数字

字母t,即表示逻辑真

该值为nil,这表示逻辑false,还有一个空的列表。

稍微介绍一下LISP形式

在前面的章节中,我们提到LISP代码计算过程中采取以下步骤:

读取器转换字符到LISP对象的字符串或 s-expressions.

求值器定义为那些从s-表达式内置的Lisp语法形式。计算第二个级别定义的语法决定了S-表达式是LISP语言形式。

现在,一个LISP的形式可以是:

一个原子

空或非名单

有符号作为它的第一个元素的任何列表

求值器可以作为一个函数,它接受一个有效的LISP语言的形式作为参数,并返回一个值。这个就是为什么我们把括号中的LISP语言表达,因为我们要发送的整个表达式/形式向求值作为参数的原因。

LISP命名约定

名称或符号可以包含任意数量的空白相比,开放和右括号,双引号和单引号,反斜杠,逗号,冒号,分号和竖线其他字母数字字符。若要在名称中使用这些字符,需要使用转义字符()。

一个名字可以包含数字,但不能全部由数字组成,因为那样的话它会被解读为一个数字。同样的名称可以具有周期,但周期不能完全进行。

使用单引号

LISP计算一切,包括函数的参数和列表的成员。

有时,我们需要采取原子或列表字面上,不希望他们求值或当作函数调用。

要做到这一点,我们需要先原子或列表中带有单引号。

下面的例子演示了这一点:

创建一个名为main.lisp文件,并键入下面的代码进去:
(write-line "single quote used, it inhibits evaluation")
(write '(* 2 3))
(write-line " ")
(write-line "single quote not used, so expression evaluated")
(write (* 2 3))
当单击Execute按钮,或按下Ctrl+ E,LISP立即执行它,返回的结果是:

single quote used, it inhibits evaluation
(* 2 3)
single quote not used, so expression evaluated
6

\subsection{黑客与画家}
\label{sec:org270665d}
\subsection{保罗-格雷厄姆的创业哲学}
\label{sec:org2307260}
他的创业公式是:

\subsubsection{(1)搭建原型}
\label{sec:orga6ce0c7}
\subsubsection{(2)上线运营(不管bug)}
\label{sec:orgb1d20eb}
\subsubsection{(3)收集反馈}
\label{sec:org1592720}
\subsubsection{(4)调整产品}
\label{sec:orgc1f4898}
\subsubsection{{\bfseries\sffamily DONE} (5)成长壮大}
\label{sec:orged9da84}
鼓励创业公司快速发布产品,因为这样可以尽早知道一个创意是否可行。其次,认为一定要
特别关注用户需要什么,这样才有办法将一个苤项目转变成好项目。“许多伟大的公司,一
开始的时候做的都是与后来业务完全不同的事情。乔布斯创建苹果公司后的第一个计划是出
售计算机零件,然后让用户自己组装,后来才变成开发苹果电脑。你需要倾听用户的声音,
琢磨他们需要什么,然后去做。” 所有学员刚刚来到YC的时候,每人都会拿到一件白色恤
衫,上面写着“Make something people want”(制造用户需要的东西),等到他们的项目
行到风险投资以后,以会收到一件黑色T恤衫,上面写着“I made something people
want”(我制造了用户需要的东西)。
比起那些令人叫好的创意,格雷厄姆更看重创始人的素质。他说:“我们从一开始就认识到,
创始人本身比他的创意更重要。”他还认为,小团队更容易成功,创始成员总数最好不要超
过三个人。其中一个原因是,创始人越多,股权越不容易平等分配,容易造成内耗。

格雷厄姆认为,我们正在进入一个创业时代。未来的社会,创业可能成为一种常态,而替别
人打工反而成了少见的事情。一方面,创业是最有效的创造财富的方法,对创始人、对投资
者、对社会都如此。“如果拉里-佩奇(Larry Page)和谢尔盖-布林(Sergey Brin)没有
创立谷歌,那么他们可能还在某个研究部门工作,写一些不会有多少人使用的代码。但是,
他们选择了创业,想一想这样做为全世界增加了多少价值?”另一方面,创业越来越简单了,
成本也越来越低。“以前创业很昂贵,你不得不找到投资人才能创业。而现在,唯一的门槛
就是勇气。”

格雷厄姆认为,对于科技公司来说,未来充满了机会,前景一片光明。“所有东西都在变成
软件。印刷机诞生后,人类写过多少个字,未来就有多少家软件公司。”多少个字,未来就有多少家软件公司。”

\subsection{保罗-格雷厄姆其人其事}
\label{sec:orgad85f90}
1964年,保罗-格雷厄姆(Paul Graham)出生于匹兹堡郊区的一个中产阶级家庭。父亲是设
计核反应堆的物理学家,母亲在家照看他和他的妹妹。

青少年时代,格雷厄姆就开始编程。但是,他还喜欢许多与计算机无关的东西,这在编程高
手之中是很少见的。中学时,他喜欢写小说;进入康奈尔大学以后,他主修哲学。后来发现
哲学很难理解,于是研究生阶段他就去了哈佛大学计算机系,主攻人工智能。

他在这个方向上进展不顺利,因此对学术感到灰心。(但是,作为研究工具的Lisp语言,对
他日后产生了重大影响。)博士读到一半,他又去哈佛艺术系旁听。拿到博士学位以后,他
报名进入罗德岛设计学院暑期班,学习绘画课程,梦想成为画家。

上完暑期班,他去了欧洲,在有500年历史的伄罗伦萨美术学院继续学习绘画。第二年,钱
花完了,他不得不返回美国,在波士顿的一家创业公司中担任程序员。那时是1992年。

此后的两三年,格雷厄姆一直过着一种动荡的生活。他栖身于纽约的一间极小的公寓,追求
自己的艺术家梦想,但是收入低而且不稳定,日子过得非常窘迫,常常入不敷出,他不得不
经常替别人编程,赚取一些生活费。

\subsection{{\bfseries\sffamily DONE} 译者序}
\label{sec:org5d4f711}
你现在拿在手里的是一本非常重要、也非常独特的书。

它的作者是美国互联网llwj举足轻重、有“创业教父”之称的哈佛大学计算机博士保罗-格
雷厄姆(Paul Graham)。本书是他的文集。

书中的内容并不深奥,不仅仅是写给程序员和创业者的,更是写给普通读者的。作者最大的
目的就是,通过这本书让普通读者理解我们所处的这个计算机时代。

1968年至1972年期间,美国出版过一本叫做《地球商品目录》(Whole Earth Catalog)的
杂志,内容从植物种子到电子仪器,无所不包,出版目的据说是要帮助读者“理解整个系
统”。多年后,草果公司的总裁乔布斯盛赞它“有点像印刷版的谷歌”。从某种意义上说,
本书也是如此,本书也是如此,作者试图从许许多多不同的方面解释这个时代的内在脉络,
揭示它的发展轨迹,帮助你看清我们现在的位置和将来的方向。

电子技术的发展,使得计算机日益成为人类社会必不可少的一部分。

每个人日常生活的很大一部分都化在与计算机打交道上面。家用电表是智能的,通信网络是
程控的,信用卡是联网的,就连点菜都会用到电子菜单。越来越多的迹象表明,未来的人类
生活不仅是人与人的互动,而且更多的将是人与计算机的互动。

想要把握这个时代,就必须理解计算机。理解计算机的关键,则是要理解计算机背后的人。
表面上这是一个机器的时代,但是实际上机器的设计者决定了我们的时代。程序员的审美决
定了你看到的软件界面,程序员的爱好决定了你有什么样的软件可以使用。

我们的时代是程序员主导的时代,而伟大的程序员就是黑客。

本书就是帮助你了解黑客、从而理解这个时代的一把钥匙。

在媒体和普通人的眼里,“黑客”(Hacker)就是入侵计算机的人,就是“计算机犯罪”的
同义词。但是,这并不是它的真正含义(至少不是原意),更不是本书所使用的含义。

要想读懂这本书,首先就必须正确理解什么是“黑客”。

为了把这个问题说清楚,有必要从源头上讲起。1946年,第一台电子计算机ENIAC在美国诞
生,从此世界上一些最聪明、最有创造力的人开始进入这个行业,在他们身上逐渐地形成了
一种独特的技术文化。在这种文化的发展过程中,涌现了很多“行话”(jargon)。20世纪
60年代初,麻省理工学院有一个学生团体叫做“铁路模型技术俱乐部”(Tech Model
Railroad Club,简称TMRC),他们反难题的解决方法称为hack。
在这里,hack 作为名词有两个意思,既可以指很巧妙或很便捷的解决方法,也可以指比较
笨拙、不那么优雅的解决方法。两者都能称为hack,不同的是,前者是漂亮的解决方法
(cool hack或neat hack),后者是丑陋的解决方法(ugly hack 或quick hack)。hack
的字典解释是砍(木头),在这些学生看来,解决一个计算机难题就好像砍倒一棵大树。那
么相应地,完成这种hack的过程就被称为hacking,而从事hacking的人就是hacker, 也就
是黑客。

从这个意思出发,hack还有一个引申义,指对某个程序或设备进行个性,使其完成原来不可
用的功能(或者禁止外部使用者接触到的功能)。在这种意义上,hacking可以与盗窃信息、
信用卡欺诈或其他计算机犯罪联系在一起,这也是后来“黑客”被当作计算机入侵者的称呼
的原因。

但是,在20世纪60年代这个词被发明的时候,“黑客”完全是正面意义上的称呼。TMRC使用
这个词是带有敬意的,因为在他们看来,如果要完成一个hack,就必然包含着高度的革新、
独树一帜的风格、精湛的技艺。最能干的人会自豪了称自己为黑客。

这时,“黑客”这个词不仅是第一流能力的象征,还包含着求解问题过程中产生的精神愉悦
或享受。也就是说,从一开始,黑客就是有精神追求的。自由软件基金会创始人理查德-斯
托你尔曼说:“出于兴趣而解决某个难题,不管它有没有用,这就是黑客。”

根据理查德-斯托尔曼的说法,黑客行为必须包含三个特点:好玩、高智商、;探索精神。
只有其行为同时满足这三个标准,才能被称为“黑客”。另一方面,它们也构成了黑客的价
值观,黑客追求的就是这三种价值,而不是实用性或金钱。

1984年,《新闻周刊》的记者史蒂文-利维出版了历史上第一本介绍黑客的著作————《黑客:
计算机革命的英雄》(/Hackers: Heroes of the Computer Revolution/)。在该书中,他
进一步将黑客的价值观总结为六条“黑客伦理”(hacker ethic),直到今天这几条伦理都
被视为这方面的最佳论述。

\subsubsection{(1)使用计算机以及所有有助于这个世界本质的事物都不应受到任何限制。任何事情都应}
\label{sec:orgfafc680}
该亲手尝试。

(Access to computers \uline{\_ and anything that might teach you something about the
way the world works \_} should be unlimited and total. Always yield to the Hands
\_ On Imperative!)
\subsubsection{(2)信息应该全部免费。}
\label{sec:org671d667}

(All information should be free.)

\subsubsection{(3)不信任权威,提倡去中心化。}
\label{sec:org850f93b}

(Mistrust Authorth ————Promote Decentralization.)

\subsubsection{(4)判断一名黑客的水平应该看他的技术能力,而不是看他的学历、年龄或地位等其他标}
\label{sec:orge970ca3}
准。

(Hackers should be judged by their hacking, not bogus criteria such as degrees,
age, race, or position.)

\subsubsection{(5)你可以用计算机创造美和艺术。}
\label{sec:org053ee7e}

(You can create art and beauty on a computer.)

\subsubsection{(6)计算机使生活更美好。}
\label{sec:org6e2a9d6}

(Computers can change your life for the better.)

根据这六条“黑客伦理”,黑客价值观的核心原则可以概括成这样几点:分享、开放、民主、
计算机的自由使用、进步。

所以,“黑客”这个词的原始含义就是指那些信奉“黑客伦理”而且能力高超的程序员。历
史上一些最优秀的程序员都是“黑客”。除了上文提到的理查德-斯托尔曼,还包括Unix操
作系统创始人丹尼斯-里奇和肯—汤普森,经典巨著《计算机程序设计艺术》的作者、斯坦福
大学计算机教授高德纳,Linex 操作系统创始人莱纳斯-托湃兹,“开源运动”创始人埃里
克-雷蒙德,微软公司创始人比尔-盖茨等。正是黑客把计算机工业推向了更高的高度。

“黑客伦理”的一个必然推论就是,黑客不服从管教,具有叛逆精神。

黑客通常对管理者强加的、限制他们行为的愚蠢三分之一不屑一顾,会找出规避的方法。一
部分原因是为了自由使用计算机,另一部分原因是为了展现自己的聪明。比如,计算机设备
的各种安全措施就是最常被黑客破解的东西。史蒂夫·利维对这一点有过段生动的描述:

“对于黑客来说,关着的门就是一种挑衅,而锁着的门则是一种侮辱。……黑客相信,只有
有助于改进现状、探索未知,人们就应该被允许自由地使用各种工具和信息。当一个黑客需
要一样东西来帮助自己进行创造、进行探索或者进行修修补补时,他不会自找麻烦,不会接
受那些财产专有权的荒谬概念。”

这就是黑客有时会入侵计算机系统的原因,他们的主要目的并不是健儿别人的利益,这与那
些计算机罪犯是不同的。

1983年,一帮密尔沃基市的青少年黑客入侵了美国和加拿大的一些计算机系统,这件事被广
泛报道,同年9月5日的《酒瘾周刊》封面报道的标题就是“小心:黑客在行动”,这是历史
上主流媒体第一次使用“黑客”这个词。在报道的时候,媒体只注意和强调黑客行为一个很
窄的方面:入侵系统。(可能因为这种行为容易引起公众的注意,提升报道的关注度。)他
们反黑客简单定义主入侵系统、破坏安全设施的人。从此,大多数人对于黑客有了错误的看
法。同时,那些入侵计算机的程序员也自称“黑客”,使得这个问题进一步复杂化。

杂志、电视剧、电影、小说都对黑客的这种形象大肆渲染。黑客成了反社会的技术高手的代
名词,仿佛只要他坐在键盘前,就有一种从犯罪活动的魔力,可以操纵任何与网络相连的机
器,从核弹到车库大门,都在黑客敲打键盘的操作之中被控制。根据这种观点,黑客在最好
的情况下是一个没有认识到自己能力的清白的人,在最坏的情况下则是一个恐怖分子。在过
去几年中,随着计算机病毒的泛滥,黑客在大众心目中已经成了一个有害的人群。

那些传统意义上的黑客不认同这样使用“黑客”这个词。他们认为,历史上确实有一些正直
的黑客,为了亲自了解系统,做过违反法规的入侵举动。但是,那些人并没有恶意,而且从
一开始恶作剧就是黑客文化的一部分,仅仅由此推断入侵和破坏系统就是黑客文化的实质完
全是错误的。真正的黑客致力于改变世界,让世界运转得更好。媒体对黑客的定义未免过于
片面。

为了澄清“黑客”这个概念,他们提出只有传统意义上的黑客才能被称为hacker,而那些恶
意入侵计算机系统的人应该被称为cracker(入侵者)。这个观点已经在程序员社区中得到普
遍认同。
\subsection{{\bfseries\sffamily DONE} \href{https://learnxinyminutes.com/docs/elisp/}{Learn lisp.el}}
\label{sec:org3e84b9d}
\lstset{language=Lisp,label= ,caption= ,captionpos=b,numbers=none}
\begin{lstlisting}

  ;; This gives an introduction to Emacs Lisp in 15 minutes (v0.2d)
  ;;
  ;; First make sure you read this text by Peter Norvig:
  ;; http://norvig.com/21-days.html
  ;;
  ;; Then install GNU Emacs 24.3:
  ;;
  ;; Debian: apt-get install emacs (or see your distro instructions)
  ;; OSX: http://emacsformacosx.com/emacs-builds/Emacs-24.3-universal-10.6.8.dmg
  ;; Windows: http://ftp.gnu.org/gnu/windows/emacs/emacs-24.3-bin-i386.zip
  ;;
  ;; More general information can be found at:
  ;; http://www.gnu.org/software/emacs/#Obtaining

  ;; Important warning:
  ;;
  ;; Going through this tutorial won't damage your computer unless
  ;; you get so angry that you throw it on the floor.  In that case,
  ;; I hereby decline any responsibility.  Have fun!

  ;;;;;;;;;;;;;;;;;;;;;;;;;;;;;;;;;;;;;;;;;;;;;;;;;;;;;;;;;;;;;;;;;;;;;;;;
  ;;
  ;; Fire up Emacs.
  ;;
  ;; Hit the `q' key to dismiss the welcome message.
  ;;
  ;; Now look at the gray line at the bottom of the window:
  ;;
  ;; "*scratch*" is the name of the editing space you are now in.
  ;; This editing space is called a "buffer".
  ;;
  ;; The scratch buffer is the default buffer when opening Emacs.
  ;; You are never editing files: you are editing buffers that you
  ;; can save to a file.
  ;;
  ;; "Lisp interaction" refers to a set of commands available here.
  ;;
  ;; Emacs has a built-in set of commands available in every buffer,
  ;; and several subsets of commands available when you activate a
  ;; specific mode.  Here we use the `lisp-interaction-mode', which
  ;; comes with commands to evaluate and navigate within Elisp code.

  ;;;;;;;;;;;;;;;;;;;;;;;;;;;;;;;;;;;;;;;;;;;;;;;;;;;;;;;;;;;;;;;;;;;;;;;;
  ;;
  ;; Semi-colons start comments anywhere on a line.
  ;;
  ;; Elisp programs are made of symbolic expressions ("sexps"):
(+ 2 2)


  ;; This symbolic expression reads as "Add 2 to 2".

  ;; Sexps are enclosed into parentheses, possibly nested:
  (+ 2 (+ 1 1))

  ;; A symbolic expression contains atoms or other symbolic
  ;; expressions.  In the above examples, 1 and 2 are atoms,
  ;; (+ 2 (+ 1 1)) and (+ 1 1) are symbolic expressions.

  ;; From `lisp-interaction-mode' you can evaluate sexps.
  ;; Put the cursor right after the closing parenthesis then
  ;; hold down the control and hit the j keys ("C-j" for short).

  (+ 3 (+ 1 2))

  ;;           ^ cursor here
  ;; `C-j' => 6

  ;; `C-j' inserts the result of the evaluation in the buffer.

  ;; `C-xC-e' displays the same result in Emacs bottom line,
  ;; called the "minibuffer".  We will generally use `C-xC-e',
  ;; as we don't want to clutter the buffer with useless text.

  ;; `setq' stores a value into a variable:
(setq my-name "Bastien")

  ;; `C-xC-e' => "Bastien" (displayed in the mini-buffer)

  ;; `insert' will insert "Hello!" where the cursor is:
  (insert "Hello!")Hello!
  ;; `C-xC-e' => "Hello!"

  ;; We used `insert' with only one argument "Hello!", but
  ;; we can pass more arguments -- here we use two:

  (insert "Hello" " world!")Hello world!
  ;; `C-xC-e' => "Hello world!"

  ;; You can use variables instead of strings:
  (insert "Hello, I am " my-name)Hello, I am Bastien
  ;; `C-xC-e' => "Hello, I am Bastien"

  ;; You can combine sexps into functions:
  (defun hello () (insert "Hello, I am " my-name))
  ;; `C-xC-e' => hello

  ;; You can evaluate functions:
  (hello)Hello, I am Bastien
  ;; `C-xC-e' => Hello, I am Bastien

  ;; The empty parentheses in the function's definition means that
  ;; it does not accept arguments.  But always using `my-name' is
  ;; boring, let's tell the function to accept one argument (here
  ;; the argument is called "name"):

  (defun hello (name) (insert "Hello " name))
  ;; `C-xC-e' => hello

  ;; Now let's call the function with the string "you" as the value
  ;; for its unique argument:
  (hello "you")Hello you
  ;; `C-xC-e' => "Hello you"

  ;; Yeah!

  ;; Take a breath.

  ;;;;;;;;;;;;;;;;;;;;;;;;;;;;;;;;;;;;;;;;;;;;;;;;;;;;;;;;;;;;;;;;;;;;;;;;
  ;;
  ;; Now switch to a new buffer named "*test*" in another window:

  (switch-to-buffer-other-window "*test*")
  ;; `C-xC-e'
  ;; => [screen has two windows and cursor is in the *test* buffer]

  ;; Mouse over the top window and left-click to go back.  Or you can
  ;; use `C-xo' (i.e. hold down control-x and hit o) to go to the other
  ;; window interactively.

  ;; You can combine several sexps with `progn':
  (progn
    (switch-to-buffer-other-window "*test*")
    (hello "you"))
  ;; `C-xC-e'
  ;; => [The screen has two windows and cursor is in the *test* buffer]

  ;; Now if you don't mind, I'll stop asking you to hit `C-xC-e': do it
  ;; for every sexp that follows.

  ;; Always go back to the *scratch* buffer with the mouse or `C-xo'.

  ;; It's often useful to erase the buffer:
  (progn
    (switch-to-buffer-other-window "*test*")
    (erase-buffer)
    (hello "there"))

  ;; Or to go back to the other window:
  (progn
    (switch-to-buffer-other-window "*test*")
    (erase-buffer)
    (hello "you")
    (other-window 1))

  ;; You can bind a value to a local variable with `let':
  (let ((local-name "you"))
    (switch-to-buffer-other-window "*test*")
    (erase-buffer)
    (hello local-name)
    (other-window 1))

  ;; No need to use `progn' in that case, since `let' also combines
  ;; several sexps.

  ;; Let's format a string:
  (format "Hello %s!\n" "visitor")

  ;; %s is a place-holder for a string, replaced by "visitor".
  ;; \n is the newline character.

  ;; Let's refine our function by using format:
  (defun hello (name)
    (insert (format "Hello %s!\n" name)))

  (hello "you")Hello you

  ;; Let's create another function which uses `let':
  (defun greeting (name)
    (let ((your-name "Bastien"))
      (insert (format "Hello %s!\n\nI am %s."
                      name       ; the argument of the function
                      your-name  ; the let-bound variable "Bastien"
                      ))))

  ;; And evaluate it:
  (greeting "you")Hello you!

I am Bastien.

  ;; Some functions are interactive:
  (read-from-minibuffer "Enter your name: ")

  ;; Evaluating this function returns what you entered at the prompt.

  ;; Let's make our `greeting' function prompt for your name:
  (defun greeting (from-name)
    (let ((your-name (read-from-minibuffer "Enter your name: ")))
      (insert (format "Hello!\n\nI am %s and you are %s."
                      from-name ; the argument of the function
                      your-name ; the let-bound var, entered at prompt
                      ))))

  (greeting "Bastien")Hello Bastien!

I am Bastien.

  ;; Let's complete it by displaying the results in the other window:
  (defun greeting (from-name)
    (let ((your-name (read-from-minibuffer "Enter your name: ")))
      (switch-to-buffer-other-window "*test*")
      (erase-buffer)
      (insert (format "Hello %s!\n\nI am %s." your-name from-name))
      (other-window 1)))

  ;; Now test it:
  (greeting "Bastien")

  ;; Take a breath.

  ;;;;;;;;;;;;;;;;;;;;;;;;;;;;;;;;;;;;;;;;;;;;;;;;;;;;;;;;;;;;;;;;;;;;;;;;
  ;;
  ;; Let's store a list of names:
  ;; If you want to create a literal list of data, use ' to stop it from
  ;; being evaluated - literally, "quote" the data.
  (setq list-of-names '("Sarah" "Chloe" "Mathilde"))

  ;; Get the first element of this list with `car':
  (car list-of-names)

  ;; Get a list of all but the first element with `cdr':
  (cdr list-of-names)

  ;; Add an element to the beginning of a list with `push':
  (push "Stephanie" list-of-names)

  ;; NOTE: `car' and `cdr' don't modify the list, but `push' does.
  ;; This is an important difference: some functions don't have any
  ;; side-effects (like `car') while others have (like `push').

  ;; Let's call `hello' for each element in `list-of-names':
  (mapcar 'hello list-of-names)Hello SarahHello ChloeHello Mathilde

  ;; Refine `greeting' to say hello to everyone in `list-of-names':
  (defun greeting ()
      (switch-to-buffer-other-window "*test*")
      (erase-buffer)
      (mapcar 'hello list-of-names)
      (other-window 1))

  (greeting)

  ;; Remember the `hello' function we defined above?  It takes one
  ;; argument, a name.  `mapcar' calls `hello', successively using each
  ;; element of `list-of-names' as the argument for `hello'.

  ;; Now let's arrange a bit what we have in the displayed buffer:

  (defun replace-hello-by-bonjour ()
      (switch-to-buffer-other-window "*test*")
      (goto-char (point-min))
      (while (search-forward "Hello")
        (replace-match "Bonjour"))
      (other-window 1))

  ;; (goto-char (point-min)) goes to the beginning of the buffer.
  ;; (search-forward "Hello") searches for the string "Hello".
  ;; (while x y) evaluates the y sexp(s) while x returns something.
  ;; If x returns `nil' (nothing), we exit the while loop.

  (replace-hello-by-bonjour)

  ;; You should see all occurrences of "Hello" in the *test* buffer
  ;; replaced by "Bonjour".

  ;; You should also get an error: "Search failed: Hello".
  ;;
  ;; To avoid this error, you need to tell `search-forward' whether it
  ;; should stop searching at some point in the buffer, and whether it
  ;; should silently fail when nothing is found:

  ;; (search-forward "Hello" nil 't) does the trick:

  ;; The `nil' argument says: the search is not bound to a position.
  ;; The `'t' argument says: silently fail when nothing is found.

  ;; We use this sexp in the function below, which doesn't throw an error:

  (defun hello-to-bonjour ()
      (switch-to-buffer-other-window "*test*")
      (erase-buffer)
      ;; Say hello to names in `list-of-names'
      (mapcar 'hello list-of-names)
      (goto-char (point-min))
      ;; Replace "Hello" by "Bonjour"
      (while (search-forward "Hello" nil 't)
        (replace-match "Bonjour"))
      (other-window 1))

  (hello-to-bonjour)

  ;; Let's boldify the names:

  (defun boldify-names ()
      (switch-to-buffer-other-window "*test*")
      (goto-char (point-min))
      (while (re-search-forward "Bonjour \\(.+\\)!" nil 't)
        (add-text-properties (match-beginning 1)
                             (match-end 1)
                             (list 'face 'bold)))
      (other-window 1))

  ;; This functions introduces `re-search-forward': instead of
  ;; searching for the string "Bonjour", you search for a pattern,
  ;; using a "regular expression" (abbreviated in the prefix "re-").

  ;; The regular expression is "Bonjour \\(.+\\)!" and it reads:
  ;; the string "Bonjour ", and
  ;; a group of            | this is the \\( ... \\) construct
  ;;   any character       | this is the .
  ;;   possibly repeated   | this is the +
  ;; and the "!" string.

  ;; Ready?  Test it!

  (boldify-names)

  ;; `add-text-properties' adds... text properties, like a face.

  ;; OK, we are done.  Happy hacking!

  ;; If you want to know more about a variable or a function:
  ;;
  ;; C-h v a-variable RET
  ;; C-h f a-function RET
  ;;
  ;; To read the Emacs Lisp manual with Emacs:
  ;;
  ;; C-h i m elisp RET
  ;;
  ;; To read an online introduction to Emacs Lisp:
  ;; https://www.gnu.org/software/emacs/manual/html_node/eintr/index.html
\end{lstlisting}


\subsection{\href{http://www.ituring.com.cn/book/miniarticle/53201}{On Lisp}}
\label{sec:org981e042}
\subsection{{\bfseries\sffamily DONE} \href{http://deathking.github.io/yast-cn/}{Scheme}}
\label{sec:orgb45acb0}

\section{时事政治学习}
\label{sec:org53d963c}

\subsection{{\bfseries\sffamily DONE} \href{http://news.ifeng.com/a/20171029/52843046\_0.shtml}{十九大后,习近平对军队高级干部提出6个“必须”}}
\label{sec:org3204cc9}
原标题:十九大后,习近平对军队高级干部提出6个“必须”

新华网记者 王子晖

【学习进行时】党的十九届一中全会决定了新一届中央军事委员会组成人员。26日,习近平
出席军队领导干部会议并发表重要讲话,对全军学习贯彻党的十九大精神和新时代党的强军
思想作出动员和部署,同时对军队高级干部提出了新要求。新华社《学习进行时》原创品牌
栏目“讲习所”今天推出文章,为您解读。

10月25日,党的十九届一中全会决定了中央军事委员会组成人员,习近平任中央军事委员会
主席。

26日,习近平出席军队领导干部会议并发表重要讲话,为全军学习贯彻党的十九大精神和新
时代党的强军思想作出动员和部署,同时对军队高级干部提出了新要求。

习近平强调,*我军高级干部必须对党忠诚、听党指挥,必须善谋打仗、能打胜仗,必须锐意
改革、勇于创新,必须科学统筹、科学管理,必须厉行法治、从严治军,必须作风过硬、作
出表率,以饱满的精神状态和奋斗姿态为党工作,忠实履行好职责。*

贯彻新时代党的强军思想从高级干部抓起

十八大以来的5年,是党和国家发展进程中极不平凡的5年,也是我军发展进程中极不平凡的5年。

“恢复了一些带根本性的东西,破解了一些深层次矛盾,取得了一些开创性成果”,纵观5
年来国防和军队建设的一系列历史性成就,贯穿的主线就是新时代党的强军思想。

坚持政治建军、改革强军、科技兴军、依法治军,更加注重聚焦实战,更加注重创新驱动,
更加注重体系建设,更加注重集约高效,更加注重军民融合,5年来,“国防和军队建设上
了一个大台阶”,在这一历史性进程中,新时代党的强军思想始终发挥着引领作用。

党的十九大为实现党在新时代的强军目标划定了时间表,提出必须全面贯彻党领导人民军队
的一系列根本原则和制度,确立新时代党的强军思想在国防和军队建设中的指导地位。

习近平强调,“我军高级干部是强军事业的中坚力量,身上千钧重担,身后千军万马”。确
立新时代党的强军思想在国防和军队建设中的指导地位,必须从军队高级干部抓起。

习近平提出的这6个“必须”从政治素质、实战能力、改革精神、管理水平、法治思维以及
工作作风6个方面提出要求,正与新时代党的强军思想一脉相承。

如此强调,就是指明了军队高级干部贯彻新时代党的强军思想的关键和带兵治军的根本遵循。
加上“必须”二字,体现的就是严字当头、一严到底。只有不折不扣地做到这6项要求,才
能“以饱满的精神状态和奋斗姿态为党工作,忠实履行好职责”。

“必须对党忠诚、听党指挥”是前提和总领

6个“必须”中“必须对党忠诚、听党指挥”排第一,体现前提、总领之意。

“政治建军是我军的立军之本”,习近平不只一次这样强调。人民军队从革命战争年代走到
今天,能够一直保持血性、保持本色,靠的就是坚强有力的政治工作。“党指挥枪”、坚持
党对军队绝对领导无疑是强军之魂。

2014年召开的古田全军政治工作会议上,习近平强调要把理想信念、党性原则、战斗力标准、
政治工作威信这4个“带根本性的东西”立起来。几年来,我党我军光荣传统和优良作风得
到恢复和发扬,人民军队政治生态得到有效治理。

迈进新时代的人民军队要不忘初心、牢记使命,最根本的还是要听党指挥,军队政治工作只
能加强不能削弱,只能前进不能停滞,只能积极作为不能被动应对。

正因如此,党的十九大重申,必须全面贯彻党领导人民军队的一系列根本原则和制度,“坚
持党对人民军队的绝对领导”作为新时代坚持和发展中国特色社会主义的一条基本方略突出
写进十九大报告。

军队领导干部会议上,习近平就此作出强力部署,强调要加强思想政治建设,围绕学习贯彻
党的十九大精神,深入贯彻古田全军政治工作会议精神,从严从紧加强军队党的建设,保持
政治整训劲头和力度。

军队高级干部是军队中的“关键少数”,政治工作首先要从高级干部做起,只有高级干部对
党绝对忠诚,坚决听党指挥,才能确保党在任何时候都能牢牢掌握军队,在中国特色强军之
路上始终步稳蹄疾。

能打仗、打胜仗是中心思想和最终目的

*战场打不赢,一切等于零。*党的十九大明确指出,军队是要准备打仗的,一切工作都必须坚
持战斗力标准,向能打仗、打胜仗聚焦。

对于军队高级干部来说,就是要不断提高谋划打仗、指挥打仗、带兵打仗能力。

习近平提出的6个“必须”,每一项都与打仗息息相关,中心思想和最终目的就是确保人民
军队招之能战,战之必胜。

其中,“必须对党忠诚、听党指挥”是根本保证,“必须善谋打仗、能打胜仗”是直接要求,
其他四项则是基于国家安全环境的深刻变化和新形势下军事战略方针提出的充满时代性和战
略性的重要课题。

军队高级干部能否做到“锐意改革、勇于创新”,事关继续深化国防和军队改革;能否做到
“科学统筹、科学管理”,事关推进军事管理革命;能否做到“厉行法治、从严治军”,事
关推动治军方式根本性转变;能否做到“作风过硬、作出表率”,事关全军将士的精神和意
志。这些无一不是实现国防和军队现代化,全面建成世界一流军队的重要因素。

“身上千钧重担,身后千军万马”,军队高级干部只有紧紧遵循6个“必须”,才能适应世
界新军事革命发展趋势和国家安全需求,让人民军队担当起党和人民赋予的新时代使命任务。

\subsection{凤凰网}
\label{sec:org10cfa56}
\url{http://www.ifeng.com}
\subsection{习近平关心的“政治能力”是什么?}
\label{sec:org1f1e8e2}
\url{http://news.ifeng.com/a/20170429/51026634\_0.shtml}
\subsection{{\bfseries\sffamily DONE} 原标题:【干部必读】党报刊文谈“政治能力”,信息量极大}
\label{sec:org4bce716}
【学习小组按】

“提高政治能力”是习近平关心的重大命题之一。

习近平在省部级主要领导干部学习贯彻十八届六中全会精神专题研讨班开班式上发表重要讲话,明确提出“注重提高政治能力”的重大命题。

但,一般人说不清楚:什么是“政治能力”,怎么拥有“政治能力”,如何提高“政治能力”?

近日,《人民日报》刊登武警部队政治委员朱生岭的《着力提高与履行领导职责相匹配的政治能力(深入学习贯彻习近平同志系列重要讲话精神)》一文,该文从充分认识提高政治能力的重大意义、正确理解提高政治能力的丰富内涵、科学把握提高政治能力的方法途径三个方面,系统论述了领导干部,尤其是军队领导干部,提高、履行与领导职责相匹配的政治能力的意义和方法。

\subsubsection{一、充分认识提高政治能力的重大意义}
\label{sec:org5b8f526}

\begin{itemize}
\item 驾驭复杂局面、维护国家政治安全需要提高政治能力。

\item 全面从严治党、巩固党的执政地位需要提高政治能力。

\item 推进部队建设发展、实现强军目标需要提高政治能力。
\end{itemize}

\subsubsection{二、正确理解提高政治能力的丰富内涵}
\label{sec:org4143b9f}

内涵要义。从内涵要义看,领导干部的政治能力主要是指其运用政治知识和政治经验从事政治活动并取得政治绩效的能力,主要包括:

把握方向、把握大势、把握全局的能力

保持政治定力、驾驭政治局面、防范政治风险的能力。

岗位要求。从岗位要求看,领导干部需要具备与其任职岗位相匹配的政治能力,即:

\begin{itemize}
\item 牢固树立政治理想,

\item 正确把握政治方向,

\item 坚定站稳政治立场,

\item 严格遵守政治纪律。
\end{itemize}

衡量标准。

从衡量标准看,领导干部要具备过硬政治能力,就要做到:

\begin{itemize}
\item 政治意识敏锐、

\item 政治态度鲜明、

\item 政治定力坚强、

\item 政治操守坚定、

\item 政治担当果敢、

\item 政治考验合格。
\end{itemize}

\subsubsection{三、科学把握提高政治能力的方法途径}
\label{sec:org3529ced}

\begin{enumerate}
\item 在深化政治理论学习中修炼。

\item 在落实最高政治要求中锻炼。

\item 在严肃党内政治生活中锤炼。

\item 在防范各种政治风险中磨炼。

\item 在执行重大政治任务中历练。

\item 这篇文章可谓当前国内谈“政治能力”最为系统的文章之一,学习小组推荐阅读。
\end{enumerate}
\subsection{{\bfseries\sffamily DONE} 习近平:在知识分子、劳模、青年代表座谈会上的讲话}
\label{sec:orgafe54af}
\url{http://news.ifeng.com/a/20160430/48641425\_0.shtml}
4月26日,中共中央总书记、国家主席、中央军委主席习近平在安徽合肥主持召开知识分子、
劳动模范、青年代表座谈会并发表重要讲话。新华社记者李涛摄


核心提示:习近平近日在安徽合肥主持召开知识分子代表座谈会,强调各级党委和政府以及
各级领导干部要就工作和决策中的有关问题主动征求他们的意见和建议,欢迎他们提出批评。
对来自知识分子的意见和批评,只要出发点是好的,就要热忱欢迎,对的就要积极采纳;即
使一些意见和批评有偏差,甚至不正确,也要多一些包容、多一些宽容,坚持不抓辫子、不
扣帽子、不打棍子。

原标题:习近平:在知识分子、劳动模范、青年代表座谈会上的讲话

(授权发布)习近平:在知识分子、劳动模范、青年代表座谈会上的讲话

新华社合肥4月30日电

在知识分子、劳动模范、青年代表座谈会上的讲话

(2016年4月26日)

习近平

大家好!我这次来安徽调研,正好是“五一”国际劳动节、“五四”青年节前夕。今天,我
们在这里召开一个座谈会,请一些知识分子、劳动模范、青年代表来座谈,主要是想当面听
听大家的意见和建议,号召广大知识分子、广大劳动群众、广大青年共同为全面建成小康社
会而奋斗,并以此纪念即将到来的“五一”国际劳动节、“五四”青年节。

首先,我代表党中央,向在座各位,并通过你们,向全国广大知识分子、广大劳动群众、广
大青年,致以诚挚的问候和节日的祝贺!

刚才,几位同志的发言,结合自己的学习和工作,谈认识、谈感受、提建议,很生动、很朴
实、很有见地,听后很受鼓舞、很受启发。

今天这个座谈会,请来的是知识分子、劳动模范、青年代表。这样安排,我们是有考虑的。
我国是工人阶级领导的、以工农联盟为基础的人民民主专政的社会主义国家。知识分子是工
人阶级的一部分,劳动人民是国家的主人,青年是中国特色社会主义事业接班人、是国家的
未来和民族的希望。我们要全面建成小康社会,进而建成富强民主文明和谐的社会主义现代
化国家,实现中华民族伟大复兴,必须依靠知识,必须依靠劳动,必须依靠广大青年。这是
我们国家和民族发展的力量所在,也是我们事业成功的力量所在。

党的十八大以来,每年“五一”国际劳动节、“五四”青年节,我都参加相关活动,也讲过
一些话。就知识分子工作,我也在不同场合讲过一些意见。我的有关讲话归结起来,核心意
思就是:经过近代以来特别是中国共产党诞生以来中国人民持续奋斗,中华民族伟大复兴已
经展现出光明前景,现在我们比历史上任何时期都更接近中华民族伟大复兴的目标,比历史
上任何时期都更有信心、更有能力实现这个目标。同时,实现中华民族伟大复兴还有很长的
路要走,前进道路并不平坦,必须坚定中国特色社会主义道路自信、理论自信、制度自信,
随时准备应对各种困难和挑战,无论遇到什么风浪我们都不能停下前进步伐;实现中华民族
伟大复兴是十分伟大而又十分艰巨的事业,需要全体中华儿女众志成城、万众一心,把一切
力量都凝聚起来,把一切积极因素都调动起来,为了共同的目标不懈奋斗。

我们正处在实现“两个一百年”奋斗目标中第一个一百年奋斗目标、全面建成小康社会的决
胜阶段。党的十八届五中全会和“十三五”规划纲要,描绘了全面建成小康社会宏伟蓝图。
现在,摆在我们面前的任务是把美好蓝图变为现实。广大知识分子、广大劳动群众、广大青
年要紧跟时代、肩负使命、锐意进取,把自身的前途命运同国家和民族的前途命运紧紧联系
在一起,努力为全面建成小康社会贡献智慧和力量。

这里,我就新形势下进一步发挥广大知识分子、广大劳动群众、广大青年的作用讲一些意见。

全面建成小康社会,我国广大知识分子能够提供十分重要的人才支撑、智力支撑、创新支撑。
希望我国广大知识分子充分发挥自身优势,勇于担当、敢于创新,服务社会、报效人民,努
力作出新的更突出的贡献。

知识分子,顾名思义,就是文化水平较高、知识比较丰富的人,其中不少是学有所长、术有
专攻、在某个领域某个方面的行家专家。知识分子对知识、对技术掌握得比较多,对自然、
对社会了解得比较深,在推动经济社会发展、推动社会文明进步中能够发挥十分重要的作用。
在我们党领导革命、建设、改革90多年的历程中,广大知识分子为党和人民建立了彪炳史册
的功勋。

伴随党和人民事业不断发展,我国知识分子队伍越来越大,遍布全社会各个领域。在全面建
成小康社会进程中,广大知识分子要肩负起自己的使命,立足岗位、不断学习、学以致用,
做好本职工作。当老师,就要心无旁骛,甘守三尺讲台,“春蚕到死丝方尽,蜡炬成灰泪始
干”。做研究,就要甘于寂寞,或是皓首穷经,或是扎根实验室,“板凳要坐十年冷,文章
不写一句空”。搞创作,就要坚持以人民为中心的创作思想,深入实践、深入群众、深入生
活,努力创作出人民群众喜爱的精品力作。一个知识分子,不论在哪个行业、从事什么职业,
也不论学历、职称、地位有多高,唯有秉持求真务实精神,才能探究更多未知,才能获得更
多真理,也才能为社会作出更大贡献。

勇立潮头、引领创新,是广大知识分子应有的品格。面对日益激烈的国际竞争,我们必须把
创新摆在国家发展全局的核心位置,不断推进理论创新、制度创新、科技创新、文化创新等
各方面创新。广大知识分子要增强创新意识,敢于走前人没有走过的路,敢于抢占国内国际
创新制高点。要把握创新特点,遵循创新规律,既奇思妙想、“无中生有”,努力追求原始
创新,又兼收并蓄、博采众长,善于进行集成创新和引进消化吸收再创新;既甘于“十年磨
一剑”,开展战略性创新攻关,又对接现实需求,及时开展应急性创新攻关;既尊重个人创
造,发挥尖兵作用,又注重集体攻关,发挥合作优势。要坚持面向经济社会发展主战场、面
向人民群众新需求,让创新成果更多更快造福社会、造福人民。

天下为公、担当道义,是广大知识分子应有的情怀。我国知识分子历来有浓厚的家国情怀,
有强烈的社会责任感。“修身齐家治国平天下”,“为天地立心、为生民立命、为往圣继绝
学、为万世开太平”,“先天下之忧而忧,后天下之乐而乐”,这些思想为一代又一代知识
分子所尊崇。现在,党和人民更加需要广大知识分子发扬这样的担当精神。这是一份沉甸甸
的责任。广大知识分子要坚持国家至上、民族至上、人民至上,始终胸怀大局、心有大我。
要坚守正道、追求真理,立足我国国情,放眼观察世界,不妄自菲薄,不人云亦云。要实事
求是、客观公允,重实情、看本质、建真言,多为推进党和人民事业发展献计出力。任何时
候任何情况下,都不能做有损国家民族尊严、有损知识分子良知的事。

知识分子工作是党的一项十分重要的工作。各级党委和政府要切实尊重知识、尊重人才,充
分信任知识分子,努力为广大知识分子工作学习生活创造更好条件。要深化科技、教育、文
化体制改革,深化人才发展体制改革,加快形成有利于知识分子干事创业的体制机制,放手
让广大知识分子把才华和能量充分释放出来。要遵循知识分子工作特点和规律,减少对知识
分子创造性劳动的干扰,让他们把更多精力集中于本职工作。要善于运用沟通、协商、谈心
等方式做好知识分子思想工作,多了解他们工作学习生活中的困难,多同他们共同探讨一些
问题,多鼓励他们取得的成绩和进步。

知识分子有思想、有主见、有责任,愿意对一些问题发表自己的见解。各级党委和政府、各
级领导干部要就工作和决策中的有关问题主动征求他们的意见和建议,欢迎他们提出批评。
对来自知识分子的意见和批评,只要出发点是好的,就要热忱欢迎,对的就要积极采纳。即
使一些意见和批评有偏差,甚至不正确,也要多一些包容、多一些宽容,坚持不抓辫子、不
扣帽子、不打棍子。人不是神仙,提意见、提批评不能要求百分之百正确。如果有的人提出
的意见和批评不妥当或者是错误的,要开展充分的说理工作,引导他们端正认识、转变观点,
而不要一下子就把人看死了,更不要回避他们、排斥他们。各级领导干部要善于同知识分子
打交道,做知识分子的挚友、诤友。

全面建成小康社会,我国亿万劳动群众是主体力量。希望我国广大劳动群众以劳动模范为榜
样,爱岗敬业、勤奋工作,锐意进取、勇于创造,不断谱写新时代的劳动者之歌。

“人生在勤,勤则不匮。”幸福不会从天降,美好生活靠劳动创造。全面建成小康社会的奋
斗目标,为广大劳动群众指明了光明的未来;全面建成小康社会的历史任务,为广大劳动群
众赋予了光荣的使命;全面建成小康社会的伟大征程,为广大劳动群众提供了宝贵的机遇。
面对这样一个千帆竞发、百舸争流、有机会干事业、能干成事业的时代,广大劳动群众一定
要倍加珍惜、倍加努力。

劳动模范是劳动群众的杰出代表,是最美的劳动者。劳动模范身上体现的“爱岗敬业、争创
一流,艰苦奋斗、勇于创新,淡泊名利、甘于奉献”的劳模精神,是伟大时代精神的生动体
现。我们要在全社会大力宣传劳动模范的先进事迹,号召全社会向他们学习、向他们致敬。
要为劳动模范更好施展才华、展现精神品格提供全方位支持,使他们的劳动技能、创新方法、
管理经验能广泛传播,充分发挥示范带动作用。劳动模范要珍惜荣誉、谦虚谨慎、再接再厉,
不断在新的起点上为党和人民创造更大业绩。

素质是立身之基,技能是立业之本。广大劳动群众要勤于学习,学文化、学科学、学技能、
学各方面知识,不断提高综合素质,练就过硬本领。要立足岗位学,向师傅学,向同事学,
向书本学,向实践学。三百六十行,行行出状元。任何一名劳动者,无论从事的劳动技术含
量如何,只要勤于学习、善于实践,在工作上兢兢业业、精益求精,就一定能够造就闪光的
人生。

人类是劳动创造的,社会是劳动创造的。劳动没有高低贵贱之分,任何一份职业都很光荣。
广大劳动群众要立足本职岗位诚实劳动。无论从事什么劳动,都要干一行、爱一行、钻一行。
在工厂车间,就要弘扬“工匠精神”,精心打磨每一个零部件,生产优质的产品。在田间地
头,就要精心耕作,努力赢得丰收。在商场店铺,就要笑迎天下客,童叟无欺,提供优质的
服务。只要踏实劳动、勤勉劳动,在平凡岗位上也能干出不平凡的业绩。

梦想属于每一个人,广大劳动群众要敢想敢干、敢于追梦。说到底,实现中华民族伟大复兴
的中国梦,要靠各行各业人们的辛勤劳动。现在,党和国家事业空间很大,只要有志气有闯
劲,普通劳动者也可以在宽广舞台上展示自己的人生价值。许多劳动模范平凡而感人的事迹,
都充分说明了这一点。我们要在全社会大力弘扬劳动精神,提倡通过诚实劳动来实现人生的
梦想、改变自己的命运,反对一切不劳而获、投机取巧、贪图享乐的思想。

各级党委和政府要关心和爱护广大劳动群众,切实把党和国家相关政策措施落实到位,不断
推进相关领域改革创新,坚决扫除制约广大劳动群众就业创业的体制机制和政策障碍,不断
完善就业创业扶持政策、降低就业创业成本,支持广大劳动群众积极就业、大胆创业。要切
实维护广大劳动群众合法权益,帮助广大劳动群众排忧解难,积极构建和谐劳动关系。

现在,我国经济发展进入新常态,经济发展方式正在深刻转变,经济结构正在深刻调整,这
对部分劳动群众就业带来了暂时的影响。各级党委和政府要落实好失业人员再就业和生活保
障、财政专项奖补等支持政策,落实和完善援助措施,创造更多就业岗位,通过鼓励企业吸
纳、公益性岗位安置、社会政策托底等多种渠道帮助就业困难人员,实现零就业家庭动态
“清零”,确保安置分流有序、社会和谐稳定。

全面建成小康社会,广大青年是生力军和突击队。希望我国广大青年充分展现自己的抱负和
激情,胸怀理想、锤炼品格,脚踏实地、艰苦奋斗,不断书写奉献青春的时代篇章。

实现中华民族伟大复兴的中国梦,需要一代又一代有志青年接续奋斗。青年人朝气蓬勃,是
全社会最富有活力、最具有创造性的群体。党和人民对广大青年寄予厚望。

广大青年要自觉践行社会主义核心价值观,不断养成高尚品格。要以国家富强、人民幸福为
己任,胸怀理想、志存高远,投身中国特色社会主义伟大实践,并为之终生奋斗。要加强思
想道德修养,自觉弘扬爱国主义、集体主义精神,自觉遵守社会公德、职业道德、家庭美德。
要坚持艰苦奋斗,不贪图安逸,不惧怕困难,不怨天尤人,依靠勤劳和汗水开辟人生和事业
前程。“看似寻常最奇崛,成如容易却艰辛。”青年的人生之路很长,前进途中,有平川也
有高山,有缓流也有险滩,有丽日也有风雨,有喜悦也有哀伤。心中有阳光,脚下有力量,
为了理想能坚持、不懈怠,才能创造无愧于时代的人生。

“人才有高下,知物由学。”梦想从学习开始,事业靠本领成就。广大青年要自觉加强学习,
不断增强本领。人生的黄金时期在青年。青年时期学识基础厚实不厚实,影响甚至决定自己
的一生。广大青年要如饥似渴、孜孜不倦学习,既多读有字之书,也多读无字之书,注重学
习人生经验和社会知识。“纸上得来终觉浅,绝知此事要躬行。”所有知识要转化为能力,
都必须躬身实践。要坚持知行合一,注重在实践中学真知、悟真谛,加强磨练、增长本领。

广大青年要自觉奉献青春,为全面建成小康社会多作贡献。青年时光非常可贵,要用来干事
创业、辛勤耕耘,为将来留下珍贵的回忆。广大农村青年要在发展现代农业、建设社会主义
新农村中展现现代农民新形象,广大企业青年要在积极参与生产劳动、产品研发、管理创新
中创造更多财富,广大科研单位青年要在深入钻研学问、主动攻克难题中多出创新成果,广
大机关事业单位青年要在提高为社会、为民众服务水平中建功立业。

广大青年要保持初生牛犊不怕虎的劲头,不懂就学,不会就练,没有条件就努力创造条件。
“志之所趋,无远弗届,穷山距海,不能限也。”对想做爱做的事要敢试敢为,努力从无到
有、从小到大,把理想变为现实。要敢于做先锋,而不做过客、当看客,让创新成为青春远
航的动力,让创业成为青春搏击的能量,让青春年华在为国家、为人民的奉献中焕发出绚丽
光彩。

各级党委和政府要充分信任青年、热情关心青年、严格要求青年、积极引导青年,为广大青
年成长成才、创新创造、建功立业做好服务保障工作。各级领导干部要做青年朋友的知心人、
青年工作的热心人。

我就讲这些。最后,祝大家工作顺利、身体健康、阖家幸福,在今后的工作中取得更大成绩!
\subsection{{\bfseries\sffamily DONE} \href{http://www.liuxue86.com/a/3351261.html}{十九大报告全文}}
\label{sec:org7b77008}
\subsection{摘要}
\label{sec:orgd2da772}

习近平同志在作十九大报告时说,中国共产党人的初心和使命,就是为中国人民谋幸福,为
中华民族谋复兴。
  习近平说,中国共产党第十九次全国代表大会,是在全面建成小康
社会决胜阶段、中国特色社会主义进入新时代的关键时期召开的一次十分重要的大会。
 
 大会的主题是:不忘初心,牢记使命,高举中国特色社会主义伟大旗帜,决胜全面建成小
康社会,夺取新时代中国特色社会主义伟大胜利,为实现中华民族伟大复兴的中国梦不懈奋
斗。
  习近平强调,不忘初心,方得始终。中国共产党人的初心和使命,就是为中国人
民谋幸福,为中华民族谋复兴。这个初心和使命是激励中国共产党人不断前进的根本动力。
全党同志一定要永远与人民同呼吸、共命运、心连心,永远把人民对美好生活的向往作为奋
斗目标,以永不懈怠的精神状态和一往无前的奋斗姿态,继续朝着实现中华民族伟大复兴的
宏伟目标奋勇前进。
  习近平指出,当前,国内外形势正在发生深刻复杂变化,我国发
展仍处于重要战略机遇期,前景十分光明,挑战也十分严峻。全党同志一定要登高望远、居
安思危,勇于变革、勇于创新,永不僵化、永不停滞,团结带领全国各族人民决胜全面建成
小康社会,奋力夺取新时代中国特色社会主义伟大胜利。
\subsection{一、过去五年的工作和历史性变革}
\label{sec:org8690c9d}

  十八大以来的五年,是党和国家发展进程中极不平凡的五年。面对世界经济复
苏乏力、局部冲突和动荡频发、全球性问题加剧的外部环境,面对我国经济发展进入新常态
等一系列深刻变化,我们坚持稳中求进工作总基调,迎难而上,开拓进取,取得了改革开放
和社会主义现代化建设的历史性成就。

  为贯彻十八大精神,党中央召开七次全会,分
别就政府机构改革和职能转变、全面深化改革、全面推进依法治国、制定“十三五”规划、
全面从严治党等重大问题作出决定和部署。五年来,我们统筹推进“五位一体”总体布局、
协调推进“四个全面”战略布局,“十二五”规划胜利完成,“十三五”规划顺利实施,党
和国家事业全面开创新局面。

  经济建设取得重大成就。坚定不移贯彻新发展理念,坚
决端正发展观念、转变发展方式,发展质量和效益不断提升。经济保持中高速增长,在世界
主要国家中名列前茅,国内生产总值从五十四万亿元增长到八十万亿元,稳居世界第二,对
世界经济增长贡献率超过百分之三十。供给侧结构性改革深入推进,经济结构不断优化,数
字经济等新兴产业蓬勃发展,高铁、公路、桥梁、港口、机场等基础设施建设快速推进。农
业现代化稳步推进,粮食生产能力达到一万二千亿斤。城镇化率年均提高一点二个百分点,
八千多万农业转移人口成为城镇居民。区域发展协调性增强,“一带一路”建设、京津冀协
同发展、长江经济带发展成效显著。创新驱动发展战略大力实施,创新型国家建设成果丰硕,
天宫、蛟龙、天眼、悟空、墨子、大飞机等重大科技成果相继问世。南海岛礁建设积极推进。
开放型经济新体制逐步健全,对外贸易、对外投资、外汇储备稳居世界前列。

  全面深化改革取得重大突破。蹄疾步稳推进全面深化改革,坚决破除各方面体制机制弊端。改革全
面发力、多点突破、纵深推进,着力增强改革系统性、整体性、协同性,压茬拓展改革广度
和深度,推出一千五百多项改革举措,重要领域和关键环节改革取得突破性进展,主要领域
改革主体框架基本确立。中国特色社会主义制度更加完善,国家治理体系和治理能力现代化
水平明显提高,全社会发展活力和创新活力明显增强。

  民主法治建设迈出重大步伐。积极发展社会主义民主政治,推进全面依法治国,党的领导、人民当家作主、依法治国有机
统一的制度建设全面加强,党的领导体制机制不断完善,社会主义民主不断发展,党内民主
更加广泛,社会主义协商民主全面展开,爱国统一战线巩固发展,民族宗教工作创新推进。
科学立法、严格执法、公正司法、全民守法深入推进,法治国家、法治政府、法治社会建设
相互促进,中国特色社会主义法治体系日益完善,全社会法治观念明显增强。国家监察体制
改革试点取得实效,行政体制改革、司法体制改革、权力运行制约和监督体系建设有效实施。

  思想文化建设取得重大进展。加强党对意识形态工作的领导,党的理论创新全面推进,
马克思主义在意识形态领域的指导地位更加鲜明,中国特色社会主义和中国梦深入人心,社
会主义核心价值观和中华优秀传统文化广泛弘扬,群众性精神文明创建活动扎实开展。公共
文化服务水平不断提高,文艺创作持续繁荣,文化事业和文化产业蓬勃发展,互联网建设管
理运用不断完善,全民健身和竞技体育全面发展。主旋律更加响亮,正能量更加强劲,文化
自信得到彰显,国家文化软实力和中华文化影响力大幅提升,全党全社会思想上的团结统一
更加巩固。

  人民生活不断改善。深入贯彻以人民为中心的发展思想,一大批惠民举措
落地实施,人民获得感显著增强。脱贫攻坚战取得决定性进展,六千多万贫困人口稳定脱贫,
贫困发生率从百分之十点二下降到百分之四以下。教育事业全面发展,中西部和农村教育明
显加强。就业状况持续改善,城镇新增就业年均一千三百万人以上。城乡居民收入增速超过
经济增速,中等收入群体持续扩大。覆盖城乡居民的社会保障体系基本建立,人民健康和医
疗卫生水平大幅提高,保障性住房建设稳步推进。社会治理体系更加完善,社会大局保持稳
定,国家安全全面加强。

  生态文明建设成效显著。大力度推进生态文明建设,全党全
国贯彻绿色发展理念的自觉性和主动性显著增强,忽视生态环境保护的状况明显改变。生态
文明制度体系加快形成,主体功能区制度逐步健全,国家公园体制试点积极推进。全面节约
资源有效推进,能源资源消耗强度大幅下降。重大生态保护和修复工程进展顺利,森林覆盖
率持续提高。生态环境治理明显加强,环境状况得到改善。引导应对气候变化国际合作,成
为全球生态文明建设的重要参与者、贡献者、引领者。

  强军兴军开创新局面。着眼于实现中国梦强军梦,制定新形势下军事战略方针,全力推进国防和军队现代化。召开古田全
军政治工作会议,恢复和发扬我党我军光荣传统和优良作风,人民军队政治生态得到有效治
理。国防和军队改革取得历史性突破,形成军委管总、战区主战、军种主建新格局,人民军
队组织架构和力量体系实现革命性重塑。加强练兵备战,有效遂行海上维权、反恐维稳、抢
险救灾、国际维和、亚丁湾护航、人道主义救援等重大任务,武器装备加快发展,军事斗争
准备取得重大进展。人民军队在中国特色强军之路上迈出坚定步伐。

  港澳台工作取得新进展。全面准确贯彻“一国两制”方针,牢牢掌握宪法和基本法赋予的中央对香港、澳门
全面管治权,深化内地和港澳地区交流合作,保持香港、澳门繁荣稳定。坚持一个中国原则
和“九二共识”,推动两岸关系和平发展,加强两岸经济文化交流合作,实现两岸领导人历
史性会晤。妥善应对台湾局势变化,坚决反对和遏制“台独”分裂势力,有力维护台海和平
稳定。

  全方位外交布局深入展开。全面推进中国特色大国外交,形成全方位、多层次、
立体化的外交布局,为我国发展营造了良好外部条件。实施共建“一带一路”倡议,发起创
办亚洲基础设施投资银行,设立丝路基金,举办首届“一带一路”国际合作高峰论坛、亚太
经合组织领导人非正式会议、二十国集团领导人杭州峰会、金砖国家领导人厦门会晤、亚信
峰会。倡导构建人类命运共同体,促进全球治理体系变革。我国国际影响力、感召力、塑造
力进一步提高,为世界和平与发展作出新的重大贡献。

  全面从严治党成效卓著。全面加强党的领导和党的建设,坚决改变管党治党宽松软状况。推动全党尊崇党章,增强政治意
识、大局意识、核心意识、看齐意识,坚决维护党中央权威和集中统一领导,严明党的政治
纪律和政治规矩,层层落实管党治党政治责任。坚持照镜子、正衣冠、洗洗澡、治治病的要
求,开展党的群众路线教育实践活动和“三严三实”专题教育,推进“两学一做”学习教育
常态化制度化,全党理想信念更加坚定、党性更加坚强。贯彻新时期好干部标准,选人用人
状况和风气明显好转。党的建设制度改革深入推进,党内法规制度体系不断完善。把纪律挺
在前面,着力解决人民群众反映最强烈、对党的执政基础威胁最大的突出问题。出台中央八
项规定,严厉整治形式主义、官僚主义、享乐主义和奢靡之风,坚决反对特权。巡视利剑作
用彰显,实现中央和省级党委巡视全覆盖。坚持反腐败无禁区、全覆盖、零容忍,坚定不移
“打虎”、“拍蝇”、“猎狐”,不敢腐的目标初步实现,不能腐的笼子越扎越牢,不想腐
的堤坝正在构筑,反腐败斗争压倒性态势已经形成并巩固发展。

  五年来的成就是全方位的、开创性的,五年来的变革是深层次的、根本性的。五年来,我们党以巨大的政治勇气
和强烈的责任担当,提出一系列新理念新思想新战略,出台一系列重大方针政策,推出一系
列重大举措,推进一系列重大工作,解决了许多长期想解决而没有解决的难题,办成了许多
过去想办而没有办成的大事,推动党和国家事业发生历史性变革。这些历史性变革,对党和
国家事业发展具有重大而深远的影响。

  五年来,我们勇于面对党面临的重大风险考验
和党内存在的突出问题,以顽强意志品质正风肃纪、反腐惩恶,消除了党和国家内部存在的
严重隐患,党内政治生活气象更新,党内政治生态明显好转,党的创造力、凝聚力、战斗力
显著增强,党的团结统一更加巩固,党群关系明显改善,党在革命性锻造中更加坚强,焕发
出新的强大生机活力,为党和国家事业发展提供了坚强政治保证。

  同时,必须清醒看到,我们的工作还存在许多不足,也面临不少困难和挑战。主要是:发展不平衡不充分的一
些突出问题尚未解决,发展质量和效益还不高,创新能力不够强,实体经济水平有待提高,
生态环境保护任重道远;民生领域还有不少短板,脱贫攻坚任务艰巨,城乡区域发展和收入
分配差距依然较大,群众在就业、教育、医疗、居住、养老等方面面临不少难题;社会文明
水平尚需提高;社会矛盾和问题交织叠加,全面依法治国任务依然繁重,国家治理体系和治
理能力有待加强;意识形态领域斗争依然复杂,国家安全面临新情况;一些改革部署和重大政
策措施需要进一步落实;党的建设方面还存在不少薄弱环节。这些问题,必须着力加以解决。
 
 五年来的成就,是党中央坚强领导的结果,更是全党全国各族人民共同奋斗的结果。我
代表中共中央,向全国各族人民,向各民主党派、各人民团体和各界爱国人士,向香港特别
行政区同胞、澳门特别行政区同胞和台湾同胞以及广大侨胞,向关心和支持中国现代化建设
的各国朋友,表示衷心的感谢!

  同志们!改革开放之初,我们党发出了走自己的路、建
设中国特色社会主义的伟大号召。从那时以来,我们党团结带领全国各族人民不懈奋斗,推
动我国经济实力、科技实力、国防实力、综合国力进入世界前列,推动我国国际地位实现前
所未有的提升,党的面貌、国家的面貌、人民的面貌、军队的面貌、中华民族的面貌发生了
前所未有的变化,中华民族正以崭新姿态屹立于世界的东方。

  经过长期努力,中国特色社会主义进入了新时代,这是我国发展新的历史方位。

  中国特色社会主义进入新时代,意味着近代以来久经磨难的中华民族迎来了从站起来、富起来到强起来的伟大飞跃,迎
来了实现中华民族伟大复兴的光明前景;意味着科学社会主义在二十一世纪的中国焕发出强
大生机活力,在世界上高高举起了中国特色社会主义伟大旗帜;意味着中国特色社会主义道
路、理论、制度、文化不断发展,拓展了发展中国家走向现代化的途径,给世界上那些既希
望加快发展又希望保持自身独立性的国家和民族提供了全新选择,为解决人类问题贡献了中
国智慧和中国方案。

  这个新时代,是承前启后、继往开来、在新的历史条件下继续夺
取中国特色社会主义伟大胜利的时代,是决胜全面建成小康社会、进而全面建设社会主义现
代化强国的时代,是全国各族人民团结奋斗、不断创造美好生活、逐步实现全体人民共同富
裕的时代,是全体中华儿女勠力同心、奋力实现中华民族伟大复兴中国梦的时代,是我国日
益走近世界舞台中央、不断为人类作出更大贡献的时代。

  中国特色社会主义进入新时
代,我国社会主要矛盾已经转化为人民日益增长的美好生活需要和不平衡不充分的发展之间
的矛盾。我国稳定解决了十几亿人的温饱问题,总体上实现小康,不久将全面建成小康社会,
人民美好生活需要日益广泛,不仅对物质文化生活提出了更高要求,而且在民主、法治、公
平、正义、安全、环境等方面的要求日益增长。同时,我国社会生产力水平总体上显著提高,
社会生产能力在很多方面进入世界前列,更加突出的问题是发展不平衡不充分,这已经成为
满足人民日益增长的美好生活需要的主要制约因素。
 
 必须认识到,我国社会主要矛盾的变化是关系全局的历史性变化,对党和国家工作提出了许多新要求。我们要在继续推动发
展的基础上,着力解决好发展不平衡不充分问题,大力提升发展质量和效益,更好满足人民
在经济、政治、文化、社会、生态等方面日益增长的需要,更好推动人的全面发展、社会全
面进步。

  必须认识到,我国社会主要矛盾的变化,没有改变我们对我国社会主义所处历史阶段的判断,我国仍处于并将长期处于社会主义初级阶段的基本国情没有变,我国是世
界最大发展中国家的国际地位没有变。全党要牢牢把握社会主义初级阶段这个基本国情,牢
牢立足社会主义初级阶段这个最大实际,牢牢坚持党的基本路线这个党和国家的生命线、人
民的幸福线,领导和团结全国各族人民,以经济建设为中心,坚持四项基本原则,坚持改革
开放,自力更生,艰苦创业,为把我国建设成为富强民主文明和谐美丽的社会主义现代化强
国而奋斗。

  同志们!中国特色社会主义进入新时代,在中华人民共和国发展史上、中华
民族发展史上具有重大意义,在世界社会主义发展史上、人类社会发展史上也具有重大意义。
全党要坚定信心、奋发有为,让中国特色社会主义展现出更加强大的生命力!

\subsection{二、新时代中国共产党的历史使命}
\label{sec:orgd32b6dc}

  一百年前,十月革命一声炮响,给中国送来了马克思列宁主义。中国先进分子从马克思列宁主义的科学真理中看到了解决中国问题的出路。在近代以后
中国社会的剧烈运动中,在中国人民反抗封建统治和外来侵略的激烈斗争中,在马克思列宁
主义同中国工人运动的结合过程中,一九二一年中国共产党应运而生。从此,中国人民谋求
民族独立、人民解放和国家富强、人民幸福的斗争就有了主心骨,中国人民就从精神上由被
动转为主动。

  中华民族有五千多年的文明历史,创造了灿烂的中华文明,为人类作出
了卓越贡献,成为世界上伟大的民族。鸦片战争后,中国陷入内忧外患的黑暗境地,中国人
民经历了战乱频仍、山河破碎、民不聊生的深重苦难。为了民族复兴,无数仁人志士不屈不
挠、前仆后继,进行了可歌可泣的斗争,进行了各式各样的尝试,但终究未能改变旧中国的
社会性质和中国人民的悲惨命运。

  实现中华民族伟大复兴是近代以来中华民族最伟大
的梦想。中国共产党一经成立,就把实现共产主义作为党的最高理想和最终目标,义无反顾
肩负起实现中华民族伟大复兴的历史使命,团结带领人民进行了艰苦卓绝的斗争,谱写了气
吞山河的壮丽史诗。

  我们党深刻认识到,实现中华民族伟大复兴,必须推翻压在中国
人民头上的帝国主义、封建主义、官僚资本主义三座大山,实现民族独立、人民解放、国家
统一、社会稳定。我们党团结带领人民找到了一条以农村包围城市、武装夺取政权的正确革
命道路,进行了二十八年浴血奋战,完成了新民主主义革命,一九四九年建立了中华人民共
和国,实现了中国从几千年封建专制政治向人民民主的伟大飞跃。

  我们党深刻认识到,实现中华民族伟大复兴,必须建立符合我国实际的先进社会制度。我们党团结带领人民完成
社会主义革命,确立社会主义基本制度,推进社会主义建设,完成了中华民族有史以来最为
广泛而深刻的社会变革,为当代中国一切发展进步奠定了根本政治前提和制度基础,实现了
中华民族由近代不断衰落到根本扭转命运、持续走向繁荣富强的伟大飞跃。
 
 我们党深刻认识到,实现中华民族伟大复兴,必须合乎时代潮流、顺应人民意愿,勇于改革开放,让
党和人民事业始终充满奋勇前进的强大动力。我们党团结带领人民进行改革开放新的伟大革
命,破除阻碍国家和民族发展的一切思想和体制障碍,开辟了中国特色社会主义道路,使中
国大踏步赶上时代。

  九十六年来,为了实现中华民族伟大复兴的历史使命,无论是弱小还是强大,无论是
顺境还是逆境,我们党都初心不改、矢志不渝,团结带领人民历经千难万险,付出巨大牺
牲,敢于面对曲折,勇于修正错误,攻克了一个又一个看似不可攻克的难关,创造了一个
又一个彪炳史册的人间奇迹。

  同志们!今天,我们比历史上任何时期都更接近、更有信心和能力实现中华民族伟大复
兴的目标。


  行百里者半九十。中华民族
伟大复兴,绝不是轻轻松松、敲锣打鼓就能实现的。全党必须准备付出更为艰巨、更为艰苦
的努力。

  实现伟大梦想,必须进行伟大斗争。社会是在矛盾运动中前进的,有矛盾就
会有斗争。我们党要团结带领人民有效应对重大挑战、抵御重大风险、克服重大阻力、解决
重大矛盾,必须进行具有许多新的历史特点的伟大斗争,任何贪图享受、消极懈怠、回避矛
盾的思想和行为都是错误的。全党要更加自觉地坚持党的领导和我国社会主义制度,坚决反
对一切削弱、歪曲、否定党的领导和我国社会主义制度的言行;更加自觉地维护人民利益,
坚决反对一切损害人民利益、脱离群众的行为;更加自觉地投身改革创新时代潮流,坚决破
除一切顽瘴痼疾;更加自觉地维护我国主权、安全、发展利益,坚决反对一切分裂祖国、破
坏民族团结和社会和谐稳定的行为;更加自觉地防范各种风险,坚决战胜一切在政治、经济、
文化、社会等领域和自然界出现的困难和挑战。全党要充分认识这场伟大斗争的长期性、复
杂性、艰巨性,发扬斗争精神,提高斗争本领,不断夺取伟大斗争新胜利。

  实现伟大梦想,必须建设伟大工程。这个伟大工程就是我们党正在深入推进的党的建设新的伟大工程。
历史已经并将继续证明,没有中国共产党的领导,民族复兴必然是空想。我们党要始终成为
时代先锋、民族脊梁,始终成为马克思主义执政党,自身必须始终过硬。全党要更加自觉地
坚定党性原则,勇于直面问题,敢于刮骨疗毒,消除一切损害党的先进性和纯洁性的因素,
清除一切侵蚀党的健康肌体的病毒,不断增强党的政治领导力、思想引领力、群众组织力、
社会号召力,确保我们党永葆旺盛生命力和强大战斗力。

  实现伟大梦想,必须推进伟大事业。中国特色社会主义是改革开放以来党的全部理论和实践的主题,是党和人民历尽千
辛万苦、付出巨大代价取得的根本成就。中国特色社会主义道路是实现社会主义现代化、创
造人民美好生活的必由之路,中国特色社会主义理论体系是指导党和人民实现中华民族伟大
复兴的正确理论,中国特色社会主义制度是当代中国发展进步的根本制度保障,中国特色社
会主义文化是激励全党全国各族人民奋勇前进的强大精神力量。全党要更加自觉地增强道路
自信、理论自信、制度自信、文化自信,既不走封闭僵化的老路,也不走改旗易帜的邪路,
保持政治定力,坚持实干兴邦,始终坚持和发展中国特色社会主义。

  伟大斗争,伟大工程,伟大事业,伟大梦想,紧密联系、相互贯通、相互作用,其中起决定性作用的是党的
建设新的伟大工程。推进伟大工程,要结合伟大斗争、伟大事业、伟大梦想的实践来进行,
确保党在世界形势深刻变化的历史进程中始终走在时代前列,在应对国内外各种风险和考验
的历史进程中始终成为全国人民的主心骨,在坚持和发展中国特色社会主义的历史进程中始
终成为坚强领导核心。

  同志们!使命呼唤担当,使命引领未来。我们要不负人民重托、
无愧历史选择,在新时代中国特色社会主义的伟大实践中,以党的坚强领导和顽强奋斗,激
励全体中华儿女不断奋进,凝聚起同心共筑中国梦的磅礴力量!

\subsection{三、新时代中国特色社会主义思想和基本方略}
\label{sec:org3dbae6c}

  十八大以来,国内外形势变化和我国各项事业发展都给我们提
出了一个重大时代课题,这就是必须从理论和实践结合上系统回答新时代坚持和发展什么样
的中国特色社会主义、怎样坚持和发展中国特色社会主义,包括新时代坚持和发展中国特色
社会主义的总目标、总任务、总体布局、战略布局和发展方向、发展方式、发展动力、战略
步骤、外部条件、政治保证等基本问题,并且要根据新的实践对经济、政治、法治、科技、
文化、教育、民生、民族、宗教、社会、生态文明、国家安全、国防和军队、“一国两制”
和祖国统一、统一战线、外交、党的建设等各方面作出理论分析和政策指导,以利于更好坚
持和发展中国特色社会主义。

  围绕这个重大时代课题,我们党坚持以马克思列宁主义、
毛泽东思想、邓小平理论、“三个代表”重要思想、科学发展观为指导,坚持解放思想、实
事求是、与时俱进、求真务实,坚持辩证唯物主义和历史唯物主义,紧密结合新的时代条件
和实践要求,以全新的视野深化对共产党执政规律、社会主义建设规律、人类社会发展规律
的认识,进行艰辛理论探索,取得重大理论创新成果,形成了新时代中国特色社会主义思想。

  新时代中国特色社会主义思想,明确坚持和发展中国特色社会主义,总任务是实现社会
主义现代化和中华民族伟大复兴,在全面建成小康社会的基础上,分两步走在本世纪中叶建
成富强民主文明和谐美丽的社会主义现代化强国;明确新时代我国社会主要矛盾是人民日益
增长的美好生活需要和不平衡不充分的发展之间的矛盾,必须坚持以人民为中心的发展思想,
不断促进人的全面发展、全体人民共同富裕;明确中国特色社会主义事业总体布局是“五位
一体”、战略布局是“四个全面”,强调坚定道路自信、理论自信、制度自信、文化自信;
明确全面深化改革总目标是完善和发展中国特色社会主义制度、推进国家治理体系和治理能
力现代化;明确全面推进依法治国总目标是建设中国特色社会主义法治体系、建设社会主义
法治国家;明确党在新时代的强军目标是建设一支听党指挥、能打胜仗、作风优良的人民军
队,把人民军队建设成为世界一流军队;明确中国特色大国外交要推动构建新型国际关系,
推动构建人类命运共同体;明确中国特色社会主义最本质的特征是中国共产党领导,中国特
色社会主义制度的最大优势是中国共产党领导,党是最高政治领导力量,提出新时代党的建
设总要求,突出政治建设在党的建设中的重要地位。

  新时代中国特色社会主义思想,
是对马克思列宁主义、毛泽东思想、邓小平理论、“三个代表”重要思想、科学发展观的继
承和发展,是马克思主义中国化最新成果,是党和人民实践经验和集体智慧的结晶,是中国
特色社会主义理论体系的重要组成部分,是全党全国人民为实现中华民族伟大复兴而奋斗的
行动指南,必须长期坚持并不断发展。

  全党要深刻领会新时代中国特色社会主义思想
的精神实质和丰富内涵,在各项工作中全面准确贯彻落实。

\subsubsection{  (一)坚持党对一切工作的领导。}
\label{sec:org0669b29}

党政军民学,东西南北中,党是领导一切的。必须增强政治意识、大局意识、核心意
识、看齐意识,自觉维护党中央权威和集中统一领导,自觉在思想上政治上行动上同党中央
保持高度一致,完善坚持党的领导的体制机制,坚持稳中求进工作总基调,统筹推进“五位
一体”总体布局,协调推进“四个全面”战略布局,提高党把方向、谋大局、定政策、促改
革的能力和定力,确保党始终总揽全局、协调各方。

\subsubsection{  (二)坚持以人民为中心。}
\label{sec:org61eb841}

人民是历史的创造者,是决定党和国家前途命运的根本力量。必须坚持人民主体地位,坚持立党为
公、执政为民,践行全心全意为人民服务的根本宗旨,把党的群众路线贯彻到治国理政全部
活动之中,把人民对美好生活的向往作为奋斗目标,依靠人民创造历史伟业。

\subsubsection{  (三)坚持全面深化改革。}
\label{sec:org5cc7862}

只有社会主义才能救中国,只有改革开放才能发展中国、发展社会主义、
发展马克思主义。必须坚持和完善中国特色社会主义制度,不断推进国家治理体系和治理能
力现代化,坚决破除一切不合时宜的思想观念和体制机制弊端,突破利益固化的藩篱,吸收
人类文明有益成果,构建系统完备、科学规范、运行有效的制度体系,充分发挥我国社会主
义制度优越性。

\subsubsection{   (四)坚持新发展理念。}
\label{sec:org382e48d}

发展是解决我国一切问题的基础和关键,发展
必须是科学发展,必须坚定不移贯彻创新、协调、绿色、开放、共享的发展理念。必须坚持
和完善我国社会主义基本经济制度和分配制度,毫不动摇巩固和发展公有制经济,毫不动摇
鼓励、支持、引导非公有制经济发展,使市场在资源配置中起决定性作用,更好发挥政府作
用,推动新型工业化、信息化、城镇化、农业现代化同步发展,主动参与和推动经济全球化
进程,发展更高层次的开放型经济,不断壮大我国经济实力和综合国力。

\subsubsection{  (五)坚持人民当家作主。}
\label{sec:org88beb1a}

坚持党的领导、人民当家作主、依法治国有机统一是社会主义政治发展的必然
要求。必须坚持中国特色社会主义政治发展道路,坚持和完善人民代表大会制度、中国共产
党领导的多党合作和政治协商制度、民族区域自治制度、基层群众自治制度,巩固和发展最
广泛的爱国统一战线,发展社会主义协商民主,健全民主制度,丰富民主形式,拓宽民主渠
道,保证人民当家作主落实到国家政治生活和社会生活之中。

\subsubsection{  (六)坚持全面依法治国。}
\label{sec:org94f528c}
全面依法治国是中国特色社会主义的本质要求和重要保障。必须把党的领导贯彻落实到依法
治国全过程和各方面,坚定不移走中国特色社会主义法治道路,完善以宪法为核心的中国特
色社会主义法律体系,建设中国特色社会主义法治体系,建设社会主义法治国家,发展中国
特色社会主义法治理论,坚持依法治国、依法执政、依法行政共同推进,坚持法治国家、法
治政府、法治社会一体建设,坚持依法治国和以德治国相结合,依法治国和依规治党有机统
一,深化司法体制改革,提高全民族法治素养和道德素质。

\subsubsection{  (七)坚持社会主义核心价}
\label{sec:orgb1782f3}
值体系。文化自信是一个国家、一个民族发展中更基本、更深沉、更持久的力量。必须坚持
马克思主义,牢固树立共产主义远大理想和中国特色社会主义共同理想,培育和践行社会主
义核心价值观,不断增强意识形态领域主导权和话语权,推动中华优秀传统文化创造性转化、
创新性发展,继承革命文化,发展社会主义先进文化,不忘本来、吸收外来、面向未来,更
好构筑中国精神、中国价值、中国力量,为人民提供精神指引。

\subsubsection{  (八)坚持在发展中保障和改善民生。}
\label{sec:org5ca5426}

增进民生福祉是发展的根本目的。必须多谋民生之利、多解民生之忧,在发
展中补齐民生短板、促进社会公平正义,在幼有所育、学有所教、劳有所得、病有所医、老
有所养、住有所居、弱有所扶上不断取得新进展,深入开展脱贫攻坚,保证全体人民在共建
共享发展中有更多获得感,不断促进人的全面发展、全体人民共同富裕。建设平安中国,加
强和创新社会治理,维护社会和谐稳定,确保国家长治久安、人民安居乐业。

\subsubsection{  (九)坚持人与自然和谐共生.}
\label{sec:orgb7adac0}

建设生态文明是中华民族永续发展的千年大计。必须树立和践行绿水
青山就是金山银山的理念,坚持节约资源和保护环境的基本国策,像对待生命一样对待生态
环境,统筹山水林田湖草系统治理,实行最严格的生态环境保护制度,形成绿色发展方式和
生活方式,坚定走生产发展、生活富裕、生态良好的文明发展道路,建设美丽中国,为人民
创造良好生产生活环境,为全球生态安全作出贡献。

\subsubsection{  (十)坚持总体国家安全观。}
\label{sec:org3662e3b}

统筹发展和安全,增强忧患意识,做到居安思危,是我们党治国理政的一个重大原则。必须坚持
国家利益至上,以人民安全为宗旨,以政治安全为根本,统筹外部安全和内部安全、国土安
全和国民安全、传统安全和非传统安全、自身安全和共同安全,完善国家安全制度体系,加
强国家安全能力建设,坚决维护国家主权、安全、发展利益。

\subsubsection{  (十一)坚持党对人民军队的绝对领导。}
\label{sec:orge35ed86}

建设一支听党指挥、能打胜仗、作风优良的人民军队,是实现“两个一百
年”奋斗目标、实现中华民族伟大复兴的战略支撑。必须全面贯彻党领导人民军队的一系列
根本原则和制度,确立新时代党的强军思想在国防和军队建设中的指导地位,坚持政治建军、
改革强军、科技兴军、依法治军,更加注重聚焦实战,更加注重创新驱动,更加注重体系建
设,更加注重集约高效,更加注重军民融合,实现党在新时代的强军目标。

\subsubsection{  (十二)坚持“一国两制”和推进祖国统一。}
\label{sec:org441e716}

保持香港、澳门长期繁荣稳定,实现祖国完全统一,是实
现中华民族伟大复兴的必然要求。必须把维护中央对香港、澳门特别行政区全面管治权和保
障特别行政区高度自治权有机结合起来,确保“一国两制”方针不会变、不动摇,确保“一
国两制”实践不变形、不走样。必须坚持一个中国原则,坚持“九二共识”,推动两岸关系
和平发展,深化两岸经济合作和文化往来,推动两岸同胞共同反对一切分裂国家的活动,共
同为实现中华民族伟大复兴而奋斗。

\subsubsection{  (十三)坚持推动构建人类命运共同体。}
\label{sec:org1964edf}

中国人民的梦想同各国人民的梦想息息相通,实现中国梦离不开和平的国际环境和稳定的国际秩序。
必须统筹国内国际两个大局,始终不渝走和平发展道路、奉行互利共赢的开放战略,坚持正
确义利观,树立共同、综合、合作、可持续的新安全观,谋求开放创新、包容互惠的发展前
景,促进和而不同、兼收并蓄的文明交流,构筑尊崇自然、绿色发展的生态体系,始终做世
界和平的建设者、全球发展的贡献者、国际秩序的维护者。

\subsubsection{  (十四)坚持全面从严治党。}
\label{sec:orgddbed0e}

勇于自我革命,从严管党治党,是我们党最鲜明的品格。必须以党章为根本遵循,把党的政
治建设摆在首位,思想建党和制度治党同向发力,统筹推进党的各项建设,抓住“关键少
数”,坚持“三严三实”,坚持民主集中制,严肃党内政治生活,严明党的纪律,强化党内
监督,发展积极健康的党内政治文化,全面净化党内政治生态,坚决纠正各种不正之风,以
零容忍态度惩治腐败,不断增强党自我净化、自我完善、自我革新、自我提高的能力,始终
保持党同人民群众的血肉联系。

  以上十四条,构成新时代坚持和发展中国特色社会主
义的基本方略。全党同志必须全面贯彻党的基本理论、基本路线、基本方略,更好引领党和
人民事业发展。

  实践没有止境,理论创新也没有止境。世界每时每刻都在发生变化,
中国也每时每刻都在发生变化,我们必须在理论上跟上时代,不断认识规律,不断推进理论
创新、实践创新、制度创新、文化创新以及其他各方面创新。

  同志们!时代是思想之母,
实践是理论之源。只要我们善于聆听时代声音,勇于坚持真理、修正错误,二十一世纪中国
的马克思主义一定能够展现出更强大、更有说服力的真理力量!

\subsection{  四、决胜全面建成小康社会,开启全面建设社会主义现代化国家新征程}
\label{sec:org3349e32}

  改革开放之后,我们党对我国社会主
义现代化建设作出战略安排,提出“三步走”战略目标。解决人民温饱问题、人民生活总体
上达到小康水平这两个目标已提前实现。在这个基础上,我们党提出,到建党一百年时建成
经济更加发展、民主更加健全、科教更加进步、文化更加繁荣、社会更加和谐、人民生活更
加殷实的小康社会,然后再奋斗三十年,到新中国成立一百年时,基本实现现代化,把我国
建成社会主义现代化国家。

  从现在到二〇二〇年,是全面建成小康社会决胜期。要按
照十六大、十七大、十八大提出的全面建成小康社会各项要求,紧扣我国社会主要矛盾变化,
统筹推进经济建设、政治建设、文化建设、社会建设、生态文明建设,坚定实施科教兴国战
略、人才强国战略、创新驱动发展战略、乡村振兴战略、区域协调发展战略、可持续发展战
略、军民融合发展战略,突出抓重点、补短板、强弱项,特别是要坚决打好防范化解重大风
险、精准脱贫、污染防治的攻坚战,使全面建成小康社会得到人民认可、经得起历史检验。


  从十九大到二十大,是“两个一百年”奋斗目标的历史交汇期。我们既要全面建成小康
社会、实现第一个百年奋斗目标,又要乘势而上开启全面建设社会主义现代化国家新征程,
向第二个百年奋斗目标进军。

  综合分析国际国内形势和我国发展条件,从二〇二〇年
到本世纪中叶可以分两个阶段来安排。

  第一个阶段,从二〇二〇年到二〇三五年,在
全面建成小康社会的基础上,再奋斗十五年,基本实现社会主义现代化。到那时,我国经济
实力、科技实力将大幅跃升,跻身创新型国家前列;人民平等参与、平等发展权利得到充分
保障,法治国家、法治政府、法治社会基本建成,各方面制度更加完善,国家治理体系和治
理能力现代化基本实现;社会文明程度达到新的高度,国家文化软实力显著增强,中华文化
影响更加广泛深入;人民生活更为宽裕,中等收入群体比例明显提高,城乡区域发展差距和
居民生活水平差距显著缩小,基本公共服务均等化基本实现,全体人民共同富裕迈出坚实步
伐;现代社会治理格局基本形成,社会充满活力又和谐有序;生态环境根本好转,美丽中国目
标基本实现。

  第二个阶段,从二〇三五年到本世纪中叶,在基本实现现代化的基础上,
再奋斗十五年,把我国建成富强民主文明和谐美丽的社会主义现代化强国。到那时,我国物
质文明、政治文明、精神文明、社会文明、生态文明将全面提升,实现国家治理体系和治理
能力现代化,成为综合国力和国际影响力领先的国家,全体人民共同富裕基本实现,我国人
民将享有更加幸福安康的生活,中华民族将以更加昂扬的姿态屹立于世界民族之林。

  
同志们!从全面建成小康社会到基本实现现代化,再到全面建成社会主义现代化强国,是新
时代中国特色社会主义发展的战略安排。我们要坚忍不拔、锲而不舍,奋力谱写社会主义现
代化新征程的壮丽篇章!

\subsection{  五、贯彻新发展理念,建设现代化经济体系}
\label{sec:org3d2338d}

  实现“两个
一百年”奋斗目标、实现中华民族伟大复兴的中国梦,不断提高人民生活水平,必须坚定不
移把发展作为党执政兴国的第一要务,坚持解放和发展社会生产力,坚持社会主义市场经济
改革方向,推动经济持续健康发展。

  我国经济已由高速增长阶段转向高质量发展阶段,
正处在转变发展方式、优化经济结构、转换增长动力的攻关期,建设现代化经济体系是跨越
关口的迫切要求和我国发展的战略目标。必须坚持质量第一、效益优先,以供给侧结构性改
革为主线,推动经济发展质量变革、效率变革、动力变革,提高全要素生产率,着力加快建
设实体经济、科技创新、现代金融、人力资源协同发展的产业体系,着力构建市场机制有效、
微观主体有活力、宏观调控有度的经济体制,不断增强我国经济创新力和竞争力。

\subsubsection{  (一)深化供给侧结构性改革。}
\label{sec:org8618c08}

建设现代化经济体系,必须把发展经济的着力点放在实体经济上,把
提高供给体系质量作为主攻方向,显著增强我国经济质量优势。加快建设制造强国,加快发
展先进制造业,推动互联网、大数据、人工智能和实体经济深度融合,在中高端消费、创新
引领、绿色低碳、共享经济、现代供应链、人力资本服务等领域培育新增长点、形成新动能。
支持传统产业优化升级,加快发展现代服务业,瞄准国际标准提高水平。促进我国产业迈向
全球价值链中高端,培育若干世界级先进制造业集群。加强水利、铁路、公路、水运、航空、
管道、电网、信息、物流等基础设施网络建设。坚持去产能、去库存、去杠杆、降成本、补
短板,优化存量资源配置,扩大优质增量供给,实现供需动态平衡。激发和保护企业家精神,
鼓励更多社会主体投身创新创业。建设知识型、技能型、创新型劳动者大军,弘扬劳模精神
和工匠精神,营造劳动光荣的社会风尚和精益求精的敬业风气。

\subsubsection{  (二)加快建设创新型国家。}
\label{sec:orgf8f679c}

创新是引领发展的第一动力,是建设现代化经济体系的战略支撑。要瞄准世界科技前
沿,强化基础研究,实现前瞻性基础研究、引领性原创成果重大突破。加强应用基础研究,
拓展实施国家重大科技项目,突出关键共性技术、前沿引领技术、现代工程技术、颠覆性技
术创新,为建设科技强国、质量强国、航天强国、网络强国、交通强国、数字中国、智慧社
会提供有力支撑。加强国家创新体系建设,强化战略科技力量。深化科技体制改革,建立以
企业为主体、市场为导向、产学研深度融合的技术创新体系,加强对中小企业创新的支持,
促进科技成果转化。倡导创新文化,强化知识产权创造、保护、运用。培养造就一大批具有
国际水平的战略科技人才、科技领军人才、青年科技人才和高水平创新团队。

\subsubsection{  (三)实施乡村振兴战略。}
\label{sec:org4bd0200}

农业农村农民问题是关系国计民生的根本性问题,必须始终把解决好“三
农”问题作为全党工作重中之重。要坚持农业农村优先发展,按照产业兴旺、生态宜居、乡
风文明、治理有效、生活富裕的总要求,建立健全城乡融合发展体制机制和政策体系,加快
推进农业农村现代化。巩固和完善农村基本经营制度,深化农村土地制度改革,完善承包地
“三权”分置制度。保持土地承包关系稳定并长久不变,第二轮土地承包到期后再延长三十
年。深化农村集体产权制度改革,保障农民财产权益,壮大集体经济。确保国家粮食安全,
把中国人的饭碗牢牢端在自己手中。构建现代农业产业体系、生产体系、经营体系,完善农
业支持保护制度,发展多种形式适度规模经营,培育新型农业经营主体,健全农业社会化服
务体系,实现小农户和现代农业发展有机衔接。促进农村一二三产业融合发展,支持和鼓励
农民就业创业,拓宽增收渠道。加强农村基层基础工作,健全自治、法治、德治相结合的乡
村治理体系。培养造就一支懂农业、爱农村、爱农民的“三农”工作队伍。

\subsubsection{  (四)实施区域协调发展战略。}
\label{sec:orgfbcaece}

加大力度支持革命老区、民族地区、边疆地区、贫困地区加快发展,强
化举措推进西部大开发形成新格局,深化改革加快东北等老工业基地振兴,发挥优势推动中
部地区崛起,创新引领率先实现东部地区优化发展,建立更加有效的区域协调发展新机制。
以城市群为主体构建大中小城市和小城镇协调发展的城镇格局,加快农业转移人口市民化。
以疏解北京非首都功能为“牛鼻子”推动京津冀协同发展,高起点规划、高标准建设雄安新
区。以共抓大保护、不搞大开发为导向推动长江经济带发展。支持资源型地区经济转型发展。
加快边疆发展,确保边疆巩固、边境安全。坚持陆海统筹,加快建设海洋强国。

\subsubsection{  (五)加快完善社会主义市场经济体制。}
\label{sec:org2dcbbf4}

经济体制改革必须以完善产权制度和要素市场化配置为重
点,实现产权有效激励、要素自由流动、价格反应灵活、竞争公平有序、企业优胜劣汰。要
完善各类国有资产管理体制,改革国有资本授权经营体制,加快国有经济布局优化、结构调
整、战略性重组,促进国有资产保值增值,推动国有资本做强做优做大,有效防止国有资产
流失。深化国有企业改革,发展混合所有制经济,培育具有全球竞争力的世界一流企业。全
面实施市场准入负面清单制度,清理废除妨碍统一市场和公平竞争的各种规定和做法,支持
民营企业发展,激发各类市场主体活力。深化商事制度改革,打破行政性垄断,防止市场垄
断,加快要素价格市场化改革,放宽服务业准入限制,完善市场监管体制。创新和完善宏观
调控,发挥国家发展规划的战略导向作用,健全财政、货币、产业、区域等经济政策协调机
制。完善促进消费的体制机制,增强消费对经济发展的基础性作用。深化投融资体制改革,
发挥投资对优化供给结构的关键性作用。加快建立现代财政制度,建立权责清晰、财力协调、
区域均衡的中央和地方财政关系。建立全面规范透明、标准科学、约束有力的预算制度,全
面实施绩效管理。深化税收制度改革,健全地方税体系。深化金融体制改革,增强金融服务
实体经济能力,提高直接融资比重,促进多层次资本市场健康发展。健全货币政策和宏观审
慎政策双支柱调控框架,深化利率和汇率市场化改革。健全金融监管体系,守住不发生系统
性金融风险的底线。

\subsubsection{  (六)推动形成全面开放新格局。}
\label{sec:org267dfcf}

开放带来进步,封闭必然落后。
中国开放的大门不会关闭,只会越开越大。要以“一带一路”建设为重点,坚持引进来和走
出去并重,遵循共商共建共享原则,加强创新能力开放合作,形成陆海内外联动、东西双向
互济的开放格局。拓展对外贸易,培育贸易新业态新模式,推进贸易强国建设。实行高水平
的贸易和投资自由化便利化政策,全面实行准入前国民待遇加负面清单管理制度,大幅度放
宽市场准入,扩大服务业对外开放,保护外商投资合法权益。凡是在我国境内注册的企业,
都要一视同仁、平等对待。优化区域开放布局,加大西部开放力度。赋予自由贸易试验区更
大改革自主权,探索建设自由贸易港。创新对外投资方式,促进国际产能合作,形成面向全
球的贸易、投融资、生产、服务网络,加快培育国际经济合作和竞争新优势。   同志们!
解放和发展社会生产力,是社会主义的本质要求。我们要激发全社会创造力和发展活力,努
力实现更高质量、更有效率、更加公平、更可持续的发展!

\subsection{  六、健全人民当家作主制度体系,发展社会主义民主政治}
\label{sec:org5456af0}

  我国是工人阶级领导的、以工农联盟为基础的人民民主
专政的社会主义国家,国家一切权力属于人民。我国社会主义民主是维护人民根本利益的最
广泛、最真实、最管用的民主。发展社会主义民主政治就是要体现人民意志、保障人民权益、
激发人民创造活力,用制度体系保证人民当家作主。

  中国特色社会主义政治发展道路,
是近代以来中国人民长期奋斗历史逻辑、理论逻辑、实践逻辑的必然结果,是坚持党的本质
属性、践行党的根本宗旨的必然要求。世界上没有完全相同的政治制度模式,政治制度不能
脱离特定社会政治条件和历史文化传统来抽象评判,不能定于一尊,不能生搬硬套外国政治
制度模式。要长期坚持、不断发展我国社会主义民主政治,积极稳妥推进政治体制改革,推
进社会主义民主政治制度化、规范化、法治化、程序化,保证人民依法通过各种途径和形式
管理国家事务,管理经济文化事业,管理社会事务,巩固和发展生动活泼、安定团结的政治
局面。

\subsubsection{  (一)坚持党的领导、人民当家作主、依法治国有机统一。}
\label{sec:org0fc857c}

党的领导是人民当家
作主和依法治国的根本保证,人民当家作主是社会主义民主政治的本质特征,依法治国是党
领导人民治理国家的基本方式,三者统一于我国社会主义民主政治伟大实践。在我国政治生
活中,党是居于领导地位的,加强党的集中统一领导,支持人大、政府、政协和法院、检察
院依法依章程履行职能、开展工作、发挥作用,这两个方面是统一的。要改进党的领导方式
和执政方式,保证党领导人民有效治理国家;扩大人民有序政治参与,保证人民依法实行民
主选举、民主协商、民主决策、民主管理、民主监督;维护国家法制统一、尊严、权威,加
强人权法治保障,保证人民依法享有广泛权利和自由。巩固基层政权,完善基层民主制度,
保障人民知情权、参与权、表达权、监督权。健全依法决策机制,构建决策科学、执行坚决、
监督有力的权力运行机制。各级领导干部要增强民主意识,发扬民主作风,接受人民监督,
当好人民公仆。

\subsubsection{  (二)加强人民当家作主制度保障。}
\label{sec:org501922d}

人民代表大会制度是坚持党的领导、
人民当家作主、依法治国有机统一的根本政治制度安排,必须长期坚持、不断完善。要支持
和保证人民通过人民代表大会行使国家权力。发挥人大及其常委会在立法工作中的主导作用,
健全人大组织制度和工作制度,支持和保证人大依法行使立法权、监督权、决定权、任免权,
更好发挥人大代表作用,使各级人大及其常委会成为全面担负起宪法法律赋予的各项职责的
工作机关,成为同人民群众保持密切联系的代表机关。完善人大专门委员会设置,优化人大
常委会和专门委员会组成人员结构。

\subsubsection{  (三)发挥社会主义协商民主重要作用。}
\label{sec:orgf6b7d0a}

有事好商
量,众人的事情由众人商量,是人民民主的真谛。协商民主是实现党的领导的重要方式,是
我国社会主义民主政治的特有形式和独特优势。要推动协商民主广泛、多层、制度化发展,
统筹推进政党协商、人大协商、政府协商、政协协商、人民团体协商、基层协商以及社会组
织协商。加强协商民主制度建设,形成完整的制度程序和参与实践,保证人民在日常政治生
活中有广泛持续深入参与的权利。

  人民政协是具有中国特色的制度安排,是社会主义
协商民主的重要渠道和专门协商机构。人民政协工作要聚焦党和国家中心任务,围绕团结和
民主两大主题,把协商民主贯穿政治协商、民主监督、参政议政全过程,完善协商议政内容
和形式,着力增进共识、促进团结。加强人民政协民主监督,重点监督党和国家重大方针政
策和重要决策部署的贯彻落实。增强人民政协界别的代表性,加强委员队伍建设。

\subsubsection{  (四)深化依法治国实践。}
\label{sec:org5816011}

全面依法治国是国家治理的一场深刻革命,必须坚持厉行法治,推进科学
立法、严格执法、公正司法、全民守法。成立中央全面依法治国领导小组,加强对法治中国
建设的统一领导。加强宪法实施和监督,推进合宪性审查工作,维护宪法权威。推进科学立
法、民主立法、依法立法,以良法促进发展、保障善治。建设法治政府,推进依法行政,严
格规范公正文明执法。深化司法体制综合配套改革,全面落实司法责任制,努力让人民群众
在每一个司法案件中感受到公平正义。加大全民普法力度,建设社会主义法治文化,树立宪
法法律至上、法律面前人人平等的法治理念。各级党组织和全体党员要带头尊法学法守法用
法,任何组织和个人都不得有超越宪法法律的特权,绝不允许以言代法、以权压法、逐利违
法、徇私枉法。

\subsubsection{  (五)深化机构和行政体制改革。}
\label{sec:orgbbb9ef3}

统筹考虑各类机构设置,科学配置党
政部门及内设机构权力、明确职责。统筹使用各类编制资源,形成科学合理的管理体制,完
善国家机构组织法。转变政府职能,深化简政放权,创新监管方式,增强政府公信力和执行
力,建设人民满意的服务型政府。赋予省级及以下政府更多自主权。在省市县对职能相近的
党政机关探索合并设立或合署办公。深化事业单位改革,强化公益属性,推进政事分开、事
企分开、管办分离。

\subsubsection{  (六)巩固和发展爱国统一战线。}
\label{sec:orgc0dc6e9}

统一战线是党的事业取得胜利的
重要法宝,必须长期坚持。要高举爱国主义、社会主义旗帜,牢牢把握大团结大联合的主题,
坚持一致性和多样性统一,找到最大公约数,画出最大同心圆。坚持长期共存、互相监督、
肝胆相照、荣辱与共,支持民主党派按照中国特色社会主义参政党要求更好履行职能。深化
民族团结进步教育,铸牢中华民族共同体意识,加强各民族交往交流交融,促进各民族像石
榴籽一样紧紧抱在一起,共同团结奋斗、共同繁荣发展。全面贯彻党的宗教工作基本方针,
坚持我国宗教的中国化方向,积极引导宗教与社会主义社会相适应。加强党外知识分子工作,
做好新的社会阶层人士工作,发挥他们在中国特色社会主义事业中的重要作用。构建亲清新
型政商关系,促进非公有制经济健康发展和非公有制经济人士健康成长。广泛团结联系海外
侨胞和归侨侨眷,共同致力于中华民族伟大复兴。

  同志们!中国特色社会主义政治制度
是中国共产党和中国人民的伟大创造。我们完全有信心、有能力把我国社会主义民主政治的
优势和特点充分发挥出来,为人类政治文明进步作出充满中国智慧的贡献!

\subsection{  七、坚定文化自信,推动社会主义文化繁荣兴盛}
\label{sec:orgaa1ab2b}

  文化是一个国家、一个民族的灵魂。文化兴国运
兴,文化强民族强。没有高度的文化自信,没有文化的繁荣兴盛,就没有中华民族伟大复兴。
要坚持中国特色社会主义文化发展道路,激发全民族文化创新创造活力,建设社会主义文化
强国。
  中国特色社会主义文化,源自于中华民族五千多年文明历史所孕育的中华优秀
传统文化,熔铸于党领导人民在革命、建设、改革中创造的革命文化和社会主义先进文化,
植根于中国特色社会主义伟大实践。发展中国特色社会主义文化,就是以马克思主义为指导,
坚守中华文化立场,立足当代中国现实,结合当今时代条件,发展面向现代化、面向世界、
面向未来的,民族的科学的大众的社会主义文化,推动社会主义精神文明和物质文明协调发
展。要坚持为人民服务、为社会主义服务,坚持百花齐放、百家争鸣,坚持创造性转化、创
新性发展,不断铸就中华文化新辉煌。

\subsubsection{  (一)牢牢掌握意识形态工作领导权。}
\label{sec:orgd3dcfe5}

意识形态
决定文化前进方向和发展道路。必须推进马克思主义中国化时代化大众化,建设具有强大凝
聚力和引领力的社会主义意识形态,使全体人民在理想信念、价值理念、道德观念上紧紧团
结在一起。要加强理论武装,推动新时代中国特色社会主义思想深入人心。深化马克思主义
理论研究和建设,加快构建中国特色哲学社会科学,加强中国特色新型智库建设。高度重视
传播手段建设和创新,提高新闻舆论传播力、引导力、影响力、公信力。加强互联网内容建
设,建立网络综合治理体系,营造清朗的网络空间。落实意识形态工作责任制,加强阵地建
设和管理,注意区分政治原则问题、思想认识问题、学术观点问题,旗帜鲜明反对和抵制各
种错误观点。

\subsubsection{  (二)培育和践行社会主义核心价值观。}
\label{sec:orgb702b98}

社会主义核心价值观是当代中国
精神的集中体现,凝结着全体人民共同的价值追求。要以培养担当民族复兴大任的时代新人
为着眼点,强化教育引导、实践养成、制度保障,发挥社会主义核心价值观对国民教育、精
神文明创建、精神文化产品创作生产传播的引领作用,把社会主义核心价值观融入社会发展
各方面,转化为人们的情感认同和行为习惯。坚持全民行动、干部带头,从家庭做起,从娃
娃抓起。深入挖掘中华优秀传统文化蕴含的思想观念、人文精神、道德规范,结合时代要求
继承创新,让中华文化展现出永久魅力和时代风采。

\subsubsection{  (三)加强思想道德建设。}
\label{sec:org70f9e16}

人民有
信仰,国家有力量,民族有希望。要提高人民思想觉悟、道德水准、文明素养,提高全社会
文明程度。广泛开展理想信念教育,深化中国特色社会主义和中国梦宣传教育,弘扬民族精
神和时代精神,加强爱国主义、集体主义、社会主义教育,引导人们树立正确的历史观、民
族观、国家观、文化观。深入实施公民道德建设工程,推进社会公德、职业道德、家庭美德、
个人品德建设,激励人们向上向善、孝老爱亲,忠于祖国、忠于人民。加强和改进思想政治
工作,深化群众性精神文明创建活动。弘扬科学精神,普及科学知识,开展移风易俗、弘扬
时代新风行动,抵制腐朽落后文化侵蚀。推进诚信建设和志愿服务制度化,强化社会责任意
识、规则意识、奉献意识。

\subsubsection{  (四)繁荣发展社会主义文艺。}
\label{sec:org8334af4}

社会主义文艺是人民的文艺,
必须坚持以人民为中心的创作导向,在深入生活、扎根人民中进行无愧于时代的文艺创造。
要繁荣文艺创作,坚持思想精深、艺术精湛、制作精良相统一,加强现实题材创作,不断推
出讴歌党、讴歌祖国、讴歌人民、讴歌英雄的精品力作。发扬学术民主、艺术民主,提升文
艺原创力,推动文艺创新。倡导讲品位、讲格调、讲责任,抵制低俗、庸俗、媚俗。加强文
艺队伍建设,造就一大批德艺双馨名家大师,培育一大批高水平创作人才。

\subsubsection{  (五)推动文化事业和文化产业发展。}
\label{sec:org7bc57ec}

满足人民过上美好生活的新期待,必须提供丰富的精神食粮。要
深化文化体制改革,完善文化管理体制,加快构建把社会效益放在首位、社会效益和经济效
益相统一的体制机制。完善公共文化服务体系,深入实施文化惠民工程,丰富群众性文化活
动。加强文物保护利用和文化遗产保护传承。健全现代文化产业体系和市场体系,创新生产
经营机制,完善文化经济政策,培育新型文化业态。广泛开展全民健身活动,加快推进体育
强国建设,筹办好北京冬奥会、冬残奥会。加强中外人文交流,以我为主、兼收并蓄。推进
国际传播能力建设,讲好中国故事,展现真实、立体、全面的中国,提高国家文化软实力。

  同志们!中国共产党从成立之日起,既是中国先进文化的积极引领者和践行者,又是中
华优秀传统文化的忠实传承者和弘扬者。当代中国共产党人和中国人民应该而且一定能够担
负起新的文化使命,在实践创造中进行文化创造,在历史进步中实现文化进步!

\subsection{  八、提高保障和改善民生水平,加强和创新社会治理}
\label{sec:org4388c75}

  全党必须牢记,为什么人的问题,是检
验一个政党、一个政权性质的试金石。带领人民创造美好生活,是我们党始终不渝的奋斗目
标。必须始终把人民利益摆在至高无上的地位,让改革发展成果更多更公平惠及全体人民,
朝着实现全体人民共同富裕不断迈进。

  保障和改善民生要抓住人民最关心最直接最现
实的利益问题,既尽力而为,又量力而行,一件事情接着一件事情办,一年接着一年干。坚
持人人尽责、人人享有,坚守底线、突出重点、完善制度、引导预期,完善公共服务体系,
保障群众基本生活,不断满足人民日益增长的美好生活需要,不断促进社会公平正义,形成
有效的社会治理、良好的社会秩序,使人民获得感、幸福感、安全感更加充实、更有保障、
更可持续。

\subsubsection{{\bfseries\sffamily DONE}   (一)优先发展教育事业。}
\label{sec:org5e266bc}
建设教育强国是中华民族伟大复兴的基础工程,必须把教育事业放在优先位置,加快教育现
代化,办好人民满意的教育。要全面贯彻党的教育方针,落实立德树人根本任务,发展素质
教育,推进教育公平,培养德智体美全面发展的社会主义建设者和接班人。推动城乡义务教
育一体化发展,高度重视农村义务教育,办好学前教育、特殊教育和网络教育,普及高中阶
段教育,努力让每个孩子都能享有公平而有质量的教育。完善职业教育和培训体系,*深化
产教融合、校企合作。*加快一流大学和一流学科建设,实现高等教育内涵式发展。健全学
生资助制度,使绝大多数城乡新增劳动力接受高中阶段教育、更多接受高等教育。支持和规
范社会力量兴办教育。加强师德师风建设,培养高素质教师队伍,倡导全社会尊师重教。办
好继续教育,加快建设学习型社会,大力提高国民素质。


\subsubsection{  (二)提高就业质量和人民收入水平。}
\label{sec:org512a2c7}

就业是最大的民生。要坚持就业优先战略和
积极就业政策,实现更高质量和更充分就业。大规模开展职业技能培训,注重解决结构性就
业矛盾,鼓励创业带动就业。提供全方位公共就业服务,促进高校毕业生等青年群体、农民
工多渠道就业创业。破除妨碍劳动力、人才社会性流动的体制机制弊端,使人人都有通过辛
勤劳动实现自身发展的机会。完善政府、工会、企业共同参与的协商协调机制,构建和谐劳
动关系。坚持按劳分配原则,完善按要素分配的体制机制,促进收入分配更合理、更有序。
鼓励勤劳守法致富,扩大中等收入群体,增加低收入者收入,调节过高收入,取缔非法收入。
坚持在经济增长的同时实现居民收入同步增长、在劳动生产率提高的同时实现劳动报酬同步
提高。拓宽居民劳动收入和财产性收入渠道。履行好政府再分配调节职能,加快推进基本公
共服务均等化,缩小收入分配差距。

\subsubsection{  (三)加强社会保障体系建设。}
\label{sec:orgf50409e}

按照兜底线、织密
网、建机制的要求,全面建成覆盖全民、城乡统筹、权责清晰、保障适度、可持续的多层次
社会保障体系。全面实施全民参保计划。完善城镇职工基本养老保险和城乡居民基本养老保
险制度,尽快实现养老保险全国统筹。完善统一的城乡居民基本医疗保险制度和大病保险制
度。完善失业、工伤保险制度。建立全国统一的社会保险公共服务平台。统筹城乡社会救助
体系,完善最低生活保障制度。坚持男女平等基本国策,保障妇女儿童合法权益。完善社会
救助、社会福利、慈善事业、优抚安置等制度,健全农村留守儿童和妇女、老年人关爱服务
体系。发展残疾人事业,加强残疾康复服务。坚持房子是用来住的、不是用来炒的定位,加
快建立多主体供给、多渠道保障、租购并举的住房制度,让全体人民住有所居。

\subsubsection{  (四)坚决打赢脱贫攻坚战。}
\label{sec:orgad4da45}

让贫困人口和贫困地区同全国一道进入全面小康社会是我们党的庄严
承诺。要动员全党全国全社会力量,坚持精准扶贫、精准脱贫,坚持中央统筹省负总责市县
抓落实的工作机制,强化党政一把手负总责的责任制,坚持大扶贫格局,注重扶贫同扶志、
扶智相结合,深入实施东西部扶贫协作,重点攻克深度贫困地区脱贫任务,确保到二〇二〇
年我国现行标准下农村贫困人口实现脱贫,贫困县全部摘帽,解决区域性整体贫困,做到脱
真贫、真脱贫。

\subsubsection{  (五)实施健康中国战略。}
\label{sec:org7809f61}

人民健康是民族昌盛和国家富强的重要标志。
要完善国民健康政策,为人民群众提供全方位全周期健康服务。深化医药卫生体制改革,全
面建立中国特色基本医疗卫生制度、医疗保障制度和优质高效的医疗卫生服务体系,健全现
代医院管理制度。加强基层医疗卫生服务体系和全科医生队伍建设。全面取消以药养医,健
全药品供应保障制度。坚持预防为主,深入开展爱国卫生运动,倡导健康文明生活方式,预
防控制重大疾病。实施食品安全战略,让人民吃得放心。坚持中西医并重,传承发展中医药
事业。支持社会办医,发展健康产业。促进生育政策和相关经济社会政策配套衔接,加强人
口发展战略研究。积极应对人口老龄化,构建养老、孝老、敬老政策体系和社会环境,推进
医养结合,加快老龄事业和产业发展。

\subsubsection{  (六)打造共建共治共享的社会治理格局。}
\label{sec:org34e5273}

加强
社会治理制度建设,完善党委领导、政府负责、社会协同、公众参与、法治保障的社会治理
体制,提高社会治理社会化、法治化、智能化、专业化水平。加强预防和化解社会矛盾机制
建设,正确处理人民内部矛盾。树立安全发展理念,弘扬生命至上、安全第一的思想,健全
公共安全体系,完善安全生产责任制,坚决遏制重特大安全事故,提升防灾减灾救灾能力。
加快社会治安防控体系建设,依法打击和惩治黄赌毒黑拐骗等违法犯罪活动,保护人民人身
权、财产权、人格权。加强社会心理服务体系建设,培育自尊自信、理性平和、积极向上的
社会心态。加强社区治理体系建设,推动社会治理重心向基层下移,发挥社会组织作用,实
现政府治理和社会调节、居民自治良性互动。

\subsubsection{  (七)有效维护国家安全。}
\label{sec:orgfc7d5c0}

国家安全是安
邦定国的重要基石,维护国家安全是全国各族人民根本利益所在。要完善国家安全战略和国
家安全政策,坚决维护国家政治安全,统筹推进各项安全工作。健全国家安全体系,加强国
家安全法治保障,提高防范和抵御安全风险能力。严密防范和坚决打击各种渗透颠覆破坏活
动、暴力恐怖活动、民族分裂活动、宗教极端活动。加强国家安全教育,增强全党全国人民
国家安全意识,推动全社会形成维护国家安全的强大合力。

  同志们!党的一切工作必须
以最广大人民根本利益为最高标准。我们要坚持把人民群众的小事当作自己的大事,从人民
群众关心的事情做起,从让人民群众满意的事情做起,带领人民不断创造美好生活!
\subsection{  九、加快生态文明体制改革,建设美丽中国}
\label{sec:org46e58eb}

  人与自然是生命共同体,人类必须尊重自然、
顺应自然、保护自然。人类只有遵循自然规律才能有效防止在开发利用自然上走弯路,人类
对大自然的伤害最终会伤及人类自身,这是无法抗拒的规律。

  我们要建设的现代化是
人与自然和谐共生的现代化,既要创造更多物质财富和精神财富以满足人民日益增长的美好
生活需要,也要提供更多优质生态产品以满足人民日益增长的优美生态环境需要。必须坚持
节约优先、保护优先、自然恢复为主的方针,形成节约资源和保护环境的空间格局、产业结
构、生产方式、生活方式,还自然以宁静、和谐、美丽。

\subsubsection{   (一)推进绿色发展。}
\label{sec:org1cad666}

加快建
立绿色生产和消费的法律制度和政策导向,建立健全绿色低碳循环发展的经济体系。构建市
场导向的绿色技术创新体系,发展绿色金融,壮大节能环保产业、清洁生产产业、清洁能源
产业。推进能源生产和消费革命,构建清洁低碳、安全高效的能源体系。推进资源全面节约
和循环利用,实施国家节水行动,降低能耗、物耗,实现生产系统和生活系统循环链接。倡
导简约适度、绿色低碳的生活方式,反对奢侈浪费和不合理消费,开展创建节约型机关、绿
色家庭、绿色学校、绿色社区和绿色出行等行动。

\subsubsection{  (二)着力解决突出环境问题。}
\label{sec:org3dfa87a}

坚持全民共治、源头防治,持续实施大气污染防治行动,打赢蓝天保卫战。加快水污染防治,实
施流域环境和近岸海域综合治理。强化土壤污染管控和修复,加强农业面源污染防治,开展
农村人居环境整治行动。加强固体废弃物和垃圾处置。提高污染排放标准,强化排污者责任,
健全环保信用评价、信息强制性披露、严惩重罚等制度。构建政府为主导、企业为主体、社
会组织和公众共同参与的环境治理体系。积极参与全球环境治理,落实减排承诺。

\subsubsection{  (三)加大生态系统保护力度。}
\label{sec:orgdb77b8e}

实施重要生态系统保护和修复重大工程,优化生态安全屏障体系,构
建生态廊道和生物多样性保护网络,提升生态系统质量和稳定性。完成生态保护红线、永久
基本农田、城镇开发边界三条控制线划定工作。开展国土绿化行动,推进荒漠化、石漠化、
水土流失综合治理,强化湿地保护和恢复,加强地质灾害防治。完善天然林保护制度,扩大
退耕还林还草。严格保护耕地,扩大轮作休耕试点,健全耕地草原森林河流湖泊休养生息制
度,建立市场化、多元化生态补偿机制。

\subsubsection{  (四)改革生态环境监管体制。}
\label{sec:org230f92c}

加强对生态文
明建设的总体设计和组织领导,设立国有自然资源资产管理和自然生态监管机构,完善生态
环境管理制度,统一行使全民所有自然资源资产所有者职责,统一行使所有国土空间用途管
制和生态保护修复职责,统一行使监管城乡各类污染排放和行政执法职责。构建国土空间开
发保护制度,完善主体功能区配套政策,建立以国家公园为主体的自然保护地体系。坚决制
止和惩处破坏生态环境行为。

  同志们!生态文明建设功在当代、利在千秋。我们要牢固
树立社会主义生态文明观,推动形成人与自然和谐发展现代化建设新格局,为保护生态环境
作出我们这代人的努力!

\subsection{  十、坚持走中国特色强军之路,全面推进国防和军队现代化}
\label{sec:org701b4e1}
  国防和军队建设正站在新的历史起点上。面对国家安全环境的深刻变化,面对强国强军
的时代要求,必须全面贯彻新时代党的强军思想,贯彻新形势下军事战略方针,建设强大的
现代化陆军、海军、空军、火箭军和战略支援部队,打造坚强高效的战区联合作战指挥机构,
构建中国特色现代作战体系,担当起党和人民赋予的新时代使命任务。

  适应世界新军
事革命发展趋势和国家安全需求,提高建设质量和效益,确保到二〇二〇年基本实现机械化,
信息化建设取得重大进展,战略能力有大的提升。同国家现代化进程相一致,全面推进军事
理论现代化、军队组织形态现代化、军事人员现代化、武器装备现代化,力争到二〇三五年
基本实现国防和军队现代化,到本世纪中叶把人民军队全面建成世界一流军队。

  加强
军队党的建设,开展“传承红色基因、担当强军重任”主题教育,推进军人荣誉体系建设,
培养有灵魂、有本事、有血性、有品德的新时代革命军人,永葆人民军队性质、宗旨、本色。
继续深化国防和军队改革,深化军官职业化制度、文职人员制度等重大政策制度改革,推进
军事管理革命,完善和发展中国特色社会主义军事制度。树立科技是核心战斗力的思想,推
进重大技术创新、自主创新,加强军事人才培养体系建设,建设创新型人民军队。全面从严
治军,推动治军方式根本性转变,提高国防和军队建设法治化水平。

  军队是要准备打
仗的,一切工作都必须坚持战斗力标准,向能打仗、打胜仗聚焦。扎实做好各战略方向军事
斗争准备,统筹推进传统安全领域和新型安全领域军事斗争准备,发展新型作战力量和保障
力量,开展实战化军事训练,加强军事力量运用,加快军事智能化发展,提高基于网络信息
体系的联合作战能力、全域作战能力,有效塑造态势、管控危机、遏制战争、打赢战争。

  坚持富国和强军相统一,强化统一领导、顶层设计、改革创新和重大项目落实,深化国
防科技工业改革,形成军民融合深度发展格局,构建一体化的国家战略体系和能力。完善国
防动员体系,建设强大稳固的现代边海空防。组建退役军人管理保障机构,维护军人军属合
法权益,让军人成为全社会尊崇的职业。深化武警部队改革,建设现代化武装警察部队。

  同志们!我们的军队是人民军队,我们的国防是全民国防。我们要加强全民国防教育,
巩固军政军民团结,为实现中国梦强军梦凝聚强大力量!

\subsection{  十一、坚持“一国两制”,推}
\label{sec:org8feeb4e}
进祖国统一

  香港、澳门回归祖国以来,“一国两制”实践取得举世公认的成功。事实
证明,“一国两制”是解决历史遗留的香港、澳门问题的最佳方案,也是香港、澳门回归后
保持长期繁荣稳定的最佳制度。

  保持香港、澳门长期繁荣稳定,必须全面准确贯彻
“一国两制”、“港人治港”、“澳人治澳”、高度自治的方针,严格依照宪法和基本法办
事,完善与基本法实施相关的制度和机制。要支持特别行政区政府和行政长官依法施政、积
极作为,团结带领香港、澳门各界人士齐心协力谋发展、促和谐,保障和改善民生,有序推
进民主,维护社会稳定,履行维护国家主权、安全、发展利益的宪制责任。

  香港、澳
门发展同内地发展紧密相连。要支持香港、澳门融入国家发展大局,以粤港澳大湾区建设、
粤港澳合作、泛珠三角区域合作等为重点,全面推进内地同香港、澳门互利合作,制定完善
便利香港、澳门居民在内地发展的政策措施。

  我们坚持爱国者为主体的“港人治港”、
“澳人治澳”,发展壮大爱国爱港爱澳力量,增强香港、澳门同胞的国家意识和爱国精神,
让香港、澳门同胞同祖国人民共担民族复兴的历史责任、共享祖国繁荣富强的伟大荣光。

  解决台湾问题、实现祖国完全统一,是全体中华儿女共同愿望,是中华民族根本利益所
在。必须继续坚持“和平统一、一国两制”方针,推动两岸关系和平发展,推进祖国和平统
一进程。

  一个中国原则是两岸关系的政治基础。体现一个中国原则的“九二共识”明
确界定了两岸关系的根本性质,是确保两岸关系和平发展的关键。承认“九二共识”的历史
事实,认同两岸同属一个中国,两岸双方就能开展对话,协商解决两岸同胞关心的问题,台
湾任何政党和团体同大陆交往也不会存在障碍。

  两岸同胞是命运与共的骨肉兄弟,是
血浓于水的一家人。我们秉持“两岸一家亲”理念,尊重台湾现有的社会制度和台湾同胞生
活方式,愿意率先同台湾同胞分享大陆发展的机遇。我们将扩大两岸经济文化交流合作,实
现互利互惠,逐步为台湾同胞在大陆学习、创业、就业、生活提供与大陆同胞同等的待遇,
增进台湾同胞福祉。我们将推动两岸同胞共同弘扬中华文化,促进心灵契合。

  我们坚
决维护国家主权和领土完整,绝不容忍国家分裂的历史悲剧重演。一切分裂祖国的活动都必
将遭到全体中国人坚决反对。我们有坚定的意志、充分的信心、足够的能力挫败任何形式的
“台独”分裂图谋。我们绝不允许任何人、任何组织、任何政党、在任何时候、以任何形式、
把任何一块中国领土从中国分裂出去!

  同志们!实现中华民族伟大复兴,是全体中国人
共同的梦想。我们坚信,只要包括港澳台同胞在内的全体中华儿女顺应历史大势、共担民族
大义,把民族命运牢牢掌握在自己手中,就一定能够共创中华民族伟大复兴的美好未来!
 
\subsection{   十二、坚持和平发展道路,推动构建人类命运共同体}
\label{sec:org4d77a9f}

  中国共产党是为中国人民谋幸
福的政党,也是为人类进步事业而奋斗的政党。中国共产党始终把为人类作出新的更大的贡
献作为自己的使命。

  中国将高举和平、发展、合作、共赢的旗帜,恪守维护世界和平、
促进共同发展的外交政策宗旨,坚定不移在和平共处五项原则基础上发展同各国的友好合作,
推动建设相互尊重、公平正义、合作共赢的新型国际关系。

  世界正处于大发展大变革
大调整时期,和平与发展仍然是时代主题。世界多极化、经济全球化、社会信息化、文化多
样化深入发展,全球治理体系和国际秩序变革加速推进,各国相互联系和依存日益加深,国
际力量对比更趋平衡,和平发展大势不可逆转。同时,世界面临的不稳定性不确定性突出,
世界经济增长动能不足,贫富分化日益严重,地区热点问题此起彼伏,恐怖主义、网络安全、
重大传染性疾病、气候变化等非传统安全威胁持续蔓延,人类面临许多共同挑战。

  我
们生活的世界充满希望,也充满挑战。我们不能因现实复杂而放弃梦想,不能因理想遥远而
放弃追求。没有哪个国家能够独自应对人类面临的各种挑战,也没有哪个国家能够退回到自
我封闭的孤岛。

  我们呼吁,各国人民同心协力,构建人类命运共同体,建设持久和平、
普遍安全、共同繁荣、开放包容、清洁美丽的世界。要相互尊重、平等协商,坚决摒弃冷战
思维和强权政治,走对话而不对抗、结伴而不结盟的国与国交往新路。要坚持以对话解决争
端、以协商化解分歧,统筹应对传统和非传统安全威胁,反对一切形式的恐怖主义。要同舟
共济,促进贸易和投资自由化便利化,推动经济全球化朝着更加开放、包容、普惠、平衡、
共赢的方向发展。要尊重世界文明多样性,以文明交流超越文明隔阂、文明互鉴超越文明冲
突、文明共存超越文明优越。要坚持环境友好,合作应对气候变化,保护好人类赖以生存的
地球家园。

  中国坚定奉行独立自主的和平外交政策,尊重各国人民自主选择发展道路
的权利,维护国际公平正义,反对把自己的意志强加于人,反对干涉别国内政,反对以强凌
弱。中国决不会以牺牲别国利益为代价来发展自己,也决不放弃自己的正当权益,任何人不
要幻想让中国吞下损害自身利益的苦果。中国奉行防御性的国防政策。中国发展不对任何国
家构成威胁。中国无论发展到什么程度,永远不称霸,永远不搞扩张。

  中国积极发展
全球伙伴关系,扩大同各国的利益交汇点,推进大国协调和合作,构建总体稳定、均衡发展
的大国关系框架,按照亲诚惠容理念和与邻为善、以邻为伴周边外交方针深化同周边国家关
系,秉持正确义利观和真实亲诚理念加强同发展中国家团结合作。加强同各国政党和政治组
织的交流合作,推进人大、政协、军队、地方、人民团体等的对外交往。

  中国坚持对
外开放的基本国策,坚持打开国门搞建设,积极促进“一带一路”国际合作,努力实现政策
沟通、设施联通、贸易畅通、资金融通、民心相通,打造国际合作新平台,增添共同发展新
动力。加大对发展中国家特别是最不发达国家援助力度,促进缩小南北发展差距。中国支持
多边贸易体制,促进自由贸易区建设,推动建设开放型世界经济。

  中国秉持共商共建
共享的全球治理观,倡导国际关系民主化,坚持国家不分大小、强弱、贫富一律平等,支持
联合国发挥积极作用,支持扩大发展中国家在国际事务中的代表性和发言权。中国将继续发
挥负责任大国作用,积极参与全球治理体系改革和建设,不断贡献中国智慧和力量。

  
同志们!世界命运握在各国人民手中,人类前途系于各国人民的抉择。中国人民愿同各国人
民一道,推动人类命运共同体建设,共同创造人类的美好未来!

\subsection{  十三、坚定不移全面从严治党,不断提高党的执政能力和领导水平}
\label{sec:org390d657}

  中国特色社会主义进入新时代,我们党一
定要有新气象新作为。打铁必须自身硬。党要团结带领人民进行伟大斗争、推进伟大事业、
实现伟大梦想,必须毫不动摇坚持和完善党的领导,毫不动摇把党建设得更加坚强有力。

  全面从严治党永远在路上。一个政党,一个政权,其前途命运取决于人心向背。人民群
众反对什么、痛恨什么,我们就要坚决防范和纠正什么。全党要清醒认识到,我们党面临的
执政环境是复杂的,影响党的先进性、弱化党的纯洁性的因素也是复杂的,党内存在的思想
不纯、组织不纯、作风不纯等突出问题尚未得到根本解决。要深刻认识党面临的执政考验、
改革开放考验、市场经济考验、外部环境考验的长期性和复杂性,深刻认识党面临的精神懈
怠危险、能力不足危险、脱离群众危险、消极腐败危险的尖锐性和严峻性,坚持问题导向,
保持战略定力,推动全面从严治党向纵深发展。

  新时代党的建设总要求是:坚持和加
强党的全面领导,坚持党要管党、全面从严治党,以加强党的长期执政能力建设、先进性和
纯洁性建设为主线,以党的政治建设为统领,以坚定理想信念宗旨为根基,以调动全党积极
性、主动性、创造性为着力点,全面推进党的政治建设、思想建设、组织建设、作风建设、
纪律建设,把制度建设贯穿其中,深入推进反腐败斗争,不断提高党的建设质量,把党建设
成为始终走在时代前列、人民衷心拥护、勇于自我革命、经得起各种风浪考验、朝气蓬勃的
马克思主义执政党。

\subsubsection{  (一)把党的政治建设摆在首位。}
\label{sec:orga2a837d}

旗帜鲜明讲政治是我们党作为马
克思主义政党的根本要求。党的政治建设是党的根本性建设,决定党的建设方向和效果。保
证全党服从中央,坚持党中央权威和集中统一领导,是党的政治建设的首要任务。全党要坚
定执行党的政治路线,严格遵守政治纪律和政治规矩,在政治立场、政治方向、政治原则、
政治道路上同党中央保持高度一致。要尊崇党章,严格执行新形势下党内政治生活若干准则,
增强党内政治生活的政治性、时代性、原则性、战斗性,自觉抵制商品交换原则对党内生活
的侵蚀,营造风清气正的良好政治生态。完善和落实民主集中制的各项制度,坚持民主基础
上的集中和集中指导下的民主相结合,既充分发扬民主,又善于集中统一。弘扬忠诚老实、
公道正派、实事求是、清正廉洁等价值观,坚决防止和反对个人主义、分散主义、自由主义、
本位主义、好人主义,坚决防止和反对宗派主义、圈子文化、码头文化,坚决反对搞两面派、
做两面人。全党同志特别是高级干部要加强党性锻炼,不断提高政治觉悟和政治能力,把对
党忠诚、为党分忧、为党尽职、为民造福作为根本政治担当,永葆共产党人政治本色。
 
\subsubsection{   (二)用新时代中国特色社会主义思想武装全党。}
\label{sec:orgcbbb2f9}

思想建设是党的基础性建设。革命理想高
于天。共产主义远大理想和中国特色社会主义共同理想,是中国共产党人的精神支柱和政治
灵魂,也是保持党的团结统一的思想基础。要把坚定理想信念作为党的思想建设的首要任务,
教育引导全党牢记党的宗旨,挺起共产党人的精神脊梁,解决好世界观、人生观、价值观这
个“总开关”问题,自觉做共产主义远大理想和中国特色社会主义共同理想的坚定信仰者和
忠实实践者。弘扬马克思主义学风,推进“两学一做”学习教育常态化制度化,以县处级以
上领导干部为重点,在全党开展“不忘初心、牢记使命”主题教育,用党的创新理论武装头
脑,推动全党更加自觉地为实现新时代党的历史使命不懈奋斗。

\subsubsection{  (三)建设高素质专业化干部队伍。}
\label{sec:org1cfee6d}

党的干部是党和国家事业的中坚力量。要坚持党管干部原则,坚持德才兼备、
以德为先,坚持五湖四海、任人唯贤,坚持事业为上、公道正派,把好干部标准落到实处。
坚持正确选人用人导向,匡正选人用人风气,突出政治标准,提拔重用牢固树立“四个意
识”和“四个自信”、坚决维护党中央权威、全面贯彻执行党的理论和路线方针政策、忠诚
干净担当的干部,选优配强各级领导班子。注重培养专业能力、专业精神,增强干部队伍适
应新时代中国特色社会主义发展要求的能力。大力发现储备年轻干部,注重在基层一线和困
难艰苦的地方培养锻炼年轻干部,源源不断选拔使用经过实践考验的优秀年轻干部。统筹做
好培养选拔女干部、少数民族干部和党外干部工作。认真做好离退休干部工作。坚持严管和
厚爱结合、激励和约束并重,完善干部考核评价机制,建立激励机制和容错纠错机制,旗帜
鲜明为那些敢于担当、踏实做事、不谋私利的干部撑腰鼓劲。各级党组织要关心爱护基层干
部,主动为他们排忧解难。

  人才是实现民族振兴、赢得国际竞争主动的战略资源。要
坚持党管人才原则,聚天下英才而用之,加快建设人才强国。实行更加积极、更加开放、更
加有效的人才政策,以识才的慧眼、爱才的诚意、用才的胆识、容才的雅量、聚才的良方,
把党内和党外、国内和国外各方面优秀人才集聚到党和人民的伟大奋斗中来,鼓励引导人才
向边远贫困地区、边疆民族地区、革命老区和基层一线流动,努力形成人人渴望成才、人人
努力成才、人人皆可成才、人人尽展其才的良好局面,让各类人才的创造活力竞相迸发、聪
明才智充分涌流。

\subsubsection{  (四)加强基层组织建设。}
\label{sec:orgf4996ed}

党的基层组织是确保党的路线方针政策和
决策部署贯彻落实的基础。要以提升组织力为重点,突出政治功能,把企业、农村、机关、
学校、科研院所、街道社区、社会组织等基层党组织建设成为宣传党的主张、贯彻党的决定、
领导基层治理、团结动员群众、推动改革发展的坚强战斗堡垒。党支部要担负好直接教育党
员、管理党员、监督党员和组织群众、宣传群众、凝聚群众、服务群众的职责,引导广大党
员发挥先锋模范作用。坚持“三会一课”制度,推进党的基层组织设置和活动方式创新,加
强基层党组织带头人队伍建设,扩大基层党组织覆盖面,着力解决一些基层党组织弱化、虚
化、边缘化问题。扩大党内基层民主,推进党务公开,畅通党员参与党内事务、监督党的组
织和干部、向上级党组织提出意见和建议的渠道。注重从产业工人、青年农民、高知识群体
中和在非公有制经济组织、社会组织中发展党员。加强党内激励关怀帮扶。增强党员教育管
理针对性和有效性,稳妥有序开展不合格党员组织处置工作。

\subsubsection{  (五)持之以恒正风肃纪。}
\label{sec:orgaa9fbdc}

我们党来自人民、植根人民、服务人民,一旦脱离群众,就会失去生命力。加强作风建设,
必须紧紧围绕保持党同人民群众的血肉联系,增强群众观念和群众感情,不断厚植党执政的
群众基础。凡是群众反映强烈的问题都要严肃认真对待,凡是损害群众利益的行为都要坚决
纠正。坚持以上率下,巩固拓展落实中央八项规定精神成果,继续整治“四风”问题,坚决
反对特权思想和特权现象。重点强化政治纪律和组织纪律,带动廉洁纪律、群众纪律、工作
纪律、生活纪律严起来。坚持开展批评和自我批评,坚持惩前毖后、治病救人,运用监督执
纪“四种形态”,抓早抓小、防微杜渐。赋予有干部管理权限的党组相应纪律处分权限,强
化监督执纪问责。加强纪律教育,强化纪律执行,让党员、干部知敬畏、存戒惧、守底线,
习惯在受监督和约束的环境中工作生活。

\subsubsection{  (六)夺取反腐败斗争压倒性胜利。}
\label{sec:org6c78f89}

人民群众
最痛恨腐败现象,腐败是我们党面临的最大威胁。只有以反腐败永远在路上的坚韧和执着,
深化标本兼治,保证干部清正、政府清廉、政治清明,才能跳出历史周期率,确保党和国家
长治久安。当前,反腐败斗争形势依然严峻复杂,巩固压倒性态势、夺取压倒性胜利的决心
必须坚如磐石。要坚持无禁区、全覆盖、零容忍,坚持重遏制、强高压、长震慑,坚持受贿
行贿一起查,坚决防止党内形成利益集团。在市县党委建立巡察制度,加大整治群众身边腐
败问题力度。不管腐败分子逃到哪里,都要缉拿归案、绳之以法。推进反腐败国家立法,建
设覆盖纪检监察系统的检举举报平台。强化不敢腐的震慑,扎牢不能腐的笼子,增强不想腐
的自觉,通过不懈努力换来海晏河清、朗朗乾坤。

\subsubsection{  (七)健全党和国家监督体系。}
\label{sec:orgd6e60ff}

增强
党自我净化能力,根本靠强化党的自我监督和群众监督。要加强对权力运行的制约和监督,
让人民监督权力,让权力在阳光下运行,把权力关进制度的笼子。强化自上而下的组织监督,
改进自下而上的民主监督,发挥同级相互监督作用,加强对党员领导干部的日常管理监督。
深化政治巡视,坚持发现问题、形成震慑不动摇,建立巡视巡察上下联动的监督网。深化国
家监察体制改革,将试点工作在全国推开,组建国家、省、市、县监察委员会,同党的纪律
检查机关合署办公,实现对所有行使公权力的公职人员监察全覆盖。制定国家监察法,依法
赋予监察委员会职责权限和调查手段,用留置取代“两规”措施。改革审计管理体制,完善
统计体制。构建党统一指挥、全面覆盖、权威高效的监督体系,把党内监督同国家机关监督、
民主监督、司法监督、群众监督、舆论监督贯通起来,增强监督合力。

\subsubsection{  (八)全面增强执政本领。}
\label{sec:org11736b6}

领导十三亿多人的社会主义大国,我们党既要政治过硬,也要本领高强。要增强
学习本领,在全党营造善于学习、勇于实践的浓厚氛围,建设马克思主义学习型政党,推动
建设学习大国。增强政治领导本领,坚持战略思维、创新思维、辩证思维、法治思维、底线
思维,科学制定和坚决执行党的路线方针政策,把党总揽全局、协调各方落到实处。增强改
革创新本领,保持锐意进取的精神风貌,善于结合实际创造性推动工作,善于运用互联网技
术和信息化手段开展工作。增强科学发展本领,善于贯彻新发展理念,不断开创发展新局面。
增强依法执政本领,加快形成覆盖党的领导和党的建设各方面的党内法规制度体系,加强和
改善对国家政权机关的领导。增强群众工作本领,创新群众工作体制机制和方式方法,推动
工会、共青团、妇联等群团组织增强政治性、先进性、群众性,发挥联系群众的桥梁纽带作
用,组织动员广大人民群众坚定不移跟党走。增强狠抓落实本领,坚持说实话、谋实事、出
实招、求实效,把雷厉风行和久久为功有机结合起来,勇于攻坚克难,以钉钉子精神做实做
细做好各项工作。增强驾驭风险本领,健全各方面风险防控机制,善于处理各种复杂矛盾,
勇于战胜前进道路上的各种艰难险阻,牢牢把握工作主动权。

  同志们!伟大的事业必须
有坚强的党来领导。只要我们党把自身建设好、建设强,确保党始终同人民想在一起、干在
一起,就一定能够引领承载着中国人民伟大梦想的航船破浪前进,胜利驶向光辉的彼岸!

 
 同志们!中华民族是历经磨难、不屈不挠的伟大民族,中国人民是勤劳勇敢、自强不息的
伟大人民,中国共产党是敢于斗争、敢于胜利的伟大政党。历史车轮滚滚向前,时代潮流浩
浩荡荡。历史只会眷顾坚定者、奋进者、搏击者,而不会等待犹豫者、懈怠者、畏难者。全
党一定要保持艰苦奋斗、戒骄戒躁的作风,以时不我待、只争朝夕的精神,奋力走好新时代
的长征路。全党一定要自觉维护党的团结统一,保持党同人民群众的血肉联系,巩固全国各
族人民大团结,加强海内外中华儿女大团结,团结一切可以团结的力量,齐心协力走向中华
民族伟大复兴的光明前景。

  青年兴则国家兴,青年强则国家强。青年一代有理想、有
本领、有担当,国家就有前途,民族就有希望。中国梦是历史的、现实的,也是未来的;是
我们这一代的,更是青年一代的。中华民族伟大复兴的中国梦终将在一代代青年的接力奋斗
中变为现实。全党要关心和爱护青年,为他们实现人生出彩搭建舞台。广大青年要坚定理想
信念,志存高远,脚踏实地,勇做时代的弄潮儿,在实现中国梦的生动实践中放飞青春梦想,
在为人民利益的不懈奋斗中书写人生华章!

  大道之行,天下为公。站立在九百六十多万平方公里的广袤土地上,吸吮着五千多年中
华民族漫长奋斗积累的文化养分,拥有十三亿多中国人民聚合的磅礴之力,我们走中国特色
社会主义道路,具有无比广阔的时代舞台,具有无比深厚的历史底蕴,具有无比强大的前进
定力。全党全国各族人民要紧密团结在党中央周围,高举中国特色社会主义伟大旗帜,锐意
进取,埋头苦干,为实现推进现代化建设、完成祖国统一、维护世界和平与促进共同发展三
大历史任务,为决胜全面建成小康社会、夺取新时代中国特色社会主义伟大胜利、实现中华
民族伟大复兴的中国梦、实现人民对美好生活的向往继续奋斗!



\subsection{  ---------------   十九大报告:习近平直抵人心的这19句话   }
\label{sec:orgc569934}
2017年10月18日,中国共产党第十九次全国代表大会在北京隆重召开,习近平代表第十八届
中央委员会向大会作报告,新华网摘编了报告中直抵人心的19句话,以飨广大网友!

\subsubsection{     1、中国共产党人的初心和使命,就是为中国人民谋幸福,为中华民族谋复兴。}
\label{sec:orgf643fc8}

\subsubsection{    2、中国特色社会主义进入了新时代。}
\label{sec:orgda64bd6}

\subsubsection{    3、中华民族迎来了从站起来、富起来到强起来的伟大飞跃。}
\label{sec:org090392e}

\subsubsection{    4、我国社会主要矛盾已经转化为人民日益增长的美好生活需要和不平衡不充分的发展}
\label{sec:org6cd6864}
之间的矛盾。


\subsubsection{    5、中华民族伟大复兴,绝不是轻轻松松、敲锣打鼓就能实现的。}
\label{sec:orged2c1ec}

\subsubsection{    6、把人民对美好生活的向往作为奋斗目标,依靠人民创造历史伟业。}
\label{sec:org5722b46}

\subsubsection{    7、乘势而上开启全面建设社会主义现代化国家新征程,向第二个百年奋斗目标进军。}
\label{sec:org326f0b5}

\subsubsection{    8、中华民族将以更加昂扬的姿态屹立于世界民族之林。}
\label{sec:org853148a}

\subsubsection{    9、中国开放的大门不会关闭,只会越开越大。}
\label{sec:org0113d04}

\subsubsection{    10、有事好商量,众人的事情由众人商量,是人民民主的真谛。}
\label{sec:org08766b9}

\subsubsection{    11、把人民利益摆在至高无上的地位。}
\label{sec:org34bc3ab}

\subsubsection{    12、让全体人民住有所居。}
\label{sec:org062becb}

\subsubsection{    13、打赢蓝天保卫战。}
\label{sec:org0aca712}
  
\subsubsection{      14、我们的军队是人民军队,我们的国防是全民国防。}
\label{sec:orga5c56a8}

\subsubsection{    15、我们有坚定的意志、充分的信心、足够的能力挫败任何形式的“台独”分裂图谋。}
\label{sec:orga9a25a9}

\subsubsection{    16、人民群众反对什么、痛恨什么,我们就要坚决防范和纠正什么。}
\label{sec:org15b396c}

\subsubsection{    17、不管腐败分子逃到哪里,都要缉拿归案、绳之以法。}
\label{sec:org0b28b88}

\subsubsection{    18、党始终同人民想在一起、干在一起。}
\label{sec:org2c25e14}

\subsubsection{    19、中国梦是我们这一代的更是青年一代的。}
\label{sec:orge43bec5}

该文章《习总书记十九大报告全文》来源于出国留学网,网址:\url{http://www.liuxue86.com/a/3351261.html}
\subsection{{\bfseries\sffamily DONE} \href{http://news.ifeng.com/a/20171027/52824191\_0.shtml}{中央政治局会议研究部署学习宣传贯彻十九大精神}}
\label{sec:org0263e62}
原标题:中共中央政治局召开会议研究部署学习宣传贯彻党的十九大精神审议《中共中央政
治局关于加强和维护党中央集中统一领导的若干规定》和《中共中央政治局贯彻落实中央八
项规定的实施细则》 中共中央总书记习近平主持会议

央视网消息(新闻联播):十九届中共中央政治局27日召开会议,研究部署学习宣传贯彻党
的十九大精神,审议《中共中央政治局关于加强和维护党中央集中统一领导的若干规定》和
《中共中央政治局贯彻落实中央八项规定的实施细则》。中共中央总书记习近平主持会议。

会议强调,党中央集中统一领导是党的领导的最高原则,从根本上关乎党和国家前途命运、
关乎人民根本利益。加强和维护党中央集中统一领导是全党共同的政治责任,首先是中央领
导层的政治责任。中央政治局要带头树立政治意识、大局意识、核心意识、看齐意识,严格
遵守党章和党内政治生活准则,全面落实党的十九大关于加强和维护党中央集中统一领导的
各项要求,自觉在以习近平同志为核心的党中央集中统一领导下履行职责、开展工作,坚决
维护习近平总书记作为党中央的核心、全党的核心的地位,凝聚全党意志,激发全国各族人
民充满信心朝着实现“两个一百年”奋斗目标、建设社会主义现代化强国、实现中华民族伟
大复兴中国梦的宏伟目标奋勇前进。

会议一致同意中央政治局关于加强和维护党中央集中统一领导的若干规定。中央政治局全体
同志要牢固树立“四个意识”,坚定“四个自信”,主动将重大问题报请党中央研究,认真
落实党中央决策部署并及时报告落实的重要进展;要带头执行党的干部政策,结合分管工作
负责任地向党中央推荐干部;要对党忠诚老实,自觉同违反党章、破坏党的纪律、危害党中
央集中领导和团结统一的言行作斗争,认真履行所分管部门、领域或所在地区的全面从严治
党责任;要坚持每年向党中央和总书记书面述职;要严格遵守有关宣传报道的规定。中央书
记处和中央纪律检查委员会、全国人大常委会党组、国务院党组、全国政协党组、最高人民
法院党组、最高人民检察院党组每年向中央政治局常委会、中央政治局报告工作。

会议认为,作风建设永远在路上。贯彻执行中央八项规定是关系我们党会不会脱离群众,能
不能长期执政、能不能很好履行执政使命的大问题。党的十九大对持之以恒正风肃纪作出新
部署,我们必须坚持以上率下,巩固和拓展落实中央八项规定精神成果,坚持不懈改作风转
作风,让党的作风全面好起来,确保党同人民想在一起、干在一起,始终保持党同人民群众
的血肉联系。

会议指出,修订后的实施细则,坚持以习近平新时代中国特色社会主义思想为指导,贯彻落
实党的十九大对党的作风建设的新部署新要求,坚持问题导向,根据这几年中央八项规定实
施过程中遇到的新情况新问题,着重对改进调查研究、精简会议活动、精简文件简报、规范
出访活动、改进新闻报道、厉行勤俭节约等方面内容作了进一步规范、细化和完善,更加切
合工作实际,增强了指导性和操作性。中央政治局的同志要带头弘扬党的优良作风,严格执
行中央八项规定,为全党作出表率。

会议强调,学习宣传贯彻党的十九大精神是当前和今后一段时期全党全国的首要政治任务。
要把全党全国各族人民思想统一到党的十九大精神上来,把力量凝聚到实现党的十九大确定
的各项任务上来。要引导广大干部群众认真研读党的十九大报告和党章,准确领会把握党的
十九大精神的思想精髓、核心要义,原原本本、原汁原味学习好党的十九大精神。要在全面
系统的基础上突出重点、抓住关键,把着力点聚焦到习近平新时代中国特色社会主义思想是
党必须长期坚持的指导思想上,聚焦到5年来党和国家事业取得历史性成就和发生历史性变
革上,聚焦到作出中国特色社会主义进入了新时代、我国社会主要矛盾已经转化为人民日益
增长的美好生活需要和不平衡不充分的发展之间的矛盾等重大论断的深远影响上,聚焦到贯
彻落实党的十九大的重大决策部署上,聚焦到习近平总书记是全党拥护、人民爱戴、当之无
愧的党的领袖上。

会议强调,要把学习党的十九大精神作为党的理论武装工作的重点任务,面向全体党员开展
多形式、分层次、全覆盖的学习培训。要把学习党的十九大精神作为党校、干部学院、行政
学院教育培训的必修课,作为学校思想政治教育和课堂教学的重要内容。要组织开展内容丰
富、形式多样的宣传教育活动,注重宣传各地区各部门学习贯彻的具体举措和实际行动,注
重反映基层干部群众学习贯彻的典型事迹和良好风貌,广泛吸引干部群众积极参与,在全社
会形成学习贯彻党的十九大精神的浓厚氛围。要加强对外宣传,针对国际社会关切,积极宣
介党的十九大精神。

会议指出,要组织开展集中宣讲活动,推动党的十九大精神进企业、进农村、进机关、进校
园、进社区、进军营、进网站。领导干部要带头学、带头讲、带头干,把党的十九大精神讲
清楚、讲明白,让老百姓听得懂、能领会、可落实,推动党的理论创新成果走近群众,凝聚
党心民心、扩大社会共识。

会议强调,学习宣传贯彻党的十九大精神,要推动全党牢固树立政治意识、大局意识、核心
意识、看齐意识,在政治立场、政治方向、政治原则、政治道路上同以习近平同志为核心的
党中央保持高度一致,自觉维护以习近平同志为核心的党中央权威和集中统一领导。要大力
弘扬马克思主义学风,切实提高推动发展、解决问题的能力,坚定自觉地把党中央各项决策
部署落到实处。
\subsection{{\bfseries\sffamily DONE} \href{http://news.ifeng.com/a/20171028/52827456\_0.shtml}{中共高层怎样选人用人?这些细节有意味}}
\label{sec:org442298e}
【解局】中共高层怎样选人用人?这些细节有意味

昨天我们就一份“不太好懂”的名单排序做了一些分析,虽然是技术性的,但是后台岛友的
反响非常热烈。的确,十九大这个重要的历史事件,是观察分析中共政治运转的极佳窗口。

近期还有两篇重磅通稿具有这样的“窗口”功效。一篇是26日的《党的新一届中央领导机构
诞生记》,一篇是24日的《新一届中共中央委员会和中央纪律检查委员会诞生记》。

从新一届中央领导机构(包括总书记、中央政治局常委、政治局委员、书记处成员、中央军
委成员、中央纪委领导机构),到中央委员会(委员204人、候补委员172人)、中央纪律检
查委员会(133人),构成了未来一段时间内中共党内的“顶层架构”。通过党内民主选举
进入这个行列的,都是将在未来一段时间对中国各地方各领域举足轻重的人物。

对于一个政党、一套政治体制来说,选人用人是相当核心的环节。因此,通过新华社这两篇
《诞生记》,我们就可以了解很多中共政治运转的规则和“密码”。

\subsection{标准}
\label{sec:orgc457b85}

选人用人,首先要有个标准。综合两篇通稿,大致可以用一个词来概括:“政治家集团”。

《“两委”诞生记》里写,“党中央明确提出,新一届中央委员会肩负着带领全党和全国人
民全面建成小康社会的使命”,跟了三段描述语,但落脚点都是“应当是政治家集团”。

《中央领导机构诞生记》里则更明确地写道,“坚持政治家集团标准,坚持五湖四海、任人
唯贤……”;推荐人选应具备的条件中,排在首位的就是,推荐人选应当是对党忠诚、信念
坚定、与党中央保持高度一致的“合格马克思主义政治家”。

也就是说,无论是选“两委”成员、还是选中央领导机构成员,都是要致力于把中共党内的
领导层,建设成一个“政治家集团”。

这个政治家集团是怎样的呢?通稿里的原话表述是:“习近平总书记明确指出……选拔一批
政治强、懂专业、善治理、敢担当、作风正的领导骨干”。

兼具这几条的,无疑是事业需要的好干部。能够进入“两委”、中央领导机构考察视野的人
选,无疑也都是中共党内的政治精英。而在他们走上更高的领导位置,为党和国家做出更大
贡献之前,还需要“过三关”——政治关、廉洁关、能力关。



\subsection{过关}
\label{sec:orgdf50eab}

排在首位第一关,“政治标准”。政治上“不过关”的,“一票否决”。文中说得明白:
“不能同党中央保持高度一致、自觉维护党中央权威和集中统一领导的,一票否决;对党中
央决策部署态度暧昧甚至心怀不满、另搞一套的,一票否决;骨头不硬、见风使舵、爱惜羽
毛、当所谓’开明绅士’、不敢担当的,一票否决……”

根据这些“红线”、“硬杠杠”,“那些在大是大非面前立场坚定的干部,那些政治可靠、
敢于担当的干部,那些强力推动本单位改革事业的干部,脱颖而出,进入两委人选考察范
围”;“凡是政治上靠不住的,不仅没有被选上来,而且还被坚决调整下去”。

第二,廉洁关,“铁的原则”。“对每个人选都做到干部档案必审、个人事项报告必核、纪
检监察机关意见必听、线索具体具有可查性信访举报必查……”《诞生记》透露,“在某地,
考察组了解到两名呼声较高的干部涉及廉政问题后,坚决将其排除在会议推荐参考名单外。
考察结束后,这两名干部因涉嫌职务犯罪被立案侦查。”

第三,能力关,“以事择人”。《诞生记》透露,某地一名女干部,原被列为考察对象,但
考察组感到其能力素质离“两委”人选要求有一定差距,最终没有将其列入,而是将另一名
政治素质好、长期在艰苦地区、成绩比较突出的干部列入名单。

“实事求是、以事择人”、“革除了唯结构、唯比例的旧观念”,“带来了重人才、重专业
的新变化”,这几句话是有深意的。



\subsection{程序}
\label{sec:org3337747}

有了标准,还要遵循严格的组织程序。

先说“两委”。通稿披露,“两委”人事工作从去年2月就启动,前后经过一年半左右时间。

其程序,首先是开中央政治局常委会专门研究,成立“十九大干部考察领导小组”,习近平
任组长;2016年6月,常委会、政治局会审议通过《关于认真做好十九届“两委”人事准备
工作的意见》;2016年7月,领导小组审议通过《十九届“两委”人选考察工作总体方案》,
对提名名额分配、考察方法步骤、组织实施等作出具体安排。

走完这几步,才是“组建考察组”。从去年7月到今年6月,考察组分批次对31个省区市、
124个中央和国家机关、中央金融企业、在京中央企业、全军29个大单位和军委机关战区级
部门进行考察。

这关键的考察环节,还细分为4个步骤:1、“综合分析研究,确定考察单位”;2、“谈话
调研和推荐,确定考察对象”;3、“深入考察,提出遴选对象”;4、“听取考察组汇报,
提出建议名单”。

再说中央领导机构。

今年初,习近平听取常委同志意见,形成共识,启动这项工作;接着,是4月24日常委会,
专门研究并讨论通过《关于十九届中央领导机构人选酝酿工作谈话调研安排方案》,确定此
工作在习近平直接领导下进行,明确原则和条件。之后,才是中央有关领导在一定范围内开
始进行谈话调研。

这两项工作到今年9月份后,开始进入一个进度节奏。9月25日,中央政治局常委会提出新一
届中央领导机构组成人选方案,以及“两委”候选人预备人选建议名单。9月29日,习近平
总书记主持召开中央政治局会议,审议通过这两份建议名单。

此时,一位中共政治精英是否能够进入“两委”,还要看最后两步。

尤为关键的一步是,十九大上各代表团以差额选举方式进行预选。数据显示,“提名十九届
中央委员候选人222名,当选204名,差额比例8.8\%;提名候补中央委员候选人189名,当选
172名,差额比例9.9\%;提名中央纪委委员候选人144名,当选133名,差额比例8.3\%。”

只有不被差额掉的候选人,才能真正走到正式选举的最后程序。



\subsection{革除}
\label{sec:orge6295b1}

选贤任能,是中国政治传统中一个非常突出的优点、也是每个政权力求达到的目标。中共继
承发展了这一传统。

体现在这次换届中,是一个重大创新:以“谈话调研”的方式,在一定范围内面对面征求意
见建议。以前的一些做法,因为其弊端突出而被摒弃——主要是“大会海推”、“划票打勾”
和中央领导机构酝酿时的“会议推荐”等办法。

以往,“两委”人选考察工作的第一步,就是召开省区市党委全委(扩大)会议,进行投票
推荐,“动辄几百人’大呼隆’投票”。上来就投票,可想而知对推荐人的了解情况一定是
不够的。

用通稿的话说,“由于信息不对称,很多人投票随意、导致民意失真,还有很多人投关系票、
人情票,过程中还有人拉票、贿选,甚至催生出’期权’’期货’交易”。通稿专门点出了
曾经在这样的环节中出问题的落马高官——“如周永康、孙政才、令计划等人,就曾利用会议
推荐搞拉票贿选等非组织活动。”

“期权”、“期货”交易这样的经济术语用在投票选人领域,很直白了,可以想见那种选举
和回报的“潜关系”。“这样的民主变了味,走偏了方向。干部不负责任,党组织卸掉了责
任,党的领导被弱化。”如果读者还记得《巡视利剑》里对于辽宁贿选案当事人的采访拍摄
细节,应该会更有感触。因此,十九大报告明确写:“自觉抵制商品交换原则对党内生活的
侵蚀,营造风清气正的良好政治生态”。

通稿披露,习近平总书记明确提出,不搞“大会海推”“划票打勾”,选人用人,党组织必
须加强领导、把好关。



\subsection{创新}
\label{sec:orgba23809}

不搞“大会海投”那一套,是不是就不要投票了?是不是不民主?

“不是不要民主,而是要将民主的真实性、有效性充分发挥出来,进一步提升党内民主的质
量和实效。”

通稿里有这样一个例子:在对某省一名干部进行考察时,有同志对其工作方式持有不同看法。
但考察组经过调查发现,这名干部“不怕得罪人”,一举关停1100多家污染小企业、小作坊,
“引起一些既得利益人士的非议,却得到了广大干部群众的认可”。“如果按过去先进行投
票推荐,这名干部可能就被挡在考察视线之外了。”

投票看上去很民主。但是如果只是拘泥于形式,就会“以票取人”、“唯票论”,既可能过
分看重选票导致选举走形变味,也可能唯票取人导致用人疏漏。因此,“不是不要票,而是
不’唯票’”。

所以,本次采取的核心方式是“谈话调研”:先进行谈话调研、听取意见,提出参考名单后,
再进行会议推荐。

具体怎么做?“面对面谈话,广泛深入听取意见”,“真名实姓、实名推荐”,“非常深入,
不定调子、也不限时间,问得非常细、非常耐心”。据统计,考察组平均每个省谈话1500多
人次,比过去参加会议推荐的人数大大增加。

为酝酿中央领导机构人选,2017年4月下旬至6月,习近平专门安排时间,分别与现任党和国
家领导同志、中央军委委员、党内老同志谈话,充分听取意见,前后谈了57人;中央相关领
导同志分别听取了正省部级、军队正战区职党员主要负责同志和其他十八届中央委员共258
人的意见;中央军委负责同志分别听取了现任正战区职领导同志和军委机关战区级部门主要
负责同志共32人的意见。

“严格标准、事业为上”,“党和国家领导职务也不是’铁椅子’’铁帽子’,符合年龄的
也不一定当然继续提名,主要根据人选政治表现、廉洁情况和事业需要,能留能转、能上能
下”。



谈话调研这种创新方式,和严格的组织程序配合起来,体现的就是党内民主和民主集中制原
则。先民主、后集中,再民主、再集中,在这个过程中,把真正对党和国家有心有力的政治
精英按照程序挑选出来,放在合适的重要位置上。

“没有暗潮涌动,始终风清气正。充分沟通酝酿,凝聚全党意志”。

中国这么大的国家,能够保持稳定发展强大,根本原因在于执政党的强大。执政党的强大,
首先是党中央的强大,保证有一个统一的方向和意志。同时,也需要组织运转的科学有序有
效,否则中央意志得不到有效贯彻。革命年代也好,改革开放也好,新时代也好,说中共强,
就强在这里。一个团结在核心周围的政治家集团、形成统一意志和行动的战斗集体,也是值
得期待的。

两份诞生记中有很多细节,大多来自当事人的经历和感受,非常有趣。已经写了很多了,限
于篇幅,岛叔就不展开了,大家有时间可以找来原文好好读读。

\subsection{{\bfseries\sffamily DONE} \href{https://mp.weixin.qq.com/s?\_\_biz=MjM5MDIwODkyMA==\&mid=2649730116\&idx=1\&sn=dd8919eb9459463c3dfa10b25db2a9c7\&chksm=be534ab38924c3a5b8fbe520293c36a1a4f47752a23e5be835d87c7ba8ed10ab03a51c561e08\#rd}{这篇新华社报道,每段信息量都好大}}
\label{sec:orgc13d25d}
编者按:10月26日,新华社刊发重磅文章《领航新时代的坚强领导集体——党的新一届中央领
导机构产生纪实》,披露了党的新领导班子的产生过程。

\subsection{文章提到,党的十七大、十八大探索采取了会议推荐的方式,但由于过度强调票的分量,带}
\label{sec:org8813259}
来了一些弊端:*有的同志在会议推荐过程中简单“划票打勾”,导致投票随意、民意失真,
甚至投关系票、人情票。*周永康、孙政才、令计划等就曾利用会议推荐搞拉票贿选等非组织
活动。从2017年初开始,习近平就“如何酝酿产生新一届中央领导机构人选问题”认真听取
了中央政治局常委同志的意见,*最终大家一致赞成采取谈话调研的方式进行选拔。*

2017年5月下旬,一位省部级领导干部接到通知,来中南海参加组织谈话。一到候谈室,3份
材料已经摆在桌上——《谈话调研有关安排》《现任党和国家领导人党员同志名册》《正省部
级党员领导干部名册》。这位干部事后感慨到:“没有限定推荐人数,了解多少就谈多少,
怎么想就怎么谈,实事求是,畅所欲言。”

以下为新华社全文:

新华社北京10月26日消息,在决胜全面建成小康社会、夺取新时代中国特色社会主义伟大胜
利的征程上,这一刻无疑具有标志性的历史意义——

2017年10月25日上午,北京人民大会堂。

党的十九届一中全会选举产生了25人组成的十九届中央政治局,选举习近平、李克强、栗战
书、汪洋、王沪宁、赵乐际、韩正为中央政治局常委,选举习近平为中央委员会总书记;通
过了中央书记处成员;决定了中央军事委员会组成人员;批准了十九届中央纪委一次全会选
举产生的领导机构。


10月25日,中国共产党第十九届中央委员会第一次全体会议在北京人民大会堂举行。 新华
社记者 刘卫兵 摄

\subsection{这是8900多万名党员的领路人,这是13亿多人民的主心骨。}
\label{sec:orgb4dc052}

\subsection{不忘初心,继续前进!以习近平同志为核心的新一届党中央领航中国,扬帆再出发。}
\label{sec:orgd6aa664}

万山磅礴看主峰——在强国强军新征程上,在民族复兴关键当口,确立习近平总书记为党中央
和全党的核心,确立习近平新时代中国特色社会主义思想为党的指导思想,是党心所向、民
心所向

\subsection{一个波澜壮阔的年代,必定会有激荡人心的时刻。}
\label{sec:orgc0bbec2}

一年前的金秋北京,党的十八届六中全会确立了习近平总书记在党中央和全党的核心地位,
人民大会堂雷鸣般的掌声音犹在耳。

这是全党同志发自内心的崇敬爱戴,这是亿万人民追求梦想的情感认同。

党的十八大以来,以习近平同志为核心的党中央迎难而上,开拓进取,取得了改革开放和社
会主义现代化建设的历史性成就,推动党和国家事业发生了历史性变革:

经济建设取得重大成就,全面深化改革取得重大突破,民主法治建设迈出重大步伐,思想文
化建设取得重大进展,人民生活不断改善,生态文明建设成效显著,强军兴军开创新局面,
港澳台工作取得新进展,全方位外交布局深入展开,全面从严治党成效卓著……

5年来的成就是全方位的、开创性的,5年来的变革是深层次的、根本性的。中国特色社会主
义进入了新时代。

新的实践孕育着新的思想。在坚持马克思主义基本原理的基础上,以习近平同志为核心的党
中央,创造性地坚持和发展了科学社会主义,开辟了21世纪中国马克思主义发展的新境界,
创立了习近平新时代中国特色社会主义思想。

\subsection{办大事、解难题,挽狂澜、开新局。}
\label{sec:org7f06e6d}

进行伟大斗争、建设伟大工程、推进伟大事业、实现伟大梦想,一系列伟大实践和理论创新,
倾注着中国共产党人的理想信念和人民情怀,彰显着党中央的政治勇气和责任担当,展现着
习近平总书记的雄才伟略和领袖风范。习近平总书记励精图治、力挽狂澜,统筹内政外交国
防,统领治党治国治军,为党和国家长治久安不畏艰险、殚精竭虑,赢得了党心、军心、民
心,是新时代中国共产党当之无愧的坚强核心。

在选举党的十九大代表时,习近平同志全票当选。在选举十九届中央委员会委员时,习近平
同志全票当选。在十九届一中全会选举新一届中央领导集体时,习近平同志再次全票当选中
央委员会总书记。雷鸣般的掌声一次次响起,经久不息……

一张张选票代表党心民意、一次次掌声传递信任期望——有习近平总书记这个党的核心、军队
统帅、人民领袖,是党之大幸、军之大幸、民之大幸!是实现“两个一百年”奋斗目标、实
现中华民族伟大复兴的中国梦、实现人民对美好生活的向往的希望所在、力量所在、胜利所
在!

\subsection{以党的十九大胜利召开为标志,中国步入“两个一百年”奋斗目标的历史交汇期——}
\label{sec:orgd1fe812}

\subsection{到2020年,既要全面建成小康社会、实现第一个百年奋斗目标,又要乘势而上开启全面建设}
\label{sec:orgef31654}
社会主义现代化国家新征程,向第二个百年奋斗目标进军。

与此同时,国内外形势正在发生深刻复杂变化,我国发展仍处于重要战略机遇期,前景十分
光明,挑战也十分严峻。迎难而上,进行具有许多新的历史特点的伟大斗争,走好新时代的
长征路,党的领导至关重要,领导核心尤为重要。

沧海横流显砥柱,万山磅礴看主峰。

在这个承前启后的关键时期,全党全军全国各族人民都有一个共同期盼,就是希望选出一个
好的中央领导集体,在习近平总书记领导下,持续巩固成果、攻坚克难、奋勇向前,谱写社
会主义现代化新征程的壮丽篇章。

千秋伟业聚英才——着眼于党的事业继往开来和国家长治久兴,以习近平同志为核心的党中央
统筹谋划新一届中央领导机构人选酝酿提名工作

\subsection{千秋大业,关键在人。治国之要,首在用人。}
\label{sec:org6a9d2e8}

新的时代,呼唤新的坚强领导集体。

如何产生以习近平总书记为核心的新一届中央领导集体,领航中国继往开来,全党期待,全
民关注,世界瞩目。

党章规定,党的中央政治局、中央政治局常务委员会和中央委员会总书记,由中央委员会全
体会议选举。

根据这一规定,党的十九大要选举产生新一届中央委员会,十九届一中全会要选举产生新一
届中央领导机构。

对此,党中央高度重视。

习近平总书记指出,我们党是一个拥有8900多万名党员的大党,在一个十几亿人口的大国执
政,肩膀上的担子重、责任大,必须组成一个政治坚定、团结统一、坚强有力、奋发有为的
中央领导集体。

在以习近平同志为核心的党中央统筹谋划下,新一届中央领导机构人选的酝酿提名工作有序
展开……

2017年从年初开始,习近平总书记就如何酝酿产生新一届中央领导机构人选问题,认真听取
中央政治局常委同志的意见。

大家一致赞成,在总结党的十六大、十七大、十八大有关做法的基础上,借鉴十九届“两
委”人选和省级党委换届考察工作的做法和经验,采取谈话调研的方式,就新一届中央政治
局、常委会、书记处组成人选,中央军委组成人选以及需要统筹考虑的国务院领导成员人选
和全国人大、全国政协党内新提拔人选等,在一定范围内面对面听取推荐意见和建议。

2017年4月24日,习近平总书记主持召开中央政治局常委会会议进行专门研究,讨论通过了
《关于十九届中央领导机构人选酝酿工作谈话调研安排方案》。谈话调研和人选酝酿工作在
习近平总书记直接领导下进行。主要遵循以下原则:

——着眼于统筹推进“五位一体”总体布局和协调推进“四个全面”战略布局,贯彻落实新发
展理念,全面建成小康社会,不断推动新时代中国特色社会主义事业向前发展;着眼于提高
党的领导水平和执政能力、保持和发展党的先进性和纯洁性,推进国家治理体系和治理能力
现代化,巩固党的执政地位;着眼于党的事业后继有人、兴旺发达,确保党和国家长治久安。

\subsection{——坚持政治家集团标准,坚持五湖四海、任人唯贤,坚持德才兼备、以德为先,坚持事业为}
\label{sec:org0a5f162}
上、公道正派,严把政治关和廉洁关,精准科学选人用人。

——进一步改进完善党和国家领导人产生机制,积极稳妥地推进党和国家高层领导的新老交替。

——坚持党管干部原则,贯彻民主集中制,充分发扬党内民主,提高民主质量和实效。谈话调
研重在集思广益、统一认识,不限定推荐人数,人选推荐票数作为参考,不以票取人。根据
干部条件、一贯表现和班子结构需要,研究提出新一届中央领导机构人选。

新一届中央军委组成人选方案,应突出强调坚持政治标准,聚焦备战打仗,优化结构布局,
注重老中青梯次配备。

按照这些原则,中央提出了推荐人选应具备的条件:

\subsection{——对党忠诚,信念坚定,牢固树立“四个意识”,坚定“四个自信”,坚决贯彻习近平新时}
\label{sec:orgcba6b98}
代中国特色社会主义思想,与以习近平同志为核心的党中央保持高度一致,是合格的马克思
主义政治家。

\subsection{——领导能力强,实践经验丰富,有强烈的革命事业心,有改革创新和实事求是精神,敢于担}
\label{sec:org6a816f6}
当,有正确的政绩观,工作业绩突出。

\subsection{——带头执行民主集中制,自觉维护以习近平同志为核心的党中央权威和集中统一领导,善于}
\label{sec:orgb342120}
团结同志,公道正派,心胸宽广。

\subsection{——具有共产党人的世界观、人生观、价值观,带头坚持原则,带头遵守党的纪律和规矩,作}
\label{sec:org2b70f35}
风过硬,清正廉洁,在党内外有较高威信和良好形象。

参照往届做法,根据党和国家事业发展需要和中央领导机构建设的实际,中央还对推荐人选
的范围、年龄和结构提出明确要求。

大家一致认为,中央关于新一届中央领导机构人选酝酿工作的原则科学合理,推荐人选的标
准条件清晰明确,推荐范围、年龄杠杠和结构要求符合实际,体现了党中央的远见卓识。

民主科学凝共识——新一届中央领导机构人选的产生,采取了一系列新方式、新举措,体现了
选人用人新机制、新导向,展示了党的新作风、新形象

2017年5月下旬,一位省部级领导干部接到通知,来京参加组织谈话。

谈话地点安排在中南海。一到候谈室,3份材料已经摆在桌上——《谈话调研有关安排》《现
任党和国家领导人党员同志名册》《正省部级党员领导干部名册》。

按照谈话调研工作程序,给参加谈话干部安排充分时间阅读材料,独立认真思考准备。

在此基础上,中央领导同志以面对面谈话的方式,听取了这位干部关于新一届中央领导机构
人选的推荐意见。

“没有限定推荐人数,了解多少就谈多少,怎么想就怎么谈,实事求是,畅所欲言。”这位
干部事后感慨,“作为一名在地方工作的同志,有机会、有资格对新一届中央领导机构人选
发表意见、进行推荐,这是党中央对我的高度信任,充分体现了我们党的民主作风和宽广胸
怀,体现了我们党善于集中全党智慧的优良传统。”

用个别谈话调研的形式,在一定范围内面对面听取对中央领导机构人选的推荐意见和有关建
议,这是十九届中央领导机构人选酝酿提名工作的重大创新。

这一重大创新,体现在借鉴历史经验、探索选人用人新方式新举措上——

在党和国家高层领导人选产生方面,我们党有着优良传统,不断进行积极探索,有经验也有
教训。党的十七大、十八大探索采取了会议推荐的方式,但由于过度强调票的分量,带来了
一些弊端:有的同志在会议推荐过程中简单“划票打勾”,导致投票随意、民意失真,甚至
投关系票、人情票。中央已经查处的周永康、孙政才、令计划等就曾利用会议推荐搞拉票贿
选等非组织活动。

坚持问题导向,中央对新一届中央领导机构人选的产生方式进行创新和改进,强调坚持民主
方向、改进民主方法、提高民主质量,决定在对十九届“两委”委员人选深入考察、严格把
关基础上,通过谈话调研、听取意见、反复酝酿、会议决定等程序逐步酝酿产生中央领导机
构人选。

从2017年4月下旬至6月,习近平总书记专门安排时间,分别与现任党和国家领导同志、中央
军委委员、党内老同志谈话,充分听取意见,前后谈了57人。

根据中央政治局常委会的安排,中央相关领导同志分别听取了正省部级、军队正战区职党员
主要负责同志和其他十八届中央委员共258人的意见。中央军委负责同志分别听取了现任正
战区职领导同志和军委机关战区级部门主要负责同志共32人的意见。

这种采取个别谈话调研、面对面听取意见建议的方式,得到了参加谈话同志的一致赞誉。大
家普遍感到,方案考虑周全,工作安排细致,程序设计周密,纪律要求严格,这样反映出的
意见更全面、更真实、更准确。

这一重大创新,体现在坚持以事择人、形成组织工作新机制新导向上——

在谈话推荐工作中,中央明确了推荐人选的条件,坚持以马克思主义政治家集团标准选人,
注重知行合一;坚持事业为上、任人唯贤,注重工作能力与实践经验;坚持严把人选廉洁关
和作风关,注重形象口碑。

严格标准、事业为上,参加谈话的同志对此高度评价、一致赞同。

大家认为,党和国家领导职务也不是“铁椅子”“铁帽子”,符合年龄的也不一定当然继续
提名,主要根据人选政治表现、廉洁情况和事业需要,能留能转、能上能下。

大家反映,十九届中央领导机构人选的产生,健全了科学的用人机制,对进一步形成良好的
党内政治生态、增强干部选任的科学性和公信力具有深远意义。

许多同志说,这次在这么大范围内就党和国家高层人事安排问题广泛听取意见,是新形势下
充分发扬党内民主的有效方式,是改进和完善党和国家领导人产生机制的成功实践,倡导了
新的正确用人导向。

这一重大创新,体现在坚持风清气正、再塑党的新作风新形象上——

肩负光荣的政治使命和沉甸甸的政治责任,本着对党和国家事业高度负责的态度,参加谈话
的同志严肃认真,知无不言、言无不尽,讲心里话,公正表达意见。

参加谈话的同志在思考准备时都非常认真、十分慎重,有的拟好谈话提纲;在谈话中都能畅
所欲言,不仅充分发表推荐意见,许多同志还对中央领导班子建设提出了很好的建议;有的
同志在谈话回去后又打来电话补充意见,有的还专门补交了书面材料……

“谈话过程也是对本人的一次考验和党性教育,是高级领导干部参与党内政治生活的生动实
践。”大家反映,这种方式克服了以往“大会海推”“划票打勾”带来的种种弊端,没有暗
潮涌动,始终风清气正。

充分沟通酝酿,凝聚全党意志。

在综合大家意见建议的基础上,2017年9月25日,中央政治局常委会提出了新一届中央领导
机构的组成人选方案。

新一届中央纪委领导成员人选建议方案,由中央纪委、中央组织部有关方面经过酝酿讨论,
向中央提出。新一届中央军委组成人选建议方案,由中央军委经过集体讨论,向中央提出。

9月29日,中央政治局会议审议通过了新一届中央领导机构人选建议名单,决定提请党的十
九届一中全会和中央纪委一次全会分别进行选举、通过、决定。

新一届中央领导机构产生的过程,是坚持党的领导和充分发扬民主相结合、凝聚全党智慧的
过程,是严格按党章、按制度、按程序办事的过程,是党和国家领导人产生机制不断改进完
善的过程,充分显示出我们党更加团结统一、成熟自信。

乘风破浪扬帆进——新一届中央领导机构汇集了全党各方面优秀的执政骨干,他们将团结带领
全党全国各族人民,豪情满怀、意气风发进行伟大斗争、建设伟大工程、推进伟大事业、实
现伟大梦想

10月24日,党的十九大选举产生了新一届中央委员会和中央纪律检查委员会。

10月25日,党的十九届一中全会选举产生了新一届中央领导机构。


10月25日,中国共产党第十九届中央委员会第一次全体会议在北京人民大会堂举行。习近平、
李克强、栗战书、汪洋、王沪宁、赵乐际、韩正等出席会议。新华社记者 鞠鹏 摄

这是一个体现全党意志、凝聚全党共识、反映人民期待,值得全党全军和全国各族人民充分
信赖的领导集体——

新一届中央领导机构由符合马克思主义政治家标准,能够适应统筹推进“五位一体”总体布
局、协调推进“四个全面”战略布局需要,具有破解改革攻坚难题、应对各种风险能力的专
业素质,具有丰富领导经验和群众工作本领,忠诚、干净、担当,在干部群众中有很高威信
的各方面党的执政骨干组成。

这是一个承前启后、继往开来,充分体现当代中国共产党人风貌的领导集体——

十九届中央政治局由25名熟悉各方面、各领域工作的同志组成,都有较高学历和专业知识,
结构比较合理,有在地方工作的,有在中央和国家机关工作的,也有军队的同志,还有女同
志。其中,10名同志是十八届中央政治局委员继续提名,3名同志是全国人大和国务院领导
同志转任,12名同志是新提拔的。

这是一个朝气蓬勃、富有活力,能够适应党和国家事业长远发展需要的领导集体——

新一届中央领导机构进退比例比较适当,保持了人员和工作的连续性,积极稳妥地实现了党
和国家高层领导的新老交替。一批德才兼备、年富力强的领导干部进入新一届中央政治局,
充分反映了我们党的兴旺发达、后继有人。

在新一届中央领导机构酝酿人选和征求意见时,一些党和国家领导同志以党和人民利益为重,
以对国家发展和民族振兴高度负责的精神,主动表示退下来,让相对年轻的同志上来,表现
出了共产党人的宽阔胸怀和高风亮节。

\subsection{领航新时代,再启新征程。}
\label{sec:orgd718cb2}

2017年10月25日上午,人民大会堂东大厅华灯璀璨,气氛热烈。11时54分,中国共产党第十
九届中央委员会总书记习近平和中央政治局常委李克强、栗战书、汪洋、王沪宁、赵乐际、
韩正步入东大厅,同采访党的十九大的中外记者亲切见面。



步履矫健、姿态从容,中央领导同志在镜头和闪光灯前展示出沉稳豪迈的气度和锐意进取的
精神。

习近平总书记代表新一届中央领导机构成员衷心感谢全党同志的信任,表示一定恪尽职守、
勤勉工作、不辱使命、不负重托。他说,过去的5年,我们做了很多工作,有的已经完成了,
有的还要接着做下去。党的十九大又提出了新目标新任务,我们要统筹抓好落实。只要我们
深深扎根人民、紧紧依靠人民,就可以获得无穷的力量,风雨无阻,奋勇向前。

这是新时代领航者的自信,这是一个执政党的担当,这是伟大民族复兴的希望。

以习近平同志为核心的新一届中央领导集体,必将团结带领全党全国各族人民,高举中国特
色社会主义伟大旗帜,锐意进取,埋头苦干,决胜全面建成小康社会,夺取新时代中国特色
社会主义伟大胜利,引领承载着中国人民伟大梦想的航船破浪前进,驶向光辉的彼岸。

本文来自新华社
\section{教育教学}
\label{sec:org42a87e8}
\subsection{\href{http://www.sohu.com/a/229047360\_114778?\_f=index\_chan30news\_28}{为什么全球顶尖成功人士都会遵循“5小时法则”?}}
\label{sec:orge4fb6ed}
全球最顶尖的5家公司的创始人比尔·盖茨、史蒂夫·乔布斯、沃伦·巴菲特、杰夫·贝佐斯、
拉里·佩奇,他们都是博学的通才,他们都有两种不同寻常的特质。在研究了白手起家的亿
万富翁多年之后,我发现有两种特质在他们获得如今的财富、成功、影响力和名声方面发挥
了关键作用。事实上,我自己也非常相信这两种特质,所以我在自己的生活中创办公司、成
为一个更好的作家、做一个更好的丈夫、实现财务安全的过程中都会利用这两种特质。这些
成功人士拥有的两种共同特质分别是:

(1)他们都是如饥似渴的学习者。

(2)他们都是通才。

下面让分别阐述一下这两个特质,并分享一些简单的技巧,从而让你自己在日常生活中也能
利用它们。

首先,这两种特质的定义。我把一个如饥似渴的学习者定义为一个遵循“5小时原则”的人,
即每周至少花5个小时来学习。我将通才定义为一个能在至少三个不同领域里都能胜任的人,
并将这几个领域的技能都整合到一个技能组合中,使他们成为所在领域内前1\%的顶尖人才。
如果你不断学习和模拟这两个特质,并且认真对待它们,我相信它们会对你的生活产生巨大
的影响,并会加速你获得成功的脚步。当你变成一个如饥似渴的学习者时,那么你过去所学
到的所有知识的价值就会呈现复合效应。当你成为一个通才时,你就能开发出综合技能的能
力,并开发出一套独特的技能组合,这能够帮助你获得竞争优势。

根据比尔盖茨自己的估计,他每周都会读完一本书,这个习惯已经坚持了52年,其中很多书
籍都是与软件或商业无关的书籍。此外,在他的整个职业生涯中,他每年还会有一个为期两
周的阅读假期,这两周时间专门用来阅读。在1994年的接受《花花公子》的采访中,我们发
现他已经把自己当成了一个博学的通才了:

花花公子:你不喜欢自己被称为是一名商人吗?
盖茨:不喜欢。在我的所有思考时间中,我将10\%的思考时间用于商业思考。商业并没有那
么复杂,我不想将商人的身份放在我的名片上。
花花公子:那么你会将什么写在你的名片上呢?
盖茨:科学家。当我读到一些伟大的科学家的故事,比如克里克和沃森是如何发现DNA的时
候,我就会非常兴奋。商业成功的故事无法让我获得同样的乐趣和快感。
盖茨将自己视为一位科学家,这是非常有意思的,因为他从大学辍学了,并在辍学后就一直
扎根于软件行业。

有趣的是,埃隆·马斯克也不认为自己是个商人。在最近一次接受CBS的采访中,马斯克说他
认为自己更像是一个设计师、工程师、技术专家,甚至是巫师。

这样的例子不胜枚举。众所周知,拉里·佩奇会花时间与谷歌的每一个人进行深度交流,从
谷歌的门卫到核聚变科学家,并总是希望自己能从他们身上学到些什么。

沃伦·巴菲特是这样描述自己获得成功的关键的:“每天阅读500页书。知识的运作方式是:
知识会慢慢累积呈复合效应,就像复利一样。”

杰夫·贝佐斯是通过实验进行大规模学习的方式来打造他的整个公司的,并且他在一生中都
是一个如饥似渴的阅读者。

最后,史蒂夫·乔布斯将各种学科结合在一起,并将之视为苹果的竞争优势,他曾经这样说
过:“单靠科技是远远不够的,必须要让科技与人文科学以及人性相结合,其成果必须能够
让用户产生共鸣。”

当然,上面五家顶尖公司的创始人并不是唯一拥有这两种特质的成功人士。正如我之前写过
的,如果我们把这份名单扩大到其他白手起家的亿万富翁,我们很快就会看到奥普拉·温弗
瑞、雷·达利奥、大卫·鲁本斯坦、菲尔·耐特、霍华德·马克斯、马克·扎克伯格、埃隆·马斯
克、查尔斯·科赫和其他很多拥有类似特质和习惯的成功人士。

为什么世界上最忙碌的人会把最宝贵的时间投入到学习与和他们所在领域看起来无关的知识
上,比如核聚变、字体设计、科学家传记和医生回忆录?

他们每个人都掌管着一个由全世界成千上万最聪明的人组成的公司。他们把自己生活和公司
业务中的几乎每一项任务都委派给了最优秀、最聪明的人去负责。那么他们为什么还要坚持
学习大量知识呢?

在写了几篇试图回答这些问题的文章之后,这就是我最终得出的结论:“在最高层次上,学
习并不是一件你为工作做准备的事情,学习本身就是最重要的工作。它是你要打造的一个核
心能力。这是一件你永远不能委派给别人去做的东西。这是关系到企业进步和长远成功的终
极驱动力之一。“当我意识到这一点的时候,我在想:在如今这样一个日益复杂、快速变化、
先进的知识经济时代,为什么大家没有在‘我们在一生中都应该成为贪婪的学习者和博学的
通才’上达成共识并认为这是我们理所应当做的事情呢?为什么大部分普通人都将可以学习
视为一种可做可不做的事情呢?

我认为,这主要和我们在学校、大学和社会上被反复灌输的三个强有力信息有关,这些信息
在过去可能是事实,但现在已经不再是了,它们现在已经变成三条谎言了。下面我们就来看
看如何一一打破这三条谎言:

(1)谎言1:学科是分类知识的最好方法。

(2)谎言2:大部分学习都发生在学校/大学。

(3)谎言3:你必须选择一个领域并且专攻这个领域。

这些观念是如此有害,它们摧毁了我们对学习和知识的直觉,最终阻碍我们创造我们想要的
成功。如果我们能意识到它们,我们就能像世界上最成功的人所做的那样去纠正它们。

谎言1:学科是分类知识的最好方法。

我们的教育体系建立在这样一种模型上,即将知识分为不同学科——数学、阅读、历史和科学
等。从幼儿园开始,我们接收到的信息就是,这些科目最好是能各自单独学习。

我们甚至将这些学科细分为更小的学习领域,例如,将经济学进一步细分为微观经济学和宏
观经济学。这种将学科领域进行分解并分开教学的范例叫做简化论。尽管它仍然是我们社会
的标准,但它实际上已经开始在一些更进步的国家发生变化了。

简化论能带来很多益处。在关系紧密领域内,想法能够得到迅速而高效地传播,因为每个人
都属于同一种文化,都说着同样的语言。研究一个系统的各个部分要比研究一个复杂的整体
系统要更为容易。这种范式在很多重大的发现中发挥了非常重要作用。

但简化论的一个关键缺点在于,不同领域之间的连接变得非常模糊,所带来的结果就是所谓
的“负学习迁移”,即学习一件东西会使学习其他东西变得更加困难,因为我们所学到的概
念与特定的研究领域是如此紧密地联系在一起。如果你在学习第二语言的时候陷入了困境,
因为你所学的第二语言的语法、语序、时态或复数的规则与你的母语的规则不匹配,你就会
经历负学习迁移。

简化论的另一个缺点是,在一个专业领域之外的人很难理解这个领域内发生的事情。不妨想
象一下一个神经外科医生试图向一个平面设计师解释大脑手术的进展,会发生什么。每个领
域都有各自的语言和文化,所以在一个领域里独特的见解并不适用于另一个领域,尽管不同
领域内的见解经常可以而且应该是彼此适用的。这就导致了回音室的出现。

生物学家James Zull在他的书《The Art of the Changing Brain》里解释了为什么学习迁
移如此复杂。“通常情况下,我们没有将一个主题和另一个主题连接起来的神经网络。它们
是分开建立的,特别是如果我们在将知识分解成数学、语言、科学和社会科学等不同部分的
标准课程中学习。” 因为没有人教我们去看所有这些知识的共同根源,所以我们看不到它
们之间的内在联系。

埃隆·马斯克强烈地感到我们的教育体系无法教给孩子们学习所有这些学科知识的共同根源,
他创建了自己的学校,并让自己的所有孩子都进入这个学校。

埃隆·马斯克在接受北京电视台采访时称,由于不喜欢他的孩子的学校,因此他便自己建了
一所学校。这所学校被命名为Ad Astra,意思是“奔向星辰”。这所学校规模极小,相对比
较隐秘。它没有自己的网站,也没有创建社交网帐号。曾经撰写过一篇关于洛杉矶私立学校
的文章的克里斯蒂娜·西蒙(Christina Simon)深入了解了Ad Astra学校。她说她认识了一
位与马斯克的孩子上同一所学校的孩子的母亲。这位母亲对西蒙说,相对比较新的Ad Astra
学校规模很小,还处于测试阶段。这所学校目前只接纳了少数孩子,他们的父母亲都是
SpaceX的员工。马斯克在接受采访时说,Ad Astra学校创立了一年多,现有14名学生。Ad
Astra学校未区分年级,一年级和三年级的学生之间是没有明显区别的。他说:“让所有的
孩子在同一时间通过相同的年级考试,就像装配线一样。”

他说:“有人喜欢英语或语言,有人喜欢数学,有人喜欢音乐。人各有志,各有所长。因此,
根据他们的态度和能力来因材施教就显得非常重要了。”马斯克为他的孩子办了退学手续,
为了创办Ad Astra学校甚至还挖走了一名教师。他说:“我认为这些事情都是应该做的,但
我没有看到普通学校去做这些事。”马斯克认为,普通学校在教授学生如何解决问题上犯了
一个根本性的错误。马斯克说:“教授解决问题的方法或者讲解问题本身而非解决问题的工
具,这一点很重要。”

他说:“假设你想教别人引擎的工作原理,传统的做法是说,’我们将讲授有关螺丝刀和扳
手的所有知识。’这是一种截然不同的教学方式。” 相反,马斯克认为直接给学生们提供
一台引擎然后在学生们面前拆卸它,这种教授方式会更有意义。马斯克解释说:“我们如何
将它分解开来?你需要一把螺丝刀,然后一件非常重要的事情随之发生了,那就是工具的关
联性变得很明显了。”

多年来,我了解到有一种更深层的方法来分类知识,一种学习基本原理的方法,它适用于所
有领域,并教授让人终生受用的技能。这些基本原理被称为思维模型。

让我们来看看一个思维模型,叫做“压力和恢复”。在锻炼中,压力和恢复的现象是锻炼使
我们变得更强壮的原因所在:它暂时使我们的肌肉和心血管系统超过了它们现在的承受能力,
并在恢复过程中重建自己。有了这种思维模式,我们就可以在其他领域寻找到这种思维模型。
例如,它解释了为什么经历了某些类型的困难能够帮助我们变得更加强大。在心理学领域,
这被称为创伤后成长。在社会心理学领域,这些痛苦的经历被称为多样化的经历。在成年人
的发展中,他们被称为最优冲突。通过这些例子,我们可以看到同一个底层思维模型在不同
的应用领域里是如何被赋予不同的名称的。

思维模型是将学科联系在一起的无形的思想网络。



这正是很多世界顶尖学习者和通才在如今的知识经济中获得成功的秘密所在。

真相:将知识按思维模型分类与将知识按学科分类是同样重要的,因为思维模型是将不同学
科连接起来的底层基础。

谎言2:大部分学习发生在学校/大学。

“自学是唯一的教育形式。”——Mark Twain

我确信,教育最根本的问题之一就是学校与学习的归并融合。事实上,学校只是学习发生的
一个环境。在我们的生活中,几乎所有的学习都发生在学校之外:在家里,在操场上,在运
动场上,在旅行中,在我们读的书和我们喜欢的爱好里,特别是在我们的工作中。然而,我
们却被训练成只将正规教育视为真正的教育。

在军事和执法部门,把在学校里学到的东西和现实世界里发生的事情混为一谈都被认为是
“训练伤痕”。在《Algorithms To Live By》这本书中,引用一些有关这些伤痕的极端例
子,显示它们会带来多严重的后果:在现实枪战中,警察只开枪两次就把枪放进枪套里(就
像他们在训练中所做的那样)。在一个令人惊恐的例子中,一名警察从袭击者手中夺过枪,
然后本能地将枪还给他,就像他在警察学院的训练过程中一次又一次做的那样。同样的道理,
我们在学校里学习到的一些技能,这些技能要么无法转移到现实世界中,要么会阻碍我们在
现实世界中的表现。

例如,相信我们中的大多数人都会认同,在课堂上,那些遵守纪律和规则的人会因此得到奖
励。但在现实世界中,关键的领导特质包括冒险和原创思维,这两个特质都是与我们在课堂
上学的东西相违背的。简而言之,我们接受的大部分正规教育都是为了将我们培养成追随者,
而不是领导者。

下面我分享一下我是如何在我自己的生活中揭露“学校等于学习”这个谎言的。

我童年的大部分时间都是在学习中度过的,所以我的GPA绩点很高,也就顺理成章进入一所
很好的大学。因为从小到大,大人们都一直给我灌输这样的理念:一所好大学是通往美好生
活的门票,我对这一点也深信不疑。我的求学之路也非常顺利,我被理想中的学校录取了:
沃顿商学院和纽约大学斯特恩商学院。刚开始的时候,我非常努力地完成每一项学习任务,
并在每一项研究课题上都付出了大量心血。但后来我读到一份研究报告,这份报告显示,大
多数美国总统、国会议员、参议员甚至大学校长毕业时的GPA绩点都很低。一项对700名百万
富翁的调查显示,他们的平均GPA绩点是2.9。事实上,我所在大学的校长是以2.1的GPA绩点
毕业的。此外,我了解到,我的大多数创业榜样都没能完成大学学业,即使他们完成了大学
学业,他们也不会将其视为他们所获得的成功的重要因素。

“这是咋回事?我被骗了一辈子。我想成为一名企业家,但分数并不重要。”我暗自纳闷。

自那一天起,我就停止了为了分数而读书的日子。我的平均GPA绩点下降到2.9,当教授置仅
仅为了让我不要闲下来而给我布置作业的时候,我就会直接跳过这些作业。相反,我设计了
自己的学习书籍、要参加的会议,并旁听了我感兴趣的课程。幸运的是,我做了一个很好的
赌注。当我面试实习的时候,只有一次面试是让我提供GPA绩点的。尽管我的GPA绩点很低,
但我还是得到了实习机会。在工作场合,我从来没有被问过我是在哪所学校就读的。

这些年来,我的思维变得更加微妙。下面是我目前对正规教育的看法:

(1)最具影响力的领导者、艺术家和科学家几乎都对学习有一种内在的热爱,那种近乎痴
迷地热爱,这种痴迷贯穿了他们的一生。无论他们有多忙,他们都会抽出时间来学习。1991
年,比尔·盖茨在接受采访中曾自豪地这样分享到,虽然他经常工作到深夜,但深夜回到家
后还是会花点时间用来阅读。当埃隆·马斯克想要雇佣世界上最聪明的人时,他更关心的是
候选人的技能组合,而不是学位。

(2)中学和高等教育一般都没有鼓励学生自主学习或培养终身学习的习惯。事实上,为了
考试或者是进入一所好的大学而学习往往会培养外在的动机,而这实际上会阻碍内在的动机。
正规的教育通常不擅长向学生展示不同学科之间的联系,也不擅长教学生如何在现实世界中
应用他们所学到的东西去得到他们想要的结果。

(3)正规教育的最重要成果是让学生热爱学习并拥有自主学习的能力。一个自主学习的学
习者能够识别他们所面临的问题并对问题进行优先级划分,识别那些他们可以学习的用来解
决这些问题的知识并对这些知识进行优先级划分,坚持每周至少学习5个小时的时间,并将
他们学到的东西应用到现实世界的挑战中。一旦一个人爱上了学习,他们就会终身学习。当
然,几十年终身学习的积累要比四年大学积累的东西要多得多。

(4)妖魔化正规教育并不是真正的解决之道。多年来,我已经在数百所高中和大学发表过
演讲,演讲对象从最精英的人群到最弱势的人群都有,我的观点已经软化了。我有两个孩子
在上小学。他们的老师改变了我们孩子的生活。我在教育系统中遇到过许多了不起的老师,
他们提供了变革的经验。这些教师中,很多人收入严重不足、不受尊重,却是社会中最有价
值的贡献者。在整个系统中,有一些机构也很了不起。立法者制定了更严格的考试规定,导
致了为了考试而教学和学习的文化的出现,这仅仅是因为他们希望学校系统更加可靠。这是
一个复杂而重要的挑战。

真相:在我们的生活中,大多数的学习都是在学校之外完成的,在获得成功的过程中,自主
学习能力是比分数和学位更重要的因素。

谎言3:你必须选择一个领域并且专攻这个领域。

在亚当·斯密的代表作《国富论》这本书的第一页中,他以一个别针工厂为例,说明了专业
化的力量。在这个工厂里,只有10名员工,每天却能生产出令人震惊的48000个别针,这就
是分工的结果,每个人只专注于制作流程中的其中一个部分。史密斯估计,如果这10个工人
中的每一个人都亲自参与整个制作流程中的每一步的话,那么这个工厂每天只能制作出200
个别针。换句话说,专业化将他们的效率提高了240倍。

我们几乎所有人都被灌输这样的观念,要想在生活中出人头地,就必须专攻一个领域。当你
看到上面的别针工厂的例子,这就不足为奇了。在工业时代,生产力是通过定量产出来衡量
的。对于那些仍在制造业工作的人来说,这种模式仍然适用。

但我们大多数人现在都生活中知识经济中,在这个经济中,生产力不是用数量来衡量的,而
是用创造性的产出来衡量的。产生创造性想法的最好方法之一就是学习和综合你所在领域内
的其他人还不知道的有价值的技能和概念。在知识经济中,学习跨领域的不同兴趣和技能,
然后将你的见解运用到你的核心专业里,换句话说,也就是成为一个现代的博学通才,这才
能让你真正脱颖而出。

在《为什么“兴趣广泛”的通才更有可能获得成功?详解通才的7大优势》这篇文章中,我
详细解释了为什么在现代知识经济的大环境中每个人都应该成为一个通才,并列出了通才的
7大优势,这7大优势分别是:(1)通过融合两种及以上的技能能够让你成为一流的人才;
(2)大多数创造性突破是通过非典型技能组合实现的;(3)学习并精通一项新技能比以往
更加容易和迅速;(4)比以往更加容易开创一个新领域、新行业或者全新的技能组合;(5)
它可以为你的未来职业发展保驾护航,让你的技能不会过时;(6)它可以帮你解决更为复
杂的问题;(7)它可以帮你脱颖而出,在全球经济中有效竞争。

真相:专业化是工业经济的关键。在当前的知识经济中,学习掌握了至少三个领域的知识并
能将它们整合到一个技能组合中的现代型通才将会有更大的优势,这能够使他们成为自己所
在领域中前1\%的顶尖人才。

需要记住的三大新真相

综上所述,我们过去的学习方式已经不再适用于快速变化的知识经济了。相反,要记住这些
新的真相:

(1)除了按照学科来分类知识外,按照思维模型来分类知识同样重要甚至更重要。因为思
维模型是将不同学科连接起来的底层基础。

(2)大部分学习都是在学校之外完成的,在获得成功的过程中,自主学习能力是比成绩和
学位更重要的因素。

(3)在当前的知识经济中,学习掌握了至少三个领域的知识并能将它们整合到一个技能组
合中的现代型通才将会有更大的优势,这能够使他们成为自己所在领域中前1\%的顶尖人才。

这就是为什么那些喜欢阅读和学习的通才以及那些研究思维模型的人都能变得如此成功的原
因所在。这也解释了为什么世界上这么多顶尖CEO、亿万富翁、科学家和成功人士都具有这
些共同的特质。

现在就下定决心,不要把所有的时间都花在一个狭窄领域的细节知识上,因为这只能让自己
看不到世界其他地方、其它领域发生的事情,这将阻碍你快速适应新发展的能力。

相反,要在终身学习上进行投入,每周至少花五个小时在你的领域之外进行学习和探索,学
习你的同事还不知道的技能和概念。此外,一定要学习思维模型,这些底层基础知识是所有
领域知识的基础,而且基本不会随时时间的推移发生变化。将自己训练成为一名对思维模型
有深入了解的自学成才型通才,做到这些将是在现代知识经济中获得成功的关键。

原文链接:\url{https://medium.com/the-mission/the-founders-of-the-worlds-five-largest-companies-all-follow-the-5-hour-rule-and-they-re-9ca82e93f3fc}

编译组出品。
\subsection{\href{http://www.sohu.com/a/227869340\_497891?\_f=index\_chan25news\_192}{六个习惯走向精彩人生}}
\label{sec:org78b4e3b}
有梦想有目标的人都希望自己的人生过得既高效又充实。那么,就随小编一起来看看下面的
文章中哪些聪明的习惯会让你的生活充满成就感。



Do you ever lay in bed at night in a state of insomnia and look back over your
choices in life and think, "I wish I did that differently?"

你是否曾经在失眠的状态下躺在床上,回顾自己在生活中的选择,想一想:“我希望自己做
得不同?”

Perhaps it's feeling like you should've made that transition into a personal
calling long ago; chosen better friends, peers, or work habits; or learned
better leadership skills to advance up the corporate ladder.

也许这是因为你觉得自己早就应该转变成一个人;选择更好的朋友、同事或工作习惯;或者
学会更好的领导技巧来提升公司的地位。

While I won't promise this list is the path to increasing wealth or instant
success, I will say this: Every successful person arrived there by choosing to
repeat the same habits over and over until they reached the top.

虽然我不承诺这份清单是增加财富或即时成功的途径,但我会这样说:每个成功的人都会选
择一遍又一遍地重复同样的习惯,直到他们达到顶峰。

Let me leave you with some key success lessons that so many of us miss until
much later in life.

我留下一些关键的成功经验教训,这些经验教训是我们很多人都错过的,直到很久以后。

01

Your success is only as good as the people whom you surround yourself with.

你的成功只和你身边的人一样好。



Pick your network wisely, it could make or break you. Billionaire Warren Buffett
once told a 14-year-old kid at a Berkshire Hathaway annual meeting one of the
keys to his success: "It's better to hang out with people better than you. Pick
out associates whose behavior is better than yours and you'll drift in that
direction."

明智地选择你的关系网,它可以造就你,也可以摧毁你。亿万富翁沃伦·巴菲特曾在伯克希
尔哈撒韦公司年会上告诉一个14岁的孩子,他成功的关键之一是:“最好和比你更优秀的人
在一起”。挑选那些行为比你好的同事,你会朝那个方向漂移。”

He taught a good life lesson for all of us about absorbing the very qualities
and traits of successful people--those further down the path than us--that will
elevate and make us better as leaders, workers, and human beings.

他给我们所有的人上了一堂很好的人生课,学习了成功人士的品质和特质--他们比我们走得
更远--这将提升我们作为领导者、工人和人类的地位,使我们变得更好。

02

Create value to capture people's attention.

创造价值来吸引人们的注意力。



Wharton professor Adam Grant says that successful people capture others'
attention by creating something of value, which will yield great returns and
enlarge their network.

沃顿商学院教授Adam Grant说,成功人士通过创造有价值的东西来吸引他人的注意,这将带
来巨大的回报,扩大他们的关系网。

The example he cites is that of Sara Blakely, founder of Spanx. For two and a
half years, Ms. Blakely ambitiously sold fax machines by day so that she could
build her prototype of footless pantyhose by night.

他举的例子是Spanx创始人萨拉·布莱克利。两年半的时间里,布莱克利一直雄心勃勃地白天
销售传真机,以便在晚上制作出她的无脚连裤袜原型。

With good intentions and without ever stalking, Blakely sent one from the first
batch to none other than Oprah Winfrey, who chose it as one of her favorite
things of the year. And the rest, as they say, is history.

出于良好的意愿,并且没有任何跟踪,布莱克利从第一批中送给了奥普拉·温弗莉,她把这
份礼物选为这一年中她最喜欢的东西之一。剩下的,就像他们说的,都是历史。

03

"Do what you love, and put your whole heart into it, and then just have fun."

“做你喜欢做的事,全心投入其中,然后享受快乐。”



That's a direct quote from Apple CEO Tim Cook. In all its simplicity, it seems
almost like an anomaly to the conventional wisdom of sacrifice and hard work.

这是直接引用苹果首席执行官蒂姆·库克的一句话(其实出于乔布斯)。尽管它很简单,但
似乎与传统的牺牲和努力工作的智慧不同。

Doing what you love, however, is what gives your life purpose. And that purpose
is exactly what you can't help but keep doing. Even if there are low monetary
rewards, you would probably do it anyway because of your love for it. When you
discover what this is for you, it's the thing that makes you come alive.

然而,做你喜欢做的事才是你人生的目标。而这正是你不得不继续做的事情。即使有很低的
金钱回报,你也可能会因为你的爱而去做。当你发现这对你来说意味非凡的时候,你就真正
活了过来。

Tim Cook understands this deep down. "My advice to all of you is, don't work for
money--it will wear out fast, or you'll never make enough and you will never be
happy, one or the other," Cook told students at the University of Glasgow last
year after receiving an honorary degree from the school.

蒂姆·库克深知这一点。“我对你们所有人的建议是,不要为了钱而工作——它会很快磨损,
或者你永远赚不到足够的钱,也永远不会快乐,”库克在去年获得格拉斯哥大学荣誉学位后
告诉学生们。

04

Do whatever it takes to develop your communication skills.

尽一切可能提高你的沟通能力。



In my work coaching executives and entrepreneurs, communication issues are
common and every one of them agrees that mastering this ability is a necessity
for the success of their business.

在我的工作指导中,管理人员和企业家,沟通问题很常见,每个人都认为掌握这种能力是他
们业务成功的必要条件。

Well, not just my own clients but the world's most successful billionaire
entrepreneurs as well.

不仅是我的客户,还有世界上最成功的亿万富翁。

In a post on his own Virgin blog in which he lists his top 10 quotes on
communication, Richard Branson writes, "Communication makes the world go round.
It facilitates human connections, and allows us to learn, grow, and progress.
It's not just about speaking or reading, but understanding what is being
said--and in some cases what is not being said." He adds, "Communication is the
most important skill any leader can possess."

理查德·布兰森在他自己的维珍博客上的一篇文章中列出了他在交流方面的前10条名言,他
写道:“交流使世界运转。它促进人类的联系,让我们学习、成长和进步。这不仅仅是说话
或阅读,而是要理解所说的--在某些情况下还要理解没有说的。”他补充道:“沟通是任何
领导者都能拥有的最重要的技能。”

05

Make more heart decisions instead of head decisions.

做更多的内心决定而不是大脑的决定。



The most important decisions you'll ever encounter will always be based on your
feelings--it's a heart thing, not a head thing.

你会遇到的最重要的决定总是基于你的感觉---这是一个心灵的东西,而不是一个头脑的东西。

Not sure if you can rely on your heart just yet? OK, do this: Document every
decision you make over the next three months. Look over which decisions were
spot-on because you chose to rely on that "inner voice."

你还不确定你能不能依靠你的心?好的,做这个:记录下你在接下来的三个月里所做的每一
个决定。仔细看看哪些决定是正确的,因为你选择依赖“内在声音”。

The better the outcome of those decisions, the more accurate your intuition is
becoming--going with your heart. Learning to go with your heart is a much more
effective way to make decisions than to get stuck in analysis paralysis. It's
empowering, and your peers and close friends and family will look at you in a
whole new way.

这些决定的结果越好,你的直觉就越准确---学会用心去做决定,比陷入瘫痪在分析状态要
有效得多。它赋予你权力,你的同龄人,亲密的朋友和家人都会以全新的方式看待你。

06

Listen more.

多听。



Imagine going about your business thinking that "this is the right way" but
realizing later you were horribly mistaken. I see this in clients all the
time--a tendency to plow ahead like lone rangers, convinced they have all the
answers. Show me a person who does not solicit the sound advice and wisdom of
others, and I'll show you an ignorant fool.

想象一下你在工作中认为“这是正确的方式”,但后来意识到你大错特错了。我一直在客户
身上看到这一点---倾向于像单独的游侠一样前进,相信他们拥有所有的答案。给我一个不
征求别人意见和智慧的人,我就给你一个无知的傻瓜。

The most successful people I've ever encountered practice the skill of listening
and accepting feedback because they know it will make them better. In Seven
Pillars of Servant Leadership, authors Don Frick and James Sipe describe these
helpful approaches:

我所遇到过的最成功的人练习倾听和接受反馈的技巧,因为他们知道这会使他们变得更好。
在仆人领导的七个支柱中,作者Don Frick和James Sipe描述了这些有用的方法:

Listen without interruption, objections, or defensiveness. Be willing to hear
the speaker out without turning the table. Ask questions for clarification.

倾听时不要打断、反对或辩解。愿意在不生气情况地下把演讲者听完。问问题以澄清。

Make it clear what kind of feedback you are seeking and why it is important to
you. Offer a structure for the feedback--questions, rating scales, stories.
 明确你在寻求什么样的反馈,以及为什么反馈对你很重要。提供一个反馈的结构-问题,分
级,故事。

Be clear with your commitment. Describe how you have benefited from the feedback
and what specific steps you will take toward improvement. This builds bridges
and trust with others.
明确你的承诺。描述你是如何从这些反馈中受益的,以及你将采
取哪些具体的改进措施。这会与他人建立桥梁和信任。


\subsection{\href{http://www.sohu.com/a/227957908\_200190?\_f=index\_chan25news\_124}{2018中国职业教育技术展望:地平线项目报告}}
\label{sec:org8c85674}

2018年3月19日,《2018中国职业教育技术展望:地平线项目报告》(2018 Technology
Outlook for Chinese Vocational Education: A Horizon Project Report)(下简称《报
告》)在第三届中美智慧教育大会上发布,国家开放大学校长、中国教育技术协会会长杨志
坚,教育部中外人文交流中心副主任杨晓春,国家自然科学基金委员会政策局局长郑永和,
美国教育传播与技术协会主席Eugene G. Kowch,美国教育部教育技术办公室前主任Joseph
South,美国国家科学基金会项目主任Amy L. Baylor等中美教育界和教育信息化领域的专家
出席了发布会。

九大关键趋势

短期(信息化对职业教育的推动作用,更多应用混合式学习设计,开放教育资源快速增加)
中期(转向深度学习方法,重设学习空间,跨学科研究兴起)
长期(学生从消费者转变为创造者,推进变革和创新文化,反思院校的运作模式)
九项重大挑战

可应对的(转变社会大众对职业教育的偏见,教育技术和不断变化的教师角色,创造真实性学习机会)
有难度的(将学校教育与岗位学习相结合,整合正式与非正式学习,支撑个性化学习)
严峻的(领导变更中的创新可持续性,推广跨学科实证方法,推进有效学习技术的探索)
十二项教育技术的重要发展

一年之内(立体化教材,翻转课堂,微课,在线学习)
二至三年(虚拟现实、增强现实与混合现实技术,移动学习,云计算,学习分析及适应性学习)
四至五年(下一代学习管理系统,人工智能,虚拟和远程实验室,信息可视化)
《报告》所确定的教育技术应用的九大关键趋势、九项重大挑战及十二项重要发展将对中国
职业教育产生重要影响。黄荣怀教授在发布会上提出报告的十大要点,从整体上体现了中国
职业教育特有的技术变革方向:

“立体化教材”“在线学习”“微课”和“翻转课堂”将于一年之内在职业教育教学中得以
普及,表明当前职业教育信息化更侧重于数字教学资源的建设、资源形态的多样化和信息化
教学方式的不断探索,这与近年来国家大力支持职业院校专业建设与课程开发相关。
“虚拟现实、增强现实和混合现实技术”“云计算”等技术将于未来二至三年中在职业教育
领域广泛应用,“人工智能”“信息可视化”等智能技术会在未来四至五年中得以推广。这
一观点与人工智能发展及智能时代的发展趋势相一致。
信息化对职业教育的推动作用、开放教育资源快速增加以及更多应用混合式学习设计是职业
教育中教育技术应用的短期趋势,这与“十三五”以来职业教育信息化的发展规划相吻合,
表明职业教育信息化政策和战略得到了很好的推广。
重设学习空间、转向深度学习方法和跨学科研究兴起将逐渐成为职业教育信息化未来三至五
年的主要趋势特征。而学生从消费者转变为创造者、推进变革和创新文化与反思院校的运作
模式则将在未来五年或更长时间中逐渐成为职业教育信息化发展的趋势。这些趋势与基础教
育及普通高等教育信息化发展趋势基本一致。
创造真实性学习机会、教育技术和不断变化的教师角色、转变社会大众对职业教育的偏见被
认为是职业教育领域教育技术应用可应对的挑战,从职业教育的定位和发展战略来看,前者
与实际情况较为吻合,后两者或许与一般常识有悖,这一问题值得进一步深化研究。
“十三五”以来,职业院校的数字校园和智慧校园建设逐渐普及,信息技术应用和数字资源
利用的频度呈上升趋势,不断提升了保障支撑队伍的技术服务能力和教师的信息化教学能力。
职业教育过去三十余年“工学结合”理念的渗透和“校企合作”经验的积累推动了部分地区
职业教育“产教融合”的发展,特别是信息技术及信息化的行业,如电子与计算机、网络与
通讯、智能制造等领域。区域经济与职业院校的发展相互支撑、相得益彰。
随着国家“十三五”教育信息化建设工作的稳步推进,职业教育信息化战略部署初步形成,
数字资源日益丰富,网络学习空间逐渐推广,在一定程度上推动了职业教育的均衡发展以及
学习方式的变革。
地区间的经济与社会发展差异导致职业院校的发展存在明显差异,信息化正助力职业教育理
念的分享与更新,促进了优质教育资源的共建共享和教育教学模式的创新,在一定程度上推
动了区域间的协调发展。
随着“欧洲工业4.0”与“中国制造2025”战略的提出,与基础教育和普通高等教育相比,
职业技术教育与培训在虚拟现实、人工智能等新技术领域发展将更为迅速,以应对新兴产业
发展的需要。

\subsection{\href{http://www.sohu.com/a/227757055\_120943?\_f=index\_chan12news\_27}{白居易:少年经不得顺境,中年经不得闲境,晚年经不得逆境。}}
\label{sec:org047a59a}
白居易:少年经不得顺境,中年经不得闲境,晚年经不得逆境。
2018-04-10 07:48 白居易/成长

作者:洞见Fern

朗诵:简宁

来源:洞见(ID:DJ00123987)

编辑:慈怀读书会(ID:cihuai\(_{\text{dushuhui}}\))

少年人要心忙,忙则摄浮气;老年人要心闲,闲则乐余年。

曾国藩有一句名言:

“少年经不得顺境,中年经不得闲境,晚年经不得逆境。”

人生在世谁不愿意事事顺利,又能安享时光呢?

可是,“不经一番寒彻骨,怎得梅花扑鼻香。”

综观大诗人白居易的一生,恰恰印证了曾国藩的“人生三境”。



少年经不得顺境

白居易出生的时候,正赶上安史之乱后藩镇割据的乱局,一家人颠沛流离分居5处,过了很长一段时间穷困潦倒的生活。

但是,他却没有被苦难打倒,从小便立下“兼济天下”的志向,刻苦攻读。

后来,他给好友元稹写信,其中详细回忆了他年少苦读的经历:

我白天写赋,晚上读经,中间还要学诗,不敢有一丝懈怠;

而且常常读书读到嘴边上、舌头上都长了疮;

写字写到胳膊肘上、手心里都长了老茧;

年纪轻轻,头发都白了;

以至于成年以后,身体都要比同龄人衰弱一些。

古语云:

“艰难困苦,玉汝于成。”

就是在这样的成长环境中,白居易像一枚璞玉,被磨砺得愈发光彩夺目。

他三岁识字,五岁学诗,十来岁已是文采斐然,我们从小就会背诵的那首《赋得古原草送别》就是他16岁时所作。

离离原上草,一岁一枯荣。

野火烧不尽,春风吹又生。

远芳侵古道,晴翠接荒城。

又送王孙去,萋萋满别情。

古龙说过:

“一个人在少年得意,未必是福,而少年时的折磨,却往往使得日后能有更大的成就。一块美玉,不经琢磨,不能成器,人之一生,何尝不是如此?”

少年白居易,就像他诗中描写的野草一样,用他那坚韧不拔的意志,彰显出一个少年顽强不屈的生命力。

多年后,当他进士及第,在大雁塔上写下“慈恩塔下题名处,十七人中最少年”的时候,恐怕就是对那个逆境中坚强生长的少年最好的回答吧。

梁启超在《少年中国说》中写道:

“少年智则国智,少年富则国富,少年强则国强。”

一个人如果在少年时期,沉湎于玩乐,不敢面对生活的挑战,那么他的人生必定无所作为,更别说以后要肩负起家庭,乃至社会的责任了。





中年经不得闲境

刚刚过去的2017年有一个关键词叫做“油腻”,和他搭配在一起的被称作“中年人”。

还有一句话非常流行,叫做“人到中年不如狗”。

其实,这些都是对“人到中年”的误读。

白居易被贬江州(今江西九江)的时候已经43岁,按理说在这个年纪遭遇人生重创,恐怕早就意志消沉。

而且江州司马还是个闲职,用某些体制内的话说,“混吃等死就好”。

但是白居易官闲人不闲,那首家喻户晓的《琵琶行》就是在这个时候面世的。

三年后,他调任中州刺史、然后又先后担任杭州刺史和苏州刺史。

从中央到地方,纵使遭受打压,可白居易不仅没有郁郁不得志,反倒更加勤勉,将政务处理得有声有色。

杭州的“白堤”和苏州的“白公堤”就是他为官一任、造福一方的最好见证。

此后,“苏杭白堤”成为中国历史上的一段千古佳话,白居易也世世代代为人们所景仰。

这不就是一个中年人“闲不住”的结果吗?

卢梭说过:

“青年是学习智慧的时期,中年是付诸实践的时期。”

纵观整个中华历史,像白居易这样的中年人比比皆是,他们用其忙碌的身影,书写下一个社会,乃至一个时代的华章。

曾国藩组建湘军的时候已经40岁,此后南征北战十数载,到了54岁还主办洋务,连他自己后来都不得不慨叹:此生中年不得闲。

而他这一“不得闲”的结果怎样呢?

于己而言,他功高盖世、位极人臣;于国家而言,他强行为一个皇朝续命一甲子。

其人更是被后世奉为“古今第一完人”。

汪国真的《人到中年》里有这样一段关于中年人的描述:

“到了中年,生命已经流过了青春湍急的峡谷,来到了相对开阔之地,变得从容清澈起来。花儿谢了不必唏嘘,还有果实呢。”

人到中年,年富力强,但凡您不闲着,无论身处何等境遇都能干出一番成绩。





晚年经不得逆境

陈继儒在《小窗幽记》中写道:

“少年人要心忙,忙则摄浮气;老年人要心闲,闲则乐余年。”

人生苦短,到了老年,已知天命,正所谓“夕阳无限好,只是近黄昏”。

此时,若非迫不得已,依旧在逆境中摸爬滚打,不但会损害你的身心健康、甚至还会有危及性命的厄运。

白居易晚年的时候,得到新皇帝的赏识,三番五次将他调入京师,可他最终选择远避朝堂。

宦海沉浮多年,使他清醒地认识到朝堂的波峰诡谲、尔虞我诈,他本就无心权力,还不如做个地方官为老百姓办些实事来得痛快。

后来的事情也证明了他这一选择的正确性:

白居易63岁那年,中国历史上最为惨烈的宫廷流血事件“甘露事变”爆发,一日之内600多名朝臣被杀,其中有很多都是当年他同朝为官的老同事,而他却成功躲过了这场灾厄。

“晚年经不得逆境”,这句话在白居易身上表现得淋漓尽致,但这并不是意味着要你苟且偷安、无所事事。

孔曰:

“六十耳顺,七十而从心所欲。”

人到晚年,应该学会遵从自己的本心,选择最适合自己的生活姿态。

就像渡边和子在《心是一切温柔的起点》中的自白:

人都会老去,而我想漂亮地老去。老年人经常被人形容为“老而丑陋”,我觉得也可以用“老而美丽”这个词来形容。

“老而美丽”,与其说是具有半老徐娘般美丽的面貌,还不如说是一种具有年轮之美、心灵不起皱纹的生活姿态。

人生暮年,白居易创作出了大量的讽喻诗、闲适诗,流传至今的多达3000首,数量为唐代诗人之冠。

这些诗歌后来传到四域八方,一时间洛阳纸贵。

白居易也被后世尊称为“诗魔”;日本人更是对其推崇备至,奉其为“诗神”。



人生七十古来稀,白居易虽然体弱多病,但是晚年的他过着无拘无束的生活,像他的字号一样,白乐天,乐天知足,反倒活得比常人更加丰富多彩,福寿延绵,终年75岁。

人生是一场不得不走的远行,就算我们无法选择它的起点,但是我们却能选择如何走过生命的终点。

若是你在人生的旅途中偶尔感到困惑和迷茫,可以抬眼望一望走在你前面的白居易:

少年经不得顺境,中年经不得闲境,晚年经不得逆境。

在每一个年龄阶段做自己该做的事情,那么你的人生也许就会少点“逆境”,有点“闲境”,多点“顺境”。

愿有一天,我们真正远去,依旧可以给这个世界留下一个漂亮的背影。

(1)白居易在中年写下千古绝句:“青衫细马春年少。”追忆似水青葱年华,愿做一名少年郎!兹心想,中年油腻最好的解药,其实就是做一名不老的少年,保持纯粹的少年感。岁月无情,愿你出走半生,归来仍是少年!

(2)人生的大境界在于成长,少年在困境中成长,中年在闲境中成长,老年在逆境中成长,不停止成长,不停打磨内心,人生必然光亮。

(3)从别人的人生里明白事理是一种智慧,哪怕我们做不到白居易这样的大境界,但了解人生的这种可能性,也会潜移默化丰富我们的内心,并将这种影响传递甚至运用在我们的生活中!

*作者:洞见Fern,来源:洞见(ID:DJ00123987)——不一样的观点,不一样的故事,数百万人订阅的微信大号。

*朗诵:简宁,声音控。世界如此喧嚣,愿用声音给你这一刻心灵的安宁。个人微信:jianning20171114

不经一番寒彻骨,怎得梅花扑鼻香


声明:本文由入驻搜狐号的作者撰写,除搜狐官方账号外,观点仅代表作者本人,不代表搜狐立场。
阅读 (2348) 不感兴趣  投诉

\subsection{\href{http://www.sohu.com/a/226720023\_112783?\_f=index\_chan25news\_40}{能吃这四种苦,你也可以上清华! }}
\label{sec:org5cb2fd0}
Beauty will buy no beef.

——漂亮不能当饭吃。

提起清华,有的同学很向往,有的同学不敢想,有的同学出于吃不到葡萄就说葡萄酸的心态,
对清华不屑一顾,还说自己是清华永远清华学子直言高中生太幸福?能吃这四种苦,你也可
以上清华!我们先不要考虑能不能考上清华,而要反问自己,能不能吃得了清华人的苦?

有的同学可能会说,能考上清华的都是学霸级的,他们智力超群,学习起来既轻松又高效,
还有时间参加各种社团活动,我们智力一般,即便埋头苦学,也赶不上。然而,事实并非如
此。他们都是因为吃得了各种苦,才成为佼佼者的。下面,我们一起来细数清华学子吃过的
几种苦。

\subsection{第一苦}
\label{sec:orgb2db15e}

\subsubsection{禁得住诱惑,争分夺秒狂刷题}
\label{sec:orgb35c885}

考入清华的双胞胎姐妹说:上了高中,一天差不多学到了13个小时。甚至连假期也是如此。
这根弦,一绷就是六年。与分相比,电视、电脑、手机、iPad对我们全都没有诱惑力。

一位高考状元说:高中的时候,到校后,立刻从书包里拿出书来,开始一天的学习。除了上
厕所和跑操,基本不会离开自己的凳子。对我而言,一天只有四节课,上午、下午、晚自习
和回家后的自习。只要市场上能买到的习题集我都做过。

这看似很恐怖的题海战术,绝对是学好高中课程的好方法。

伟大的思想家马克思说过:“天才就是勤奋。”大家如果在上课期间去清华参观,就可以看
到所有的学生骑车如飞,走路也是小跑。因为早进教室,就可以比别人多看一会书,多做一
道题。

清华学子每年做的题甚至比高三的时候还要多。他们可以为了自己的目标放弃任何诱惑。就
算在大年三十,清华的自习教室也会人满为患。

用一位美国教授的话说:“Students of Tsinghua,no Saturday,no Sunday,
noholiday!”就是这种精神铸造了清华的神话。

\subsection{第二苦}
\label{sec:org861dc65}

\subsubsection{经住逼和压,练就超级学习能力}
\label{sec:org2d9baa3}

清华大学教授许纪敏先生说:“清华的人才怎么培养出来的?是压出来的!”

清华学子回忆说:

清华的老师在上完一节两小时的课程后,便给学生留下了一次课程设计,说做不出来的就是
挂科。而挂科到一定数量就要被退学处理。结果,两周时间,学生们就从入门菜鸟成为了可
以自主设计的大神。要知道,在别的大学这门课是要学一个学期的!

有一次上微机原理课,老师说,今天回去用Protel把课上的电路模拟一下。同学们说,我们
第一次听说这个软件。老师说:“这是电子工程人员必备的软件。”转身就走了。没办法,
大家回去在图书馆熬了3天终于把这个软件学会了。

不是被逼,谁都没想到这么短的时间可以做这么多的事。

清华学子说,相比之下,高中的学生实在太幸福了,老师会在考试前帮学生整理题目、常考
考点;会早早起床,督促学生上早自习;某一科出现下滑,老师会不遗余力地帮助寻因、补
习。但有些同学依然寻找借口,不愿意给自己太大压力。殊不知,巨大的压力带来的是挑战,
更是成长。正是得益于老师们的严格要求和教育方法,才缔造了一代又一代超级学习能力的
清华学子。

\subsection{第三苦}
\label{sec:orgfd7d3e1}

\subsubsection{“强制”锻炼,长跑练就好身体}
\label{sec:orge0f1f0c}

体育锻炼的风气贯穿了清华的百年历史。清华的口号是“为祖国健康工作50年”。所以清华
的体育课要求很高、很严格,每年要测3000米长跑,跑不过不许毕业,取消推研资格。

每年8月底9月初的一个午夜,清华大学新生军训都要上演20公里拉练。这拉练的距离之长在
各大高校的军训中很少见。

一位清华的学生说:

我们刚上大学的时候身体非常差,学习很容易累。现在,我们班今年有20个人报名参加北京
国际马拉松(全国报名的大学生只有5000人,清华就有3000人)。

\subsection{第四苦}
\label{sec:org044e5be}

\subsubsection{苦行僧,练就安身立命的本身}
\label{sec:orgac8a55f}

高中阶段的感情是非常美好的,那种感觉很甜蜜,但太麻烦了———你得偷偷摸摸的,不能被
老师和家长知道,最后只顾着烦了,什么甜蜜的感觉都没有了。就算是上了大学,清华的情
侣们到了大四也基本都分手了,好多同学觉得学一点安身立命的本事比寻求那些短暂的甜蜜
要有意义得多。

于是,清华就有了“本科僧”“研究僧”的说法。正是这些过着苦行僧生活的学生使得清华
的学风在世界上都有口皆碑。

同学们都想考入清华大学,可是清华学子每天的快节奏,你能适应的了吗?清华学子们吃的
这些苦,你能受得了吗?你能坚持下来吗?无论有没有考上清华,都请谨记清华的校训:自
强不息,厚德载物。将来的我们,也许不能顺心如意,也许处境艰难,但只要有清华这股自
强不息的劲头,一定可以让自己更优秀,更强大。

\subsection{\url{http://www.sohu.com/a/224851011\_557383?\_f=index\_chan25news\_55}[为什么美国精英中学距离中国孩子越来越远?]}
\label{sec:org381fd27}
美国精英教育的终极目标是培养未来能够影响世界并推动人类进步的社会卓越贡献者。——此
话多数中国人以为很虚,但这却是美国精英家庭孩子从小耳濡目染的价值追求。

两种不同的价值观,会塑造出孩子完全不同的人生角色:一种是精致的利己主义者,另一种
则是以天下为己任的社会卓越贡献者。



中国富裕家庭最早看到了两种命运的不同,抢占先机到美国找“名师”变“高徒”,从世界
至高点俯视全球。美国的多元文化,能够让孩子们更好地理解世界多样性,寻找到更丰富的
生命意义。精英圈子的“发小”情谊,也是人生的宝贵财富。而这些价值,也正是吸引高净
值家庭送孩子低龄留学美国的根源。

当出国留学变成潮流时,竞争态势自然水涨船高,录取条件也日益严苛。为了避开申高(8
申9、9申9)的惨烈竞争,也为了让孩子尽早适应美国校园文化,达到良好融入的目的,在
小学毕业后就送孩子赴美留学(6申6、6申7)的现象已蔚然成风。



硬币都有正反两面:年龄小、可塑性强固然是优势,但随着中国小留学生问题层出不穷,
“硬币的反面”逐步被美国精英学校和中国家长群体体认。美国学校要挑选的人,是能够在
异域文化中保持心理健康、在教师指导下善于自省并完善人格、在学术上与美国教育体系能
够接轨的孩子。但是,由于中美两国的不同教育标准,导致中国孩子若无充分的准备和持久
的训练,普遍不可能达到美国小学毕业生的认知能力和语言水平,更会导致美国精英中学离
他们渐行渐远。

中美教育标准有哪些显著不同?

“传授知识”VS“启发智慧”

中国人认为孩子天生就是一张白纸,只有在老师的教授下才能获得知识并掌握技能;而美国
人认为,每个孩子天生就具备卓越的才华,老师的作用只是通过启发将孩子的智慧引导出来。

因此,中国的教育关注学生具体“学到”什么,学习的内容重在对“外部世界”的认知,而
对自我认知和人际世界的认知,基本空白;而美国的教育更在意学生在学习过程中是否真的
“会学”?除了认识“外部世界”,更要求孩子借由认识外部世界去更好的认知人际世界和
自我内心世界。这种教育本源认知上的不同,造成了中美在教育标准、教育方法、教育策略
乃至教育结果上的根本不同。而且,习惯于外部驱动的中国学生和习惯于内部驱动的美国学
生相比,缺乏美国精英学校极为重视的遵纪并自律、好奇心、创新能力、积极主动和独立精
神。

“知识记忆” VS“批判思维

中国的教育以知识记忆为主,而美国的教育更关注在学习过程中培养孩子的批判性和创造性
思维进而去进行研究型学习:即要求学生整合阅读能力、分析能力、合作能力于一体去完成
项目制学习研究。美国学校因此对小学阶段思考力和研究力近乎空白的中国留学生有着巨大
的挑战!

侧重知识记忆的中国教育的重点——记忆、理解和应用与侧重思维培养的美国教育标准——分析、
综合、评价相比,在思考能力、思维模式上有很大的劣势。长期在“死记硬背”教育环境中
的孩子,又独自在异域文化的环境下,想要短期内完成思维模式的转变跃升,可以想象到是
多么艰难,甚至可能会适得其反。

中美教育,不仅是教育理念的不同,更是在具体标准上有巨大差异。与美国的同龄人相比,
中国的小学生存在哪些“先天性”差距,导致美国学校远离中国学生?

阅读能力

1.阅读的广度和深度差距巨大,无论是质的要求还是量的要求,都是数量级的差别。

按CCSS全美教学标准要求,美国小学毕业生至少需要阅读1404本课外读物,阅读量占到全部
K12年级阅读量的77.6\%,且涉及故事、诗歌、戏剧、信息类等13种文体。在阅读深度上,美
国要求小学毕业生不仅能够对文章细节与主旨进行精准理解,还要求对文章结构与内在逻辑
进行严密分析!

反观中国小学5-6年级的语文教学大纲,不仅对课外阅读量要求低(五年制阅读总量不少于
100万字即可),而且在阅读深度上只要求做到“初步理解”句、段、篇之间的联系和分析
概括文章主旨即可。中国对孩子阅读基本功的教育可谓具有“先天缺陷”!



2.阅读技能、阅读策略及落实手法也差距极大

中国小学阶段对阅读能力的要求是记忆、理解和应用,而美国小学生在毕业时已经能够对阅
读材料进行一定的分析综合与评价。例如通过讨论式阅读,来增强对人际世界的认知和社交
能力;通过阅读进行反思,提高自我认知能力等。因此,进入初中后,以分析、综合、评价
为主的阅读作为最基本的能力,老师默认大家都会,没人会专门再去教。中国的低龄留学生,
如果不事先将阅读“童子功”补足,不仅难以进入美国精英初中,即便侥幸进入也难以在美
国教育体系中成为佼佼者。

阅读作为两个输入项之一,中美之间的巨大差异,导致写作方面的差距也是惊人。

写作能力

\begin{enumerate}
\item 写作题材的数量和写作难度存在量级差距
\end{enumerate}

与美国小学生要写出观点鲜明、逻辑严谨、结构清晰的全文体(记叙文、说明文和议论文)
才算达标的写作要求相比,中国小学生的写作要求是写出有中心思想、有条理、有真情实感
的简单记叙文!

从小学开始,美国的孩子就已经学会了“老师布置课题—学生分组分解课题—分头研究子课题—自
行阅读延展材料和参考教材—小组开会总结提炼观点—修正原始课题和观点—补充材料—撰写报
告—修正报告—提交作业”的研究性写作。而中国的小学生从小极度缺乏这种运用批判性和创
造性思维去进行研究性写作的机会,还只停留在对观察的事物进行描述的浅层思维写作上!
这样的孩子如果没有接受系统性训练,几乎不可能在美国初中拔得头筹!

2.写作时长与耐力的差距

美国小学生既能快速完成短时间基础性写作,也会根据特定写作任务进行长时间(一周或一
月)大论文写作。而中国的小学生只具备基本写作能力,根本不被要求长时间写作和研究性
写作!

这种要求上的差别导致学生的思维方式、写作能力和写作耐力存在巨大差距。而大项目写作,
是美国初高中越来越要求的教学重点,中国的孩子若没有在这个方面打下坚实基础,去了美
国面对要求用流畅、正规甚至严谨又优美的语句去完成Essay作业,不啻于一场噩梦!为什
么在美国一些地方出现“代写作业”的荒唐服务,与中国学生低下的写作能力也有一定的关
系!

听说能力

中国小学生的听说能力只局限在“信息接收——关键信息复述”的单点思维上,无法对所听到
的内容进行概括总结并在逻辑思辨后用自己的观点进行“输出”。这恰恰是美国初中所要求
的重点能力!

试想,这样的中国孩子在美国以讨论与观点表述为主的课堂学习中,怎能不成为“沉默的隐
形人”(silent student)呢?

总之,中美教育标准的巨大差异,导致同为小学毕业生的中国孩子和美国同龄人在学术与社
交能力上存在较大的差距!来自基础教育的盲点,是造成中国留学生在海外学习和发展过程
中持续发生系统性误差的根源!而且出去留学的时间越晚,差距越大!这也是为什么聪明勤
奋肯吃苦的中国孩子来到美国后,不仅在世界一流大学很少有人能够出类拔萃,而且在国际
高科技领域中也罕有卓越成就的原因之一。

本文转自《三联生活周刊》2016年7月刊
\subsection{\href{http://culture.ifeng.com/a/20180311/56639527\_0.shtml}{呼吁减负?读书不是“天下第一好事”又是什么呢}}
\label{sec:org6e76cf4}
呼吁减负?读书不是“天下第一好事”又是什么呢?
2018年03月11日 12:09:41
来源:凤凰文化 作者:季羡林

18人参与 17评论
编者按:新学期开始以来,减负话题愈演愈烈。一方面,以《教育部,请不要给我的孩子减
负》为代表,反对减负的声音引发越来越大的回响,“不让中国孩子学习高难度的知识,怎
么选拔和识别天才呢?你在放松的时候,别的国家可没有放弃。”

“减负其实就是国家从教育领域退出,把教育的责任从学校推给家庭。自从中国开始给学生
减负,学校从教育领域往后退以来,中国的家长们是越来越累了,负担越来越重了,各种教
育机构大赚特赚。”

而另一边,教育部依然继续强推减负,力挺的人不在少数。对于老生常谈的“减负”,我们
理应有一些基本共识。不合理的负担当然应该减下去,但应该承认、至少不去贬低勤奋刻苦、
努力读书的价值。而反对减负、不想孩子输在起跑线上的家长又想孩子实现怎样的人生价值
呢?

“教育读书”的话题始终是贯穿一生的。连在大学里,每逢考试季,有学生会用季羡林先生
写的日记为自己的懒惰开脱,“没做什么有意义的事——妈的,这些混蛋教授,不但不知道自
己泄气,还整天考,不是你考,就是我考,考他娘的什么东西?”可是他们却故意看不见季
老另一句发自肺腑的话,“总而言之,‘天下第一好事,还是读书’。”

季老多少是有先见之明的,今日不妨听听他理解中的天才、读书和所谓的人生意义。



季羡林

我害怕一类“天才”

人类的智商是不平衡的,这种认识已经属于常识的范畴,无人会否认的。不但人类如此,连
动物也不例外。我在乡下观察过猪,我原以为这蠢然一物,智商都一样,无所谓高低的。然
而事实上猪们的智商颇有悬殊。我喜欢养猫,经我多年的观察,猫们的智商也不平衡,而且
连脾气都不一样,颇使我感到新奇。

猪们和猫们有没有天才,我说不出。专就人类而论,什么叫做“天才”呢?我曾在一本书里
或一篇文章里读到过一个故事。某某数学家,在玄秘深奥的数字和数学符号的大海里游泳,
如鱼得水,圆融无碍。别人看不到的问题,他能看到;别人解答不了的方程式之类的东西,
他能解答。于是众人称之为“天才”。但是,一遇到现实生活中的问题,他的智商还比不了
一个小学生。比如猪肉三角三分一斤,五斤猪肉共值多少钱呢?他瞠目结舌,无言以对。

因此,我得出一个结论:“天才”即偏才。

在中国文学史或艺术史上,常常有几“绝”的说法。最多的是“三绝”,指的是诗、书、画
三绝。所谓“绝”,就是超越常人,用一个现成的词儿,就是“天才”。可是,如果仔细分
析起来,这个人在几绝中只有一项,或者是两项是真正的“绝”,为常人所不能及,其他几
绝都是为了凑数凑上去的。因此,所谓“三绝”或几绝的“天才”,其实也是偏才。

可惜古今中外参透这一点的人极少极少,更多的是自命“天才”的人。这样的人老中青都有。
他们仿佛是从菩提树下金刚台上走下来的如来佛,开口便昭告天下:“天上天下,唯我独
尊。”

这种人最多是在某一方面稍有成就,便自命不凡起来,看不起所有的人,一副“天才气”,
催人欲呕。这种人在任何团体中都不能团结同仁,有的竟成为害群之马。从前在某个大学中
有一位年轻的历史教授,自命“天才”,瞧不起别人,说这个人是“狗蛋”,那个人是“狗
蛋”。结果是投桃报李,群众联合起来,把“狗蛋”的尊号恭呈给这个人,他自己成了“狗
蛋”。

这样的人在当今社会上并不少见,他们成为社会上不安定的因素。

蒙田在一篇名叫《论自命不凡》的随笔中写道:

对荣誉的另一种追求,是我们对自己的长处评价过高。这是我们对自己怀有的本能的爱,这
种爱使我们把自己看得和我们的实际情况完全不同。

我决不反对一个人对自己本能的爱。应该把这种爱引向正确的方向。如果把它引向自命不凡,
引向自命“天才”,引向傲慢,则会损己而不利人。

我害怕的就是这样的“天才”。

1999年7月25日

天下第一好事,还是读书

古今中外赞美读书的名人和文章,多得不可胜数。张元济先生有一句简单朴素的话:“天下
第一好事,还是读书。”“天下”而又“第一”,可见他对读书重要性的认识。

为什么读书是一件“好事”呢?

也许有人认为,这问题提得幼稚而又突兀。这就等于问:“为什么人要吃饭”一样,因为没
有人反对吃饭,也没有人说读书不是一件好事。



季羡林

但是,我却认为,凡事都必须问一个“为什么”,事出都有因,不应当马马虎虎,等闲视之。
现在就谈一谈我个人的认识,谈一谈读书为什么是一件好事。

凡是事情古老的,我们常常说“自从盘古开天地”。我现在还要从盘古开天地以前谈起,从
人类脱离了兽界进入人界开始谈。人成了人以后,就开始积累人的智慧,这种智慧如滚雪球,
越滚越大,也就是越积越多。禽兽似乎没有发现有这种本领,一只蠢猪一万年以前是这样蠢,
到了今天仍然是这样蠢,没有增加什么智慧。

人则不然,不但能随时增加智慧,而且根据我的观察,增加的速度越来越快,有如物体从高
空下坠一般。到了今天,达到了知识爆炸的水平。最近一段时间以来,“克隆”使全世界的
人都大吃一惊。有的人竟忧心忡忡,不知这种技术发展“伊于胡底” 。信耶稣教的人担心
将来一旦“克隆”出来了人,他们的上帝将向何处躲藏。

人类千百年以来保存智慧的手段不出两端,一是实物,比如长城等,二是书籍,以后者为主。
在发明文字以前,保存智慧靠记忆;文字发明了以后,则使用书籍。把脑海里记忆的东西搬
出来,搬到纸上,就形成了书籍,书籍是贮存人类代代相传的智慧的宝库。后一代的人必须
读书,才能继承和发扬前人的智慧。

人类之所以能够进步,永远不停地向前迈进,靠的就是能读书又能写书的本领。我常常想,
人类向前发展,有如接力赛跑,第一代人跑第一棒;第二代人接过棒来,跑第二棒,以至第
三棒、第四棒,永远跑下去,永无穷尽,这样智慧的传承也永无穷尽。这样的传承靠的主要
就是书,书是事关人类智慧传承的大事,这样一来,读书不是“天下第一好事”又是什么呢?





季羡林

但是,话又说了回来,中国历代都有“读书无用论”的说法,读书的知识分子,古代通称之
为“秀才”,常常成为取笑的对象,比如说什么“秀才造反,三年不成”,是取笑秀才的无
能。这话不无道理。

在古代——请注意,我说的是“在古代”,今天已经完全不同了——造反而成功者几乎都是不识
字的痞子流氓,中国历史上两个马上皇帝,开国“英主”,刘邦和朱元璋,都属此类。诗人
只有慨叹“刘项原来不读书”。“秀才”最多也只有成为这一批地痞流氓的“帮忙”或者
“帮闲”,帮不上的,就只好慨叹“儒冠多误身”了。

但是,话还要再说回来,中国悠久的优秀的传统文化的传承者,是这一批地痞流氓,还是
“秀才”?答案皎如天日。

这一批“读书无用论”的现身“说法”者的“高祖”“太祖”之类,除了镇压人民剥削人民
之外,只给后代留下了什么陵之类,供今天搞旅游的人赚钱而已。他们对我们国家竟无贡献
可言。

总而言之,“天下第一好事,还是读书”。

对大多数人,人生一无意义,二无价值

当我还是一个青年大学生的时候,报刊上曾刮起一阵讨论人生的意义与价值的微风,文章写
了一些,议论也发表了一通。我看过一些文章,但自己并没有参加进去。原因是,有的文章
不知所云,我看不懂。更重要的是,我认为这种讨论本身就无意义,无价值,不如实实在在
地干几件事好。

时光流逝,一转眼,自己已经到了望九之年,活得远远超过了我的预算。有人认为长寿是福,
我看也不尽然。人活得太久了,对人生的种种相,众生的种种相,看得透透彻彻,反而鼓舞
时少,叹息时多。远不如早一点离开人世这个是非之地,落一个耳根清净。

那么,长寿就一点好处都没有吗?也不是的。这对了解人生的意义与价值,会有一些好处的。



季羡林

根据我个人的观察,对世界上绝大多数人来说,人生一无意义,二无价值。他们也从来不考
虑这样的哲学问题。走运时,手里攥满了钞票,白天两顿美食城,晚上一趟卡拉0K,玩一点
小权术,耍一点小聪明,甚至恣睢骄横,飞扬跋扈,昏昏沉沉,浑浑噩噩,等到钻入了骨灰
盒,也不明白自己为什么活过一生。

其中不走运的则穷困潦倒,终日为衣食奔波,愁眉苦脸,长吁短叹。即使日子还能过得去的,
不愁衣食,能够温饱,然而也终日忙忙碌碌,被困于名缰,被缚于利索。同样是昏昏沉沉,
浑浑噩噩,不知道为什么活过一生。

对这样的芸芸众生,人生的意义与价值从何处谈起呢?

我自己也属于芸芸众生之列,也难免浑浑噩噩,并不比任何人高一丝一毫。如果想勉强找一
点区别的话,那也是有的:我,当然还有一些别的人,对人生有一些想法,动过一点脑筋,
而且自认这些想法是有点道理的。

我有些什么想法呢?话要说得远一点。当今世界上战火纷飞,人欲横流,“黄钟毁弃,瓦釜
雷鸣”,是一个十分不安定的时代。但是,对于人类的前途,我始终是一个乐观主义者。我
相信,不管还要经过多少艰难曲折,不管还要经历多少时间,人类总会越变越好的,人类大
同之域决不会仅仅是一个空洞的理想。



但是,想要达到这个目的,必须经过无数代人的共同努力。有如接力赛,每一代人都有自己
的一段路程要跑。又如一条链子,是由许多环组成的,每一环从本身来看,只不过是微不足
道的一点东西;但是没有这一点东西,链子就组不成。在人类社会发展的长河中,我们每一
代人都有自己的任务,而且是绝非可有可无的。如果说人生有意义与价值的话,其意义与价
值就在这里。

但是,这个道理在人类社会中只有少数有识之士才能理解。鲁迅先生所称之“中国的脊梁”,
指的就是这种人。对于那些肚子里吃满了肯德基、麦当劳、比萨饼,到头来终不过是浑浑噩
噩的人来说,有如夏虫不足以与语冰,这些道理是没法谈的。他们无法理解自己对人类发展
所应当承担的责任。

话说到这里,我想把上面说的意思简短扼要地归纳一下:如果人生真有意义与价值的话,其
意义与价值就在于对人类发展的承上启下,承前启后的责任感。

\subsection{\href{http://www.sohu.com/a/208387303\_372406}{大学四年,哈佛的学生都在学什么?}}
\label{sec:org49fc885}
2017-12-04 19:06 大学/哈佛/教育改革
陈寅恪定义大学精神的第一要义为]
大学四年,哈佛的学生都在学什么?
2017-12-04 19:06 大学/哈佛/教育改革
陈寅恪定义大学精神的第一要义为:*独立之精神,自由之思想。*现代教育之父洪堡指出:大
学应实施通识教育,而不应涉足职业教育。
作为世界最著名的高等学府之一,哈佛让本科生们学什么?哈佛传统的“自由教育”的基本
元素是什么?他们的通识教育又经历着怎样的变化呢?


对于一个美国人来说,大学岁月的重要性是怎么估计也不会过分的。它们是使他文明开化的
唯一途径……面对这样一个行将接受教育的人,我们必须思索这样一个问题:如果他能够被
称为受过大学教育,他应当学习什么?
—阿兰·布鲁姆《美国精神的封闭》
\subsection{哈佛人都在学什么?}
\label{sec:org5a1d7d1}
据说,在每个哈佛人的一生中,都会出现这样的时刻,他或她突然意识到哈佛的魅力。我想
很多人的那个瞬间是在拿到长达上千页的选课单,为自己选择在哈佛的第一门课的时候。

2013年哈佛新的通识教育计划正式推行,重新划分了学生需要涉猎的八大知识范畴领域,艺
术与诠释、文化与信仰、经验推理、伦理推理、生命系统科学、物理世界科学、世界中的社
会、世界中的美国,共计400多门课程。

哈佛学生在上一堂天文课
图片来自三联生活周刊
如果说,大学教育的价值在于为一个人的一生提供一个时间段,在此期间,他的求知欲最为
旺盛,心智最为开放,并得以远离社会求速成的压力,学习如何发问,去怀疑既定的前提,
学会天马行空的思考,那么,当几乎全人类的知识一起摆在他的眼前,而他必须从中选出32
门值得花费四年时间的课程时,除了对于这场知识的盛宴充满兴奋之外,恐怕还有巨大的困
惑与不安。
02
\subsection{何谓受过良好教育的人?}
\label{sec:org2990f01}
“21世纪前25年,何谓‘受过良好教育的人’?”这是哈佛文理学院院长威廉·科比在2003
年一次通识教育改革会议上提出的第一个问题。当一所走过了近400年历史的大学回头审视
自我时,这是一个最简单,却也最艰难的问题。从2013年开始,哈佛的本科生全面推行一套
新的通识教育计划(Gen Ed),以取代20世纪70年代末设计的“核心课
程”(Core-Curriculum)。
对哈佛学生来说,没读过莎士比亚更可耻,还是不知道染色体与基因的区别更丢人?哪些知
识是重要到必须教给每个学生的?比如足够多的经济学知识让他们看懂华尔街的财务报表,
足够多的科学素养让他们读懂《科学美国人》上的每篇论文,还是足够多的幽默感让他们看
懂《纽约客》上的笑话?
“受教育”与技术训练不是一回事。尤其在西方“自由教育”(Liberal Education)的视
野之内,一个受过教育的人,必须理解自己以及自己在世界中的位置——文化的与自然的——从
而追求一种富有意义的人生。它要求一种历史性的视角,让一个人不至于陷溺于一时一地的
现实考量,活得像一只“夏天的苍蝇”(埃德蒙德·博克)。
这样的“教育”必然包含英国19世纪著名的诗人和学者马修·阿诺德所说的“曾经被了解过
的与被述说过的最好的一切”。它必须理解整体——人类世界与它的历史,我们的文化与那些
不同于我们的文化,自然世界与探究的方法,量化的与语言的技巧,还有活泼的艺术。
就像校徽上刻着的“真理”二字一样,“自由教育”——在自由探究精神指导下的不预设目标、
不与职业相挂钩的教育,是哈佛大学在近400年的历史中一直坚持的一个理想,尽管在越来
越世俗化和功利化的今天,这种坚持已经变得越来越艰难。有一种说法是自由教育就像教堂,
专业教育则像医院。大家都知道医院是干什么的,但说不清楚教堂到底有什么用,但它确实
还有某种深远的影响和作用,比如慰藉和回答人的真正需要。
哈佛通识教育委员会主席J.哈里斯在接待两位前去取经的中国学者时说:“自由教育的特点
是又宽又深。所谓宽,是教给学生的整个知识范围‘宽’,深则意味着要深入各个专业,每
一门课都讲究深度。”

一名学生走在哈佛校园里
图片来自路透社
从20世纪40年代开始,哈佛就把本科生的课程分成三个部分:主修课、选修课、通识课(此
外还有写作课与课外活动)。按照J.哈里斯的说法,这些都是哈佛式“自由教育”的基本元
素。

主修课致力于培养学生对某一学科的深入理解,这是专业化时代的要求——只有当一个人深入
钻研了某一复杂学科之后,不仅学会分析问题,还要能合理地解释解决问题的过程,才能明
白真正的智力探究与探索是什么意思。即使一个人学生时代选择的专业与他未来的事业之间
毫无关联,或者20年后将所学的专业知识全部忘光,他至少懂得精通一门专业是怎么回事。
选修课占四分之一,是为了让学生按自己的兴趣自由探索主修专业之外的知识,比如一个文
科生偶尔也会仰望星空,追问宇宙大爆炸是怎么回事;或者一个满脑子代码的计算机系学生
可能也愿意欣赏一点贝多芬、莫扎特或印象派。
剩下四分之三则是通识课(General Education)。所谓通识课程,就是学校提供给本科生
的一系列基础课程,学生必须从中选出几门作为必修课,无论他们的专业或兴趣是什么。

03
\subsection{哈佛通识课怎么变迁?}
\label{sec:org3e2a6ab}
在哈佛,主修课可以任意选、任意换,连专业也可以换,唯有通识课属于校方指定必修的,
非选不可。这是大学主动为一个年轻人的4年求学生涯开出的一张关键处方,代表了一所大
学对于知识与教育最基本的哲学与态度:一个人在大学期间应该学些什么?什么知识或方法
是每个学生都应掌握的?大学最希望培养的是什么样的人?

比如哥伦比亚大学认为有些书是每个人毕业之前都应该读过的,不是任何一本书,而必须是
荷马、柏拉图、索福克勒斯、奥古斯丁、康德、黑格尔、马克思、伍尔夫的著作……为什么?
因为这些是最戏剧性地建构了“西方”的著作者,他们的书是一些最直接的涉及什么是人以
及人可以是什么的书,它们应该成为每个人的教养的一部分。

哈佛大学则认为,比起古典名著或者最前沿的科学知识,某些学问的方法才是学生必须掌握
的,比如你可以没读过莎士比亚的作品,但必须在教授的指导下以评论和分析的方式研读过
经典文学;你可以不了解法国大革命的历史,但你得懂得如何将历史作为一种探究和理解的
方式,观察和分析当今世界的主要问题;你可以没上过“经济学原理”,却不能没修过一门
探讨社会问题基本原理的课程。

一个哲学系的学生应该能理解物理学的基本观点:这个世界是一个理性的、可预知的系统,
我们可以通过经验发现其规律;而一个穿着白大褂在实验室里捣鼓细胞的生物系学生应该具
备最基本的道德推理能力,以应对未来可能遭遇的道德困境。这是哈佛运行了30多年的通识
教育系统——“核心课程”的基本观点:在一个知识爆炸的时代,本科教育的重心必须从具体
知识的获取转化到“获取知识的方法与途径”。

哈佛著名的“正义课”就是一门“核心课程”。在第一节课的末尾,桑德尔教授就对学生发
出了警告:这门课并没有教给你任何新的知识,而是通过将你原本熟知的事物变得陌生,给
予你另一种看待事物的方法。“这是一种风险:一旦那些熟悉的东西变陌生了,就再也不会
和以前一样了。”他说,“自我知识就像失去的天真——无论这让你多么不安,你也不可能再
回头。”

30年前,哈佛“核心课程”的设计者亨利·罗索夫斯对于“何谓一个受过良好教养的人”有
着明晰的界定:能清晰而有效地思考和写作;在某些知识领域具有较高的成就;对宇宙、社
会及人类自身有深邃的理解;勤于思考伦理道德问题,具有明智的判断力和抉择力;具有丰
富的生活经验,对于世界各种文化及时代有深刻的认识。

哈佛中心校园
图片来自三联生活周刊
今天,哈佛认为,“核心课程”已经过时了——既然只有10\%的哈佛学生会选择以学术为业,
而60\%会进入商业、律师、医学等职业领域,为什么还要花费那么多的精力试图把他们塑造
成学者、教授呢?但对于未来的律师、医生、商人们,这个汇聚了世界上最多天才的大学,
却无法为21世纪前25年的“良好教育”开出一份明确的清单。
对此,哈佛通识教育改革委员会的成员之一、英语系教授路易斯·梅纳德(Louis Menand)
是这样分析的:“在知识专业化愈演愈烈的时代,绝大部分教授都是专门学科的专家,他们
在自己的领域有足够的权威,他们能告诉你,如何才能成为英语教授、物理学家、经济学家
等等,但对于一个‘普通的知识核心’,或者‘所有人都应该知道的知识’,他们一样困惑,
不可能在任何具体的内容上达成共识,这不是他们的惯常思维。”
04
\subsection{在一个不确定的时代应该怎么生活?}
\label{sec:orga7ca51e}
2007年10月21日,哈佛现任女校长德鲁·福斯特在她的就职演讲中特别提到一封来自50年前
的信,是1951年科南特校长委托哈佛档案馆保存,并转交给“下一世纪开始时”的哈佛校长
的。在信中,他担心第三次世界大战的一触即发,“很有可能使我们所居住的城市包括剑桥
在内遭到破坏……我们都想知道,自由世界在未来的50年里会如何发展”。
“正如科南特所处的时代一样,我们也处于一个使我们有充足的理由忧虑不安的世界,我们
面对的是不确定。”福斯特校长说。

哈佛最新一轮的通识教育改革很大程度上是对这个时代的“不确定性”的一种回应。全球化
与科技革命是其中最大的两个不确定因素,所以新课程计划中加重了科学的比例,并且一再
强调“国际化视野”和“合作意识”。

作为一项古老的传统,在很长一段时间内,哈佛都认为,真正的学者“必须拥抱孤独,并把
孤独作为自己的新娘”。至少在这个世纪,他们希望培养的,绝不是象牙塔里孤独的学者,
而是能在未来世界里长袖善舞、应付各种各样挑战的人。在所有的现代心智训练中,他们尤
其强调这样一种训练:将学生置于一个陌生的环境,让他们接触超越他们理解力——甚至也超
越教师理解力——的现象,让他们失去方向,然后通过学习和思考,重新找到方向。也许这才
是21世纪前25年所谓的“良好教育”。

2007年通过的《通识教育工作组报告》这样写道:“我们在报告中所描述的通识教育计划的
理想,就是要使本科生能够在一个他们毕业后将成为什么人和他们将过什么样的生活的这样
一个视野下,在哈佛课堂的内外进行一切学习。”

也就是说,哈佛所认可的“共同的知识核心”回归到了“生活”本身。在一个不确定的时代,
我们应该怎么生活?什么是美好生活的结构?什么样的成功才包含真正的幸福?公共事务中
什么是正义,什么是不公?
按照福斯特校长的说法,哈佛校徽上的“真理”(Veritas)是指一种基于 *理性、挑战、不
安和怀疑的理解之道 *。但如果这种理解之道能帮助一个学生直面未来生活的各种变故与不确
定性,更好地与自己所生存的世界打交道,理解它的复杂性,以及自己在其中扮演的角色,
从而拥有一个更美好和富有意义的人生,有何不可呢?
本文节选自《大学的精神:教育是让一个人成为最好版本的自己》,作者为三联生活周刊主
笔陈赛。返回搜狐,查看更多

声明:本文由入驻搜狐号的作者撰写,除搜狐官方账号外,观点仅代表作者本人,不代表搜狐立场。
:独立之精神,自由之思想。现代教育之父洪堡指出:大学应实施通识教育,而不应涉足职业教育。
作为世界最著名的高等学府之一,哈佛让本科生们学什么?哈佛传统的“自由教育”的基本
元素是什么?他们的通识教育又经历着怎样的变化呢?



\subsection{\href{http://www.imooc.com/article/12768}{成长路径:送给准备入行的同学}}
\label{sec:org461c55a}
先跟大家说一个事实:

我公司招聘(小初创公司 )的时候, 曾经一个星期收到一千封简历: 600 IOS, 200 安
卓, 200 美工。 95\%都是应届生。


前些天跟几个培训机构的朋友聊天, 大家也纷纷表示, IOS现在工作不好找。

所以, 各位同学务必记住: 全栈开发才是趋势, 一人多能才是出路!

如果你只会切菜, 你就是打杂的。
如果你不但会切菜,还会煎炒蒸炸,粤菜鲁菜都精通,你就是大厨! 老板离不开你!
下面是我总结的全栈开发的路径:(按照顺序来看,提及到的具体技术,请大家自行google
官方网站来学习。)


\subsection{HTML , CSS}
\label{sec:orgd8b575b}
互联网开发的本质是web开发。 这个是基础中的基础。

不要以为android, ios中用不到CSS。 现在越来越多的app已经被页面的改动拖死了。 微信,
QQ,京东都开始越来越多的使用 native 壳 + webview了。 所以, 大家务必把 CSS 学好!


时间: 20分钟看概念。(主要是看CSS的各种属性), 然后自己动手写上3,5个静态页面。 也就熟悉了。

javascript, jQuery, jQuery的各种插件。
javascript的语法非常简单, 5分钟入门。 (基本上你学会了for, if, try, 声明数组,
hash, 以及 this的用法,也就没了)


jQuery 可以认为是javascript的人性化改版。 好多难用的,难以理解的函数,都被改造成
了对程序员和蔼可亲的函数。 这个非常棒。


jQuery 操作DOM是重点。 这里建议大家多看看书, 30分钟看书,6 个小时练习。

jQuery Plugs
光会jQuery是远远不够的。 我们在实际的工作当中,是要为web页面添加很多功能的。 所
以, 要使用各种各样的插件。


下面是我回顾过去多年的项目经验,梳理出来的最常用的若干组件:

上传图片 jquery-file-upload

弹出窗口 modal dialog

轮播图 slider

表单验证 form-validate

树状菜单 jstree

图表 highchart

时间选择器 datepicker

标签页 tab

提示框 tooltip

快捷编辑 inplace editor

超级强大的下拉框: select2

地图插件: amap 高德地图。 bmap 百度地图

标签 tag

动画 animation

限制性输入框 masked input
基本上,只要大家把上面的这些插件都“熟悉了”(也就是每个都做过demo), 那么做项目
的时候, 老板就会对你很放心了。 也许你的算法不好,没关系啊,能解决问题就行。



\subsection{数据库基础:SQL}
\label{sec:orgdee3afc}
很多参加过前端培训的同学, 我发现都是不懂数据库的。 所以我把它单独列出来。

数据库就用 Mysql. 会用之后, 发现 sqlite, postgres 等等都是一样的。

基本功:

select \ldots{} where .
update
insert
delete
创建,删除数据库,
创建,删除index
foreign key 与 1对多,多对多的关联关系,以及 join.
基本上,把这些了解过了, 你对数据库也就能上手用了。(注意, 务必都要手动敲一遍)


同时, 要熟悉 安装mysql, mysql GUI 客户端的过程。

\subsection{一门后端语言, 比如Ruby}
\label{sec:org029794f}
后端语言, 包括 Ruby, Python, Java, PHP, .NET等等。

这里我推荐Ruby. 它非常优雅, 简单。 不花哨。 同样的事儿, Ruby 语言2行代码, 用

Java写的话,你的先写3个Class再加一堆getter setter.


学习语言的基本功, 包括几个方面:

数据类型, 以及各种转换。
常见的 for, if-else, try-catch
this/self , 作用域这样的高级特性
(指针啊, static non-static, 元编程这样的特性,可以边做项目边学)
\subsection{ruby\url{http://tryruby.org/levels/4/challenges/4}}
\label{sec:org298d35a}
\subsection{web开发的框架, 比如Rails, Spring, Play.}
\label{sec:org197e991}
框架跟语言不一样。

学习了语言,你会发现 你其实只能写个算法。 连GUI都做不了。所以,我们需要借助框架
来写Web.


java世界中比较知名的有:SSH (Spring Struts Hibernate, 居然跟十年前没太大变化)
以及 Play这个借鉴了Rails的框架。 Python中有Django, Tornado, PHP中有Cake,
ThinkPhp等等。


还是用Rails吧。 那么, 无论任何一个Web框架,都需要具备下面一些基本功能:

ORM(数据持久层)
让我们免于直接书写 select from 这样的语句。 有的同学问为什么, 当你见到100行的
select from 的时候,你就会极度受不了这样的语法的。 Java中的框架叫 Hibernate.
Rails中自带这个叫ActiveRecord.


处理路由。
Java的第一个web框架Struts,05年红遍了大江南北。 它被人最推崇的地方,就是如何把一
个 "/some\(_{\text{controller}}\)/action" 这样的URL, 交给对应的controller 和 action 来处理。
Rails中叫router, 可以认为更加简单方便。


\subsection{必要的表单辅助方法。}
\label{sec:org29ddd8f}
例如, <select> 标签,就是很复杂的东西。我曾经见到一个很著名的收费项目,没有使用
表单辅助方法, 手写select标签, 写了8000多行。我都无语了。 COPY起来机器都卡。 而
在rails中, 3行代码搞定。


\subsection{良好的页面渲染。}
\label{sec:org11e9378}
比如JSP, PHP, 都是这样。 在Rails中,我们使用erb. 可以很方便的在controller中定
义变量, 在view中渲染。 也可以把公共的页面提取出来, 然后被其他页面公用。


\subsection{方便的单元测试。}
\label{sec:orgeb174f0}
当老板问起:“这个项目明天能上线吗?” 你的时候心是特别虚的。 但是,如果你有单元测
试, 就可以先运行个命令(例如 \$ rake tests)然后,告诉你的老板: “我们总共有300
个单元测试, 通过了290个, 失败的10个不影响核心功能,我觉得明天可以上线”。 是不
是就很专业?


\subsection{数据库迁移}
\label{sec:org9210153}
这个可能刚入行的同学很少听说。 据我了解,在python, java中都比较少见。 只在Rails
中见到。 这是对于“数据库的版本控制”。


我们在团队协同开发时,最大的问题是: 数据库容易不同步。 比较low的解决办法是把
sql文件放到代码仓库中,然后有了改动随时更新。


最好的办法, 是把数据库的变迁,使用代码来表示。 这样的话, 别人下载来代码, 运行
个命令, 数据库就是最新的了。


\subsection{后端的组件大集合}
\label{sec:org6f3d985}
下面是我梳理了过去十多年的项目,总结出来的最常见的web后端组件:

\begin{enumerate}
\item 分页
\end{enumerate}

把结果集合按页显示。这个东东看起来常见, 实际上实现起来特别复杂。甚至专门有个论
文: “分页设计模式”


在Rails中特别简单,使用 Karminari

\begin{enumerate}
\item 上传图片
\end{enumerate}

后端处理上传的过程也极其常见:

处理request
保存到本地
生成各种尺寸的缩略图
等等。
但是这么简单的东东在任何语言里面都很麻烦。

在Rails中, 使用 carrierwave

\begin{enumerate}
\item 上传图片到云端CDN
\end{enumerate}

仅仅把图片上传到你的本地服务器还不够。 图片在北京服务器上,西藏的朋友访问起来是
特别慢的。


最初 CDN只是极少数有财力的公司的专利。 现在越来越多的公司开始为平民提供服务了。
upyun就是其中一个。


ruby中的用法已经省略到几行代码了。 具体见 'upyun'的例子。

\begin{enumerate}
\item 发送短信
\end{enumerate}

这个是最简单的技术。但是用的也最多。基本过程就是:

申请账号,充钱
添加个短信模板
使用代码向短信服务商发送请求。
短信就发送到了用户手机上。
任何语言都可以很方便的实现它。 具体的短信提供商自行搜索吧。

\begin{enumerate}
\item 所见即所得编辑器 WYSIWYG
\end{enumerate}

图片描述
如果你的项目是 论坛,或者内容管理系统,那么就一定会用到它。

最著名的编辑器是 CKEditor。

虽然已经有人为大家做好了, 但是这个东东还是很复杂。

在Rails中使用的话, 记得用 'ckeditor' gem, 不要用 'ckeditor\(_{\text{rails}}\)' gem.

\begin{enumerate}
\item 发送 HTTP请求的包
\end{enumerate}

虽然各种语言都有内置的发送HTTP请求的工具, 但是还是不如一些专门的第三方包好用。

Ruby中使用 HTTParty。

这些第三方包一定要满足几个特点:

可以发送RESTful请求
可以设置 timout等参数
可以记录 log
\begin{enumerate}
\item 良好的日志工具: log4j, log4r
\end{enumerate}

大家一定记住: 少用断点! 这个东东太low了。
一定要使用日志!

断点: 每次都得人肉运行。 无法把信息存留下来。
日志: 每次都会自动的记录下所有日志。 把信息写到某个文件中。
我们在生产环境中, 永远会使用日志作为调试bug的工具。断点就是个玩具。在开发的时候用用还行。

举个例子: 在上亿用户的应用中, 每秒钟产生的log就几百屏。这个时候是没有机会给你
时间打断点,一个变量一个变量的分析的。


所以,在优酷,每天会产生一个G的日志。 有了问题,我们都会在日志中分析。

\begin{enumerate}
\item 要在恰当的时候使用配置项
\end{enumerate}

很多初学者会忽略这个问题, 配置项,有什么用?

下面一些东西必须提取出来:

调用的第三方的服务的域名
常见的系统常量
版本号等等
千万不要把我们的后台服务器域名 分散的写在代码中。 曾经有个项目, 请求了后台120个
接口, 每个接口的域名都是 'http://yoursite.com/interface' ,如果它有个配置项,
就可以写成: SERVER + '/interface'同学们, 配置项做起来容易,收效也大,大家务必
重视起来!


ruby中的工具叫: rails-config

\begin{enumerate}
\item 分析HTML
\end{enumerate}

可能一般的项目会用不到。 但是一旦用到,就会用的特别多。

我们这个时候,要对XML, HTML做分析的话,就得借助工具。否则用语言自带的分析XML 工
具的话,会特别复杂。


这样的工具要:

支持 CSS selector
支持 各种常见的DOM操作
传统的XPATH也要支持。
ruby中的工具叫 nokogiri.

\begin{enumerate}
\item 普通登录 和 单点登录
\end{enumerate}

普通的登录,就是在某台特性的web应用中登录。 这个在传统语言中特别麻烦。 Java中的
框架之前交Acegi. 现在好像被整合到了某个著名框架中。


总之, 需要对数据库中的用户进行密码验证, 还要有注册,解锁, 发送邮件,忘记密码
等功能。 还要记录最后一次登陆IP。特别复杂。


Rails中的解决方案是Devise.

单点登录, 是让一个用户可以在多个不同的WEB应用中登陆。 需要在目标WEB应用中都添加
一些配置,然后搭建一个单点登录服务器。


单点登录特别适合把一个大项目拆分成多个小项目。我曾经成功的把一个项目拆分成了20多
个子项目。 各个项目之间还特别独立。


Ruby 中的单点登录服务器是 ruby cas.

\begin{enumerate}
\item 定时执行的任务
\end{enumerate}

例如: 我要在每天的上午十点 发送一个提醒邮件。

Rails 中这个组件是 rufu-scheduler,下面是这个脚本的例子:


scheduler.every '180s' do

end

scheduler.in '5s' do

end

scheduler.in '10d' do

end

scheduler.at '2030/12/12 23:30:00' do

end
\begin{enumerate}
\item 延迟执行的任务
\end{enumerate}

跟“定时执行”不同, 延迟执行的任务的特点是:

不能马上执行
不能立刻结束
常见的场景是: 我要发送一万封邮件。

如果一封一封的顺序发送, 那么估计要3个小时。

所以,解决方案是: 让十个或者更多进程(也叫worker)在后台发送邮件。每发送一个邮
件,就是完成一个任务。 这十个进程每隔一段时间,就扫描一下任务表,看有没有任务要
做。


ruby中的组件叫 DelayedJob

\subsection{知道软件工程的知识。}
\label{sec:org99be872}
重构
单元测试
代码风格,代码的坏味道
设计模式
常见的实现模式
同学们, 对于上面的概念, 如果面试官问到你这样的概念时, 千万不要不懂装懂。 对于
新手, 你说不知道, 显得诚实。 我做面试官时, 最怕对方说:“重构,我知道呀! 就是
重新构造代码么”。 一旦对方让你说说 重构的常见“手法”时, 就懵了。


所以, 对于上面几个概念,大家务必有时间要多看书。多学习!

\subsection{要知道自动化工具}
\label{sec:orgbfecfce}
常见的工具
java中的: Ant, Maven, Ivy
c: Make
Rails: Rake,
javascript: npm, grunt
自动化的部署工具
Ruby: Capistran
Python: Fabric
测试的自动化工具
Selenium : 专门测试Web页面的工具。 可以实现对页面的各种操作。
Appium : 专门测试App的工具。 原理同 Selenium . 把人肉的操作过程使用代码记录下来,
然后重现。

这两个工具各位同学务必要学会! 然后在你的同事面前运行一下, 亮瞎他们的双眼。 \^{}\_\^{}

对于一些 持续集成的工具, 大家也要去学习(当然了, 要先学会单元测试,才能做持续
集成)。 不过, 可惜的一点是,我无论在大公司, 还是小公司, 都没有遇到过做持续集
成的朋友。 有些孤独。 侧面也反映出, 大家对于“实现代码” 和 “测试代码” 的平衡,不
好掌握。


\subsection{一开始就要学习的: Linux, Vim, Git}
\label{sec:org97c742e}
虽然这三个技术我放在了文章末尾, 但是它们应该在你学习的第一天就出现。

对于点亮 .NET 和 IOS技能树的同学, 你们可以继续用Windows/Mac. 但是对于Java, Php,
Android的同学, 赶紧用Linux!!! 因为,最后,你们的程序都要运行在Linux上的。


推荐大家使用 ubuntu (读音 乌班图,而不是 优版图) 桌面版。 目前最高版本是 16.04
LTS。 这个是最广泛的Linux.


推荐大家使用 win7 + ubuntu双系统。 这样对于新手比较友好。 不要使用虚拟机,特别卡。

\subsubsection{Vim}
\label{sec:org552e51b}
如果你开发的语言不是Java, Object C 这样的传统语言(需要编译), 而是 javascript,
scala, Ruby, Python这样的动态语言,推荐使用VIM. 理由很简单:


世界上只有三种编辑器: Vim, Emacs, 其他。
同学们还是直接用Vim 吧。没那么花哨,极度好用。就是学习曲线略高。 一旦用上了,你
身边的人看你敲键盘,完全是眼花缭乱的感觉。比鼠标流 效率高太多。


\subsubsection{版本控制工具: Git}
\label{sec:orged50dff}
不要用SVN. 过时了。

更不要用 CVS, VSS这样 的工具, 更加的过时。

使用Git。 使用命令行。 不要用图形化的工具, 也不要在IDE中使用GIT。

下面是必须掌握的Git命令:

clone, add, commit, push, pull, merge, branch

具体的大家自行查看吧。

\subsection{推荐唯一的一本书:《Pragmatic Programmer: 从小工到专家》}
\label{sec:org2f53c94}
图片描述
作者是 敏捷宣言的两个创始人。 世界顶级软件专家。

这本书05年9月左右开始,引领了我过去十一年的技术之路。充满了大量的方法论和安身立
命的内容。 大家务必学习。 我过去读了好多本书,很多书是我一年就可以翻完的(因为已
经具备了相应的知识和丰富的经验)。但是,这本书是特别少有的让人每次读起来都有收获
的书。


通篇没有几行代码, 都是文字,但是句句经典。

\subsection{你要有自己的个人博客!}
\label{sec:org7466c55}
作为一个互联网程序员,你务必要有自己的博客或者个人网站。 比如 \url{http://siwei.me} 就
是我最好的名片。好多朋友和机会都是从我的个人网站找到我的。


写技术博客很简单,你不需要长篇大论。 我每天写博客,无非就是把自己每天遇到的问题
和想法记录下来。 一般来说不超过20行。 有的时候甚至就几十个字。但是它一旦被记录下
来,你翻阅起来就特别方便。不需要去翻源代码了。


程序员的知识很复杂的,不是看几本书就够的。
软件开发不是一个特别系统的学科。 它的内容特别繁杂。 我梳理了过去5年我写的上千篇
文章,发现工作中用到的内容,是一个大杂烩: 有app端,有H5端,有语言层面,也有服务
器层面,有脚本,还有方法论,如何梳理需求,各种工具,等等等等。


所以,大家不要认为, 我学了一本书,(某某开发指南)我就立马是个前端开发好手了。
大家不要把自己限定成 前端 或者 后端; 也不要对自己说我就是一个java程序员, 我才
不学PHP呢。


你一定要做一个全栈好手,需要你做什么你就能顶上!

\subsection{结语}
\label{sec:org2e6ffa3}
没想到会写这么多,不过还有很多东西没能特别详细的说明。 一家之言,我的观点与市面
上流行的不太一致。 但是作为行业老兵,我坚持自己的观点。


希望对大家有用。祝各位同学顺利!

相关标签:JAVALinux职场生活

作者: 申思维
链接:\url{http://www.imooc.com/article/12768}
来源:慕课网
本文原创发布于慕课网 ,转载请注明出处,谢谢合作!
\subsection{{\bfseries\sffamily DONE} 你和高考状元之间只差这十个学习习惯}
\label{sec:orge012efd}
\url{http://edu.sina.com.cn/gaokao/2017-07-16/doc-ifyiakur8928391.shtml}
大家皆是凡人,学霸和学渣相差的可能只是学习方法。高考成绩公布后,铺天盖地都是对高考状元的报道,那么除了家庭环境、成长氛围、颜值高低,本文把对2017年高考状元的关注点重新转移回到学习方法上来,看看状元们都有哪些学习的好习惯。
\subsection{错题集——必不可少}
\label{sec:org914fff4}

  错题集是很多同学都知道,但大部分人都很难坚持,就像苏格拉底要求学生每天向后甩手臂50下,最后只有亚里士多德一人能够坚持下来;或者总结错题的方法存在误区。

  高考状元的学习习惯是在做错题集的时候可以用荧光笔进行勾画,这样子既节省时间,而且知识点也经过梳理,复习的时候也更加有针对性,而且会让你的作业本更加的美观。各种不同颜色的荧光笔,在练习本上划出错题、难题、重点等,不同的色彩代表着不同属性的题目,这样温习时,便能一目了然。既提高了复习的效率,又能愉悦心情,一举多得。

\subsection{  刷题不盲目}
\label{sec:orgd6880f1}

  题海战术≠刷题。对于高考来讲,刷题其实是一种积累,如果没有足够的练习,没有见过足够多的题型,对知识点的掌握是不能透彻的。

  刷题对一部分高考状元来说并不是一种负担,同学们也没必要将其当做“沉重的包袱”,每个星期可以制定详细的学习计划,甚至可以精确到每一天、每一个小时。做好计划,将时间合理地分配给自己的弱势科目、强势科目,有计划、有规律地进行提升。经过一段时间,你很快就会发现自己的进步!

\subsection{多读书一定没错!}
\label{sec:org2759111}

  读书多少与文科生还是理科生无关,今年出现的文理状元,无一不是爱好读书,并且阅读广泛。对于读什么书,还是要注意,如果是网文、青春小说、校园杂志等这一类,同学们尽量不要选择,这类书读起来很轻松,甚至会很愉悦,但含金量着实不高,建议尽量选择经常出现在高中语文课本中的大师们的作品,比如鲁迅、汪曾祺、三毛等,以及四大名著、国外的经典名篇等等。

  学习只是生活的一部分,学习疲劳地时候,可以通过各种方式进行调剂,比如音乐、运动、读书等,希望大家都能够读书破万卷,下笔如有神!

\subsection{  记忆的技巧}
\label{sec:orgece1d55}

  文科生一般都会担忧,这么多内容,肯定是要背诵的,那怎样才能够保持长久的记忆呢?我如果有哆啦A梦的记忆面包就好了,其实背诵记忆其实也是有技巧的!

  背诵的一大诀窍就是“理解”。理解之后会发现这个知识点的深意,这个时候其实记忆已经很深刻了,更加愿意去背。

  也有状元提到自己的方法是先背目录,背完目录之后,就可以把这本书的知识框架化,这样子每次只背三句话,实际上你就是把整本书变成了无数个三句话,这样子背起来就是比较有系统性。

\subsection{  答题技巧不可少!}
\label{sec:orga9e8bd2}

  选拔性的考试除了实力,考验的还有“巧劲”——“答题技巧”。十多年的学习和考试经验,每个人都会有自己的一套答题经验和技巧。撬动地球除了要有足够的力量,还要准确找到那个支点的位置!

  有的状元建议在高一、高二的时候认真听讲,打好基础,在高三总复习这一关键阶段,针对各科的题型特点,侧重培养适合自己的答题技巧。适合自己的学习方法,才是最有效的。

\subsection{  提高效率}
\label{sec:org45076b6}

  多年来涌现出的大量的高考状元的经历显示,除了各自的学习技巧,他们共同的一个非常优秀的学习习惯就是效率非常高。无论是学习还是娱乐,甚至是运动。

  可能大家身边也有这样的同学,比如平时没有熬夜的习惯,当天的学习任务都能够完成,这有赖于他们能够充分利用好自己的课堂时间,提高课堂效率。有什么问题尽量在课堂上解决,没有当堂解决的也要尽量在课下及时和老师同学沟通,拿下当天的内容。如果在课堂上能够及时消化和掌握知识点,课下进行一定的练习会起到事半功倍的效果。

\subsection{  和老师、家长沟通}
\label{sec:orga6b5ecb}

  大家身边也有这样的同学,他们的学习成绩很不错,并且和代课老师的关系也很不错,上下课经常会去找老师沟通。有一部分高考状元的经验就是平常每科科目在考试成绩出来后,主动找科任老师沟通。

  不过尽量还是要和老师聊一些学习中遇到的难题,比如阶段性模拟考某科目发挥失常,可以找科任老师沟通,及时找出失误的原因,并纠错总结。这种梳理和归纳对调整学习的方式方法是有很大的益处的。

\subsection{  合理利用电子产品}
\label{sec:orge28bee9}

  移动互联时代,手机、平板简直就是覆盖0到99岁的“大众玩具”,没有人能离得开电子产品。

  的确,电子产品对学习也是有帮助的,有不少状元表示“很多知识都是手机上学习的,比如我会去刷微博、微信,可以获取一些社会热点,比如一些单词不懂,也可以从手机上去查查。”包括今年高考全国卷作文提到的“共享单车”就是在微博上一直热议的话题。

  但是很多孩子沉迷于此。尤其是手机游戏,相信不少家长对此深恶痛绝,孩子们一旦玩起手机,简直就是废寝忘食。所以一定要适当使用电子产品,千万不能沉迷其中。

\subsection{  不偏科 成为全才!}
\label{sec:org4c71f52}

  如果可以选择科目进行高考,相信会涌现出无数“专才”。很多孩子在一些特定的科目上具有天赋和特殊的才华,但同时在另外一些科目上非常薄弱。但是面对当前的高考制度,孩子们目前只能去适应这一制度,而不是一味地厌弃某一门科目。毕竟进入大学,才能专业地研究特定的领域。

//d1.sina.com.cn/201709/18/1469390.jpg
  除了自己尽力弥补,家长和孩子还可以尝试聚能教育一对一的课程,专业的老师帮助孩子建立在弱势科目上的自信,找到合适的学习方法,努力成为多门功课共同发展的“全才”。

\subsection{  自制力很重要!}
\label{sec:orgfaf049a}

  玩是孩子的天性,寓教于乐也是教育追求的一点。孩子们可以放心大胆的玩,旅游、游泳、打篮球、摄影、看电影、玩密室逃脱、玩游戏,但一定要给自己定好时间。

  从小锻炼自己的自制力,做好生活、学习、娱乐的时间安排,不要因为玩而荒废了学业;也不能一直闷头苦学,要学会劳逸结合。

  比如有的状元安排自己放假期间每天玩一两个小时,既可以放松自己也不会耽误学习。

  这十个优秀的学习习惯,你都有吗?不仅要培养好的习惯,也要坚持下去,经过一段时间后就会发现这些小小的习惯会带来非常明显的进步!希望大家都能够找到适合自己的学习方法,系统地搭建思维框架,每天都进步一点!

  来源:北京聚能教育的微博

\subsection{\href{http://www.docin.com/p-1423123129.html}{Ieet认证}}
\label{sec:orgeab6922}
\subsection{\href{http://www.ieet.org.tw/InfoT.aspx?n=TeachingAward}{Ieet认证官网}}
\label{sec:orga7c1b9c}

\subsection{{\bfseries\sffamily DONE} 快速学习者的高效学习策略}
\label{sec:org51f4524}
\url{http://kb.cnblogs.com/page/545476/}
作者: 译/Jodoo  来源: 简书  发布时间: 2016-05-25 10:51  阅读: 19991 次  推荐: 97
原文链接   [收藏]


  英文原文:5 Ways to Learn and Remember Absolutely Anything

  过早地关注细节,你很可能让自己陷入一叶障目不见森林的境地。
高效学习者都有哪些学习策略值得效仿?这个问题最早出现在 Quora,本文源自对该问题的
一个答复。


该答复作者是阿莱西奥·布瑞沙尼,他在数字技术领域具有十五年的专业策略咨询经验。以
下就是这个答复的具体内容。

  你提出的这个问题真的很棒。一直以来,我对商业、个人成长以及武术搏击均保持着浓
厚的兴趣。这个问题促使我反思了过去的一些经历。

  我想告诉你,我见过不少貌似学习缓慢的人,他们对一些领域的深入理解程度远远超过
了那些所谓的『快速学习者』。事实上,慢学习者在认真和仔细方面,恰好是快速学习者所
缺乏的。所以我想对你说,那些看似低效的学习策略很可能正是你的优势,而不是缺点。在
我表达了我的个人观点之后,我自己还有一些学习与记忆的基本策略想和你分享一下,这些
策略来自我过去的工作与生活经历,我认为它们均具有一定的普遍价值。

\subsection{1) 重复}
\label{sec:orga036d2f}
  我坚信重复是通向精通的必由之路。当我们学习一种新的技能,必须经常性地对这种技
能加以练习。当我们学习一种新的知识或理论,也必须对尽可能多地对其加以应用。

  李小龙曾经讲过一句非常经典的名言,『我从不畏惧一个知道一万种踢法的人,但是我
害怕一个把一种踢法练习过一万次的对手。』

  任何一种技能,只要经过连续不断地磨练和改进,最终效果都将变得异常惊人。
\subsection{2) 专注}
\label{sec:org54ebac6}
  现代生活中的干扰因素太多 - 社交媒体、多任务、开放式学习与办公环境 - 我们总是
能不断地收到来自外部的各种刺激信号。

  我们已经丧失了专注的能力。然而专注是学习和掌握很多科目和专业必备的前提条件。
  为了学习一种新的技能,我尽量让自己处于一个没有打扰的环境之中。当我阅读一些东
西时,$\backslash$\我会听一些没有歌词的纯音乐$\backslash$\,这样的话,我的注意力就不会被分散了。

  史蒂夫·乔布斯说过,『人们通常认为专注意味着,对你正在做的事情说 Yes,实际上
根本不是这样。专注的真正含义在于,当你同时面对几百个好想法时,你必须精挑细选。』

\subsection{3) 背景与细节}
\label{sec:org8abb7ab}
  为了理解一门学科,我觉得首先你应该对这个学科的概貌有一个大致的了解。我自己就
非常喜欢探求一些事物的背景以及来龙去脉。

  所以,我在阅读一本书之前,总是先浏览一下这本书的目录。这样我就对这本书的内容
有了一个基本印象。当我阅读内容细节时,我将对内容中一个主题与另一个主题之间的关系
就更加清晰了。细节非常重要,但是要在合适的时机。
  
过早地开始关注细节,你很可能错失上下文或整体信息。当然,错失了细节,也会让你
的理解仅仅停留在一些事物的表面。

  所以,我会不停地在细节和上下文之间来回切换。这样我就能够在获取知识或技能的整

体性概念的同时,又能学到具体内容及细节。

\subsection{4) 关系}
\label{sec:org6de3955}
  这种在上下文与细节之间来回切换的学习方式,向我充分展现了信息之间的关系。这一
点对于学习与记忆的长期效果来说,非常非常重要。

  这就是为什么当我们谈论某一局牌时,有人能够绘声绘色地回忆起每一个细节。其实诀
窍就在于牌与牌之间的关系。

  在不同主题之间构建一种有意义的关系或联系,就是加速学习和强化长期记忆的最有效
手段。

\subsection{5) 节奏}
\label{sec:org462dcad}
  节奏是学习过程中最有趣的一个可变因素。

  例如,如果你正在听一场在线视频讲座,你可以加速这个讲座的播放速度(如以两倍速
率播放)。

  节奏还有一个重要功能。它能够让我们置身于完全不同的环境和压力之下。为了适应这
种变换,我们自身会自发地调节相应的学习方法。

  刻意让自己体验这种不断变化的学习节奏,能够更好地强化我们的学习效果和学习能力。
长跑运动员训练冲刺式的速跑,就是为了磨练自己对不同状况路面的适应性。

  就我个人而言,如果我正在准备一场演讲或者技术演示,我会在最后一次练习中以两倍
语速讲话。这样做就是为了确保,在我面临外部压力之下(公开场合演示),可以记起所有
的演讲内容和信息。如果在语速加倍的情况下,我都能回忆起这些信息,我自然能在正常语
速下,轻松地想起这些内容。


  变换节奏不只是与回忆或记忆相关,其实这样做在很大程度上,能够激发和改变学习的
潜能和活力 - 为你的学习工具箱添加一个灵活的新工具。

  我希望以上信息对你有所帮助。祝你在个人成长的过程中好运!

\subsection{{\bfseries\sffamily DONE} 学习如何学习}
\label{sec:org0325d65}
作者: 李忠  发布时间: 2016-06-18 17:34  阅读: 5925 次  推荐: 48   原文链接   [收藏]
  在「如何学习」这点上,一直觉得自己做的不够好,曾经想学吉他,坚持了两个礼拜就
以「手指太短,不适合」终结了,后来想学数学,却终究连翻开书的勇气都没有,工作一忙
更是顾不上这些了。所以在 Youtube 上看到 Barbara Oakley 的 Learning How to Learn
时,才发现自己在学习上的各种问题,收获颇多。如果有兴趣的话,建议直接看视频,讲解
地很有条理且生动。
  Barbara Oakley 是系统工程学博士,但对于「学习」方面也颇有研究,在 Coursera
上也有相应的课程。还出了几本书,比如 A Mind For Numbers: How to Excel at Math
and Science,这里对于她在 Google 的分享做一个简单的翻译和摘要。
摘要

  (主持人)我记得查理芒格说过,他认识的人中没有一个不每天阅读的,还把巴菲特形容
为一台学习机器。那么如何才能成为一个高效学习者呢?

  正文
  我小的时候,想要学习其他语言,但大学的助学金不太容易拿到,而我又迫切想要学习
一门语言,然后我想到了一个既可以学习语言,又能得到些收入的方法,那就是参军。然后
我确实学了一门语言:俄语。虽然俄国的环境不怎么样,但我喜欢冒险和新的视角。我回想
在西点的工程师们,他们解决问题的能力非常出众,往往能想到我想不到的。然后我就想,
我能不能也达到跟他们一样的程度?有学生提了这么个问题:如何改变你的大脑。然后我就
去了解世界顶级的教授他们是如何做到让学生更好地学习工程学、数学、化学的?他们自己
又是如何学习的?跟他们接触后发现,他们常用的有隐喻和类比。接下来我想跟大家分享下
学习的关键因素。
  
我们都知道大脑是很复杂的,那么来简化一下,可以想象成大脑以两种不同的模式工作。
第一个是「专注」,另一个则是「发散」。我们用一个弹球机来描述这两种状态。

假设大脑里有一个弹球机底部有一个触发开关,有很多的槽点密集排布。比如,你已经知道
了乘法,然后要处理一道乘法运算题。当你处于「专注」模式时,会运用已有的模型,在
「乘法」的槽点附近打转,也就是结合以前的学习经验去寻找答案。如果你要解决的问题是
基于新的模型,比如你已经知道了乘法运算,但从没有接触过除法运算,如何掌握这个新的
模型呢,这时就要用到「发散」了。当处于「发散」状态时,「槽点」之间的距离会变大,
你无法通过局限在某一点来解决问题,但至少能找到一种新的思考事物的角度。当你在解决
一个非常困难的问题时,不要逼自己长时间处于「专注」状态,这样就会局限在一个狭小的
范围。所以需要进入到另一种模式,也就是「发散」模式。简单来说,就是脱离当前的工作
环境,出去走走,冲个澡等,总之是让大脑脱离「专注」状态。我们来举一个例子吧。
  Salvador Dali 是20世纪著名的超现实主义画家,他最爱干的事情就是,当遇到一个棘
手的问题时,会躺在椅子上,放松再放松,同时手里握着一把钥匙。当足够放松到快要睡着
时,钥匙就会掉在地上,与地板碰撞的声音会把他叫醒,然后就可以带着从「发散」状态收
获的想法继续进入到「专注」模式。
  你可能会觉得这个对艺术家有用,那么对工程师是否也同样有效呢?据传,爱迪生也有
类似的行为爱好,只不过不是钥匙,而是滚珠轴承。

  当你在解决一个问题时,即使已经有成千上万的人已经解决过了,但对你来说确实第一
次,你也可以试试类似的方法。

  当你处于「专注」状态时,并没有利用到其他更多的关联,这也是为什么在两种状态间
切换是如此重要。就像你不能一下吃成大胖子一样,神经系统也是需要一段时间来适应新的
学习和思维方式。
  你或许会说,我有拖延症,那我们就来说说拖延症。拖延症的成因是当你面对不喜欢做
的事情时,大脑的「痛感中心」就会被激活。所以当你看一本不喜欢的书时,会感觉到隐痛,
这种痛就像手指被锤子砸了一样,通常会有两种做法:第一种是花大概 20 分钟去搞定它,
然后痛感就会慢慢降低进而消失。但如果你像大多数人一样,将注意力集中到其他做起来更
舒服的事情上,就会马上就会感到好些了。
  从某种程度上来说,拖延症也是会上瘾的,这对你的生活是非常有害的。最有效的方式
是使用「番茄工作法」,通常来说设置 25 分钟为一个「番茄时间」,然后关闭其他所有会
打扰你的一切。在这 25 分钟内,集中精力进入到「专注」模式。由于你专注于当前的任务,
而不是「我必须完成它」的痛苦,做起来就会容易很多。当到时间后,给自己点奖励,出去
走走或上上网、聊聊天都行。有一点要注意的是,不要以完成任务为目标,时间到了,就休
息。它能帮助你跨过痛苦期,进入 flow 状态。还有就是不要一下子做太多的「番茄」,一
步一步来,慢慢适应这套系统。
  还有跟学习很相关的一点是睡眠。常常有人说考试前要睡好觉,事实上,睡眠在各种层
面上都很重要。当处于非睡眠状态时,代谢物会在细胞间产生,它们就像垃圾一样在那,而
且越积越多,这会影响你的判断。这也是为什么当你长时间工作时,逻辑会越来越乱。当你
睡觉时,这些细胞会缩小,然后垃圾就被冲走了,就像重新打扫了房间一样。
  根据神经学上的发现,将学习分为多个短期学习、睡觉,多个短期学习、睡觉,这样的
循环系统,对于构建神经网络非常有帮助,这也是高效学习的秘诀之一。

  人每天都会长出新的神经元,有两种方式可以让这些神经元存活并成长。一个就是将自
己暴露在新的环境中,这也是为什么旅行会很有帮助,这些新元素能让新出生的神经元活下
来。还有一个让新长出来的神经元活下来的方法,很简单,就是锻炼。不需要励志成为奥林
匹克运动员或者成为马拉松选手,即使只是简单的散步也是非常有效的。但即使只有几天的
锻炼也会带来更大的效果,它会增强新神经元的存活和生长。
  接下来聊聊工作记忆(Working Memory),工作记忆就是临时记住一些信息,以前常说
有7个槽可以用,这也是为什么你能记住7位数的电话号码,但事实上大概只有4个槽可以用,
所以当你用工作记忆来记一些东西时,可以想象有一只章鱼在掌控着这几个槽,并建立连接,
这也是为什么不能一次记住太多的想法。当你多任务同时开工时,相当于把章鱼的触角从仅
有的几个槽中拿走一个或多个,这会让你变得笨一点。而发散模式则是有更多的连接。
  那如何把短期记忆变成长期记忆呢,最好的方法就是练习,练得越多,神经元就会长得
越长,扎得越深。
  如果你不练习,那么这些「蝙蝠」就会在模式形成长期记忆前把它叼走,这也是为什么
有时觉得已经理解了某个概念,然后走开了,过了两天,这些内容都被「蝙蝠」叼走了,然
后就记不起什么了。所以最好的方式就是带一定间隙的重复练习(Spaced Repetition),比
如周一、周二、周三、周五、周日练习。
  再来说说 Chunk。假设你要拼一副图,如果不明白每一块代表的含义,就会有无从下手
的感觉,就像中间的那个圆一样,你能看到它,它也是个 Chunk,但却无法与其他的 Chunk
产生联系,这也是为什么死记硬背的效果会很差的原因。
  当你在研究某个课题时,你其实在创建一个 Chunk 群,这些 Chunks 会跟其他的
Chunks 生关联,这也是伟大创意的产生之源。这时往往会得到一些支离破碎的 Chunks,如
果都学会了,就会形成一副完整的图片,即使少了其中几片。
  但如果你不重复练习,并且深刻掌握 Chunks,也能把 Chunks 拼起来,只不过是模糊
的,而且很难拼成全图。

  不同领域的 Chunks,有可能长得差不多,这样就可以借鉴原有的 Chunk 来学习新的
Chunk。比如你是一个物理学家,再去学经济学会更简单些,因为其中的一些 Chunks 非常
相像。

  最后给大家一些关于学习的建议:
  测试是必要且重要的。Test yourself on everything, all the time。同样的时间用
来做测试和学习,前者会让你收获更多。使用卡片(Flashcards),卡片不是专门用来学习语
言的,卡片是一个通用的学习手段,诗人们会用它记忆诗句,以此来更强烈地感受诗词带来
的震撼。记得做「家庭作业」,不要只做一次,挑其中的重点多做几次,即使只是在大脑里
过一遍,确保自己真的掌握了,这样你就得到了一组 Chunks。

  最有效的方法还是「回想」,尤其是当你在解决困难的问题时。当你在阅读文章时,试
着离开书本,回想一下能否记起其中的要点,这对于理解内容有很大的帮助,比反复读和思
维导图效果都要好.

  还有一个简单的技巧是向其他人阐述你所理解的东西,并且假设对方只是一个 10 岁的
小孩(费曼学习法)。如果你能找到一个简单清楚的描述,就能更加深入地理解。你甚至可
以把自己放到问题发生的场景中。当你处于专注模式时,会有一种「我已经掌握了」的感觉,
这时可以跟其他处于diffuse 模式的人交流下想法,有时会帮助你纠正错误。



  最后,我们都说要追随你的热情,但热情只是让你擅长的东西变得更擅长,而有些东西
要花很长的时间才能擅长,所以不仅仅要追随你的热情,还要扩大你的热情,然后生活质量
就会有大幅度的提升。

\subsection{{\bfseries\sffamily DONE} 如何阅读计算机科学类的书}
\label{sec:orgc6619da}
\url{http://kb.cnblogs.com/page/576251/}
作者: Joshua Nie  发布时间: 2017-09-21 13:51  阅读: 4263 次  推荐: 17   原文链接
[收藏]
  作为一个研发工程师,无论你是否喜爱阅读,相信你都一定读过不少关于计算机技术的
书籍。这其中不乏《21天学会JAVA》这样的语言入门书籍,也有《算法导论》这样的专题书
籍,也有《人月神话》这样关于软件管理学的实用性的书籍。也许你已经读过他们中的大部
分,也许你现在还在不断地购入新的书籍来补充你的知识库。但请稍等一下,你是否思考过
这样的问题,面对大量的计算机科学书籍,你是否都真正读懂了它们呢?有多少本书,当你
将他放在书架上之后,就再也没有重新打开过?有多少知识是真正被存储在你的大脑中,并
随时可以提供调用?拿到一本书后,高效阅读的正确姿势的什么?如果你有以上的疑惑,那
么接下来,我们将一起探讨一个问题,如何阅读一本计算机科学类书籍。
\subsection{  阅读的四种层次}
\label{sec:org64f19b4}
  首先,我们先要学会如何阅读。你可能会觉得不可思议,我已经接受过高等教育,怎么
可能还不会阅读。然而可悲的是,现代教育体系中,恰恰忽略了对阅读能力的训练。我们在
初中之后,阅读水平就几乎没有机会再得到提升。总体来说,阅读分为四种层次,分别是:
\begin{itemize}
\item 基础阅读
\item 检视阅读
\item 分析阅读
\item 主题阅读
\end{itemize}
  这其中的概念来源于莫提默·J·艾德勒和查尔斯·范多伦的著作《如何阅读一本书(How
To Read A Book)》,这里我必须对其中的概念做简单的总结,以便在后续的篇幅中,我们
能统一对阅读名词的理解。
\subsubsection{  基础阅读}
\label{sec:org8dd7e0c}
  当我们完成中学学业后,我们中的绝大部分人,都已经掌握了基础阅读的能力。在这个
层次中,我们关心的是,书里的每句话是什么意思。这是一个最基础的层次。
\subsubsection{  检视阅读}
\label{sec:org14e327a}
  检视阅读,我们也可以称之为快速阅读。快速浏览全书,了解书的主题,架构全书,提
出核心问题。这并不是很新鲜的概念,但很多人可能并没有思考过,为什么要做检视阅读。
检视阅读作用是为了帮助我们筛选这本书是否值得阅读,同时为接下来的分析阅读打下基础。
在这个层次中,我们关心的是,这本书在讲什么。
\subsubsection{  分析阅读}
\label{sec:org0387879}
  分析阅读是一个更为高级的阅读层次,目标让我们能充分理解本书,与作者对话。其中
包含了多个阶段,这里不再详述,有兴趣的同学可以研读原著。
\subsubsection{  主题阅读}
\label{sec:org44270c4}
  当我们跨越过分析阅读后,这本书已经被我们掌握。此时,我们会就相同的主题,阅读
不同的书籍,找出其中关联与矛盾,倾听不同的作者的不同声音,从而对某个主题产生更加
深刻的认识。这个阶段,我们关注的不再是某一本书,而是一个具体的问题。
  计算机科学书籍的特征
  原著中针对不同类型的书籍,给予了不同的阅读建议。但由于所著时间很早,就计算机
科学类图书的阅读建议,在书中并没有专门设计章节阐述。根据我的阅读经历,深感计算机
\subsection{科学类书籍,较其他类型图书有着其独特性:}
\label{sec:org5c6d023}
\subsubsection{  单本书籍的信息量大}
\label{sec:orgce0ebcc}
  相较其他学科,绝大多数计算机科学类书籍并不是以得出结论并且论证结论为核心,而
偏重于阐述方法和解释原理。有很多计算机书籍旨在剖析某个系统。这里的系统不仅仅指代
诸如操作系统这样的实体系统,还包括一门语言或者一套管理方法论这样的理论系统。而系
统通常是由多个部分组成的综合体,这其中势必包含不同组成部分的不同细节,信息量之大
可见一斑。
\subsubsection{  注重实践}
\label{sec:org3d8f6bf}
  计算机科学是一门实用性的学科。这里的实用性可以理解为,计算机科学诞生的目的就
是为了解决实际问题。因此,几乎所有的计算机科学书籍,都是以指导实践为目标而作。
\subsubsection{  更新速度快}
\label{sec:org4a7ff52}
  计算机科学的更迭速度可以准确地被描述为日新月异。有些技术很快地火爆起来,又很
快地消亡,所以有些书也就跟着很快地淹没在时代的进程中。
\subsubsection{  分类细致但同质度高}
\label{sec:orga285fdd}
  计算机科学对自己有着过分清晰的划分,不同的技术之间往往边界清晰。我们很少见操
作系统和数据库系统在同一本书中论述,也不常见集不同语言之成的大作。由于领域划分细
致,相同领域的书籍,多数时候往往论述的是同样的主题。
\subsection{  阅读计算机科学书籍的误区}
\label{sec:orge8c28fd}
  绝大多数读者的错误意识在于把所有的书籍都认为是层层推进的论述过程。这样的阅读
经验一旦沿用在计算机科学类书籍中,就会感觉举步维艰。前文说过,大多数的计算机书籍
都是在剖析系统,一个系统又是由许多相互关联的部分组成。解读这类书籍,如同拆解一个
机械,我们在拆解的过程,常常会犯下这些错误。
\subsubsection{  通读全书}
\label{sec:org7d8fe30}
  在你的头脑中没有对全书的结构有整体了解的情况下,从头至尾通读全书,意味着试图
从细节窥视一个系统的全貌。这是一种低效的读书方式。当读到中落时,你会因为没有全局
概念,而迷失在各种细节中,以至于完全失去了阅读的方向和目标。
\subsubsection{  跳过序言}
\label{sec:org69a9443}
  序言往往是很多人忽略的内容,似乎序言只是重复了正文的内容。而正因为如此,序言
以简短精炼的语言,为你分解了整本书的架构,帮助你把握系统的整体。这项工作本来应该
是读者在阅读全书之前的必备工作,绝大多数的作者都已经帮你完成了,而你需要做的仅仅
是认真的阅读它。
\subsubsection{  脱离实践}
\label{sec:org8a7eb6c}
  前文说过,计算机科学类书籍重视实践,脱离了实践,往往就不能完全理解书中所述的
理论和方法,过目就忘,纸上谈兵。
\subsubsection{  忽视基础}
\label{sec:org711f5e6}
  封装在计算机的世界中是一个非常重要的概念。计算机的发展史,总的来说就是一部封
装史:将底层不断包装,提供简单的调用方式,由此不断的扩展计算机的边界和能力。新的
技术层出不穷,而他们的很多所依赖的环境和系统,从设计之初就没有发生过质的变化。
  有时,在追逐新的技术之前,深入了解他们所在的系统;在学习新的算法之前,掌握好
其基础的数学原理。只有牢固的基础才能支撑足够结实的上层建筑。
\subsection{  阅读计算机科学书籍的建议}
\label{sec:org20a043d}
  当了解阅读误区后,你们是不是已经发现阅读这类书籍的核心原理呢?那就是将整本书
当做一个系统,从整体到局部,层层递进,逐步剖析。根据这个核心原理,我总结了一些好
的实践方式。
\subsubsection{  检视阅读}
\label{sec:org8656ac4}
  当你拿到一本计算机科学书籍,第一步就应该快速浏览序言和目录,然后用检视阅读的
方式整理出整本书的大纲。这样,你对这本书是介绍理论还是关注实践,所属什么分类,哪
些问题是本书将会讨论,而哪些问题是不被详细讨论的,这些信息你都会有整体上的认知。
这时,你就可以很轻松地判断,这本书值不值的阅读,哪些内容是你已经熟知的,哪些内容
是你关注的重点,这样做阅读的效率将会大大的提高。
  如果从来没有使用过这种阅读方式,开始实践时,会受到一定的心理上的阻力。可能你
对某个专有名词完全没有概念,以至于整章的内容都模棱两可。这时,你应该坚持继续阅读,
对不甚理解的内容,先记住有这样的概念。绝大多数的时候,经过检视阅读后,过程中的问
题都会有所释怀,剩下依然没有明白的内容,视其重要性,再决定是否对其进行分析阅读。
\subsubsection{  提取问题}
\label{sec:org1bca79d}

  当你了解了整本书的全貌,一般而言,你会发现,有些章节你已经熟悉,有些章节你全
然不知。这时就要对这些章节进行分析阅读。分析阅读的很多步骤和方法在《如何阅读一本
书(How To Read A Book)》有详细的介绍,这里不展开细说。但有时,你在阅读的过程中,
会发现阅读的兴趣在下降。信息量愈大,阅读的动力愈弱,最后你就迷失在信息的汪洋之中。
  我们应该如何避免这样的信息疲劳呢?答案就是去掉冗余的干扰信息。在上一个建议中,
我们强调了检视阅读的重要性。那检视阅读的成果是什么呢?那就是你对每个部分(不一定
是书中给你划分的章节)所提出的问题,也可以称之为阅读目标。而你要做的就是,找到这
些问题的答案,完成自己的阅读目标。
  这样做过滤了很多作者认为重要,其实和你关心的主旨没有联系的信息,减少了信息疲
劳。同时,不同部分之间有关联的问题,可以帮助你更好的串联全书阐述的核心概念,把握
整本书的主要脉络。
  例如,我在阅读《深入理解计算机系统》的异常控制流时,就提出这样的问题:进程是
如何管理内存?而部分的答案,在下一个章节虚拟内存中。当我解答这个问题时,我就会将
这两个分离的章节的内容,通过一个问题联系在一起,加深了自己的理解。
\subsubsection{  持续重读}
\label{sec:orgf5d22a8}
  一本经典优秀的计算机科学书籍,值得你反复的阅读。不要觉得整本书我已经完全理解,
就再也不需要重新回顾阅读了。因为此类书籍存在大量信息,而这些信息并没有必要占据我
们大脑有限的记忆存储空间。我们要做的就是认真做好第一条建议,当我们需要使用这些书
籍解决问题的时候,能第一时间在其中找到我们需要的信息。毫不夸张的说,计算机科学类
的书籍生来就是供人反复翻阅的。
\subsubsection{  鉴别烂书}
\label{sec:orgc4fa90e}
  作为阅读爱好者,谁能说自己没读过几本烂书呢。在计算机科学这个类别中,烂书的比
例一点也不比其他学科低。信息重复(抄袭),结构混乱,论证不清晰(作者对某个技术一
知半解)等等,都是烂书的特征。关于烂书,我们要做的就是第一时间将其鉴别出来,然后
放到自己的黑名单中。具体如何鉴别烂书,由于本篇幅太长,我可能会新开一篇文章单独讨
论。
\subsection{  结语}
\label{sec:org228dc7b}

  以上就是我对于如何阅读计算机科学类书籍的理解。本来想缩短些篇幅,但最后还是决
定保留那些我觉得应该详细论述的部分。毕竟这篇文章的初心并非是厕所读物,而是一个阅
读爱好者认真地与读者探讨一个严肃的话题。如果可以,我希望在通过我不断地探索,阅读
能力的持续提升,我还能在此宝地继续这个话题,完善我的理论。
\subsection{  我在下面列出我认为经典优秀的计算机科学书籍,也欢迎大家补充,排名不分先后。}
\label{sec:org2efb479}
《算法导论》Thomas H.Cormen、 Charles E.Leiserson
《深入理解计算机系统》Randal E. Bryant
《人月神话》Frederick P.Brooks
《编程珠玑》Jon Bentley
《高性能MySQL》施瓦茨 (Baron Schwartz)、 扎伊采夫 (Peter Zaitsev)
《代码大全》Steve McConnell
《程序员修炼之道:从小工到专家》亨特(Andrew Hunt)、 托马斯(David Thomas)
\subsection{你所知道的学习方法,都是错的!}
\label{sec:orgf17991d}
\url{http://kb.cnblogs.com/page/137524/}
来源: 果壳网  发布时间: 2012-04-03 11:05  阅读: 11966 次  推荐: 10   原文链接
[收藏]

上课的时候记笔记?哪门功课不行,就集中精力专项突击?自习的时候不要晃,选好一个地
儿安安稳稳地待那儿学习?你还在这样学习吗?不要被骗了:这些被我们奉为良好学习习惯
的东西,恰恰是冒了学习正道的大不韪。
  英文原文:Everything You Thought You Knew About Learning Is Wrong
  原文发布于 2012 年 1 月 29 日
  文 / Garth Sundem
  译 / 小老鼠汪

  前不久,我有幸采访了加州大学洛杉矶分校 “学习和遗忘实验室” 主任,心理学特聘
教授 —— 罗伯特 · 比约克(Robert Bjork)。要说往脑子里狂塞东西还不掉出来,比约克就
是这方面大大的专家。跟比约克谈过后我发现,我所知道的关于学习方法的一切,都是错的!

  一开始,比约克问我说,当我面前堆了一摞书要啃的时候,我会怎么办。
  “人通常会一块儿一块儿地整,” 比约克说, “干完这个再干那个。”

  正确地学习方法,应该是交换着学,学会儿这个,再学会儿那个。 好比你要练网球的
发球,你不应该花一个小时的时间苦练发球,而应该把反手击球、截击、扣杀和步法,结合
起来交换着练。“这就增加了难度,” 比约克说, “而人们往往容易忽略这些不是立竿见
影的效果。”

  专注地练一段时间能让你的发球水平有一个明显的提高,而交换着练习则能够使你在很
多技能上,都往前迈出小小的一步,你几乎无法察觉自己有所提高。然而,随着时间的推移,
这些小小的进步累积起来,将会比你花同样多的时间,去一项一项单独掌握每一个技能所获
得的提高多得多。

  对此,比约克表示,交换练习用得好的话,能让你把各项技能都相应的 “座” 到位。
“把一个知识点跟记忆中的其他东西联系起来学,这样的学习会更加有效,” 他说。需要
注意的一点是:交换着练习的这些小技巧,要同属于一个大的技能才行。如果你想学打网球,
那么你交换着练习的应该是发球、反手击球、截击、扣杀和步法,而不是发球、花样游泳、
背诵欧洲国家的首都和学习用 Java 编程。

  同样,只在一个固定的地方学习当然很好,前提是你只需要在那个地方才会用到你学的
那些东西。如果你想在宿舍、办公室或者图书馆二楼自习室等等以外的地方,也能回忆起你
所学的知识,比约克建议, 不妨在几个不同的地点换着进行自习 。

  无论你是学数学、学法语,还是学社交舞步,交替着学和换着地点学都将适用。类似的
还有一个叫做 “时间间隔效果”(spacing effect),这一概念最初由赫尔曼 · 艾宾浩斯
(Hermann Ebbinghaus)在 1885 年提出,学习的时候,复习要隔开一段时间,会学得更好。

  “如果你学了之后不练,研究表明,中间隔的时间越长,你忘的就越多,” 比约克说

  但有趣的是: 如果你学了之后,隔一段时间再学,这时候你隔的时间越长,复习的时
候你学到的东西就越多。 比约克表示: “当我们从记忆中提取信息的时候,我们做的不只
是说它在那里就行了。记忆不仅仅是回放。我们这次取出来了的东西,下次要取的话,取起
来就会变得更容易。我们每次取的过程越难、涉及的东西越多,整个记忆就越有效。”

  注意这里所说的是 “我们这次取出来了的东西”(没取出来是没有用的)。所以, 从
学完到你开始复习的时间,应该是你刚刚好要开始忘记的时候。 这样,你越是拼命地回忆
之前学过的东西,你复习的效果就会越好。如果你学完之后马上复习,就没有这个效果了。

  同理,比约克还建议说,笔记最好下课之后才开始记,以强迫自己回忆课上讲过的东西;
而不是在课堂上记,黑板上有啥抄啥。你必须下苦功才行。你花的工夫越多,你学到的就越
多,你当然也就越牛。
  那么,关于遗忘呢?
  “赶紧忘掉你知道的 ‘遗忘’ 的定义吧,”比约克说, “人们通常认为,学习就是
在记忆里面修东西,而遗忘呢,则是把你修起来的东西给拆了。但在某些方面,反过来说才
是对的。”
  这么说吧,只要是你学过的东西,其实是一直待在你记忆里不会忘的。你还记得你儿时
好友的电话号码吗? 记不得了? 那好,比约克说了,如果这时候提醒你一下,那么你回忆
起这个电话号码的速度和印象,会比让你重新记一个新的 7 位数电话号码,要迅速和清晰
得多。所以这个旧的电话号码不是被你忘记了 —— 它一直待在你脑海里的某个地方 —— 只是
把它取出来有点儿麻烦就是了。我们一直把遗忘当成是学习的死对头,这也算是冤案一桩。
学习和遗忘的关系有点儿像是共生,实际上遗忘对记忆还有帮助作用.
  “人脑有无限的存储量,要是什么都回忆得起那就糟了,” 比约克说, “试想一下,
你记得你住过的所有地方的所有电话号码,每当有人问你电话号码的时候,你必须把这一长
串电话号码都给理一遍才行。” 我们忘记旧的电话号码,或者把它们埋于记忆深处,回忆
够不到的地方,方便我们迅速提取出现在使用的那个电话号码。被你恨得牙痒痒的敌人(我
就是忘性大),其实更像是默默守在一旁的伙伴(吐槽:防止你因为一直忘不掉以前的糗而
自绝于寰呀!)。

\subsection{{\bfseries\sffamily DONE} 梁漱溟:学问的境界}
\label{sec:org317f244}
\url{http://www.sohu.com/a/138472452\_523168?loc=2\&cate\_id=1350}

所谓学问,就是对问题说得出道理,有自己的想法。

想法似乎人人都是有的,但又等于没有。因为大多数人的头脑杂乱无章,人云亦云,对于不
同的观点意见,他都点头称是,等于没有想法。

我从来没有想过要做学问,走上现在这条路,只是因为我喜欢提问题。大约从十四岁开始,
总有问题占据在我的心里,从一个问题转入另一个问题,一直想如何解答,解答不完就欲罢
不能,就一路走了下来。

提得出问题,然后想要解决它,这大概是做学问的起点吧。

以下分八层来说明我走的一条路:

\subsection{第一层境界:形成主见}
\label{sec:orgc7b5756}

用心想一个问题,便会对这个问题有主见,形成自己的判断。

说是主见,称之为偏见亦可。我们的主见也许是很浅薄的,但即使浅薄,也终究是你自己的意见。
许多哲学家的哲学也很浅,就因为浅便行了,胡适之先生的哲学很浅,亦很行。因为这是他
自己的,纵然不高深,却是心得,而亲切有味。所以说出来便能够动人,能动人就行了!他
就能自成一派,其他人不行,就是因为其他人连浅薄的哲学都没有。

\subsection{第二层境界:发现不能解释的事情}
\label{sec:orgccbd8ea}

有主见,才有你自己;有自己,才有旁人,才会发觉前后左右都是与我意见不同的人。
这时候,你感觉到种种冲突,种种矛盾,种种没有道理,又种种都是道理。于是就不得不第
二步地用心思。面对各种问题,你自己说不出道理,不甘心随便跟着人家说,也不敢轻易自
信,这时你就走上求学问的正确道路了。

\subsection{第三层境界:融汇贯通}
\label{sec:orgb122e33}

从此以后,前人的主张、今人的言论,你不会轻易放过,稍有与自己不同处,便知道加以注意。

你看到与自己想法相同的,感到亲切;看到与自己想法不同的,感到隔膜。有不同,就非求
解决不可;有隔膜,就非求了解不可。于是,古人今人所曾用过的心思,慢慢融汇到你自己。

你最初的一点主见,成为以后大学问的萌芽。从这点萌芽,你才可以吸收养料,才可以向上
生枝发叶,向下入土生根。待得上边枝叶扶疏,下边根深蒂固,学问便成了。

这是读书唯一正确的方法,不然读书也没用处。会读书的人说话时,说他自己的话,不堆砌
名词,不旁征博引;反之,引书越多的人越不会读书。

\subsection{第四层境界:知不足}
\label{sec:org054a2af}

用心之后,就知道要虚心了。自己当初一点见解之浮浅,不足以解决问题。

学问的进步,不单是见解有进步,还表现在你的心思头脑锻炼得精密了,心气态度锻炼得谦虚了。
心虚思密是求学的必要条件。

对于前人之学,总不要说自己都懂。因为自己觉得不懂,就可以除去一切浮见,完全虚心地
先求了解它。

遇到不同的意见思想,我总疑心他比我高明,疑心他必有我所未及的见闻,不然,他何以不
和我作同样判断呢?疑心他必有精思深悟过于我,不然,何以我所见如此而他所见如彼呢?

\subsection{第五层境界:以简御繁}
\label{sec:orge5a244f}
你见到的意见越多,专研得愈深,这时候零碎的知识,片段的见解都没有了;心里全是一贯
的系统,整个的组织。如此,就可以算成功了。到了这时候,才能以简御繁,才可以学问多
而不觉得多。

凡有系统的思想,在心里都很简单,仿佛只有一两句话。凡是大哲学家皆没有许多话说,总
不过一两句。很复杂很沉重的宇宙,在他手心里是异常轻松的——所谓举重若轻。

学问家如说肩背上负着多沉重的学问,那是不对的;如说当初觉得有什么,现在才晓得原来
没有什么,那就对了。道理越看得明透,越觉得无甚话可说,还是一点不说的好。心里明白,
口里讲不出来。

反过来说,学问浅的人说话愈多,思想不清楚的人名词越多。让一个没有学问的人看见,真
要把他吓坏了!其实道理明透了,名词便可用,可不用,或随意拾用。

\subsection{第六层境界:运用自如}
\label{sec:org34d9fce}

如果外面或里面还有解决不了的问题,那学问必是没到家。如果学问已经通了,就没有问题。

真学问的人,学问可以完全归自己运用。假学问的人,学问在他的手里完全不会用。

\subsection{第七层境界:一览众山小}
\label{sec:orgea183fa}
学问里面的甘苦都尝过了,再看旁人的见解主张,其中得失长短都能够看出来。这个浅薄,
那个到家,这个是什么分数,那个是什么程度,都知道得很清楚;因为自己从前也是这样,
一切深浅精粗的层次都曾经过。

\subsection{第八层境界:通透}
\label{sec:org91ffee4}

思精理熟之后,心里就没有一点不透的了。

\subsection{粗读一遍,体会有不深。\textit{<2017-09-27 Wed 10:31>}}
\label{sec:org052b700}

\subsection{{\bfseries\sffamily DONE} 【悦思教育】最近很火的一个故事,很有意思!}
\label{sec:org75d8059}
\url{http://www.sohu.com/a/138752203\_372511?loc=1\&focus\_pic=0}

从前,有两个好朋友,一个叫「聪明」,一个叫「诚信」。某日,两人结伴乘船出游,不巧,
在海上遇到大风暴,两人乘坐的船沉没了,救生艇上仅仅于一个位置。那个叫「聪明」的年
轻人,一看形势不好,为了争夺救生艇上的位置,就把「诚信」推进海里,自己逃生去了。

「诚信」喝了不少水,却大难不死,被海浪推到了一个小岛上,他惊魂未定,只好坐在沙滩
上等待救援的船只。不久果然听到远处传来一阵阵欢快的音乐,他马上站起来,向着音乐的
方向望去,发现有一艘小船向小岛驶来,他看见小船上有面小旗,上面写着「快乐」两个字,
原来是「快乐」的小船。

「诚信」急忙喊道:「快乐,快乐」我是「诚信」,你能救我吗?「快乐」一听,笑着对
「诚信」说,不行不行,我要是有了「诚信」就不快乐了,你看这世界上有多少人因为说老
实话而不快乐。说罢,「快乐」走了。又过一会,「地位」的小船来了。「诚信」忙喊道:
「地位」、「地位」我是「诚信」,你能带我回家吗?「地位」一听,忙把船划离小岛,一
边回头冲着「诚信」说:“不行,不行,你不能搭我的船,我的地位来之不易,要是有了诚
信,我的地位就保不住了”。


「诚信」很失望地看着「地位」离去,眼里充满着疑惑与不解,只好无奈的再小岛上再呆下去。

过不久,又来了一艘船,「诚信」一看是「竞争」的船,「诚信」又喊道:「竞争」,「竞
争」,我是「诚信」,你能不能让我搭你的船回家。「竞争」一看是「诚信」,忙说道:
「你不要给我添麻烦了,如今世界竞争这么激烈,我如果还要诚信的话,我就竞争不过人家
了」。说罢,扬长而去。

突然,海上开始电闪雷鸣,狂风卷起一波波的滔天巨浪,正当「诚信」快要绝望的时候,突
然听到一个亲切慈祥的声音喊到:“孩子,上船吧。”

「诚信」一看,原来是时间老人。“你为什么要救我呢?”「诚信」问道。
时间老人微笑着说:“只有时间才可以证明「诚信」是多么重要啊!”

在回程的路上,「时间老人」指着因巨浪翻船而落水的「聪明」「快乐」「地位」「竞争」,
意味深长地说道:“没有了「诚信」,「聪明」反而害苦了自己,「快乐」不会长久,「地
位」是虚假的,「竞争」也是失败的

\subsection{{\bfseries\sffamily DONE} 蔡元培与大学精神}
\label{sec:orgb5a43f8}
\url{https://wenku.baidu.com/view/f0861d1d5f0e7cd1842536a4.html}

一、略说著名大学的著名与校长的关系
凡一所著名大学的成长,莫不以先进的理念、有识的学者戮力一共而长期积淀作为理想结
果的必要前因。而校长的责任,便在于据适时的先进的理念聚拢一帮有责任、有真识的学
者,于学术与经国的两端作适宜的掂量,并分属先后,以教育家的眼光为民族的和人类的
事业戮力。

在中国,大学的年龄即使最大的也不过百余岁而已,而今日的办学层次便悬殊不齐如此,有
时代也有现状的原因,但最大的问题在于校长的理念以及治校的策略各有不同。  在近代中
国大学的成长过程中,凸现出来的一批著名大学,其伊始的校长大多为学术界的名流,或具
有一般教育者所不有的卓识理念与治校方案。

清华大学校长梅贻琦明确指出:“师资为大学第一要素。”就任北大校长的蔡元培在就职演
说中呼吁,大学当是“研究高深学问”的地方。李登辉执掌复旦大学时,为处理师生间出现
的不同意见,特厘定了《复旦大学师生代表联席会议组织大纲》,确定以“师生合作,发展
学校”为宗旨。张伯苓每逢新年伊始新聘教师到校,便召开新教师茶话会;逢年过节,他与
夫人邀请教师及夫人聚会联欢;每学年完毕,惯例宴请全体教职人员,以酬谢大家一年的辛
苦。梅贻琦和蔡元培深识办学理念与发展方向的重要性,而张伯苓对“安内”与“攘外”的
道理理解的极其深刻。一所大学的著名,绝不是因为几个空喊的口号,违心的誓言托扶起来
的,而需要校长在教育的真目的下对于教员与学生以深切真诚的关怀为契机,在教员与学生
的一齐的长期努力后的结果。

当然,伟大的教育家的成长本身便是一部可赞歌的历程,他们往往有一般人所难以企及的学识和地位。中国历来有“学而优则仕”的传统,所以执掌一
校最高权力的备选者,莫不以学识为第一要素。而况社会中以高等人才聚集之地,更需以超
凡绝伦的人士出任了。比如广西大学校长马君武幼承家学,习经史,曾留学日本、欧洲,获
得工学博士学位。他不仅精通有关的自然科学知识,而且精通英、法、日、德等四国语言,
对史学、哲学、政治学、经济学等都有很深的造诣。

简单的说,近代著名大学的著名,与
一些著名的教育家的努力是分不开的,他们本人即有渊博的学识和深厚的素养,而具体考究
后发现,这些教育家的学识往往以人文科学知识的涵养最为突出,尤其以哲学、美学、文学、
史学、政治学、伦理学、教育学为特色,乃至于神学、宗教、戏剧、音乐等都有不同程度的
涉猎。所以近代著名的大学,往往以人文的理念为其办学的源泉,而以理工的研究作为经国
的纹理。

二、蔡元培进北大的前后

早在1912年1月19日,身为教育总长的蔡元培启用教
育部印信的当日即发布南京中华民国临时政府教育部令:小学堂读经科一律废止。5月,教
育部再度下令:废止师范、中、小学读经科;7月,蔡元培在全国第一届教育会议上提出:
各级学校不应祭孔——“忠君与共和政体不合,尊孔与信教自由相违”„„ 正是清朝固执、陈
旧、封闭的学术体系使蔡元培反感、焦躁,导致他在二十世纪初年采纳了革命政治,而后旅
欧寻找新思想和革命的沃土。这位以读经写怪八股而登科及第的旧文人却以极大的颇力毅然
决然的取缔了例行千年的经科,足见其革命的理念已在行动中开始被证明了。

北大的前身
是清末设立的京师大学堂,既然为政府出资兴办,则其办学的理念便不言自明了,及至于民
国初,北京大学实为一所“官僚的养成所”,而且在学校的管理上有很浓厚的衙门气派。蔡
元培在接受北大校长一职前,即有人劝告,说北京大学“太腐败,进去了,若不能整顿,反
于自己的声名有碍„„”但也有“少数的说,既然知道他腐败,更应进去整顿,就是失败,也
算尽了心„„”结果是“我到底服从后说,进北京”(蔡元培:《我在北京大学的经历》)。
当然,蔡元培之所以“到底服从后说,进北京”毕竟有自己更为深刻的想法,这便与孙中山
的鼓励有关系了,他相信蔡元培可以利用北大这个平台去影响国事。蔡元培目的也在于以教
育实现救国的理想,而绝非为了政治目的,恰恰在蔡元培那里,对于官僚的厌恶,是极其痛
彻的,这在他第一次坚辞北大校长的宣言中可以清晰的看见。

袁世凯做了民国的大总统后,
蔡元培于1912年7月辞职去了法国,表示不愿与袁政府合作。即使袁声称代表“四万万人坚
留”,蔡元培也只做彬彬有礼状答道:“元培也对于四万万人之代表而辞职”。袁世凯没有
办法,只好勉强同意。1916年袁世凯死后,黎元洪出任大总统,北京政府明令恢复了民初
《临时约法》,蔡元培应邀回国,受孙中山的鼓励并于同年12月26日被任命为北京大学校长。

北大的前身是清末设立的京师大学堂,既然为政府出资兴办,则其办学的理念便不言自明了,
及至于民国初,北京大学实为一所“官僚的养成所”,而且在学校的管理上有很浓厚的衙
门气派。蔡元培在接受北大校长一职前,即有人劝告,说北京大学“太腐败,进去了,若
不能整顿,反于自己的声名有碍„„”但也有“少数的说,既然知道他腐败,更应进去整顿,
就是失败,也算尽了心„„”结果是“我到底服从后说,进北京”(蔡元培:《我在北京大
学的经历》)。当然,蔡元培之所以“到底服从后说,进北京”毕竟有自己更为深刻的想
法,这便与孙中山的鼓励有关系了,他相信蔡元培可以利用北大这个平台去影响国事。蔡
元培目的也在于以教育实现救国的理想,而绝非为了政治目的,恰恰在蔡元培那里,对于
官僚的厌恶,是极其痛彻的,这在他第一次坚辞北大校长的宣言中可以清晰的看见。

蔡元培对当时北大的学生陋习有较为深入的了解,比如为学分和证书而利用考试前的时间精
研讲义,有些教员则干脆将试题的内容早些通知学生。教员的讲义也有几年无所更变的等
等。关于师生的道德问题,如北大的部分师生有赌博嫖娼的陋习,决意一并以除之。于是,
蔡元培抱着改革教育、清除积弊的理念于1917年1月8日到北京大学上任。到校视事的第二
天,他发表了《就任北京大学校长之演说》,蔽其旨如下:一曰抱定宗旨。„„大学者,研
究高深学问者也;二曰砥砺德行;三曰敬爱师长。

至于大学的宗旨,蔡元培说的很明白,大学是为“研究高深学问者也。”而不是为升官发财而苦心毅力的。“若徒志在做官发财,
宗旨既乖,趋向自异。平时则放荡冶游,考试则熟读讲义,不问学问之有无,惟争分数之
多寡。试验既终,书籍束之高阁,毫不过问。敷衍三、四年,潦草塞责,文凭到手,即可
借此活动于社会,岂非与求学初衷大相背驰乎?”所以他认为:“我们第一要改革的,是
学生的观念”。观念的改革,也绝非一两次演讲所能起到作用的,关键还是要有一帮有真
学识,热心于学术的人的客观的濡染,这就有了陈独秀执掌文学院院长,有了胡适之的进
入,有了梁漱溟的参与,辜鸿铭的认真,同时也有了林某人的批驳。

蔡决心以“兼容并
包,思想自由”这八个字来塑造北大,是他在欧洲留学期间就已埋下的心愿。他在《〈北
京大学月刊〉发刊词》中阐述了自己对大学精神的理解:“大学者,‘囊括大典,网罗众
家’之学府也。„„各国大学,哲学之唯心论与唯物论,文学、美术之理想派与写实派,计
学之干涉论与放任论,伦理学之动机论与功利论,宇宙论之乐天观与厌世观,常樊然并峙
于其中,此思想自由之通则,而大学之所以为大也。”终其在北大的努力,其言行是一同
的。  三、蔡元培和五四运动  杜威是1919年5月初来到中国讲学的。不久五四运动爆发,
他亲眼目睹了这场学生运动的整个过程。在离开中国前夕曾对胡适说过这样的话:“拿世
界各国的大学校长来比较一下„„这些校长中,在某些学科上有卓越贡献的,固不乏其人。
但是以一个校长身份,而能领导那所大学对一个民族、一个时代起到转折作用的,除蔡元
培而外,恐怕找不出第二个。”  历史是无数的偶然化合的必然。如果没有蔡元培任北大
的校长,就很难有陈独秀和《新青年》与北大的共融,也很难有胡适、李大钊等一大批具
有革新精神的知识分子在一起切磋谈辩。正如胡适后来所说的,如果没有蔡先生,他的一
生很可能会在一家二三流的报刊编辑生涯中度过。当然,如果没有这一大批学者的淘染,
五四运动恐怕是另一番景象了。抗战后创立“九三学社”的许德衎说:“发动五四运动的
主力是北京大学,而其精神上的指导者是蔡元培。” 梁漱溟也说:“蔡先生一生的成就
不在学问,不在事功,而只在开出一种风气,酿成一大潮流,影响到全国,收果于后
世。”(《纪念蔡元培先生》)

学生的请愿活动受到了蔡元培的鼓励,因为他认识到:
“五四运动是社会的各方面酝酿出来的。政治太腐败,社会太龌龊,学生天良未泯,便不
答应这种腐败的政治,龌龊的社会,蓄之已久,进发一朝,于是乎有五四运动”他希望知
识分子能开辟自己的领地去发挥影响力,不是作为一个顺应者而是时代的责任者积极地站
出来铁肩担道义,要求学生们“读书不忘救国。”所以他在电话中回答教育总长傅增湘关
于学生游行的事说:“学生爱国运动,我不忍阻止。”  但是当五四运动发展出乎蔡元培
所料时,蔡元培又疾呼“救国不忘读书”,他说:“吾国人口号四万万,当此教育无能、
科学无能时代,得受普通教育者,百分之几;得受纯粹科学教育者,万分之几。诸君以环
境之适宜,而有受教育之机会,且有受纯粹科学之机会,所以树吾国新文化之基础;而参
加于世界学术之林者,皆将有赖于诸君。诸君之责任,何等重大,今乃为参加大多数国民
政治运动之故,而绝对牺牲之乎?”“诸君唤醒国民之任务,至矣,尽矣,无以复加矣!”
学生们显然是义气过头了,结果与政府发生了严重的冲突,蔡元培当然不是怕官僚的人,
在他的学生被捕后,他积极奔走营救被捕学生,尽到了一位大学校长的职责。并向集会的
学生承诺:“我保证尽我最大的努力”在“三天之内救出我的学生。”对于政界的压力,
蔡元培表示“愿以一人抵罪”,当场议决成立校长团,向当局请愿营救。5月7日北京政府
迫于全国舆论压力,释放了被捕学生,蔡元培亲率全体师生到北大红楼前广场迎接。  5
月8日,蔡元培为承担责任,交付辞呈,在未得到批准的情况下挂冠南归,并且发表出京
启事表明心迹说:“我倦矣!‘杀君马者道旁儿。’‘民亦劳止,汔可小休’。我欲小休
矣!北京大学校长之职,已正式辞去;其他有关系之各学校、各集会,自五月九日起,一
切脱离关系。特此声明,惟知我者谅之。”6月15日,蔡元培发表《不肯再任北大校长的
宣言》,提出:一、我绝对不能再作那政府任命的校长;二、我绝对不能再作不自由的大
学校长;三、我绝对不能再到北京的学校任校长。

蔡元培是近代教育界的典范式人物,他的教育理念,不仅影响了当时的北大,也使得近代的
中国转变起到了一定的航向的意义,这是前面提到的一些证据可以证明的。即使今天的文化
和教育的发展,也或多或少沿袭了蔡元培时代的大学精神。蔡元培的伟大在于,他“打开思
想牢狱,解放千年知识囚徒,主将美育承宗教;”而且能“推转时代巨轮,成功一世人民哲
匠,却尊自由为学风。”是蔡元培,首先在古腐的中国大地上掀起一股强劲的自由学术清风,
使更多人认识到学术的独立比于政治的独立是更高一个层次的,而一个国家是否真正的独立,
则要看它的学术是否自由,正如他说的:“大学者,研究高深学问者也”。

\subsection{{\bfseries\sffamily DONE} 大学精神}
\label{sec:orgaab9c06}
\url{http://baike.sogou.com/v7623620.htm?fromTitle=大学精神}

“大学精神”是大学自身存在和发展中形成的具有独特气质的精神形式的文明成果,它是科
学精神的时代标志和具体凝聚,是整个人类社会文明的高级形式。面临知识经济的机遇和挑
战,建设“大学精神”不仅是高等教育自身发展的需要,同时也是社会进步的需要。“大学
精神”的本质特征概括为创造精神、批判精神和社会关怀精神。

\subsection{1 内容 编辑}
\label{sec:orgf2f2187}
\subsubsection{创造精神}
\label{sec:orgdcd2d90}

创造精神是“大学精神”的大学存在的价值所在,是大学在社会有机体中保证自身地位的根
本生命力。文化的继承不能依赖遗传,只能通过传递方式继承并发展下去。教育从一开始就
成为传递和保留人类文化的重要手段。 爱因斯坦正是在这个意义上理解学校的:“学校向
来是把传统的财富从一代传到下一代的最重要的手段。” 与过去相比,这种情况更加适用
于今天。


由于经济现代化的作用,作为传统的教育的传递者——家庭,已经削弱。因此,比起以前,人
类社会的延续和健康,要在更高程度上依靠学校,大学教育通过确立教育内容,对人类文化
进行选择;对人类文化进行整理。通过更新教育观念,更新人们的价值观念,更新人们的价
值取向,改变思维方式,实现文化的再生。


从 洪堡提出教学与科研相统一的原则看,科学研究成为大学的一个基本职能,大学的科研
成果的多少也就是标志着大学对社会的贡献的大小。如果把大学为社会培养的创造性人才称
为高素质的劳动者,那么,大学的科研成果则是对社会生产力的又一直接贡献,二者共同构
成了大学的生产力与生命力的标志。“斯坦福精神” 之所以被世人称道,关键在于她拥有
众多的诺贝尔奖及全美科学奖的获得者,拥有把科学研究转化为生产力的硅谷效应。


大学是以人才培养为己任的,而创造性恰恰是人才的核心特质。曾任 哈佛大学校长40年之
久的 艾略特认为,大学文化最有价值的成果是使学生具有开放的头脑,经过训练而谨慎的
思考态度,谦恭的行为,掌握哲学研究方法,全面了解前人积累的思想。爱因斯坦更直接地
认为“学校的目标应该是培养有独立行动和独立思考的个人,不过他们要把社会服务看作自
己人生的最高目的。”“一个由没有个人独创性和个人志愿的规格统一的个人所组成的社会,
是一个没有发展可能的不幸的社会。”


另一方面,大学也创造社会理想,并把这些理想传递给社会成员,通过人们的实践,使理想
变成现实的文化实体。社会理想是社会需要的具体反映,这种需要是反映社会发展规律并以
社会发展规律为基础的。由于在文化积累方面的特殊优势,知识分子,特别是集中在大学校
园里的知识分子比其他社会成员更能认识社会发展规律。有了对社会规律的认识,就能够提
出符合社会发展规律的社会理想。

\subsubsection{批判精神}
\label{sec:org164faaa}

批判精神与社会其他结构相比,大学具有自身的优势。具体表现在:知识聚集的场所。大学
是继承传统科技文化遗产,不断创造新科技文化的场所,聚集了古今中外各种知识,具有很
强的知识容量。思想观念和学术思潮的交汇处。大学生产生新思想,包容新观念,在这里不
同的学术观念可以并存,不同的思想可以通过学术交流相互影响,具有良好的争鸣传统。追
求理想的永恒特性。


从 欧洲中世纪早期的大学开始,就有了自治的传统,并以传播知识和研究学问为最高理想,
相对超越于社会现实。大学的批判精神首先表现为大学教师在教学和科研过程中能够以科学
的态度对待传统与现实,否定非科学的内容,破除迷信与保守主义,建立科学的知识体系。
可以这样说,大学的教学与科研发展史就是科学史重要过程的展开史,是一个肯定与否定相
结合的扬弃过程。


大学批判精神的另一方面是对社会现实的理性反思和价值构建。进入技术时代后,技术性淡
化了人性,使人失去了对他人的热情和敏感,结果,人性变成了技术的牺牲品。同时,人性
又屈服于技术,把技术崇拜为神。科学与人文分离的结果就两个极端而言,出现了两种畸形
人,一种是只懂技术而灵魂苍白的“空心人”,一种是不懂技术、奢谈人文的“边缘人”。
现实社会改变这种“技术毒害”是无力的,而大学教育者,特别是 人文社会科学教育却将
其作为应有的内容。


早在本世纪初,西方一些著名的大学就注意克服这种片面性,探索科技与人文的汇通之路。
哈佛的学生在一二年级开设“通识课程”,广泛涉及人文、社会和自然科学的各个方面。
麻省理工学院的工科学生要学占总课时22\%左右的人文课程。我国现行被一再呼吁的人性教
育、全人教育、通识教育、道德教育、心理教育等无不是针对技术对人的异化进行批判的结
果。


批判精神的最后一个方面是大学知识群体对政府决策的参谋和建议。科学决策是政府决策的
关键,但是由于决策者自身素质的限制,做到科学决策并不容易,所以要倾听专家意见,请
专家参与决策成为决策机制中的一环,专家之所以成为专家,就是因为他们职业所特有的对
问题的科学态度和客观的批判精神。


\subsubsection{社会关怀精神}
\label{sec:org1c20f99}

社会关怀精神高等教育是社会发展的必然产物,社会需要是第一推动力。在工业化、信息化
的社会里,大学已经被越来越深入地卷进社会机器的运转之中。关注现实、服务社会成为高
校的第三职能,高等教育通过科学研究直接转化为社会第一生产力——科学技术;通过人才培
养,为社会提供生产力中最活跃的因素——高质量的人力资源。


社会关怀精神还表现在大学对社会精神文明的参与和建设。除了在生产力方面对社会的贡献
外,大学通过直接的人文社会科学的研究和宣传为社会提供精神产品,包括哲学研究、文学
创作与批判、思想道德建设等。知识分子在提炼和批判社会生活的同时,又把各种精神产品
投资到社会,为社会主义建设提供直接的内容。


\subsection{2 社会示范 编辑}
\label{sec:org4507dde}
大学还通过 校园文化建设为社区文明提供示范作用。担当示范角色必须具备两个条件:一
是超前性,走在时代的前列;一是完美性,具有理想价值。大学的创造力为其走在时代前列
提供了无限的动力源泉,而大学特有的思想兼容、百家争鸣的学术氛围又保证了各种思想观
念的撞击,有利于形成较为和谐的精神环境。如果说示范作用是综合的、显性的,那么价值
引导则是深层的,久远的。

社会的现实价值存在总是多元的,而且具有短期性、易变性等特征,但大学的价值观念由于
受到历史文化积淀的影响,具有摆脱短期功利狭隘性的特质,因此它可以借助于批判精神,
制衡社会负价值的发展,担当起引领主流价值的形成和推广的作用。布鲁贝克认为,60年代
以来的 美国大学“不仅是美国的教育的中心,而且是 美国生活的中心,它仅次于政府成为
社会的主要服务者和社会变革的主要工具。”它是新思想的源泉、倡导者、推动者和交流中
心。


教育是一项基础性、先导性的伟大事业,伟大事业必须有强大的力量支撑。那么,我们应该
培育什么样的大学精神﹖笔者就此问题谈点粗浅看法。


\subsection{3 定义 编辑}
\label{sec:orgf68f81c}
大学精神的核心是以育人为第一要旨,以全面人才教育为大学使命。育人的重点,首先是培
养学生对国家、对民族的责任感。培养有抱负、有政治远见、有广博知识、有责任心的人。
要教育学生以天下为己任,继承前人“国家兴亡,匹夫有责”的报国之情,学习前人“ 先
天下之忧而忧,后天下之乐而乐”鞠躬为民的品德。关心天下大事,使自己服从于社会,服
从于国家,服务于人民。其次是理想、信念教育。理想和信念是精神世界深层次问题,它取
决于世界观、人生观和价值观。要引导学生树立正确的人生目的、人生理想、人生追求和科
学的自然观、历史观、社会观和 辩证唯物主义认识论。第三是培养爱心。要教育学生爱父
母、爱生活、爱事业、爱祖国。第四是培养高尚的人格。坚持真理,胸怀坦荡,高风亮节,
严于律己,宽以待人,淡泊名利,无私奉献。第五是培养自强不息、厚德载物的精神。不但
教育学生如何认知,如何做事,更重要的是如何做人。引导学生敢于奋斗,善于成才。总之,
育人的目的就是实现江泽民同志提出的“学习科学文化与加强思想修养的统一;学习书本知
识与投身社会实践的统一;实现自我价值与服务祖国人民的统一;树立远大理想与艰苦奋斗
的统一。”使我们的大学生“成为理想远大、热爱祖国的人,成为追求真理、勇于创新的人,
成为德才兼备、全面发展的人,成为视野开阔、胸怀宽广的人,成为知行统一、脚踏实地的
人。”


\subsection{4 具体内容 编辑}
\label{sec:org0e5a0de}
\subsubsection{尊重科学}
\label{sec:org17c0417}

科学技术的力量是无法抗拒的。科技改变了人的观念,改变了人的生活方式,改变了 经济
发展模式,改变了社会发展进程。大学的主要任务是传播科学精神、培养科学素养。科学精
神是尊重规律、实事求是、勇于探索、敢于创新、坚持真理、修正错误、实证实干和独立的
精神。科学素养是指参加国家文化事务,经济生产和个人决策所必须具备的科学概念和科学
过程的知识水平和理解程度。具体地说,能认识世界的多样性和统一性;掌握科学的基本概
念和原理;了解科学、数学和技术的作用和局限性;具有用科学方法思维的能力;能够用科
学知识和科学思维方法处理和解决社会及个人问题。要对学生进行科学研究的锻炼,鼓励冒
尖,允许失败。通过科学研究的实践,逐步培养学生的科学观念、科学精神、科学方法和科
研能力。


善于创新是大学精神的灵魂。要想在教育理念、办学思想、培养模式、教学管理等方面塑造
自我,具有个性,没有创新是不行的。哈佛大学以师资雄厚,将近40名教授获诺贝尔奖而著
称,学生以学术卓越、全面发展、自信能干而闻名。 耶鲁则以教授治校、思想开放、人文
一流、盛产总统而骄傲。而 普林斯顿大学以重质量、重研究、重理论,并培养出38位诺贝
尔奖获得者而卓誉世界。 哥伦比亚大学既是一所大学校,也是一所大企业,竟然也培养出
34位诺贝尔奖获得者。年轻的 斯坦福大学以强烈的进取精神,提出不因袭任何传统,沿着
自己的路标向前,以“学术顶尖”的构想建设大学,成为“硅谷”的强大后盾。总之,凡是
有特色的大学,都因善于创新,坚持走自己的路而成名。


\subsubsection{唯才是用}
\label{sec:org6b2e196}

大学之道,在于育人,育人之道,在于大师。师强则学子成才,师惰则误人子弟。办好大学
的奥秘在于名师如林、唯才是用、兼容并包、宽容尊重。学术上需要有兼容并包的精神,要
鼓励 学术自由、民主竞争、思想碰撞、中外交流。学生既可读《诗经》,也可读《圣经》。
要引导学生“ 博学之,审问之,慎思之,善辩之,笃行之。”使大学成为科学与艺术的实
验室,成为青年学子崇拜的殿堂,成为博大精深的思想库,成为精英人才的聚集地。


\subsubsection{崇尚民主}
\label{sec:org8d1b9eb}

大学的民主精神主要体现在民主管理和民主施教上。实施民主管理必须更新教育观念,改革
教育体制,鼓励多样化,建立公平竞争环境与机制,建立规范化、法制化管理模式。要求大
学管理者的作风与品质,不是自信专横,而是从善如流;不是固步自封,而是善于进取;不
是因循守旧,而是富于想象;不是高高在上,而是深入群众;不是妄自尊大,而是对自己能
力的局限性颇有自知之明。


民主施教的关键是视学生为朋友,教学相长。倾听学生的意见,不断改革教学内容和方法。
MIT的做法值得参考:一是为了培养学生的独立探索精神,老师总是留出自由思考的时间:
二是给本科学生提供研究机会,学生或者参加老师的研究课题,或者自己设计题目请老师做
顾问;三是安排独立活动计划,由学生自己决定活动目标和实现日标的方法:四是提供科学、
技术,利用人文学科并进的“博通计划”,鼓励学生参加各种学术讨论会。这些都是改变传
统教学模式,实施民主教学的新方法,体现了大学的民主治学精神。


\subsubsection{韬光养晦}
\label{sec:orgddc7fb3}

大学为了一有大师,二有成果,必须以韬光养晦的精神,艰苦奋斗,长期积累,才能江河万
里,蓬勃发展。正如江泽民同志所说:“纵观历史,国际上的一流大学都是经过长期的建设
形成的。因此,要办成一流大学确实需要有一定的历史过程,要经过社会实践考验。对此,
既要有雄心壮志,又必须脚踏实地。”因此,大学要作长期矢志不渝的奋斗,以便造就一支
优秀的师资队伍,建立出名的重点学科,培养品学兼优的学子,创造一流的科研成果,形成
大学自己的特色。


大学精神维系着大学未来的命运;教育的理念决定着学生综合素质的高低。君子务本,其命
维新。”只有 高扬育人第一、独尊科学、善于创新、唯才是用、兼容并包、崇尚民主、韬
光养晦的大学精神,才能不愧对21世纪的辉煌。


被载入史册、流芳百世的大学应归功于她的精神,声震寰宇、名噪一时的大学也是缘于她的
精神。今天的大学,要获得长足的发展,肩负起社会的重托,完成历史赋予的使命,也必须
有自己的精神支柱。本文在诠释大学精神的内涵与特性的基础上,揭示了大学精神对大学存
在的作用,并对大学精神的塑造与发扬进行了初步的探讨。


解释

大学作为一个存在的实体,活生生地展现在人们的眼前,而寄存于这一实体中的精神却不能
仅靠视觉就能观察到,必须深入其中才可体会。“精神”一词抽象却富有魅力,大学的魅力
正在于她的精神。如何界定“大学精神”?大学精神的内涵是什么?这是本文不可回避而必
须首先论及的问题。大学精神的内核是一种不媚俗的精神,既是潜心向学的纯粹的学术精神,
又是引领社会,敢于不随波逐流的正确的批判精神。


大学精神既深藏于“大学”之中,又游离于“大学”之外。它,给大学注入了生命活力,使
大学不仅仅是教学楼、图书馆、林荫道等冷冰冰的建筑群落,也不仅仅是人才的集散地,而
是人、思想、价值观念、理性思考、创新、智慧与博大胸怀的代表。笔者认为,“大学精
神”是在某种大学理念的支配下,经过所在大学人的努力,长时期积淀而成的稳定的、共同
的追求、理想和信念,它是大学生命力的源泉,是大学文化的精髓和核心之所在,是对大学
的生存起决定性作用的思想导向。大学精神之于大学正如土壤、空气、水、阳光之于植物的
生命一样重要。大学精神本身蕴含着丰富的内涵,具体而言,表现在以下三个方面:


\subsection{5 表现 编辑}
\label{sec:orgfbce137}
\subsubsection{第一,自觉}
\label{sec:org7fac183}

大学是研究高深学问的地方,大学应有的品位是“真正培养出一些智慧的才具,培养出一些
有骨头、有广博知识,同时又有影响力的知识分子”( 李敖语)的地方。自1816年洪堡创建
柏林大学开始,学术开始进入大学的殿堂,科研在大学生活中占据着越来越重要的地位,崇
高的学术声望已成为知名大学的“通行证”。大学教师不仅仅教书育人,也必须是一个研究
者,因为他们面对的是“成熟、独立和精神已有所追求的年轻人,大学生不应单纯地接受知
识,更应以探索学问为己任。 叶恭绰在做交通大学校长时的一次演讲中曾告诫师生:“诸
君皆学问中人,请先言学问之事。……尝以为诸君修学当以三为难衡:第一,研究学术,当
以学术本身为前提以达于学术独立境界。……夫学术之事,自有其精神与范围,非以外力逼
迫而得善果者……。” 清华大学校长梅贻琦说:“大学者非谓有大楼之谓也,有大师之谓
也。”大师,素以孜孜不倦地探究学问为特质,故而,大学之高深、大学之涵阔、大学的发
展均在于有探究学术的精神。


\subsubsection{第二,永恒}
\label{sec:org0f63073}

大学是任何一个社会道德与理性的凝聚之所,具有高雅的文化品位和卓而不凡的气质,能够
出淤泥而不染,并孜孜以追求自己的理想。大学不仅以自身纯洁的德性潜移默化地影响着社
会,更以积极的姿态投入到改造社会、重塑德性的潮流中,成为社会德性的捍卫者与提升者,
领导着社会德性的发展方向。尤其在时代的变迁中,大学的道德精神就更为彰显。浙江大学
校长 竺可桢在战时西迁途中对学生说:“乱世道德堕落,历史上均是,但大学犹如海上 灯
塔,吾人不能于此时降落道德标准。切记:异日逢有作弊机会是否能涅而不淄、度而不磷,
此乃现代教育试金石也。”大学的道德精神源于大学人总体的道德精神,毋庸讳言,大学人
是社会中应该最有德性和理性的一族。正由于他们的存在,才铸成了大学精神,才使大学成
为海上的灯塔,指引着社会向着更美好的地方前进。


\subsubsection{第三,敏锐}
\label{sec:org48fbca5}

“每个国家,当其变得具有影响力时,都趋向于所处的世界上发展居领导地位的智力机构?
希腊、意大利的城市、 法国、西班牙、英国、 德国,以及现在的美国都是如此。伟大的大
学是在历史上伟大政治实体的伟大时期发展起来的。今天,教育与一个国家的质量更加不可
分割。”(Clark Kerr:《大学的作用》, 陈学飞译,江西教育出版社,1993年,第63页)
无论中西,伟大的大学必定是领时代先锋的,否则将不会有 克拉克笔下的强国。从中世纪
大学的兴起到现代大学的发展这一历史演变轨迹可以看出,大学无疑是时代的产物,并代表
着最进步的时代精神,驱动着社会向前发展。弗莱克斯纳的话一针见血:“大学不是某个时
代一般社会组织之外的东西,而是在社会组织之内的东西。……它不是与世隔绝的东西、历
史的东西、尽可能不屈服于某种新的压力的东西。恰恰相反,它是……时代的表现,并对当
时和将来都产生影响。”(Abraham Flexner:《大学:美国、英国、法国》, 牛津大学出
版社,1930,第3页)大学,作为时代的智者,能够预见并感应到社会潮流的前奏,而成为推
动社会潮流的先行者,使社会潮流之声最终成为时代的最强者。大学正是紧紧扣住了时代的
脉搏,才赢得了自身持续发展和地位的逐渐提高。


\subsection{6 存在作用 编辑}
\label{sec:org52dd38b}

我国教育学者 杨东平说,“人才辈出,大师云集,主要是一种制度文明的产物,不是急功
近利的政策能够催化出来的。”在“五四”和民国时期,北大、清华表现出来的精神和风采
至今让人留恋,其气象恢弘、学术自出、欣欣向荣的面貌正是大学精神在追求宽松的文化与
制度和谐共生的环境下孕育出的结果。创建 世界一流大学是我国高等教育改革的追求,而
“大学精神和制度的建设比资金更重要”,所以,弄清楚大学精神对大学存在与发展的作用,
无疑会加快我国大学向世界一流大学迈进的步伐。具体而言,大学精神对大学的生存与发展
的作用有以下两个方面:


\subsubsection{生命力的体现}
\label{sec:orgf077735}

大学精神对大学生存与发展的作用犹如人的精神对人的存在的意义一样,没有了精神,大学
便失去了生气,失去了发展的动力,最终也将走入穷途末路。“精神”使大学敢于迎难而上,
敢于挑战强权,敢于捍卫正义,敢于领时代所先。正因为大学拥有了学术精神,大学才成为
知识的源泉,学问的中心;正因为大学拥有了人文精神,大学就多了几分正义与正气,“一
个社会的文化底蕴和精神气质,尤其体现在大学的人文理性之中;一个人的胸襟和个性,来
源于他所受的人文精神的培养。……”(《岭南文化时报》,1996年8月28日)只要大学拥有
精神,她就不会唯唯诺诺,而像参天大树,在适其生存的环境中欣欣向荣、蓬勃发展;在逆
其生存的条件下亦能坚韧不拔,站在时代的员前沿和员顶端。在新旧文化激烈冲突的年代,
没有北大追求科学与民主的精神,就不可能有北大在世人心目中的崇高地位。在 抗日战争
硝烟弥漫的岁月,如果没有西南联大的合作精神、民主精神、自由精神,就没有西南联大的
存在,更没有出自西南联大的一大批杰出的科学家。朱利叶斯?A?斯特拉顿(Julius A
Stratton)曾评论道:“真正的大学精神与有助于进行项目组织、项目规划和昂贵设备的利
用等这些管理因素之间基本上是不相容的”。大学作为一个社会的文化存在的确与朱氏所言
的管理因素不相容.因为大学精神给予大学的是从学理和思想上关注、思考、讨论和批判社
会现实问题的权利。“现代社会科学已无可置疑地证实:经济体制和社会体制并不是一切,
它们的运作必须有另一种健全的文化精神与之配合。这种精神主要来自大学的高等教育。在
现代社会中,大学是精神堡垒,有发挥提高人的境界、丰富人的思想的重大功能。”“推倒
大学围墙,实际上是大学精神的自我否定,它可能最终取消大学的存在权利。”(《岭南文
化时报》1995年3月28日)失去了精神的大学,意味着这所大学生命力的枯竭。因此,大学不
能没有自己的精神。


\subsubsection{抵御腐蚀盾牌}
\label{sec:org146fd2d}

大学同其他社会机构一样.植根于社会,受制于社会的政治、经济和文化等。但大学与其他
社会机构在受社会影响方面的最大不同之处在于,大学具有独立的人格特质,有骨气,不随
波逐流,既能够抵御金钱的诱惑,又能够抵抗来自政治的压力和干扰,大学的这种人格特质
既是大学精神的体现,又是大学精神的内在成份之一。因此,大学精神是维护大学纯洁与独
立、平等和民主的金色盾牌。据报载, 牛津大学曾拒绝了一位 沙特富翁1000多万英镑的捐
款。原因在于这位沙特富翁在捐款时提出了附加条件,要求牛津大学办一所以他命名的商学
院。牛津大学董事会经过讨论,认为不能够因为钱而放弃大学独立自主的传统,不能开大学
受制于个人的先例,毅然拒绝了唾手可得的巨额钱财。牛津大学并不是不需要钱,而是不愿
意把自己的命运交给别人,所以,当资金的获得需以自由研究和独立决策的丧失为代价时,
牛津大学毅然地“望而却步”了。这一方面是对大学精神的守护,另一方面也是具有悠久历
史的大学精神对决策者影响的结果。 蔡元培治校时期的 北京大学,也充分体现出了强烈的
自主精神。蔡元培实行 思想自由、兼容并包的方针,聘请了不少新文化的代表人物担任教
员,如 陈独秀、李大钊、鲁迅、 胡适等。当社会反动势力攻击这些进步知识分子,要求解
聘他们时,蔡元培总是顶住压力,挺身而出保护他们。在蔡元培等一代社会精英的精心培育
下,北京大学形成浓厚的追求民主与科学的氛围。这种精神氛围不仅影响了教师,而且也深
深地影响了学生,“五四”运动的爆发正是这种精神氛围长期催化的结果。由此可以看出,
大学精神具有潜移默化的影响力,在不知不觉中使深居其中的教师和学生内化为个人品质,
表现出与大学精神的内涵相一致的行为。因此.大学精神是大学抵御诱惑与腐蚀的盾牌。恰
恰因为大学具有出淤泥而不染,超凡脱俗的品质,才为世人所敬仰,才在世人心目中占据神
圣的地位,也为自己的发展赢得了条件。


\subsection{7 塑造发扬 编辑}
\label{sec:orga14e4b5}
正因为大学精神对大学的存在与发展有着至关重要的作用,所以,每所大学都应塑造或发扬
符合本身实际的、满足时代及未来需要的精神,从而保持大学的生命之树常青。虽然不同的
大学有不同的大学精神,但在大学精神的塑造或发扬方面,却有着许多共性的条件,表现为:


\subsubsection{校长至关重要}
\label{sec:org34e28c0}

治校要有校训,校训乃一校精神风貌的体现,且与一校之长的治校理念关系甚密。校长应该
具有什么样的素质?克拉克在《大学的功用》一书中认为大学校长必须具备三种品质:决断、
勇敢、坚韧,校长是集多种社会角色于一身,既是领导者、 教育家、创新者、教导者、信
息灵通人士;又是官员、管理人、继承人、寻求一致的人、劝说者、瓶颈口;但他主要是一
个调解者,作为调解者的头等大事就是相安无事?如何使“七十二行不和谐的派别相调和”。
校长是大学的 灵魂人物和神经中枢,好的校长是带起一所好的大学的前提条件。北大没有
蔡元培不可能成为新文化的中心,清华没有梅贻琦也不可能在短时间内声名鹊起,南开没有
张伯苓也很难获得长足发展。而这些大学的声望之所以与日俱隆,关键在于拥有一批像蔡元
培、梅贻琦、张伯苓这样的校长,他们有共同的追求,有前承后继的使命感,能够维护并发
扬已确立的大学精神。而之所以能拥有一批这样的校长,是因为这些大学建立了良好的校长
选拔机制。在当时的历史条件下,校长是向社会公开招聘的,他们多为学贯中西、思想开放、
又有爱国热情的仁人志士。因此,一所好大学,必须要有好校长,而最重要的是要有选拔好
校长的运行机制。新中国成立以后,大学校长多由本校内部产生或由上级委任,开拓意识不
强,对大学的生存与发展缺乏持续性战略思考,对大学精神的内涵理解不深、重视不够。美
国学者欧内斯特博耶说,“在确认大学校长的中心作用时,我们要提出一个问题:校长是否
为大学提出了鼓舞人心的宏图大计和远景规划?”所以,重建良好的校长选拔机制很有必要,
使大学并不因一个好校长的离去而放弃大学应有的追求和使命。


\subsubsection{建设校园文化}
\label{sec:org436e522}

大学是知识分子思想自由奔放的家园,大学精神就充分体现、弥漫于校园文化中。较之于社
会的其他角落,大学校园显得更为纯净。身居其中的大学人也不知不觉地受校园文化的影响
和熏陶,而表现出不同的性格特质。正如,人们总体认为北大人好动、灵活、争强好胜,而
清华人好静、踏实、谦虚谨慎一样,特定的校园文化熏染出特定的群体个性,特定的群体个
性中透露和折射出特定的大学精神。校园文化是大学精神的载体,大学精神的塑造或发扬应
与大学校园文化的建设同步进行。值得注意的是,校园文化不仅包括物质文化,还包括制度
文化和观念文化,而且制度文化和观念文化在某种程度上比物质文化(校园环境建设)更为重
要。很多大学只重视校园环境-硬件方面建设,而相对忽视校园制度文化和观念文化-软件
方面的建设。因为校园环境的改善是看得见的,而制度和观念文化的建设却不能很快收到成
效。这种短视行为,使大学校园文化中制度文化和观念文化成为“软肋”,带来了不少显而
易见的不良现象。学生读书于校园,潜心做学问的少,意在出国深造谋好职业、浮于跟随社
会潮流的多;校外投影厅、酒吧打折优惠的海报比校园学术讲座的海报更有气势;学年伊始
各社团纷纷招兵买马一阵热乎过后,就偃旗息鼓……,校园内,除了树林中晨读的身影和图
书馆埋头苦读的情景让人心动外,还有多少值得品味的“文化”,又怎能使学生对大学产生
归宿感,怎能增强学生的使命感和责任感呢?“校园文化是通过对大学生德、智、体诸方面
的全面培养,形成健全的人格素质,把体现大学精神的科学态度、文明风范、价值观念等带
到社会,影响和感染其他人。”校园文化的核心内容是精神、价值、作风和理想追求,美丽
的校园环境只能给人留下表面印象,而校训、学风、教风、传统、讲座等价值层面的成分才
真正给人以深刻的启迪和实实在在的影响。因此,塑造或发扬大学精神也必须不断加强校园
文化的建设,尤其是制度文化和观念文化的建设。


\subsubsection{师生关系}
\label{sec:org2bf69d1}

教师和学生是大学校园永恒的主人,正由于他们的共同努力,才建设了大学精神,发扬了大
学精神,改造了大学精神。由此可见,平等和谐的师生关系,不仅有利于大学精神的形成,
而且有利于大学精神的延续。虽然大学生在生理上已成熟,独立性和自主性所增强,但他们
的进一步发展仍然离不开教师的引导;虽然现代化的教学手段为学生自我学习、自我提高、
自我教育提供了便利的条件,对教师的传统地位有一定冲击,但教师的形象会直接或间接影
响学生的思想观念和 行为举止。“传道、授业、解惑”本应成为教师的光荣职责,每一个
学生也都应具有尊师重道的基本品质,教师与学生理应在多边交流中建立亲密的、互助的合
作关系,共同探讨生活的价值、生命的意义和万物的真理。然而,随着市场经济负面影响的
冲击,以及腐朽、落后思想的传播,大学围墙里的师生关系也发生了很大的变化:教师和学
生的距离越来越远,上完课后,教师夹起讲义就走,平时几乎不和学生交流,上了一学期的
课,认不得几个学生的现象司空见惯,教师成为一个地道的“教书匠”。师生之间缺乏基本
的沟通,缺乏心与心的交流,深厚的师生情谊自然就无从谈起;不少教师放弃了两袖清风的
知识分子形象,业余兼职,下海经商,锱铢必较,言必称利,遮掩了教师头顶的神圣光环.
也在一定程度上影响了学生对教师的敬重;有的教师对学生亲疏有别,甚至做出违背原则的
事情,如不及格的学生给任课教师送点礼,就可以顺利过关,这必然会有损教师的形象;更
有甚者,个别教师师德败坏,做出违法乱纪的事情,动摇了教师在学生心目中的神圣地位,
严重破坏了教师的整体形象。在这样的大背景下,学生很难得到来自教师的关爱,教师也失
去了来自学生的敬重,师生关系渐趋冷漠。大学教师与学生感情的淡漠,既制约了大学精神
之花的盛开,又加速了大学精神之花的枯萎、凋谢。所以,在当前条件下,改造师生关系不
仅非常必要,而且还十分迫切。 雅斯贝尔斯说得好,“大学的理想要靠每一位学生和教师
来实践,至于大学组织的各种形式是次要的。如果这种为实现大学理想的活动被消解,那么
单凭组织形式是不能挽救大学生命的,而大学的生命全在于教师传授给学生新颖的、合乎自
身境遇的思想来唤起他们的自我意识。”(《什么是教育》, 邹进译,三联书店出版,
1991)大学精神的塑造是广大师生共同努力的结果,大学精神的发扬,也需要广大师生共同
维护。作为大学主人的教师和学生,应当建立自由、平等、和谐、互助、充满人情味而又不
违背原则的亲密关系,成为追求真理道路上的合作伙伴。这种师生关系的确立、巩固与代代
相传,不仅是大学精神酝酿与产生的基本条件,也是大学精神长盛不衰的根本保证。


\subsection{8 小结 编辑}
\label{sec:org6339d96}

大学精神有着丰富的内涵,对大学的生存与发展起着至关重要的作用。世界上任何一所知名
大学都有自己独特的大学精神,这不仅是一笔宝贵的财富,也是大学魅力之所在,更是大学
持续发展的动力。在我国建设世界一流大学的道路上,在大学之间竞争愈演愈烈的今天,大
学精神的塑造是必不可少且尚需加强的一个重要环节。


严谨扎实 刻苦钻研 质感生命


\subsection{{\bfseries\sffamily DONE} 世界一流大学的学生是怎样学习的?}
\label{sec:org7094900}
\url{http://blog.sina.com.cn/s/blog\_63af05000102x0h4.html?tj=edu}
世界一流大学的学生是怎样学习的?

世界著名大学既是大师云集的地方,也是培养大师的殿堂。

石毓智

\subsection{A、从毕业生的素质看大学的水平}
\label{sec:orgade10f6}

衡量一所大学的水准,一是看它的科研成果,二是看它的学生素质。我认为,评价
一所大学的学生素质的一个重要指标就是其毕业生中获得诺贝尔奖等国际顶级奖项
的人数。



世界大学排名榜五花八门,名次的出入非常大,因为它们所依据的标准不同。迄今为止,
我还没有见到哪个排名榜是把毕业生获得国际顶级奖项作为评估标准的。其实,一个大学
培养的本科生或者研究生毕业后的学术表现更能反映这个大学的教育水准,更能说明一所
大学的教育理念是否符合教育规律。


那些世界公认的世界一流大学,都是教师队伍强大,毕业生优秀。按照毕业生获得
诺贝尔奖人数来排名,世界前五位的大学分别是:




第一名,哈佛大学,其毕业生有76人获得诺贝尔奖。

第二名,剑桥大学,其毕业生有65人获得诺贝尔奖。

第三名,哥伦比亚大学,其毕业生有44人获得诺贝尔奖。

第四名,麻省理工学院,其毕业生有34人获得诺贝尔奖。

第五名,加州大学伯克利分校,其毕业生有32人获得诺贝尔奖。



上面这几所大学我都去过,走马观花了他们的校园文化。斯坦福大学建校历史比较
短,其毕业生获得诺贝尔奖的人数只有13位,然而其现今的教师和科研队伍中有20
余人获得诺贝尔奖,特别是进入新世纪以来该校就有11人获得诺贝尔奖,名列世界
第一。最近几年斯坦福大学本科生入学竞争的激烈程度已经超过哈佛大学等,其学
生素质可想而知。


其他一些世界名校,同样以培养出众多杰出人才作为立身之本。2008年我到德国开会,顺
路参观了海德堡大学,到那里才知道这所大学的毕业生中有17人获得诺贝尔奖,教师队伍
中也有十几人获得此殊荣。德国还有更牛的大学,诸如洪堡大学、慕尼黑大学等,都培养
出了极多的杰出人才。


单看亚洲,按照毕业生获得诺贝尔奖人数这个标准,日本东京大学第一,有11人获
得诺贝尔奖;日本京都大学第二,有6人获得诺贝尔奖;日本名古屋大学和以色列
的希伯来大学并列第三,都有5个毕业生获得诺贝尔奖。这些数据更能说明问题,
获得诺贝尔奖的毕业生人数,是衡量一个大学水准的可靠而稳定的指标。




\subsection{B、充分信任学生的“荣誉考试制度”}
\label{sec:orgf144614}

诚信关系着教育的成败,影响着优秀人才的培养。世界一流大学的学生的首要特质
就是讲究诚信。


美国有少数几所名牌大学实行“荣誉考试制度”,斯坦福是其中之一。这种制度规
定,不用老师监考,完全信任学生。考试的时候,老师把考卷发完后就离开考场。
办公室远的老师,搬个凳子坐在考场门外,学生有问题就出来问。办公室近的老师,
就回到自己的办公室,学生有不清楚的地方就去办公室找老师问。学生可以带任何
自己的东西到考场,包括作业本、教材、词典等,没有任何限制,而且你爱放哪儿
就放哪儿,搁在自己的考卷旁边也行。


考试中间,学生想上厕所,甚至想到室外透透风,不需要向任何人请示,也不需要
作任何登记。学生做完考卷后,把它放在桌子上就可离开,到时候老师就会来收卷
子。很多人会想,这不会乱套吗?其实,这种做法在诚信较好的社会里,比有监考
老师、有摄像头监视给人的压力还大、还可怕,让你觉得周围的同学都是“监考官”,
任何不轨的行为都会招来鄙视的眼光。


“荣誉考试制度”就是充分信任学生,认为每个学生都是诚实的、优秀的。那么,
每个学生也要用自己的行动来维护自己的尊严和名誉。我在斯坦福读博士期间,经
历了很多场闭卷考试,没有遇见一次作弊的事,也没有一次听说有人作弊,更无因
作弊被学校通报处罚的新闻。斯坦福的学生都很傻、很单纯,谁也不会往这方面想。
大家平时都在努力学习,考试的时候也就老老实实地来证明自己的真实能力。这种
看似平淡的事情,却有非凡的效果,让每个人在轻松愉快的环境中把自己的能力发
挥到极致。


2010年,我在斯坦福访学期间,修读了数学系的一门《现代代数》,是本科生课程。
这门课有一个期中考试,考试场地和时间由学生自己选择,根本不占课堂时间,考
试方式自由得令人吃惊。老师提前一个星期就在学校的这门课的教学网络上把考题
公布出来,学生可以自己任选一个地方,用两个小时把题目做好,到了规定的那一
天,学生把答好的考卷交给老师就行。


这次访学期间,我还修读了计算机系开设的《信息论》。这门课没有闭卷考试,就
是根据平时三次大作业评定成绩。学生交作业那天,教这门课的教授把所有题目的
答案打印好,厚厚的一摞子放在讲台上。在上课之前,这位教授宣布:“今天交作
业的同学可以拿一份答案回去对照,看自己答得如何,而今天因故不能交作业的同
学则下次再拿答案。”


那些当天交不了作业的同学,就自觉下个星期交作业时再拿答案。老师也不担心那些未完
成作业的学生借同学那一份答案回去抄,学生也不会想这个点子。这是一种信任的契约,
它是师生心目中最神圣的东西,谁也不会去违背。




\subsection{C、教室座位和吃饭时间所反映的精神面貌}
\label{sec:org805583a}

从课堂学生选择座位的情况,可以看出大学的学习风气。我在新加坡上过这么多年
的课,又在国内几十所大学讲过课,发现一个学生上课座位的分布规律,前两三排
一般是没有人的,而后几排座位的人最多。在斯坦福大学,这种情况恰好相反,前
几排的人最多,后面依次减少。这种学生上课选择座位的情况,一方面说明学生的
学习热情;另一方面也说明学生只上自己真正喜欢的课。


充分利用吃饭的时间学习,也是世界一流大学的普遍现象。这里讲一个我在斯坦福
的一次经历。2011年2月的一天午饭时间,我打好饭一个人坐在一张空桌子上。一
位亚裔学生端着饭问我是否可以跟我坐在同一张桌子上,这也是美国人的一种礼貌,
遇到这种情况都要征求一下先坐下者的意见。我当然同意了。当时我心里在想,看
看这位学生会不会拿出书来边吃边看。果不其然,这位同学一坐下就马上拿出一本
书来,边吃边看,一直到我走的时候,都还是低着头。这种事情在这里太正常了。


2011年夏天,我到普林斯顿大学参观时,看见校门口的一个小花园里有一个街头艺
术,名字叫“午餐时间”,是一尊铜像,一个学生坐在地上,一手拿着刚咬了一口
的三明治,另一手捧着书专注地阅读着。这尊铜像是世界一流大学学生的学习和生
活的定格。或许,边吃饭边看书的习惯并不好,但从中却折射出世界一流大学学生
的学习风尚。


斯坦福的校园又大又漂亮,到处都是凳子、花丛,应该是最适合谈情说爱、花前月下的地
方。但我在斯坦福生活了这么多年,一到晚上校园里静悄悄的,从来没有见搂搂抱抱、亲
亲热热这种现象。




\subsection{D、学生在“教授门诊”排队问问题}
\label{sec:orge510b10}

一流大学的学生都善于思考,因而必然会有这样那样的学习上的问题,主动而频繁
地找老师问问题便是著名大学的一个特殊的景观。斯坦福大学的教授,每周都会抽
出两个固定时间,在办公室里待着,专门解答学生的学习疑难问题。教授办公室的
门口,常常摆着若干张凳子,因为学生经常来问问题,教授一时接待不过来,就让
同学坐在门口排队等候,一个一个来解决他们的问题。这是地地道道的“教授门
诊”!


此外,世界一流大学的管理者都懂得,只有从事大生产的技术工人才可以集体培训,
批量生产,而高端人才是需反复打磨、精雕细刻才能产生的。因此,他们都实行小
班教育。我统计了几所著名大学的师生比例,包括哈佛大学、斯坦福大学、普林斯
顿大学、麻省理工学院、加州理工学院,它们的教师与本科生之比,一般在1:5上
下,加上研究生,老师和学生之比不超过1:10。

小而精办学最成功的典范是加州理工学院,本科生和研究生加起来也就2000人刚出
头,学校的教师大概300人,然而它的成绩斐然,培养出了许多著名的学者,其毕业
生中就有20人获得诺贝尔奖,按照比例甚至超过了哈佛大学。中国杰出的科学家钱
学森先生就是在该校获得博士学位的。


从世界一流大学的宣传策略也可以看出他们的教育理念。斯坦福大学在宣传自己的
一则广告中,特别说明,它70\%左右的本科班级都在20个学生之下,这意味着学生有
更多的机会接触老师,教学质量更能得到保证。




\subsection{E、拥有国际视野,及早接触前沿科学问题}
\label{sec:orgb73374d}

世界一流大学都很注重培养学生的国际视野,让他们及早了解科学的最前沿问题。

首先,让自己学校的世界一流大师站在教学第一线,让刚入校的学生就能零距离接
触世界级大师,消除神秘感,这有利于增强学生的信心,树立远大的志向。


其次,鼓励学生踊跃参加学校组织高水平的国际会议。2010年我在斯坦福大学访学
期间,他们的化学系举办一年一度的学术会议,报告者的资格是诺贝尔奖获得者,
作报告的人数只有10位左右,听的人很多。这些与会者除了报告自己的最新研究成
果外,还与大家讨论哪些是本领域的最前沿问题。如果一个人能在读书期间就能接
触这些杰出学者,开始思考本学科的最前沿问题,肯定有利于日后做出革命性的成
就。


再次,频繁邀请其他知名大学的最杰出的学者来作学术演讲。我在那里访学期间,
就听过数学系和物理学系的系列讲座,被邀请来的嘉宾不少都是菲尔兹奖和诺贝尔
奖获得者。


此外,美国大学非常注重学生眼光和胸怀的培养。著名大学的开学典礼和毕业典
礼的校长讲话很体现他们的教学理念。在开学或者毕业典礼这种场合,这些大学
校长总是把当今世界最有挑战性的难题拿出来让学生思考,比如气候变暖问题、
能源危机问题,如此等等,鼓励学生要有勇气去迎接挑战。


在这种讲话中,校长们常谈到一个话题,就是培养学生的自由精神、冒险勇气、国
际眼光以及智慧开发等。不难理解,在这种视野下培养出的学生更容易成为世界级
的大师。


为了培养学生的国际视野和博大的胸襟,斯坦福大学还有各种各样的基金,每年可
以 资助学生到海外考察访问。我认识一位本科生,她说在斯坦福大学的四年里,每
年暑假她都能申请到基金到国外考察,所以她到过很多国家。这位学生曾去过日本
学日语,也来过中国学汉语,两种语言说得都很流利。这样的教育方式培养出来的
学生,见识就会不一般。




\subsection{F、五花八门的学术团体和读书会}
\label{sec:org2666bc7}

课堂只是大学学习的一部分,世界一流大学里五花八门的学术团体组织的各种活动,
也成为学习的重要组成部分。斯坦福大学的每个系科的学术团体,几乎天天都组织
学术讲座,像计算机系,从中午12点到下午6点,每个时间段都安排有学术讲座。
这些学术活动又分等级,有的是针对同方向的少数专家的,有的是针对本系所有师
生的,有的则是面向全校的乃至对社区大众公开的。


针对大众的讲座,即使数学、物理、生物这些高深的学科,一般人也能听得懂。我
的经验是,不管听什么讲座,或多或少都会有所得。


在2010年访学期间,我参加了一个叫“复杂系统研究组”的学术团体。这个学术团
体每次活动有五六十人参加,既有资深的教授,也有本科生、研究生。参加者的系
科背景什么都有,有来自生物学、化学、物理学、语言学、心理学等系科的,也有
来自文学、历史学这些传统人文学科的。组织者热情四溢,张开双臂,欢迎每一个
新来者,不论你来自哪个学科,也不管你来自哪个国度。每一个新来者,组织者都
让你留下电邮,之后的所有活动都会通知你。这个研究小组开始就是几个人,后来
就像滚雪球那样,越滚越大,现在已经成了上百人的大学术团体。


大学里这种学术团体完全是自发的,完全是出于兴趣,没有任何学校的领导指使、
分配,没有功利可图,顶多向学校申请一点儿活动经费,买些开会时用的点心和饮
料,或者支付外校专家的交通费。组织者投入大量的时间和精力,往往没有任何经
济回报。这种自觉自愿的献身精神,在科学探索的道路上是不可少的。这些团体的
发起者既有老师,亦有学生。我们绝不要小觑这样的活动,它们很可能就是某个重
大科学发现的契机。


还有一帮华裔子弟组织一个《论语》学习小组,有一二十个人,他们看不懂中文原
文,就学习英语译本,每星期三晚上聚会交流自己的学习心得。他们听说我写过一
本《孔子和他们的弟子》的书,就请我去跟他们座谈了一次。


给我留下印象深刻的一件事,是很多系科的学生每个星期五下午都有一个一小时的
“美好时光”同学聚会,交流自己这个星期阅读的心得。由于每个人阅读的文献不
同,理解角度各异,说出的心得各式各样,这种活动既开阔眼界,又激发灵感,大
大提高学习效率。




\subsection{G、足够大的生活和学习空间}
\label{sec:org61434fe}

脑力劳动需要一个足够大的生活和学习空间。拥挤的环境、嘈杂的气氛都很影响脑
力工作,所以世界一流大学都特别注重给学生提供足够的学习和生活空间。


住宿条件很重要,它既是学生休息的地方,也是学生学习的场所。斯坦福大学本科
生是两个人一个房间,研究生都是一个人一个房间,有家属的研究生还是一家一套
小洋房。我在斯坦福读博士时,是带着家属的,分配给我的学生宿舍楼上有两个房
间,楼下是一个客厅和一个厨房,还有阳台和后花园。这样的住房条件就可保证同
学之间互不干扰,有家属的学生也能专心学习。我也去过圣地亚哥加州大学、圣巴
巴拉加州大学,都具备跟斯坦福大学一样的居住条件。


美国像样一点儿的大学都给研究生提供办公室。1993年,我到圣地亚哥加州大学读
书,那里的研究生是两人一间办公室。斯坦福大学的计算机系是最出人才的地方,
培养了大批IT行业的精英,这与他们的学习条件分不开的。我到过他们的系,每个
博士生都有一间办公室。连我这个临时去访问一年的访问学者,斯坦福大学也给提
供一间小小的办公室,这大大提高了我的学习工作效率。我的《为什么中国出不了
大师》一书就是在这间办公室写成的。


在学习上,有一个容易被人忽略的因素,那就是空间。这包括休息的空间、讨论的
空间、吃饭的空间、散步休闲的空间,只有具备了这些空间,才能保证学生的思想
空间。




\subsection{H在大学读书是一个探险的旅程,不是逛公园}
\label{sec:org9c1b88e}

我在不少大学讲学,学生常问我“能不能给他们一个忠告”?我的回答是:“把大
学学习看作一个探险的旅程,千万不要把它当作逛公园。”


斯坦福大学的商学院在美国也是数一数二的,培养出了许多大企业家。老院长罗伯
特·琼斯教授给经管学院的学生作了一次报告,其中一个忠告就是“不要停留在令你
舒服的环境中时间太久”。的确,一个人要成就一番事业,就要有点儿跟自己过不
去的精神,敢于挑战自己。


有一个统计,诺贝尔奖获得者中,绝大多数的人在大学学习成绩都是B。牛顿和爱因
斯坦甚至被老师认为是问题学生,他们在老师眼里并不是好学生。而那些成绩都是A
者,后来干什么很少有人知道。这个是很正常的。那些敢于挑战自己的学生,容易
被看成离经叛道,一般不会在成绩上表现自己,而科学真理的发现正是青睐这种离
经叛道者。


在2010年耶鲁大学的开学典礼上,理查德·莱文校长这样告诫新生:“耶鲁大学开设
有2000多门课供你选择,但是你不得不错过98\%的课程。但是我要督促你们多尝试不
同的课程。每一个学科代表着人类的不同经验,任何一个学科都能够给你提供不同
的窗口,去领略自然界和社会的文化积累,让你能够从不同角度看世界。如果让我
给你们一个忠告选课的话,兴趣尽量广泛,尽可能多涉猎各种学科。不要老抱着这
样的信念,你来大学之前选定的学科是最适合你的。选一些完全超越你以前知识经
验的课。这样不仅可以扩大你的知识面,还可以发现你意想不到的巨大潜力,这甚
至可以改变你的人生。”
接待讯方公司

\subsection{{\bfseries\sffamily DONE} 一位学霸的学习感悟}
\label{sec:org94562c9}
\url{http://blog.sina.com.cn/s/blog\_92ac95290102v3qm.html?tj=edu}

一位985高校在读博士,回首十几年的求学路以及自身和同学的经历,颇多感悟。

1、小学时代如果能写一手工整的字,具有准确的数学运算能力,OK,完美了。对
以后的学业生涯够用了,所以尽量给孩子五彩缤纷的童年吧。


2、小学和中学这十二年的学习内容,都是几百年甚至几千年以前(阿基米德啊、
牛顿啊、笛卡尔)人类创造以前的东西,思辨性不高,真的不难。


3、如果想要拿诺贝尔奖或者当选两院院士,这个要看天赋和智商,但是学那些几
百年以前的东西,考个好的大学,基本和智商无关。和什么有关?情商!


4、学习不好的同学,基本都是严重拖延症患者,今天的事能拖到下个学期。

5、勤奋永远是真理吗?!教育学理论里面有个“有效时间”的概念,看你的心用
在学习上面的时间是多少。所以看到班上很多拼命学的学不好,玩的反而学的好的,
不要惊讶。


6、总是期待天才,我就读的都算是不错的高中大学,读书读到现在都没有看到无
师自通的天才。同学的差距是有的,差距在哪里?接受能力和专注程度,这些都是
情商的范畴。


7、时代发展的当今,似乎城市里面的孩子更容易在学习方面出人才,我大学的同
学只有不到三分之一的农村孩子。


8、但是农村的孩子要么不读,读就会读的非常好。“寒门出才子”是真理!高中、
大学里面学习拔尖的一般家境都不好。我就是地道的农家子弟(是不是自恋了,原
谅我哈)。


9、女孩子小学一般成绩都不错,到了初中就不是太好了。这个是什么原因?!教
育学给出的是生物学解释,Thelawofthenature.


10、有鉴于第9条,女孩子上了初中一定要对数学引起高度重视。哎,多少曾经优
秀的女同学最后学业毁在数学上。。。。呜呜。


11、对于成绩不好的同学,家长总是期待出现奇迹,成绩突然“冒起来了”。什么
样的同学容易冒起来?似乎一般是男生并且一般很调皮。


12、初二真的很关键。基本是分水岭,所以要重视初二的学习啊。。。

13、初中时形成思想观和价值观的时候,这个时候有人带坏就带坏了。与其多花时
间给孩子找家教找辅导班,不如多注意孩子身边的玩伴和朋友。有句土话叫做“人
搀不走,鬼搀飞奔!”多关注孩子的生活吧!


14、家长如果发现孩子突然学业成绩掉得厉害,请参照13条。

15、“网络是是把双刃剑,有好有坏”。我的看法是:千万不要任由孩子沉溺网络!
沉溺于网络,学业必然受影响!不过,现在的孩子,已经视网络为我们时代的电视,
父辈时代的收音机,网络已经常态化、工具化,节制是关键。


16、孩子没考上理想的高中,该不该交择校费让他继续读?!这个要分类讨论。如
果的确是自然灾害,孩子临场没发挥好,那砸锅卖铁都要交。如果是确实已经不想
学了,那就不用拿血汗钱养活一帮教书先生吧,此类择校生考上好大学的是特例,
自己申请退学或是被开除的到是很多。。。。


17、刚才说“砸锅卖铁都要交”是不是过分了?!我觉得,小学、初中在哪个学校
读不重要。高中真的太重要了。。。。为什么?因为高考是选拔性考试,其他的不
是。


18、英语怎么学好?我的经验是多听,就像我们从小听方言所以就会说一样。英语也是一门语言,不要过分强调语法啊、结构啊、我认识的母语是英语的外国友人没人搞得懂自己的语法(我们中国人又有多少知道汉语里面的主谓宾定状补?!),请不要妖魔化英语。

19、我真的要强烈建议多听英语,每晚睡觉前听半个小时。长期以往,英语保证不会差的.

20、插播一段自己的小故事:记得我小时候是个调皮大王,初一的时候英语总是不
及格,后来家里发生变故,母亲重病一下子家境窘迫一贫如洗,十四岁的时候母亲
去世对我打击很大,自己开始知道好丑了。恶补英语,每晚都在听,我喝学校免费
的照见影子的稀饭,总是把早上买包子的钱拿去买电池(话说那个时候电池质量不
好),看着别人吃包子,肚子真的好饿啊……后来我中考英语满分,高中英语接近
满分,大学获得了全国大学生英语竞赛一等奖。非英语专业第一人哈。由于专业排
名第一,我跳过硕士,被直接保送博士了。


很多人问我有没有学习英语的窍门?我只是给两个字:多听(话说我现在还在每晚
听英语).


21、孩子要不要住不住宿呢?!我的建议是还是住宿好(批注:这个要因人而异,
损友一堆,也许会带坏),集体的氛围有利于孩子更加培养团队意识,知道怎样和
其他人交往。这个社会,死读书的人不招待见的。但是要关心孩子的成长,请参见
13条。


22、如果读书好算是成才的话,那么成才的人毕竟是少数,不要太苛责孩子。只要
努力的都是好孩子,但是一定要成人,人品不管什么时候,都是最重要的。


23、应试教育的体制下,你不会还真的相信“素质教育”吧?!小学时代可以过的
五彩缤纷,但是上了中学……还是现实点吧。


24、我和我学生的一段对话:

“我对数学这门科目没兴趣……呜呜。”

“孩子你二十岁了,都高三了,不到一年就高考了,你和我说你对数学没兴趣,我
给你讲讲科学家小时候的故事?!再给你培养兴趣?!”


25、我都读博士了,至今不知道兴趣为何物。而那些天生对数理化感兴趣,而对玩
游戏打篮球不感兴趣的名人典故,从人性角度出发,我更愿意相信只不过是讹传。
在应试教育的体制下,我的解释是“不讨厌就是兴趣”。所以咯,其实考大学就是
看谁对自己更狠……你信不信?!


26、早恋都是坏事?!我不觉得,在我身边就有一对高中同学,一起努力互相勉励
互相加油走过苦难的高三,最后都考上了很好的大学(我怎么可能会告诉你那个男
主角就是我!)。


27、我在提倡早恋?!错了。我还没说完,就我身边的例子而言,这种单纯美好励
志向上的恋爱发生的概率基本和你买彩票中了五百万一样。所以,还是花花肠子收
起来,好好读书吧。但是还是有许多人会买彩票……爱情来了真是挡也挡不住啊。


28、数学怎么学?!我数学一直都不错,我觉得这是一门技巧性的学科,小学要求
要运算准确就行,中学主要由四个思想方法:数形结合、分类讨论、函数思想、划
归与转化。


29、那个四个思想方法,是我上大学在一个月黑风高之夜,总结高中学业和高考经
历悟出来的(夸张啦?!哈哈哈),可惜已经迟了。我曾经断言,只要把这四种方
法掌握了的,永远不要为数学而担心。话说我本科时期辅导过一个高中学生三年,
本来数学基础一般,最后这么一灌输,考上了中科大!擦,比我考得还好。


30、如果英语的秘籍是多听,那么数学就是整理错题。

31、太重要了!整理错题。我最辉煌的战绩,是辅导一个初二的女生,一开始期末
数学只有36分,最后期末考试考了115分!!那位家长期末奖励了我五千大
洋。。。。。我是怎么做到的?我勒令她订正所有遇到的每一个错题,最后考试时
候就基本没有错题可订正了……


32、为什么订正错题这么重要?!因为其实初高中数学所有的题型就那么多,把盲
点都找出来就无敌了。。。。这是为什么。请参见第2条。


33、你以为你数学只能考一般,你只是学会了一般的知识?!错了,你掌握了一大
部分知识,只是有几个没掌握。就是那个没掌握的,总是做错。而考试,考的都是
综合题,一个知识点没掌握基本就一票否决了、、、、、错题啊,真的是太重要了。


34、话说我初高中数学错题本写了五本,那个我辅导三年的最后考上中国科学技术
大学的高中生写了七本。大概这就是为啥他高考数学考的比我好的原因
吧。。。。。。


35、数学学得不好的同学,一般都很两个明显的缺点:粗心、没有毅力。英语学得
不好的同学有个共性的缺点:懒!


36、我至今没看到学习很刻苦,但是英语差的,如果您遇到过请您给我引见一下,
增加我的阅历。学习认真但是数学差的有几个,主要还是不得要领,做的是无用功。
参见第6条。


37、英语的学习,有很多方法,但是多听无疑是最快最有效的。这个要坚持,而且
真的要坚持。每次想到自己切身学英语的经过,再看看一些教育砖家们总是喜欢把
学英语上升到形而上学或者上升到方法论的境界,真是玄之又玄。我,只能莞尔一
笑。


38、英语的提高,真的是个长久的过程,提高的速度较慢,但是考试的稳定性能好。
尤其是高中,发现没有,班上拿英语第一的总是她或者他?!


39、你问我英语听什么好?我的回答是:小学随便听听,不要太有目的性,培养兴
趣为主;初中听课文,高中听历年高考题。


40、还是有人想和我聊聊兴趣。小学的时候,强烈还是不要太压迫孩子了,真的。
著名的“起跑线”理论不知道扼杀了多少美好的童年,童年刚至心先老,长使英雄
泪满襟啊……在我大学同学中,不乏从小家境很好的同学,我看到的真相是:凡是
小时候的家人强迫学(钢琴、小提琴、六弦琴等等各种琴)的几乎没有坚持下去,
把其当成事业的。而他们回忆起来的时候,只剩下一段灰暗压迫的岁月,这就是艺
术的熏陶?!真的有这种必要么?


41、初中强烈要听课文啊!!初中是义务教育阶段,要知道,中考是必须保证很高
的及格率的!!所以,很多中考试卷的真题都是来自课文原句的改写或者同一题材
的改写,主要考固定搭配、语法(主要是从句)、一词多义、习惯表达,因为知识
点就那么多。如果哪一次出卷老师出的都是课本以外的知识点,这先生必火无疑,
他没法向全市人民交代。。。。。


42、当年明月在,曾照彩云归啊。记得那时我把自己的早饭钱全部奉献给了社会主
义现代化电池厂事业,每晚都听课文,以致最后课文每篇都能背上,后来你知道了,
不自吹自擂了……


43、我按照这样的英语学习方法,勒令我的学生背课文,真的要求严格啊,一开始
他说自己很痛苦,各种尥蹶子,我还是能HOLD住他的,他初三的课文现在背的滚瓜
烂熟。教了不到一年,现在已经稳定在115+,毫无压力(过几天他就参加中考了,
我在重庆教的最后一个学生娃,祝他好运!)。


44、其实中考英语拿高分真的很简单:如果哪位参加中考的同学,能把初二上学期
到初三下学期的所有英语课文背的滚瓜烂熟,也每天都在听英语,正常发挥的状况
下中考竟然没有考到110+,我绝对要振臂一呼,号召不明真相的群众们去教育局与
英语出卷老师当面交涉。不肖生立此贴为据。


45、高中其实和初中区别很大的,因为即使在大学如此扩招的今天,重点大学录取
率依然不足百分之十。初高中最大的差距是:初中只要能把课本看明白就能考得不
错,高中即使把书本看烂可能也只是及格分。。。。


46、高中切记要多听听高考历年真题,太重要了,能背诵更好。高考很坑爹的,真
的,选拔性考试压力太大。,我高三后期看到每一个选择题我都能说出这是哪一年
哪一个省份的。高中想考名牌大学的(全国前二十是名牌。全国前五十是重点。全
国前一百是知名),一定要认真研究高考真题,你会发现一些规律性的东西,江苏
高考每年出题目的几乎都是那帮名师们,能没规律吗?!但是不要到了大学才后知
后觉。很神奇的,你信不信?!


47、该说数学了,数学真的是个大问题。做家教时我宁愿那个孩子语文英语物理化
学地理政治生物历史乃至音乐体育美术都不好但是唯独数学好,也不希望孩子数学
很不好其它都好的。因为,经验表明,数学的提高似乎我要花更多的功夫,尤其是
把数学提高到一个稳定发挥的水平,实非一日之功。


48、小学数学没啥说的,家长多管管吧,我甚至认为小学生根本没必要找家教找辅
导班。我以前开家教班,从来不招小学生,也没教过小学生家教。不想扯淡,误人子弟,参
见第1条。

49、那些总是忙着做生意、忙事业不顾小孩子的家长们,总让我联想到小时候玩坦
克战,自己在外面杀敌,老家即将被人轰了还蒙在鼓里,继续开心的驰骋疆场。不
知道他们奋斗是为的啥?!近年来留守儿童屡次出现各种端倪,我只能说,社会有
时候真无奈。


50、如果硬是要我给一个小学数学的建议的话:学学奥数吧,对于开拓思维确实会
起到作用。没有坏处的,而且要是一不留神,发现了自己孩子原来是个难得一见的
数学神童,直接保送大学自不在话下,要是给国家争光了,岂不是举家幸甚、万民
幸甚。。。。


51、初中几何基本上可以作为中国教育的缩影,玩文字游戏和脑经急转弯更多余实
用性。我这学期还有过一次,做一道初三几何证明题没做出来的(好没面子啊),
那个题目是圆和相似性的结合,不容易想到辅助线的做法。属于脑经急转弯一类的,
奇技淫巧,不足道也。


52、高中是不学几何(空间几何除外)的,不会遇到初中那种坑爹的数学证明题,
方法性和方向性比初中突出很多,不会让你因为没想到辅助线的做法,就被判0分。
感谢党,感谢政府,感谢YCTV,感谢JSTV,将来还有可能感谢CCTV,对寒窗苦读苦
求功名的学子们厚爱与仁慈!!


53、即使那种坑爹型证明题也只有五分(一般是试卷倒数第三题的第二小问)。初
中没啥特别难的,二次函数基本算最难得了吧,二次函数屡次被选作中考的压轴题,
很锻炼思维,这个一定要多练。尤其是想考名牌高中正取生的同学
们。。。。。。、、、


54、语文这是个尴尬的学科,尤其是上了中学。先插播一段坊间流传的一则轶事:
话说北京某年中考选用了巴金先生的一篇文章作为现代文阅读材料,有好事者就把
这篇现代文阅读让巴金先生做(巴金先生05年去世,缅怀一下),最后这篇现代文
阅读满分二十分的情况下,巴金只得了七分,最难理解的是,有一个题目问“此处
作者这样说的深层含义是什么?!”,巴金作为文章的作者,给出的答案与标准答
案竟然相去甚远,被判不得分。。。。。


55、语文基本只要把字写好了,基础知识(拼音、名句、成语、病句)掌握了,就成
功一大半了。语文的改卷主观性很大,不要花太多时间,这个意义不大,实践表明这
个科目只要认真发挥不怎么拉分的。


56、我在担心上面一段话,会不会引起语文老师们集体对我口诛笔伐。我在让大家
忽视语文的学习?!我们可能不会要求早餐店的师傅给你做个周长为十六内切角为
八十度的正七边形黄桥饼,我们可能一辈子也不会达到和外国人交流无碍的境界
(要那些学英语翻译专业的干啥?!),但是我们每天都在写汉字,说汉语。其实,
语文是人生中最重要的一门学科,一个文采斐然、出口成章的人,走到哪里都会受
欢迎。要怪的是这种工业化大生产的、急于求成的教育体制。


57、腹有诗书气自华。多读课外书吧,天文地理历史人文都要读一些。书籍会给你
打开一翻新的世界,中国古典文学让人陶冶情操,心清气静。我发现理科学习特别
好的都喜欢读课外书,而且知识全面,充满灵气,我觉得两者不无关系。


58、如果培养孩子的兴趣爱好,我弱弱的建议让孩子学一些中国风的吧,民族的才
是世界的。不要一窝蜂的学钢琴、小提琴、舞蹈,学学书法、戏剧、国画吧。尤其
建议写书法,学习学的心气浮躁,写上一段名人字帖会让人清静,我一直热衷于临
摹赵孟頫,但是仍不得要领惭愧惭愧,淮剧也能唱上《谈寒窑》《珍珠塔》《买油
条》那么经典的几段,艺术细菌作祟啊。


59、我从来都不觉得初中的物理、化学属于理科的范围,尤其是初中化学。因为题
型之固定、题材之简单、思维方式之单一、计算之粗略让人不忍心把他们归类到理
科的范围。


60、如果哪位孩子初三化学没学好,和英语一样,基本就是一个字:懒。

61、读书从来就是一件很单纯的一个人的事情,和你的家境、出身、人品、父母关
系不大。我上了大学,慢慢接触社会,我常想如果所有事情都能像高考那么单纯简
单就好了。


62、读书真的和有没有钱关系不大,这一段是给家境不好的孩子看的,衣食无忧的
孩子们可以绕过去,直接下一条。从初中到大学,我在班上一直家境是最差的,母
亲早逝,父亲在外地打杂工。我是爷爷奶奶把我培养大的,和他们打电话他们常常
戏言,我就像他们的小儿子一样。我大学学费自己贷的国家助学贷款,我这人重脸
面,从不接受助学金或者开口向同学借钱,大一最辛苦的时候一天只吃两个馒头就
白开水(现在想来真的好傻啊!!!)。后来可能家教做的还算用心细致有效果吧,
酬劳都给的不低,“出场费”也水涨船高。加上每年的国家奖学金(八千大洋),
生活费自己足够花,有时还会给家里寄一点。现在我有幸拿到了全额奖学金攻读硕
博连读,我想可能我以后再也不用为没钱读书而担忧了,但我很感谢这段日子,以
及那些曾经在我最困难的时候帮助过我的人,我这辈子都不会忘记的。和我以前的
同学相比,虽然现在我没房没车没家庭,但是我真的很知足。我喜欢单纯的读书学
习写字,我把自己的兴趣当作自己的事业,夫复何求?!


而这一切,都是自己一个人努力奋斗起来的,我已经没啥遗憾了。任何时候对明天
都要充满美好的希望,乐观点,豁达点。天下寒门学子勉乎哉!!


63、英语听什么,这个是问的最多的一个问题。这个第二篇里面已经阐述了:小学
随意听什么、初中听教材课文、高中听历年高考题。


64、不要让孩子太早接触网络,这个真的不是什么好东西。

65、有人发消息问我“我孩子严重偏科,其他科目能考130几,但是英语只能考50
几,马上中考了,怎样让他中考成绩不偏科呢?!”我开玩笑的回答道:“让他其
他科目都考50几就可以不偏科了啊”。我想说的是:这是一个长期的过程,应该早
点关注,采取有效地学习方法,把偏科扼杀在摇篮之中。


66、我敢说:如果您的孩子能够每天做个学习计划,每天列出需要完成的任务,睡
觉前逐一打勾,他的学习效率会快两倍,也会更加喜欢学习。你信不信?!


67、实践表明,调皮自负的孩子比沉默自卑的孩子要更好带些,成绩提高的更快。

68、关于报考志愿:除非家庭经济情况已经到达财团的境界、或者关系已经强硬到
出路已经内定了、或者对某一门专业抱着非此专业不读大学的决心,我不是很建议
男生读文学、英语、哲学、政治、环境、生物工程等“形而上”的专业。


69、我不否认三百六十行行行出状元。但是我更不否认在社会分配如此不公、就业
压力如此大的今天,男生入错行,怎么能说是悲剧呢?!那简直是惨剧啊!我在的
本科大学是一所综合型大学,这几乎是所有毕业生的共识。


70、大型考试比如高考、中考、研究生入学考试,那种感觉和平时考试是完全不一
样的,必须承认临场发挥的重要性,你平时一定要刻意训练这一点。怎么样算是训
练合格了呢?!平时就是中考,中考就像平时。


71、粗心只是你做的还不够,熟练程度还没达到!我从来不知道什么是粗心,我也
不觉得世界上有粗心这回事。“粗心”二字,不知道多少次被当做借口掩盖了事实
的真相,害死了多少英雄好汉。你以为你会做了,其实你还没有那么熟练很容易
“粗心”,这在大型考试里面会害死人的!!!


72、我和我学生的对话:学生说“老师,我本来可以考140+的,因为粗心最后只考
了120+”我问“1加1等于几?”他回答道“等于2。”我说:“题目只有会做和不
会做两种,不得分就是不会,这个在大型考试里面从来都是这样。为什么你1加1等
于2没有粗心?!所以你要做的是把你的熟练程度和对知识点的认识再提高一个境
界,这些题目对于你来说都是1加1等于2的问题你就无敌了,把这些错题认真的誊
写到错题本上,这个就是你最宝贵的财富。”话说这孩子后来数学考试只要会的就
是对的,再也没有因为“粗心”丢过分。


73、你现在知道为什么你的错题本总是那么几张纸就没有再订正了吧?“哎呀,这
个题目其实我是会做的,只是粗心了,计算出了点问题,下次注意就行了。没必要
誊写到错题本上吧”。那我只能起到文殊菩萨保佑你中考高考不粗心吧!


74、坊间似乎总是把考试临场发挥的作用扩大化,常见的夸张版本就有:“额,那
个细小的额,本来能考上清华大学的,高考肚子泄头昏,最后没考上清华大学,考
上清华厨师培训职业学校学厨师了”。“额,我宝宝一个同学咯,本身成绩不如我
家宝宝的,高考我家宝宝没发挥好。最后他考上了南京大学,我家宝宝考了南京职
业大学了”。


75、经济学里面有个理论叫做“价格围绕价值上下波动”,具体例子就是一盒火柴
再怎么涨价,也不可能比一辆汽车贵,因为固有劳动时间不一样。这个可以用来说
明这个大型考试发挥的问题。我读书读到现在,没看到过黑的发亮的马或者白的刺
眼的马,小黑马或者小白马很常见。


76、高考、中考实际上是对你学习、心理素质、抗压能力、协调能力等综合能力的
考察,不光考察学习,我想这也是高考中考存在的合理性之一吧。


77、那些抱怨自己因为考试当天过度紧张、腹泻、失眠、头疼、失恋、遭人暗算等
总总原因没发挥好而与理想学校擦肩而过的同学,如果有这种认识,是不是那些念
兹在兹的纠结和遗憾释怀了很多?!


78、中考、高考几乎是我所有已知的国内考试中最公正公平的了,尤其是对于农村
的孩子,是一个很好的也几乎是唯一的改变命运的机会。所以我看到很多农村的、
家境一般的孩子也跟着叫嚣取消高考,改变现行人才选拔方式的时候,我,表示不
能理解。


79、我有点后悔在以前过度强调英语多听,实际上除了多听以外还有一个大问题:
词汇量。不背单词只听英语的人想学好英语,我不大相信有这种可能。


80、我敢说,高中英语得词汇量者,得天下。我曾经把一个暑假的时间背完了一本
星火英语单词册,我高考的时候,一份高考英语试卷没有我不认识的单词,与看中
文别无二致。我都是这么要求我高三学生背诵课外常用单词,起码要达到大学四级
单词水平,效果很好,一劳永逸,发挥稳定,高三后期英语根本不用操心。每一个
高中生都被要求去背课本单词,但是高考是选拔性考试,你知道为什么只背课本单
词HOLD不住高考了吧?!


81、一份模拟考试卷子,假如你考了一百四十分,你是不是很开心?!我不觉得,
因为实际上你花了两个多小时去考试,又花了很多时间听老师改卷、评奖,而实际
上对你学习进步有意义的只是那个丢掉的十分,知道了这一点,你还会轻视错题
么?!


82、改写一位伟人的名句。‘错题本的步伐应该再迈大一点“,你觉得只有数学有
必要订正错题?!


83、在错题本订正错题的时候,请用不同颜色的笔,注明题目当时为什么写错了,
以及心得体会,不要干巴巴的只是错题而已。


84、学自然科学的都知道,哲学是一切自然科学的本源,数学思想方法其实属于哲
学方法论的范畴。我们国家的中等教育很忽视数学思想方法的讲解与提炼,却经常
考察一些涉及数学思想方法的题目,这个本身就是一对矛盾。数形结合、分类讨论、
函数思想、化归与转化这四种方法,高三我会专门讲一个学期,初中我会讲半个学
期。说实话,效果好的有时候我自己都不敢相信,就是能明显感觉到学生突然开窍
了。


85、是不是经常有一些题目(比如大型考试的压轴题),会出现没有思路,不知道
怎么下手的情况?如果基础过关的情况下,那就是数学思想方法还不到位,就是常
说的“不开窍”。


86、我如果是数学教师,我一定不按教学计划讲课,我会专门讲一个月的思想方法,
磨刀不误砍材工,真的太有用了。


87、对数学史和数学思想方法掌握后,学生能够高屋建瓴的角度看题目,几何和代
数已经没有明确的界限,有的时候看到一些题目,就有思想方法像虫子一样在脑子
里蠢蠢欲动,比如看到数列我就想到用函数的方法去解答,看到一个函数解析式,
我就试图做出它的图像,数形结合看看它的性质。大有裨益啊!


88、小学五六年级其实挺重要的,承上启下,尤其是一些数学应用题,考验你的抽
象思维,而智商最重要的考核标准就是抽象思维。


89、字一定要写好啊!从小就要写好,这个也是一劳永逸。我所在的大学,每年都
被选作高考中考的阅卷点,同样的卷子,卷面整洁与否判分也迥然不同,这是我有
幸进高考阅卷现场后观察得出的结论。


90、奋斗永远都是一个连续的过程。我的意思是没有包括中考、高考在内的任何一
场考试可以作为学业的一个节点,如果你觉得过完了中考或者过完了高考我就轻松
了,那你真的需要再去成熟一下。路遥《平凡的世界》一句话一直激励着我:把辛
勤的耕作当作生命的必要,即使没有收获的指望依然心平气和的继续耕种。


91、话说我高中那些发挥不好的同学,研究生都考的是清华大学、复旦大学、上海
交大、浙江大学等名校。所以高考很重要,但是也没“一考定终生”那么离谱。


92、学途漫漫,考试多如牛毛,发挥或好或坏,或喜或悲也很常见。即使不能做到
“不以物喜,不以己悲”的境界,也要好好考虑坚持的价值。

\subsection{{\bfseries\sffamily DONE} MBA评论:读书才是这个时代最暴利的赚钱方式}
\label{sec:orgeae8c08}
\url{http://www.sohu.com/a/137995889\_600190?loc=4\&tag\_id=67429}

最近一段时间来,总是有人在说读书无用,今天是世界读书日,笔者作为一个书虫依然要为
读书正名,我们就用经济学的视角来看一下读书到底有没有用?为什么笔者说读书才是最赚
钱的生活方式?让我们一起来探寻读书的真谛吧!

不知道从什么时候开始,社会上流传起了“读书无用论”的论调,这个论调往往以一些段子
为起点,说用功读书的,比不上买房购楼的,努力学习的,比不上初中创业的。之后又有着
大量的文章出来,秉持着说读书不是为了赚钱而存在的,甚至引用马云的话“我读的书基本
上都和赚钱无关”,从而来驳斥所谓的“读书无用论”。

从经济学的角度来说,笔者认为说读书就不是为了赚钱为目的的说法是不现实的,因为经济
基础决定上层建筑,你不可能饿着肚子还说我只要精神追求吧。另一方面,这两种观点说的
有道理吗?笔者觉得还是有道理的,但是这些都是一些比较浅显的看待问题方式,今天是一
年一度的世界读书日,笔者就想从经济学和进化学的角度来和大家讨论一下,读书真的没用
了吗?

\subsection{一、读书是怎么变得没用的?}
\label{sec:orge603932}
一直以来,中国人都秉持着宋真宗赵恒的一句至理名言:这就是书中自有黄金屋,这就是读
书有用论的起源。仿佛读好书了黄金屋、颜如玉、千钟粟都可以直接得到,读书就是等于赚
钱,而秉持着这样价值观的人,现在似乎又碰到了价值观崩塌,这就是辛辛苦苦寒窗苦读出
来,来到这个社会之后又发现空有满腹经纶却没有赚钱的办法,而最有学问的大学教授似乎
也过的苦哈哈的,所以读书无用论又开始盛行。

面对着这两种截然不同的观点,笔者其实有些哭笑不得,我们先来说说,读书到底是怎么变
得无用的?纵观中国历史五千年的发展长河,我们无论是五千年前没有信史时代的尧舜禹,
还是有了历史的夏商周直至清朝末年,在这个五千年的历史长河里面中国社会都是在一个标
准的农耕社会,或者说小农社会中,小农社会的特点就是生产力较不发达,资源流动较慢,
只有大一统的国家机器才能够凝聚出足够的资源,创造出足够的财富,所以在这个社会中社
会的发展依靠的是资源的驱动,这个逻辑是我们理解读书能否赚钱问题的根源。

然后,我们来看读书,在古代什么人能够读书?相信大家都听说过一个读书人的统一称谓
“士子”,这个“士”就是古代标准的一个贵族等级,由于生产力极度不发达,在造纸术和
活字印刷术发明之前,书籍是一个极为昂贵的奢侈品,普通的穷人想要买得起书可谓是难之
又难,再加上当时的社会生产力水平,如果一个农村家族一个人每天只读书不去干农活,十
有八九就会被饿死。所以,读书在相当长的一段时间内是贵族的特权。

之后,但是孔子一直强调有教无类,鼓励普通人学习,所以的确还有部分较为富裕的农民取
得了读书的机会,再加上之后隋朝开辟了全世界最先进的一种以读书为衡量标准的社会阶层
进阶机制:科举制。通过科举考试,即使你是贫苦的农民,只要你读书读得好,就可以进行
社会阶层的进阶,古人说“朝为田舍郎,暮登天子堂”就是这个道理,通过读书,通过科举
鲤鱼跳龙门直接成为了国家的官员,成为了学而优则仕。

这个时候我们回到了问题的根源,为什么在五千年的中国历史长河中读书可以赚钱呢?这就
和农业社会的资源集中度密切相关了,因为资源匮乏,只有能够集中资源的地方才可能赚钱,
那么什么地方可以集中资源呢?就是政府或者说国家机器了,所以无论你是贵族还是官员,
因为你在古代的条件下,你可以集中资源,所以你才能赚钱,读书的意义是你通过科举的敲
门砖所以你取得了资源的控制权,所以才会有书中自有黄金屋的观点,大名鼎鼎的《儒林外
史》里面范进中举就是这样的一个故事,因为多年苦读的范进终于中举了,所以他取得了资
源分配的权利,瞬间脱贫致富了。

进入了工业文明社会之后,资源的控制权开始从原先的单一主体进行转移了,大量的企业、
资本开始取得了资源的控制权,这个时候学而优的问题就开始变化了,由于资源的控制主体
发生了改变,读书不再是唯一可以控制资源的手段了,所以出现了所谓的脑体倒挂现象,表
现往往就是造导弹的不如卖茶叶蛋,读书开始变得无用了。

此外,在中国的改革开放初期,由于经济发展尚不成熟,只要你胆子大,无论读不读书,都
能够找到资源配置中的机会,从而实现发家致富,于是很多人出现了白手起家的现象,资源
配置主体的改变以及白手起家的案例共同在为大家形成了读书无用的观点。

\subsection{二、读书才是最暴利的赚钱方式}
\label{sec:org165a1f5}
笔者一直承认读书的高贵意义,但是从经济学的角度出发,我们不妨世俗一点,我们就来说
说读书和经济的关系,我们之前说了读书是怎么变得没用的,但是这一切都是发展中的问题。
因为当社会由工业社会进入信息社会之后,赚钱的方式正在发生根本性的变革,原先由于生
产力水平低下只能依靠资源或者资本驱动的赚钱方式正在发生改变了。

我们回顾中国三十年来的经济发展方式,我们在经济发展的本质上依然没能摆脱资源驱动或
者劳动力驱动的根本模式,所以才会有那么多的暴发户出现,有那么多的没有知识却取得了
巨富的人,因为在资源驱动的发展方式下获取资源的途径只要和读书脱轨的话,那么读书就
不再能成为分配财富的标准。

目前,资源驱动的企业或者城市正在一个又一个的面临着发展的难题,世界各国出现了资源
枯竭型的城市,依靠资源驱动的生产方式正在遭到前所未有的挑战,这个时候随着信息技术
的蓬勃发展,随着高精尖科技的发展,资源驱动的发展方式正在向技术与知识驱动的方向转
变,所以我们如果再固守着资源驱动经济的老观念可能就会变得不合时宜了。

我们来看现在的福布斯排行榜,大量的像比尔盖茨、保罗扎克伯格、贝佐斯、埃隆马斯克这
样的科技大佬存在,他们有着统一的特点就是高知识水平,而在中国也是如此无论是马云、
马化腾、李彦宏没有人是没读过书的,像李彦宏甚至是绝对的高学历,因为他们都是依靠科
技驱动企业的发展。

未来,随着互联网的快速发展,一方面,知识传播的成本正在急速下降,原来昂贵的书籍,
变成了每个人都能够买得起的东西,大量的知识甚至在被免费的传播,另一方面,社会对于
知识的要求却在不断提高,越是高科技的企业就越是需要高知识素养的人才。

从长期来看,依靠体力的生产方式正在被逐渐的淘汰,甚至像富士康的机器人生产车间里面,
基本上连你出卖体力都不需要了,因为这种粗笨的体力劳动根本不需要人类去做,这个时候
就要依靠脑力来创造,因为体力的时代已经过去了,体力的增长极限已经出现,但是脑力的
天花板还不知道在哪里,所以脑力劳动的生产效率正在与日俱增,最终将有可能完全超越体
力劳动,所以用读书来提升脑力的生产力水平将会成为未来最暴利的赚钱方式。

那么为什么还是有人觉得读书没用呢?这就是因为你读的是什么书?知识也是具有鲜明的时代
特征,很多人都还在阅读上个时代的书,用着不合时宜的知识那么肯定会觉得读书无用,再
加上我们很多人读书读得都是所谓的教科书,这些知识的积累远远达不到社会需要的地步,
所以当知识积累不足以达到社会要求的时候,那么肯定会觉得读书无用。

因此,读书是这个时代必须的方式,但是如何读书,读什么书才是这个时代想要赚钱必须考
虑的东西,希望大家不要被表面的假象所误导,真正开始为时代需求来读书吧!

\subsection{{\bfseries\sffamily DONE} 请远离低质量的勤奋,那比懒惰更可怕!}
\label{sec:orgdcb446c}
培养选择性忽视的能力,养成和保持低信息食谱的习惯,少做无意义的工作……
相信很多勤奋努力的朋友都有这样的经历:

你分秒必争,坚持把白天到晚上的每段时间都安排的很紧凑,甚至有时候晚上睡觉后半夜醒
来,都要逼着自己读上几十页的书。

这样做长久下去的结果是,工作时间越来越长,休息时间越来越短,人的情绪也会越来越焦
躁。而且只要有十分钟的无作为,你就会变得非常慌张。同时你的社交时间也不得不尽量地
缩短,你甚至不再有时间交朋友。

更可怕得是,你的工作量明明没有变化,可看起来每一天它都在成倍地递增着,你开始害怕
夜幕降临的那一刻,因为那意味着这一天有更多的事情被贴上了“没完成”的标签。

因此,请远离低质量的勤奋,因为那比懒惰更可怕。勤奋努力不是一种看上去很美的姿势,
而是需要看最终所取得的结果。这也是分辨一个人是不是真的勤奋的一个衡量标准。


\subsection{01 你必须远离低质量的勤奋。}
\label{sec:org671d487}
很多人的勤奋,是一种低质量的勤奋,或者说,是在用战术上的勤奋来掩盖战略上的懒惰——
表面上很刻苦,实际上却刻意回避了真正需要解决的问题。

就拿学习来说,最重要的是做对题,要充分理解每一题的思路。如果不经常去分析真正需要
解决的问题,拒绝做出调整,那么只能叫磨洋工或者叫重复劳动。重复劳动最大的特征就是
反复地去做对事情结果没有实质影响的事情。

正如读大学的朋友都有过这样的疑惑,“我每天除了上课,剩下的时间都泡在自习室,为什
么期末考试的时候,成绩还是没有那个每天下课就去玩的室友好呢?”

其实,忙碌和效率是有着天壤之别的。那些整天忙忙碌碌的人——忙完一个活动又忙另一个——
事实上可能是以牺牲有效性为代价而维持忙碌状态的。

他们也许需要重视一些其他的能力:培养选择性忽视的能力,养成和保持低信息食谱的习惯,
少做无意义的工作……

\subsection{02 你的目标应该是富有成效。}
\label{sec:org9321e84}
最笨的努力,就是没有效率的勤奋,你的目标应该是富有成效。

一些不易察觉的小习惯,制约着我们所拥有的时间发挥出最大功效,让每天用14个小时去工
作的我们,一边“勤奋”地推掉朋友的聚会,一边“懒惰”地刷新着无意义的新闻,而结果
就成为“我没有时间做任何事,我甚至没有时间完成我的工作”。

当然,选择在自己最有精力的时候做最重要的事。有效的时间管理不仅仅是寻找一天中的额
外时间,还要将你的有效时间和你的有效精力匹配起来。

比如,夜猫子最好选择把最重要的事情放在晚上来做,晨起人士就在清晨处理重要事情,了
解自己的精力巅峰并进行合理利用,这是用更少的时间达到更好的效果的高效原则之一。

同时,对目标最有帮助的事,就是最有价值的事。不要一味地“为了工作“而“工作”,一
个人工作的目的是为了达成目标,除此之外,要尽可能地“扩大再生活”。

很多人都觉得运动是浪费时间,但研究显示,运动可使一个人的工作效率提升15\%。比如说,
跑步,骑车,瑜伽,普拉提,游泳……每天只需要30-60分钟,就可以使人的管理能力,精
神状态和承压能力都得到显著的提高。

\subsection{03 按计划地去完成你的清单。}
\label{sec:org35815cf}
每一个小时的计划,抵得上五个小时的执行,而一笔一画写出来的计划更令自己有想征服的
欲望。买一个小黑板用来写自己每一天的待办事件,并把它放在办公桌上,按照计划专注地
去完成清单。

你一天看完1本书,100天看完了100本,而你的记忆力理解力远不够支撑你的这种走马观花,
然后问你记得什么,你1本都不记得,你不勤奋。也许你只适合专心读一本书,不仅看完了,
还看懂看透了,而且还能把书中的内容、写法应用到自己的写作上,丰富自己的心灵和精神,
那么你是勤奋的,不是“勤奋的懒惰”。

我们总是“做不完”一件事,也许是因为我们并没有真心实意地“想做完它”。如果想真正
达成目的,那去自习室就不要带手机,处理棘手问题时就不要回复朋友的邮件,熬夜加班时
不要随手开了淘宝,写着购物清单时就不要不自觉地看起了肥皂剧……

很多保持高强度的工作以及高效率的态度的人,靠的就是关掉网络,锁起手机的“强行专注
力”,这也是大多数“勤奋”着的低效率人士所缺乏的能力。

因此,想要真正勤奋努力起来,那么你所做的事情,必须能够让你更加接近你的目标。
\subsection{04 拖垮你的不是勤奋,而是无用功 。}
\label{sec:orgc02eb6b}
有人邀请我在“知乎”上回答一个问题,这个问题非常具有普遍意义,一个简单的回答不足
以展开,因此我决定单独写一篇文章来彻底地分析阐述它。


提问者是一名国际会计专业的大学生,她看到自己平时都在努力上课、认真笔记、好好复习,
平均成绩都在九十分以上,而她的很多同学平时都不好好学习、只搞考前突击。这些人虽然
学习上少了认真,但胜在活得多姿多彩,学也学了,玩也玩了,有的参加了学生会,还考了
雅思,马上就要出国了。

所以提问者很困惑:她自己努力学习,却对未来出国和找工作都没有太大的帮助;别人不努
力学习,却更有可能获得好的人生。为什么自己那么辛苦地泡在自习室,却不能得到更好的
结果?

对这道问题有各种各样的回答,那些答案有的鞭策,有的鸡汤,有的循循善诱,有的冷嘲热讽。
如果让我答,我只想引用雷军的一句话:“永远不要试图用战术上的勤奋,去掩饰你战略上
的懒惰。”

在这里,天天泡自习室可以先暂定为“战术上的勤奋”;泡了四年以后的结果却依旧缺乏明
确的学习目标和动机,那就是战略上的懒惰。

我上本科的时候,周围的好学生很多,其中不乏各个省市的文理科状元。然而在这些尖子生
当中,学习最好的从来都不是一天到晚苦泡自习室的人,而往往是那些把生活安排得多姿多
彩的人——他们智商与情商皆高,往往懂得所谓努力并非流于表面;而那些貌似学习很刻苦的
人,其实效率并不高。曾有过一个学期,我制订了学习计划,强迫自己每晚都去自习室学到
熄灯前半小时,不过我很快就发现,这种泡自习室背笔记所带来的学术收获,远没有在图书
馆里大量阅读带来的多,我就放弃了。泡自习室对我最大的意义,只不过是营造了一个“苦
苦努力”的幻象,换取我对自己心灵的安慰罢了。

回到之前那个提问者,她每天除了上课,就是去泡自习室,一直泡到熄灯才回宿舍,第二天
一睁眼又前往图书馆去自习。

你如果问她:“你这么努力学习是为了什么?”
她会回答是为了考试有个好成绩。
那么好成绩有什么用呢?
是为了能顺利保研。
保研后又怎样呢?
保研后读研究生,就可以有个高一点的学历,然后再有个好成绩,今后好找一份好工作。
可是,如果好好学习的最终目的是为了找到一份好工作,那么为什么不一开始就向着“找好
工作”而努力呢?

如果在大一的时候,就定下阶段目标为“找到好工作”,那么显然她努力的路径就不应该单
单是学习那么简单,还应该配合参与学生活动、参与社会实践、寻找实习的机会、学习如何
制作修饰简历、补充课堂上学不到的工作技能……而提问者却是在用天天泡自习室的勤奋,
来掩饰她多年来不思考的懒惰。

当然,许多人学习的目标并没有那么功利,毕竟对于很多人而言,学习就是为了更好地获取
知识这一纯粹的目的。但怕就怕很多怀着这样纯粹理想的人在受了几年教育之后,会突然质
疑国内高等教育的水准,看不起教授的水平,也看不上同学所做的研究,然后觉得自己多年
的辛苦研究和学习被白白耽误了。

这是一种高级的战略型懒惰,希望利用外在环境的缺失和不足,来解释自己为何努力却达不
到相应的目标。

无论你接受的是僵化保守的体制内教育还是新颖独特的体制外教育,如果你真的勤奋钻研、
静心思考过,那么你从学习中获取的就绝对不仅仅是知识点的堆积,而应该是批判性思考的
能力、理性逻辑分析的能力,以及去伪存真的辨识能力。

即使你真的进入了一个很差的学校,难道要花费整整四年时间才能认识到这点吗?显然不需
要。如果你早就认清了这个现实,为什么不自己做点儿什么来改变学习路径呢?

也许你会摊摊手说:“可是我没有好导师、好教授、好课程、好同学啊!”
那么我只能说:“可是也没有人阻止你去找到好导师、好教授、好课程、好同学啊!”
要知道,我们正身处互联网的世界,从美国到英国再到我们自己国家,所有最顶尖最优秀的
大学,都在竞相向世界敞开免费的高质量在线公开课;国内外优秀的研究机构,从十几年前
就对外部分开放了大量汇聚尖端研究成果的数据库。如果你真的愿意,可以花很少的钱就在
网上学习许多专业技能,还能跟着有经验的人一步步地学做案例,线上线下的图书馆里还有
浩瀚的知识海洋……

在二十年前,可以说“我没有学习的条件啊”,可在这样一个信息流动速度极快、提倡知识
资源共享的社会,一个真正用心努力的人,是绝不会被当前所处的条件禁锢的。

就拿学英语这件小事儿来说,我们身边有许多工作以后坚持学习英语的人,这需要很大的决
心,需要很大的毅力,能坚持下来确实很不容易。

我曾经有个前台同事,每天都利用工作之余坚持背单词,背单词的软件会有打卡功能,我每
天都能看到她在朋友圈内的分享:今天十个、明天十个,后天是周末,也有十个……从不间
断。有次她问我:“怎么才能在短时间内快速提高英文水平?”

我反问她:“你为什么要在短时间内快速提高英文水平呢?”
她回答,因为公司经常有其他国家办公室的人到访,公司内部还有很多美国高管,他们和前
台的沟通虽然都是最基本的,但她希望在这些基本交流中能做得更好。

我说:“我有一个好方法给你,你没有一个全英文的语言环境,不如给自己报一个近期的考
试,考试的压力会逼迫你在听说读写四个方面齐头并进。”

妹子说:“可是我觉得以自己现在的水平根本考不了好成绩。”
我说你的目标不是为了考试成绩,而是给自己一些压力,并循着一些有据可依的轨道进行某
种系统的学习和提升。它虽然不是唯一的提高英语水平的途径,但确实是短期内最有效的方
法。

妹子说:“可是我觉得压力好大,让我再想想。”
回去再想想以后,这个提议就被搁置了。她仍然回去老老实实地背单词,今天十个,明天十
个,后天十个……一年坚持下来,非常不易,单词量涨了三千五百个,却依然不能和外国同
事更好地交流。

类似的例子每天都在我们身边发生着,这些例子带给我们的启示是:
01 .如果不在开始努力之前就设定一个目标,你的努力就很容易陷入“我这么努力有什么
用”的自怨自艾中。

02 .如果你的努力不和结果挂钩,那么你就只能沉浸在“我已经很努力了”的幻想当中,并
错把受苦的体验当成努力的过程。

03 .比不努力更可怕的,是你自以为“已经很努力了”,却“没有任何实质的进展”,导致
你反过头来质疑“应不应该努力”这件事,甚至把问题引向了拷问社会的公平性问题,这对
你的人生其实没有任何意义。

努力的幻象对已经进入社会的人而言,远比还处于校园中的学生们要大得多。因为成年人要
用更加有限的精力和时间,去应付更为复杂熙攘的世界,而努力的幻象,会长时间地消磨我
们的时间、精力,以换取少得可怜的结果。

所以,在努力之前,每个人都应该先问问自己:
“我努力的目标是什么?我现在所付出的努力和我的目标有因果关系吗?为了达成这个目标,
真的需要我埋头学习八小时吗?这八小时里我已经全力以赴了吗?”

人生不应该是充满痛苦疲劳的拉锯战,而应该是有的放矢地走走停停。

作者:念夕

\subsection{{\bfseries\sffamily DONE} 特级教师:没读过这50本书的老师,很难变优秀}
\label{sec:orgceb2305}
\url{http://www.sohu.com/a/139474140\_385655?loc=1\&focus\_pic=0}
小编说
“如何上好一堂课”,“班级管理的十个金点子”\ldots{}\ldots{}.相信这些文章已成为很多老师的
收藏。但看了这些,老师就离优秀不远了?

“没读过50本名著,很难成为优秀教师。”从普通教师到全国著名特级教师,张祖庆笃定地
说,正是阅读成就了今天的自己。

本文,他梳理了多年阅读经验,为老师们列出了50本精品书单。
作者简介:
张祖庆,全国著名特级教师
我以为,语文老师一定要做一个书生教师。
我以自己的成长经历,来谈一谈读书的重要性,同时和大家分享我读书中的一些观点、心得
和做法。

没读过50本文学著作很难成为优秀老师
文学当中也许并没有语文教材的教法,但是文学的功底恰恰是一个语文老师最需要的精神之
“钙”。

钱穆说:“文学即人生,人生即文学,作家与作品融而为一。”读文学,就是读人生。读大
量优秀的文学书,就是在与优秀的人对话。久之,我们的精神生命会臻于新的境界。

现实生活当中,经常我们会遇到触目惊心的各种各样奇葩的事件。幸运的是,我们拥有文学。
在阅读中,我们可以暂时忘记现实生活中的种种不如意。文学,是我们心灵栖居的后花园。

我以为没有读过50本名著,你很难成为一个很优秀的语文老师。
当然,这50本是一个虚拟的数字。你要想成为一个优秀的语文老师,非大量阅读不可。文本
解读能力,对语文老师来说,很重要。

我的阅读史:从教师必读的文学作品谈起
\subsection{1少年时代的消遣性阅读,也能提高语文水平}
\label{sec:orgbbfd86a}
我童年时代的阅读非常贫乏,基本上没什么课外书,孩提时代,最让我着迷的是听书。
语文老师杨大寿,语文课教得一般,但他的音乐课非常有特色,只上15分钟,就把一首歌教
完了。剩下的时间就给我们讲故事:瓦岗寨起义,薛刚反唐、薛仁贵、岳飞、武松、林
冲……就这样,我们在杨老师的音乐课上,听了近百个故事。

后来,我就去乡镇府文化站借连环画。连环画,是我的启蒙读物。初二那会儿,武侠小说进
入了我的阅读视野。我从梁羽生开始,到金庸到古龙,把能找到的武侠小说都找来阅读,读
得昏天黑地。
这个时候的阅读,完全是消遣性的,好玩,有意思,深深着迷。虽说消遣,但也有着巨大作
用,我发现语文水平在渐渐提高。

\subsection{2经典的文学作品,永远值得反复阅读}
\label{sec:orgce67fa2}
我最喜欢读的,甚至至今以为是世界上最棒的长篇小说是《卡拉马佐夫兄弟》。大作家毛姆
认为这是“人类最壮丽的小说”,我信。我翻来覆去读了好多遍,这本书直抵人的灵魂深处。

意大利作家卡尔维诺的好多书我也读过。
他在《为什么读经典》一书中提出十四条关于经典的定义,很经典。他认为经典就是你读第
一遍就好像读了好多遍的书;经典不是我们正在读而是我们正在重读的书;经典就是我们读
了好多遍,仍然像读第一遍的书……确实,经典的作品是值得反复阅读的。

《追忆似水年华》,我大概用了一年的时间深入阅读。普鲁斯特是一个才子型的作家,每一
页都充满着精妙的比喻,他用意识流的方式写自己的生命。

朱生豪翻译的《莎士比亚全集》——《莎士比亚全集》有梁实秋的版本,也有朱生豪的版本,
一般我会推荐大家读朱生豪的版本。

《呼啸山庄》和《霍乱时期的爱情》,我以为是外国文学当中写爱情写得最棒的,当然也有
人说《情人》也写得很好。

狄更斯的作品,我也非常喜欢,幽默而博大精深。
再比如,卡夫卡,后现代主义的代表人物,他的书越读越觉得深刻。尤其是他的《城堡》。
主人公就是走不进城堡,无论怎么努力,都是徒劳。它有着一种寓言的味道。

当然我也会读一些比较流行的,比如说《荆棘鸟》,讲神父的爱情,也感天动地。
《红与黑》也是非常棒的书。再比如,《堂吉诃德》《战争与和平》《约翰克利斯朵夫》这
样的经典。

罗曼.罗兰《翰克利斯朵夫》的翻译非常美,文字具有音乐性。
其他诸如《悲惨世界》《复活》《安娜卡列尼娜》《简爱》都是那个时候读的。还有好多老
师提到的《基督山伯爵》,我也很喜欢读。

光读文学还不行!优秀教师必须得进行专业阅读
这个专业性的阅读既指一些通识的教育名著、心理学。也有语文方面,甚至儿童阅读方面的。
这里和老师们讲一讲我读过的印象比较深刻的书。

\subsection{国外必读}
\label{sec:orga34c258}
14本强烈推荐的国外教育名著
卢梭的《爱弥儿论教育》,它其实是以小说的形式来承载教育理念。有人说,如果把西方的
教育专著都烧掉的话,只剩下三本,《爱弥儿》是其中一本。这本书有一定的挑战性,需要
老师们沉下心来阅读,里面呈现的自然主义教育和夏山教育是一脉相承的。我们可以静下心
来,用两个月到半年的时间把这本书啃下来。

另外,像苏霍姆林斯基的《给教师的建议》,这本书大家耳熟能详,其实当下教育中碰到的
大部分困惑,大抵都可以从这本书中找到解决问题的方法。

还有前苏联的《学校无分数三部曲》(《孩子们,你们好!》《孩子们,你们生活得怎么
样?》《孩子们,祝你们一路平安!》)我也非常喜欢读,故事体,案例非常多,很诗意,
写得尤其动人。

现当代的一些著作,如《教学勇气》,这本书会直抵你的内心,引导你反求诸己,引导我们
反思当下的教育生活。它会让你直面内心的挣扎,让我们在内心伤痛处,寻找教育的突破口。
这是一本值得深入阅读的书。

还有一本《优秀是教出来的》,这是克拉克的一本书,这本书尤其适合年轻老师阅读,里面
有很多管理班级的一些小技巧,小规则。如果你把《优秀是教出来的》这本书和电影《热血
教师》配套起来看的话,你会有更多的收获。这本书适合年轻老师,特别是年轻班主任老师
阅读。

推荐佐藤学的两本著作。一本是《静悄悄的革命》,还有一本是《学习的快乐-走向对话》,
佐藤学的学习共同体的理念现在已经在我们中国落地生根,以福州林莘老师为代表的实践,
把共同体的理念推到了一个新的高度。这两本书值得一线老师深入阅读。

当然,国外的教育名著还有很多。比如《教育的目的》(怀特海)、《民主主义教育》(杜
威)、《儿童心理学》(皮亚杰)、《教育漫话》(洛克)、《爱的教育》(弗洛姆)、
《多元智能新视野》(加德纳)、《童年的消逝》(尼尔.波兹曼》……

\subsection{现代必读}
\label{sec:orgc406101}
中国现代教育两座高山——陶行知和叶圣陶
中国现代教育专著,我认为陶行知和叶圣陶两个人是绕不开的。陶行知的平民教育,叶圣陶
的语文教育的理论,都值得我们深入阅读。

《陶行知教育文集》和《叶圣陶教育文集》,很好读,很深刻。这两个人的著作,对我们今
天的教育依然具有深刻的指导的意义。作为班主任和语文老师,这两座高山是绕不开。

\subsection{当代必读}
\label{sec:orgeb8e053}
18部中国当代教育中的名师名作
当代的教育作品也有一些不错的,向大家推荐两本,一本是郑金洲的《教师如何做研究》,
写得特别接地气,老师如何通过案例研究和行动研究来提升自我,这本书值得大家阅读。

还有一本书是李政涛的《重建教师的精神宇宙》,李政涛老师的文章写得非常棒,很值得大
家深入阅读。

王荣生教授的《语文科课程论基础》是当代语文课程论的抗鼎之作,这本书有一点深度,不
沉下心来看,容易看几眼就扔掉了。

福建师大的潘新和教授的两本厚厚的《语文:表现与存在》也必须在这里提一下,这本书的
学术勇气值得赞赏。他提出了我们要以言语表现为核心来重新建构我们的语文课程,他的学
术勇气影响着许多人,包括王崧舟、管建刚等。

还有两本可以关注一下。一本是《语文教育哲学导论》。哲学是一切学科的基础,这本书从
我们言语的本能,即人为什么学语文的高度去阐述语文教育的哲学。读了这本书,我们会站
在更高的层面去审视我们当下的语文教育。

杭师大的叶黎明博士写的《写作教学内容新论》,这本书对写作教学的内容作了一些重新梳
理,逻辑性非常强,是我看过的博士写作当中表达最清晰的一本。这本书推荐给研究写作的
老师。

荣维东教授的《交际语境写作》,提出了全新的写作教学观点,这本书有些难读,但我个人
预计,将会影响今后一个时期的写作教学。

李海林老师的《言语教学论》,站在人的言语生命的角度去研究语文教学,李老师写文章的
逻辑性特别强,大家可以看看他是怎样层层深入地论述他的观点。

王尚文先生的《语感论》,王先生最近有一系列作品,大家都可以找来读。《语感论》是他
的代表作,影响了很多人。

推荐吴忠豪教授的《小学语文教学内容指要》(三本)。吴忠豪教授最近一直在讲本体性教
学,他的这几本书从不同的文体角度,构建教学内容体系。

还有我最钦佩的支玉恒老师,支老师的教学艺术可以说是达到了出神入化的地步。
施茂枝老师对支玉恒老师进行深度的研究,他写的《语文教学艺术研究》是我看到的研究支
玉恒老师的书中最好的一本,没有之一。

正如周一贯老师说的,“支玉恒老师的教学艺术是尚未被破译的黑匣子”,值得我们深入关
注。这是我隆重向大家推荐的一本书。

另外,教育随笔写得好的有很多,像吴非老师的教育随笔写得很好,《不跪着教书》等。
在这里我隆重推于永正老师的《教海漫记》,这本书被誉为中国的苏霍姆林斯基的类似的著
作,于老师的文字非常轻浅、简洁,但是非常有味。这本书,可以说是我多年的案头教育圣
经。

\subsection{教师实际工作中有用的心理学书籍推荐}
\label{sec:org5c0d065}
老师也要读点心理学,尤其是我们中国的语文老师,仿佛是天然的班主任,所以我们得让自
己有一个健康的心理,同时还要掌握一些开发学生精神宇宙方面的书。

推荐给老师们:《少有人走的路》,关于自我认识方面的书,第二本是《顺应心理,孩子更
合作》,第三本是《养育男孩》。这本书值得家里有男孩的教师家长关注,也值得老师关注。

当下,男生女性化是一个非常严重的问题。尤其是我们中国,女老师成了教师队伍的百分之
八九十,女老师在教育男孩的过程中是有先天缺失的。所以,我们可以找来这本书看一看。

李子勋的《陪孩子长大》,讲一个父亲和一个母亲怎样陪着孩子长大。
好书是读不完的,我这里举的也只是沧海一粟。当然,任何理论的建构都不可能是完美无缺
的,他人的观点也需要我们要辨证地去思考,哪些观点是可以接受的,哪些观点是可以商榷
的,这样我们的脑袋才不会成为别人的跑马场。

\subsection{教给教师阅读的5大建议}
\label{sec:orgdc98c2c}
\subsubsection{1多读经典,少读烂书}
\label{sec:orgd6b818d}
叔本华曾说:读好书的前提,是不读坏书。大学问家金克木也曾经说过:书读完了。他的意
思是书是读不完的。如果我们读书中之书,书是读得完的。

\subsubsection{2敢于闯进阅读伸展区}
\label{sec:orgab79bb2}
阅读需要把自己引向一个伸展区。伸展区是有一定的挑战意义的。但是也有一些书是恐惧书,
这些书唯一的好处是拿起来看着看着就睡着了。

我个人认为,老师不要害怕深度阅读,那是你通往未知世界的路。
一本书就是一个未知的世界,一本有一点深度的书也许就是为你打开了一条通往未知世界的路。
当然,打发时光也是一种阅读。少读烂书,并不是不读烂书。偶尔读点烂书,你就会知道什么是好书。
\subsubsection{3做一个荤素不忌的阅读者}
\label{sec:org3a4f0b0}
有老师一开始提到自己读了很多杂书。我是非常赞同一个老师读杂书的,甚至可以把很多书,
混搭在一起读。《禅与摩托车维修艺术》,这本书到底讲什么?好玩。

还有一本书叫《为了报仇看电影》,看这题目很有意思。经济学可以看,哲学类可以看,古
典文学的可以看,哲学简史、建筑史等都可以看。读书要杂,读的进就读;读不进,暂时放
一边,不要去硬读。

\subsubsection{4偶尔批注,懒得摘抄}
\label{sec:orgec71bde}
这是我读书的一个态度。有的人问我:张老师,你读书摘录吗?我从来不摘录。
我以为好的句子,印象深刻的观点,它一定像钉子一样钉在我的心里,我就记住了。记不住
的句子,说明他还没有让我记住,我就不刻意地去记。我读书偶尔做一些批注,有感而发,
随意涂鸦,不亦快哉。

\subsubsection{5迷恋TA,就读透TA}
\label{sec:org3e80c58}
最近几年,我迷上一个作家,就会把这个作家所有的书都找来阅读。中国有一个作家叫阎连
科,个人以为他是不逊色于莫言、贾平凹、余华等作家。他写的书我都找来读。

此外,汪曾祺、王仁宇、刘绪源等人的书,我也一本一本地读。有些作家,不喜欢。不喜欢,就不读。
没有一本书是必读的,也没有一本书是必不可读的。
(包括以上关于读书的观点,你可能不一定认同,因为,我们的年龄、经历、兴趣、爱好,
是不同的,因此,有些我喜欢的书,你不一定喜欢,有些你喜欢的书,我不一定喜欢.)

教师生涯,最终靠的是自己的底蕴
我一直以为,一个老师教书生涯前八年,可能靠的是灵气、机遇。但是到后期,他跟别人比拼的是什么?底蕴!
底蕴是怎么来的?是书堆起来的。
也许会有老师问,张老师,你平时一堆工作,课余时间要写文章,要看电影,周末要上公开
课。那么忙,阅读时间从哪里来?老师们,我以为每天坚持阅读半小时,一年下来,你能够
精读20本书,十年你就可以精读两百本书。

一个老师,如果精读了两百本书,那是很了不起的。
出差时,我都会在飞机或高铁上看完一本薄薄的书。每次开会,我都会带一本书。对读书人
来说,开会是一件很好的事情。开会之前,开会间隙,我都会挤时间阅读。

有的老师说,我喜欢看消遣类的书。看消遣类的书当然可以,每个人都会有感到寂寞无聊的
时候,不想看学术类的书,怎么办?

找一本好看、好玩的书看,如东野圭吾的书我也看。像林清玄、龙应台这些浅浅的书,读着
不累,挺好。

最后,顺带提一下如何寻找书。关于读书的书,我推荐这几本。
《读书是教师最美的修行》(常生龙著)、《给教师的阅读建议》(闫学著)、《教师阅读
地图》(魏智渊)、《迷人的阅读》(朱煜编著)。

\subsection{附录:张祖庆老师推荐的教师成长书单}
\label{sec:orgd296ae2}
\subsubsection{文学名著:}
\label{sec:org00cc149}
《卡拉马佐夫兄弟》陀思妥耶夫斯基著
《追忆似水年年华》马塞尔·普鲁斯特著
《莎士比亚戏剧》莎士比亚著朱生豪译
《呼啸山庄》艾米莉·勃朗特著
《霍乱时期的爱情》加西亚·马尔克斯著
《悲惨世界》雨果著
《复活》列夫·托尔斯泰著
《简·爱》夏洛蒂·勃朗特著
《基督山伯爵》大仲马著
《堂吉诃德》塞万提斯著
《战争与和平》列夫·托尔斯泰著
《约翰·克利斯朵夫》罗曼·罗兰著
《双城记》查尔斯·狄更斯著
《大卫·科波菲尔》查尔斯·狄更斯
《卡夫卡文集》卡夫卡著
《荆棘鸟》考琳·麦卡洛著
\subsubsection{专业阅读:}
\label{sec:org64e3e46}
《爱弥儿:论教育》卢梭著
《给教师的建议》.苏霍姆林斯基著
《孩子们你们好》阿莫纳什维利著
《教学勇气》帕克·帕尔默著
《第56号教室的奇迹》雷夫·艾斯奎斯著
《优秀是教出来的》卡尔·威特、塞德兹、铃木镇一等著
《静悄悄的革命》佐藤学著
《学习的快乐-走向对话》佐藤学著
《陶行知教育文集》陶行知著
《叶圣陶教育文集》叶圣陶著
《教师如何做研究》郑金洲著
《重建教师的精神宇宙》李政涛著
《语文课程论基础》王荣生著
《语文教育哲学导论》潘庆玉著
《写作教学内容新论》叶黎明著
《言语教学论》李海林著
《语感论》王尚文著
《小学语文教学内容指要》吴忠豪著
《支玉恒语言教育艺术研究》施茂枝
《教诲漫记》于永正著
《语文:表现与存在》潘新和著
《教育的目的》怀特海著
《民主主义教育》杜威著
《儿童心理学》皮亚杰著
《爱的教育》弗洛姆著
《多元智能新视野》加德纳著
《童年的消逝》尼尔.波兹曼著
《教育漫话》洛克著
\subsubsection{心理学类:}
\label{sec:org107a487}
《少有人走的路》M·斯科特·派克著
《顺应心理,孩子更合作》维尼老师
《养育男孩》史蒂夫·比达尔夫著
《陪孩子长大》李子勋著
\subsubsection{最喜欢的作家:}
\label{sec:org03b6f98}
叶嘉莹《唐宋词十七讲》、《中国古典诗歌评论集》、《中国词学的现代观》
资中筠《士人风骨》、《读书人的出世与入世》,《斗室中的天下》
齐邦媛《巨流河》
钱穆《先秦诸子系年》、《中国金三百年学术史》、《国史大纲》、《中国文化史导论》
王鼎钧作文四书“《讲理》、《作文七巧》、《作文十九问》、《文学种子》”《左心房漩涡》
高尔泰《寻找家园》
木心《上海赋》、《文学回忆录》、《木心谈木心》
薛忆沩《文学的祖国》、《通往天堂的最后那一段路程》
\subsection{校长视角}
\label{sec:org74b6bd1}
\subsection{{\bfseries\sffamily DONE} 日月光华有你闪亮的眼睛|专访复旦原校长杨玉良}
\label{sec:org00b5ce9}
\subsubsection{校长与大学}
\label{sec:orge4040fd}
\subsubsection{您觉得一所大学的校长,尤其是复旦大学这样的学校的校长,对一所大学来说意味着什么?}
\label{sec:org861fb0d}


杨玉良:我跟别人的看法有点不一样。有一种流行的说法:大学校长的任务就是找钱、找人。
但我认为这两件事情不是最核心的。
\subsubsection{我认为最核心的是:}
\label{sec:orgeeab3ea}
\begin{enumerate}
\item 第一,一所大学的校长应该是这所大学的象征。
\label{sec:org50f013e}
所以校长的一言一行,实际上就代表了这所学校的基本状态。
\item 第二,一所大学的校长是这所大学整个学科的宏观布局的设计师。
\label{sec:orgdaa7d82}
所以任何大学校长如果只懂得发展自己所在的这个学科的话,我认为是不合格的。因为校长
要为整所学校着想,要能在宏观上掌控全局。别的工作,找人也好找钱也好,不是不做,我
们也做。但是如果忘了前面这几条的话,是绝对不行的。一所大学不是什么钱都要的,而且
根本不在乎钱的多少,比如哈佛大学曾经收到过一笔巨额捐款,要在哈佛大学造一栋楼,但
是哈佛大学拒绝了这笔捐款,因为这块地方不能被一栋楼占掉,它有一定的历史意义。当年
复旦拒绝周立波的捐款,也是同样的道理。所以,一个校长的言行与这所大学是要契合统一
的.
\end{enumerate}



\subsubsection{杨玉良与蔡元培像}
\label{sec:org93f2d6e}

\subsubsection{刚刚您说到一所学校的学科总体发展的布局,在您担任的校长其间,关于复旦的学科总体发}
\label{sec:org2f11099}
展的布局,您做了些什么呢?

杨玉良:
上海医学院并入复旦的时候,医学院和复旦的关系之前一直没有处理好,原因是复旦人没有
理清楚什么是医学,医学的特点是什么。当时整个医学院的管理碎片化,医学的整体性不强,
没有人去为整个医学院来考虑问题。所以我认为,合并首先要保证医学的整体性,在保证其
整体性的这个基础上,然后再来把医学里头的每一个子学科,跟复旦原有的学科的交叉融合。
所以我首先恢复了医学院的建制。

我还成立了人文学科的管理委员会。对于社会科学,我恢复了发展研究院。因为发展研究院
在邓时代和江时代,一直是国家的最重要的智库,为国家做出了很多贡献,很有必要把它恢
复过来。

我在美国加州和北欧的哥本哈根成立了“当代中国研究中心”。在校内,成立了中华文明国
际研究院。成立这两个海外中心,就是想扩大中华文化的影响力,同时增强国内智库与国外
智库的交流。我们要把真正的中国特色,和有深度的文化传递出去,而不仅仅是包包饺子,
写写毛笔字什么的。


\subsubsection{您也谈到很多学科建设的问题,我们想知道您当年大力推行通识教育是为了什么呢?}
\label{sec:org015e29a}

杨玉良:

通识教育是个“理念”,当然实现这个理念有具体的做法,我当时有一个做法就是四年一贯
制,这个是我认为的重大的亮点。为什么我说通识教育是一种理念,且这个理念应该贯穿的
所有的教学当中去呢?我举个例子,比如说你学理科,学量子力学,我可以把量子力学的方
程告诉你,解法告诉你,然后可能哪些有用我都告诉你,如果你学会了你可以去用,用量子
力学去解一些题你就会了。这个就不是用通识教育的理念在上量子学课。人类的知识永远处
在一个不断发展的状态,我对复旦的学生就有要求,因为复旦是一所文理综合性的学校,所
以我就要问:如果当量子力学这一学科的内在矛盾显现出来,即将面临一个科学革命的时代,
彼时,复旦的学生有没有能力参与这场新的科学革命中去?

比如说我经常举这样一个例子。我问过一个物理系的学生,我说你觉得量子力学怎么样?他
说,当时的这些人很伟大,写出了薛定谔方程,他觉得很奇妙。这给我的印象是很深,我眼
睛一闭就能想象当时这个情景。但是我对他说:相对论量子力学刚刚建立的时候有这样一句
话,在这个时代,哪怕是一个三流的学生,他做的工作可能以后都会永久地写进教科书。那
么如果想要具备这样的素质,这门量子学课要怎么上?有几个要素很重要。第一你应该把这
个学科的范式告诉学生们,然后当然有一部分的习题。但是更重要的是你应该告诉他量子学
是怎么诞生的。我专门有讲一个词叫“祛魅”(disenchant),就是说把那种神秘感去掉,
可能你处在那个时代,你也会去做这件事情。所以通识教育要贯穿到包括专业课里头,比如
量子力学的思想简史,如果掌握了这些,你就能敏锐地感知到一门学科可能面临的革命,那
么你就知道该如何参与其中。如果你只是教会一些术,那他能在一定程度上使用这些知识,
但是他不可能在人类知识思想的发展过程当中起到非常重要的作用。

所以通识教育的理念要贯穿到所有的课程里面去,但很遗憾这个目标很难达到,因为教师本
身也不是这样训练出来的。但是如果有人能够理解的话,他就知道应该去学什么东西,应该
怎么去学,应该关注整个学术和科学研究的前沿。前沿到底在哪里,是不是时髦就是前沿?
时髦没几年就下去了,又一个时髦时代又下去了,但有些东西是永远在路上的。

为了实现这个想法,我们做了复旦通识教育课程的重新梳理,在这方面,孙向晨教授做了很
重要的工作。但是这是一条很长的路,当时哈佛大学花了十年,出了那本《红皮书》,但是
我只给了孙向晨教授两年时间,他确实把它梳理得相当不错了,但是要贯彻实施,仍是一个
艰苦的过程。很多教师关于通识教育的讨论都落在浅表上,因为没有认识到通识教育是一种
教育理念,一种符合教育规律的教育理念。当然这种理念更适合于文理综合性大学,工程技
术类大学或许有所不同。同一门课在工程技术类院校的教法会很不一样,或许这种教法可以
让学生出来就能服务生产的第一线,但其实也不一定。复旦教育的追求应该更深远,应该具
有思想的引领性,无论是文科、理科。科学也有科学思想,但这一方面,我们缺的太多。

\subsubsection{卸任之后}
\label{sec:org2a41188}

复旦录取通知书

\subsubsection{杨校长,作为13级的学生,我们对您有很特殊的感情,因为我们的通知书是您签署的,作为即将进入大学的高中生,这张沉甸甸的通知书是我们当年最有分量的记忆。所以我们很关心您任后的状态,可以谈谈您在卸任校长之后的工作吗?}
\label{sec:org1458bda}





杨玉良:

卸任以后,本来我觉得可以有比较空闲的时间,结果发现比当校长的时候还要忙。首先,我
在担任中国科学院科学普及和科学教育委员会的主任,每个月至少去两次北京。第二个工作
是担任中华古籍保护研究院院长。现在中国的好多古籍都需要修复,其中会涉及到很多问题,
首先就是纸张。我的专业是高分子,纸张本身是一种高分子材料,所以我承担了这个工作。
我主要有两个任务,一个是把古籍保护研究院建起来,尤其是里面实验室的建设。第二个是
部分经费的筹集问题。第三个工作是利用我的专业知识帮助一些企业解决实际问题,比如之
前江苏淮安的芳纶(防弹衣原材料)项目。除了这些工作以外,在我自己的专业研究领域,
我想把原来当校长之前没有解决的学术问题继续思考下去。这些问题基本上是纯理论,一支
笔一张纸一台电脑就可以做的。原来我没有时间和精力来做这些问题,现在我利用碎片化的
时间来做,比如在机场等飞机的时候。总的说来,现在我对我的工作状态是乐在其中的。


\subsubsection{那么杨校长,在成为复旦大学校长之前,您的求学生涯又是怎么样的?比如,您怎么进的复旦,您在复旦读书的时候,复旦是什么样的?}
\label{sec:org691476e}


杨玉良:

你们对我们这一代人的了解很少。我出生在农村,父亲在上海当工人,到我上小学的年龄,
我就到上海来跟我父亲生活了。但是我初中一年级时,就到了1966年,我的学业只能停掉。
我的父亲和母亲都不主张我去参加运动,我父亲只上过私塾,我母亲是文盲,他们对我的家
庭教育非常简单,就是不要去害别人。这是一种非常简单的,中国传统农村的家庭教育。当
时我什么活动也没参加,不打人,不去开批斗会批评别人,就待在家里,当时称为“逍遥
派”。过了几年以后,到了1968年年底,我就下乡去了,又回到了自己的老家。尽管我只上
了初一,但是当我回家的时候,家乡人总觉得我是从上海回来的,而且读了点书。所以刚劳
动一年的时候,他们就叫我当赤脚医生了。那时候的赤脚医生,就是县里面培训两个礼拜。
给每个人发一本很厚的,32开的一本《赤脚医生手册》,学了两个礼拜,我就拿了这本手册
回来就开始当医生。在当时缺医少药的情况之下,赤脚医生在农村十分管用。因为当医生,
在那个时候我就养成了读书的习惯,经常到上海来找和医药有关的书,然后就地仔细看。说
句实在的,我救了不少人的命,为普通的老百姓做了不少贡献。所以六年以后,1974年,我
就被推荐到复旦来上大学,那时候叫工农兵大学生。对于我来说,在进大学之前,大学对我
来讲是极其神秘的。因为我的家族里头也没有大学生,所以根本不知道大学到底怎么回事儿。

我们那时候读大学,专业不是由你自己选的,是分配给你的,我被分配到了复旦的化学系,
里面的高分子专业。但是我个人的爱好其实是哲学。因为当时在农村,没有什么书看,唯一
一本我找到的书,就是艾思奇的历史维物主义和辩证唯物主义。所以当时我对哲学很感兴趣,
但是没有读成这个专业。

说到做科学研究,其实我的第一选择也不是化学,更不是高分子,我更喜欢的是数学或者理
论物理。我觉得我唯一有点优点的地方,就是我干一行爱一行钻一行。在国内,在化学这个
学科里,在我们这代人里面,或许我的数学和物理是可能算最好的之一。我把我的数学物理
基础揉进了我的高分子的学习,在我进入高分子领域之前,中国的高分子物理理论基本完全
是空白。当时我正好遇到了世界上的凝聚态物理的快速发展,所以我很快地把凝聚态物理和
高分子物理结合起来。所以对中国的高分子物理的理论,我觉得我是有一点开创性的贡献的。

进入大学以后,工农兵学员可以不去上课,也没有什么考试,教授们不敢考工农兵学员,因
为当时有个说法叫工农兵来上大学,那是要上大学、管大学、改造大学,叫“上管改”。当
然我也没去管也没去改,但是这样宽松的条件,给我创造了一个打开眼界的机会。图书馆里
全是我之前没有见过的书,那个时候我就天天“寝室-食堂-图书馆”,花了大部分的时间来
自学。我觉得现在大家都非常怀念的所谓的80年代,因为那是一个非常开放的时代。当时翻
译了大量的西方的思想和科学著作,这是原来不太会有的,对打开眼界有极大的好处。

1977年,我这个工农兵大学生就毕业了。正好那一年中国恢复高考制度,新一批通过考试上
大学的学生进入校园。相反,工农兵大学生在社会上的印象极差。原因很简单,因为大家觉
得工农兵大学生没学什么东西。但其实我是学了很多东西,所以我想要甩掉这个工农兵大学
生的帽子,正好也恢复了研究生考试。我就报考了研究生。在博士毕业以后,我又当了两年
博士生的辅导员。在这期间,一个美国的科学家到复旦来做报告,不仅是这个科学家要作报
告,复旦也要有个人来作报告,当时就是我来代表复旦作报告,那是我第一次用英语来作报
告。这个报告作得很成功,我也提升了自信,第一我可以用英语跟人家对话,第二我可以用
英语来讲述我自己的研究报告,第三我也体会到中国跟国外研究差距。于是我就决定要出国
进修,去的是德国。在德国的时候,我做的领域是固体核磁共振。三个月以后,德国的教授
感觉到了我的能力,他知道我的数学和物理都很好,也知道我很有创造力。当时进了一台新
仪器,他就把这台仪器交给我,让我在上面做实验。我就做了一个研究,也在理论上做了严
格的推导,刚开始他以为我做出来的这个结果是错误的,最后证明我的结果是正确的,他一
下子就对我刮目相看。在那之后,这个教授开始把学生交给我来带,我就开始带博士生了,
也确定了我在研究所里的基本地位。

在两年学习结束的时候,我面临着三个选择:回国、留在德国、去美国。我考虑再三,还是
选择了回国,因为国家需要我这样的人才去建设。1988年我回到了复旦,我一边给研究生上
课,一边做各种项目研究,比如973项目、863项目。这些项目都是结合我们国家的产业上的
重大问题去做的。我当时希望能够从实践中抽提出相关的基础理论问题,把理论问题解决以
后,能够举一反三的用到各种相关的同类型问题上。后来这些项目都做得非常成功,我也被
推荐获得了“求是杰出科学家奖”。这就是我基本的人生历程。


\subsubsection{杨玉良与导师于同隐教授}
\label{sec:org4cb509a}
\subsubsection{重 塑复旦精神}
\label{sec:org80a453c}

\subsubsection{您从八十年代至今经历了复旦的风风雨雨,那么您觉得在这么多年复旦的变化在哪里?}
\label{sec:org972734c}
杨玉良:

从我刚进复旦的时候,我就感觉到,即使是在工农兵学员的时候,我觉得复旦的学生,跟其
他大学的学生还是有差别。首先,复旦人比较低调稳重。其次,在当时,和其他学校的人比
较,复旦的人还是比较崇尚学术的,对学术问题有着良好的鉴赏能力。这也是我喜欢复旦的
一个非常重要的原因。

复旦的变化还是很大的。1976年之后,全体老师学生的学习和研究的热情突然间就释放出来
了。因为在“四人帮”时期,大家什么都做不了,当时天天要政治学习。改革开放以后,在
高校里面做的第一个改变,就是每个礼拜保证五天时间用在科学研究和教学上,以前这个时
间都被政治学习占用,这是一个巨大的变化。

那时的每天晚上,校内很多楼都是灯火通明,教师同学们都在学习。正好许多国外的优秀学
术著作被引进和翻译出来,大家都在如饥似渴地阅读。那时候的学生,没有什么功利的心态,
也没有市场经济的诱惑。对于很多老师,他们都自发的在教学楼贴上海报,给各位学生免费
上课。学生们可以把阶梯教室的地上都坐满。不仅仅是上课,如果在当时有什么问题想跟别
人交流或者想开个讲座,也可以贴个海报,借个教室,来交流的人也很多。这和今天的氛围
有很大的不同。


杨玉良与邓景发院士、杨芃原教授

\subsubsection{您觉得复旦在当代中国,扮演的是一个怎样的角色?}
\label{sec:org4430730}
杨玉良:
你可以回望历史。比如说前两天在纪念五四,在南方复旦实际上是起了很重要的作用,因为
当时离南京政府比较近嘛。所以我就说复旦是这样一所学校:在国家有难之时,你能够挺身
而出。世界上很少有国家几所大学对这个国家的走向起到十分重要的作用,如果从大学和国
家命运的联系来讲的话,唯有中国的大学起到这种作用,世界上好多国家都没达到。所以用
这个标准来评判世界一流大学的话,我觉得我们在这方面是可以算的。

从这个角度来看,你对复旦就会有一种期待,如果国家处在很平稳的阶段,教师也好,学生
也好,认真做好自己的本职工作就对了。为了要能够应付国家出现的各种情况,作为大学生,
就要做好各种各样的知识的储备,平时一直在研究,到需要的时候就可以拿出来。比如我们
现在搞的智库就是这样。然后我相信,大学里总应有一种理想主义的气氛,因为如果连大学
都没有理想主义的话,社会乃至整个世界就会迷失方向。当然,我说的理想主义不是不食人
间烟火,而是说你有理性的思维,有情感的投入,对国家、民族的发展甚至对全人类的前途
命运你是有关切的。只是因时而异,在什么时候你表现为什么形态而已。

另外,我认为大学历来就是一个多元化的地方,各种思想都可以在在校内进行讨论,乃至争
论,大家可以争论一个对和错,如果达不成一致的,你也可以在争论下去,世界上很多的问
题,都没有简单的对与错的概念,但是通过争论会得到深入的理解。

所以你就可以看复旦,老的人文学者对我的一些想法更能理解,理工科的同仁的看法我不能
确定。因为每个人都会受到自己学科的局限,学科的划分决定了其中人员的知识结构,就决
定了认识上的局限性。我就职的时候,在一个全校的干部会上做过一个讲话,我说复旦一直
主要是理科的人士当校长,有时这会使文科得不到充分的发展。因为一个国家发展进入到一
定的阶段后,科学技术当然重要,但是人文社科的重要性会越来越凸现,这就是为什么世界
上发达国家的大学校长里面,学政治学的、学经济学的,学法律的比较多,而不是理工科出
身。这种事情的效果不是一下子出来。你看现在我们的发展研究院,在国际上的影响非常大,
这个是复旦首先跨出的这一步。那么对人文学科的话,我知道的,我们学校实际上比一些人
文学科强校毫不逊色,只是我们不张扬,实际上我们文科的实力是非常强的,可以真正地和
国际上平等地对话。

\subsubsection{日月光华中有你闪亮的眼睛}
\label{sec:org94f64ff}

复旦校园
Q
\subsubsection{作为一名考上复旦大学的学生,我们经过了复旦四年的培养,您觉得经过复旦四年的培养,}
\label{sec:org555e2c8}
一个走向社会的复旦大学的学生,他应该具备怎样的品质,跟一般的社会人有怎样的不一样?

杨玉良:
你是复旦毕业的,身上当然带有复旦的独特性。我举个例子,有一次我在火车上——那个时候
我也是青年学生——一帮学生大家互相不认识,只知道大家都是大学生,因为都到一个地方去
开会。但是基本上看得出来,比如说这个是复旦的,这个是北大的,准确率蛮高,这就表明
某种印迹就在你身上,这种印迹是靠文化氛围的熏陶造成的。复旦的文化传统,就是使学生
成为一个有能力但不功利的人。

他当然应该有引领的作用,但不是野蛮的强迫,比如夸夸其谈那种,相对说比较内敛。我觉
得对毕业的学生来讲,大学是一个比较理想的环境,但社会比大学有更大的复杂性,学校是
小社会,到了外面是大社会,正因为如此,我们要安排学生进行各种社会实践活动,如到西
部去支教等等,我们都鼓励。我灌输给你的是理想主义,但同时我也告诉你,世界并不是那
么样的理想,理想主义是彼岸性的东西,虽不可至,心向往之。但理想主义是一种坐标的定
位,我总希望复旦的学生出去,能够让这个世界变得更加美好。

另外,不是说每个学生出去,一定马上就能取得所谓的世俗意义上的成功。但是如果在大学
里面是认认真真的掌握了应该有的良好的思维方式,守住了道德的底线,提升了你的精神状
态,那么我相信这样的人,到社会上人们总会需要,总会喜欢。当然,每个人的追求不一样,
你可以去做白领,你可以创业,你可以做官,你可以做学术。但是最本质的东西,对于一个
受过高等教育应该具有的这个道德素养,这个都应该具有。你是一个独立的个人,你有你自
己的思想,你有你自己的能力,不管你做哪一行,你总希望人类社会变得更美好一点。你到
了复旦,你就应该有一个更高的境界。

甚至,对国家民族的前途你要关心,对人类的前途你要关注,现在是全球化时代,任何小的
举动都会影响整个人类的前途和命运,如果你看着自己眼前的一亩三分地,好像日子过得很
好,但是作为一个真正的知识分子,历来是有这么一种情怀在,尽管每个人力量都很小,但
也要为社会的进步尽一份绵薄之力。最简单的就是说,不好的事情首先自己不干,这是最简
单的你应该可以做得到的事情。

\subsubsection{特别致谢}
\label{sec:org2ac9f43}
既2又38团队:
朱海啸 赵明节 朱朋朋 李雅婷 宋喆 张润坤
复旦通识 关注课程建设与学生学习
微信号:复旦通识教育
如有意见或建议,请联系我们:fudanhexin@fudan.edu.cn
\subsection{{\bfseries\sffamily DONE} 徐小平博客}
\label{sec:org3e26f23}
\url{http://blog.sina.com.cn/s/blog\_46cf3d4501017og8.html}
\subsection{“原来在国内我上的不是大学”}
\label{sec:org614d4ff}
原文地址:“原来在国内我上的不是大学”作者:bqxiong

  在哈佛做讲座之后,我写了一篇感想文章,说哈佛的学习氛围很浓。文章发表后,一位
哈佛学生给我发来一封邮件,说哈佛校园中,学生们的讨论很热烈,是事实,但这可不是学
生们都爱讨论,他们也是没办法,因为老师有这样的要求,不参加讨论会,可能很快就跟不
上课程。


  他想告诉我的是,大学的学习氛围,是由学校对人才培养严格要求、重视教学所带来的,
不是靠学生们的自觉。这与国内大学近年来反复强调“学风建设”,很是不同。在“学风建
设”中,学校、老师反复教育学生们,要主动学习、要爱学习。可似乎学生就是主动不起来,
该逃的课还是逃。


  其实,我在几所美国大学参观时,陪同的学生,都谈到了这一点。一位在MIT(麻省理
工学院)读大二的中国留学生告诉我,他是世界奥林匹克物理竞赛金奖获得者,高中毕业时
保送进了北大,进了北大之后,一个学期选10门课,甚至更多,都没问题,因为只要学期结
束,考试通过,就可以了。他觉得大学的日子基本上是在混中度过。


  一个同样和他保送进北大的同学,混到大二,因为长期缺课去打游戏,最终多门考试不
及格被退学。他觉得这样混下去,实在对不起自己,因此申请MIT,到这里继续读大二。到
了MIT,虽然只选了五门课,但他感到异常的繁忙,因为每门课,老师都要求要阅读大量的
书,有的课,还必须做大量的实验,稍微掉以轻心,就跟不上,在他们的同学中,晚上在图
书馆熬夜看书到深夜的情形,十分普遍。


  这位同学告诉我,他曾经对大学十分失望,而到了MIT,才发现自己以前上的哪是大学
啊。他说,他对此的感受最有说服力,因为其他的本科生,大多是直接高中毕业后来留学;
而研究生们,感受不到本科生教育。听了他的话,我在想,这可以解释,为何近年来会出现
中国留学生捐巨资给自己的美国母校却不捐给中国母校。


  我国大学,也反复在谈提高本科教育的质量,要对学生提出严格的要求,可是,这只是
说说而已。重视本科教育,意味着必 须要求教授们把大量的时间用到课程设计、课程教学
中,可我国大学的教授们,在高校强调论文、课题、经费的现实中,是不愿意在这方面花
“无谓”的精力的。这样的教育教学环境,几乎在一夜之间,就可摧毁学生们对大学的美好
期望。我国很多高中毕业生,怀着美丽的大学梦进入校园,上完第一节课之后,就从梦想回
到“现实”——原来梦想中的大学竟是这个样子。而对于学生们在大学里的“不认真”,大学
将其原因归为学生的学习态度不端正,进入大学没有进入角色。


  我也曾对大学新生分析过,大学的教学环境与中学不同,要求学生自主管理、自我规划。
但这次观察哈佛学生的学习,才明白,学生自主管理、自我规划没有错,可老师教育教学的
高度投入、负责,对课程的严格要求,是学生进行有效自我管理、规划的基本前提。哈佛本
科生的课,安排得并不多,但学生们忙,就忙在到图书馆看书、查资料、准备讨论会,以及
撰写课程论文。


  最近,数所世界名校的网上公开课受到国内大学生、白领的追捧,很多人的感慨是,看
了这些课,才知道自己以前上的不是大学。这和那位来自北大的留学生的感受很相近。但一
位哈佛学生告诉我,这些网上公开课并不是最精彩的,精彩的是讨论课。哈佛的讨论课,一
般只有八九人(学校规定,课程参加人数不得超过八九人,人数太多,有的学生无法参与,
这与我们课程强调规模,以数量来论课程欢迎程度完全不同)。在讨论课上,教师只是引导
者,学生自由发言讨论,大家在讨论中,能相互学到很多知识,思维能力、表达能力也在这
一过程中得到培养和锻炼。


  我在卫斯理女子学院参观时,一位同学说,她们有很多学生自己组织的活动,比如在上
一周,就讨论利比亚的局势,我想,这些讨论,是属于学生自发的。而他们的“自发”,不
正是源自于已经养成的学习与讨论习惯,以及鼓励学生所有话题均可自由发表意见的整体氛
围吗?


  有什么样的学校定位,有怎样的学校风格,就有怎样的学生。美国的大学,可以让学生
在忙碌中热爱学校,融入大学,而我国的大学,却让一个个好学生把“混”字挂在嘴边。如
果要说差距,这是我国大学与国外名校最大的差距。如何赶上这些名校,国外学生们的感受
最深处,也就是我国大学最应该做的。

\subsection{李开复的博客}
\label{sec:org9fead92}
\url{http://blog.sina.com.cn/kaifulee}
\subsection{{\bfseries\sffamily DONE} 我为什么热衷中国教育}
\label{sec:org87c9157}
一月八日,我非常有幸地在新浪办的"新浪2008中国教育盛典",得到两个奖项,"2008年度
十大教育人物"和"2008年十大教育博客",并且参加了于丹老师主持的教育高端论坛。


  我的博客内容一向是我自己撰写,但是一直很愧疚更新不够,有时一个月没有新的内容,
新鲜内容不够,文章总是太长,而且没有时间回答读者的众多问题,只有请他们到"我学网"。
虽然如此,新浪还是不遗余力地推广我的博客,这次还颁奖给我。我只能感谢新浪教育频道
的厚爱,用我得奖的感言:"继续努力",希望以后能够够做得更好。

  我为什么热衷中国教育
  新浪的活动办得很成功,但是更重要的是它能够引起更多的人对教育的关注。

  我深深地相信:我们虽然不能改变教育,我们必须关注教育,必须为教育贡献,因为这
是我们的后代,我们的民族,我们的国家的未来。


  我也深深地相信:如果每个人都关注教育,对教育贡献,那么教育就会改变,因为教育
的本质并不是"传道",而是"学习的能力";并不是"解惑",而是"从经验中得到领悟";并不
是"指出成功之道",而是"找到个人心中的声音"。而"分享学习的能力","分享我们的经验",
"帮助找到他人的声音"都是我们每个人可以做的事情。


  下面附上三段我在这次盛典上的讲话。

我的致辞:我为什么热衷中国教育
我为什么热衷中国教育

  作为一个热衷中国教育的公益人,我今天非常荣幸参加"新浪2008中国教育盛典"颁奖典
礼。把我自己称为是一个教育界的公益人算是讲得比较客气,有些人觉得我是不务正业。其
实虽然一方面花很多时间在我的公司上,但是在过去的九年中我也把很多时间都投入了教育。


  过去九年中我一共给学生写了七封信、三本书、办了一个网站,我最近三年每年面对十
万学生做演讲,演讲的题目都是有关教育有关成长方面的。可能在座很多朋友会奇怪为什么
我会这么热衷中国的教育?其实理由很简单,我记得当我在1990年第一次踏上中国大陆的土
壤的时候,在学校教了三个星期的书,当时让我感触特别特别深的就是看到一些和我差不多
是同龄的青年,他们和我一样聪明或者更聪明,他们绝对比我更努力,他们比我更好奇,但
是很遗憾的是他们没有一个足够良好的教育环境,达到我当时对一些科技方面的看法、认知
和成长、成才方面的价值观。我就觉得,同为炎黄子孙,我一定要花更多的时间帮助这些人
发挥他们的潜力。


  在1998年是我第一次有机会回到中国来工作,当时我就花了很多时间和中国学生做一个
对话,而更深入地理解了在过去的这一段时间,中国的教育其实已经成长很多,这是值得庆
幸的。但是还是让我感觉到感慨的就是,我更深入地认识了这批年轻人,他们确实是那么的
优秀,那么的聪明,那么的努力,那么的渴望成功,但是他们面临的挑战却非常得多,他们
有着高期望的家长,有着很快的进步,但是还没有达到世界水平的教育,他们面临着很多诱
惑,而且很多一元化的价值观挑战的社会,所以他们迷茫了,他们需要指路人,他们需要更
好的教育。


  我们应该更广义地来看教育,教育不只是智商的培养,也是情商的培养,不只是学习如
何成功,也是学习如何成长,如何做人。于是我就认为我应该花一些时间帮助他们,让我非
常欣慰的是他们也非常愿意得到我的帮助。这样从2000年开始我开始我的写作、演讲和办网
站的工作。从2000年到今天这九年,我看到了中国的教育继续在成长,当然还有很多可以做
的更好的地方,我也会不遗余力地帮助中国的青年人真正达到、发挥他们的潜力。我觉得今
天中国站在这个平坦的世界上没有任何的事情比教育更重要,我觉得中国未来的崛起、成功,
关键就是中国的青年能否发挥他们的潜力,能否得到他们应得到的成长。


  所以今天的这样一个新浪办的教育盛典我觉得非常有意义,这么多朋友参与,也代表了
我们社会上还是非常重视,非常认可教育。所以我希望以后我们也会一起携手努力,让中国
的教育做得更好,让中国的青年能够得到更好的成长,能够发挥他们的潜力。


高端论坛:求职者应该在金融危机时好好充电

  其实我觉得金融危机带来的未尝不是一个机会,过去大学生的就业往往考虑的是什么样
的品牌公司,能赚多少钱,而忽视了我的兴趣在什么地方,我的天赋在什么地方,而且也没
有充分地做一个职业的规划,就是说五年之后我想在什么公司做什么职位,如何走好第一步,
才能达到未来的理想。第一步选择比较有限,或者没有过去那么好,但是未尝不是一个机会,
让众多大学生今年能够好好想想,我要去什么样的企业才能够有更好地学习成长,去什么地
方,今天或未来才能做我感兴趣的职位,好好充电,因为金融危机现在看起来虽然很糟糕,
但是历史告诉我们它一定会过去。如果过去的时候你充好电,做好准备,也许在未来能够找
到更好的机会。


高端论坛:工科学生必须注重动手能力
我为什么热衷中国教育

  从IT这个产业来说,确实整体的情况是比较严峻,比如说谷歌我们确实还在招人,但是
可能要求会比以前更高,招的人数比以前少。但是其他IT企业是一样的。我觉得无论是本科、
硕士或者博士,动手能力都还是非常重要的,也许不在文科或者其他方面,但至少是在工科
方面,一个很会写论文,甚至能够得奖的论文人来到 IT企业去应征一个高级研究人员,高
级工程师的职位,他如果不能跟本科生一样竞争,做编程的工作,他是没有竞争力的。所以
我说一定要动手能力强,而且往往我去高校做招聘的时候,我们对动手能力的要求,比如说
写过十万代码,学生都不敢相信,说只写过两千。有时候还会埋怨说学校没有这个机会,但
是写代码是自己可以选择的,暑期可以工作的。具备很强的动手能力,我觉得最后这个机会
还是可以得到的。

\subsection{{\bfseries\sffamily DONE} 孙云晓:教育的有效方法就是让孩子不断体验成功}
\label{sec:org515bbbb}
\url{http://blog.sina.com.cn/s/blog\_475b16640102x0gb.html?tj=edu}
成功离我们并不遥远,并不是只有学习成绩取得全班第一、全年级第一才叫成功,也不是只
有考上名牌大学才是成功,更不是只有成为亿万富翁或领导人才叫成功。当你实现一个又一
个的小目标,在原来的基础上获得发展,有了提高,就是成功。每天的一点点进步让每个孩
子都能体验到成功,这样孩子就会越来越自信,从而形成反复成功




反复成功的孩子也会越来越好

你是好学生吗?当面对这样一个提问的时候,许多成绩不理想的学生很难抬起头来。
在他们看来,自己是失败者,因为学习成绩不好就不是好学生,成功似乎永远与自己
无缘。


其实,任何一个人都是成功者。只要你能来到这个世界上,你就是一个冠军。从生理
学的角度来说,每一个人都是父母体内最健康的精子细胞和卵细胞的成功结合。也就
是说,在出生之前,你已经赢得了许多场战役的伟大胜利。


遗传进化学家舍菲尔德说过:“在整个世界史中,没有任何别的人会和你一模一样,
在将来到未来的全部无限的时间中,也绝不会有像你一样的另一个人。”


所以,我们可以自豪地说:

“我来了!”

“我是与众不同的人!”

“我是一个成功的人!”孩子们像一个个天使来到人间,给父母带来了从未有过的喜
悦,这便是成功的喜悦。可是,好景不长,许多喜悦变成了从未有过的烦恼。




北京的一位母亲曲兰,曾经历过这样难以言说的烦恼。她的儿子中考成绩平平,只能
进入一所中专就读。可是,母子俩很快又陷入失望中,因为计算机技术发展日新月异,
而学校的教材早已过时,学这些陈旧的知识岂不是浪费青春年华?虽然,教育改革了,
中专生也可以考大学,但为了一张文凭,是否值得付出一生中最珍贵的时光呢?


常识告诉曲兰,人应当做自己最喜欢的事情,走最适合自己发展的道路。

细想下来,儿子最喜欢的是计算机和英语,因为四年的上网经历,使他在与外国网友
的交流中,已经娴熟地掌握了这两门技能。于是,在儿子的请求之下,她做了一个大
胆的决定:退学!回家自学!寻找适合自己发展的道路!


曲兰忐忑不安地说:“儿子,妈妈也不知道退学这条路是对还是错,也许这是一条捷
径,也许会因此让你一生备受折磨。从现在开始,你就要自己给自己当教授,安排自
己的学习了。”


不料,16岁的儿子却如释重负,像一只出笼的雄鹰,自信地抖动着翅膀。他一边给某
报纸的网络版当管理员,一边准备考MCDBA(微软认证数据库管理员)。


曲兰得悉后大吃一惊。当时,敢去报名参加这项考试的大部分是研究生,而绝大多数
中学生恐怕还不知道MCDBA是什么东西。中专退学的儿子是否不自量力呢?


谁知,一年后,当同龄人忙于高考时,曲兰的儿子真的通过了MCDBA(微软认证数据
库管理员)的全部考试。不久,他又通过了MCSD(微软解决方案认证专家)的考试。


这两项考试是当时微软认证考试中难度最大的两项国际性考试项目。由于对英语和计
算机水平要求较高,因此18岁以下的中国青少年中几乎没有人参加这两项考试;即使
在中国的IT(互联网技术)界,参加这两项考试并全部通过者也是凤毛麟角。在那一
年,亚洲地区20岁以下的青少年中,仅有两人通过考试,而曲兰之子便是其中之一
(另一位为19岁的印度青年)。


后来,曲兰之子被调往一家香港公司担任数据库主管。由于做着自己喜欢的事情,他
每天都过着快乐的日子,而他的同龄人大多还在为文凭苦读。


曲兰感慨万千地说:“儿子的经历使我第一次对今天的教育模式产生疑问——是不是每
个孩子都得上大学,或者一定要经过16年甚至20年的教育才能成才?”




我想,人类教育史上最重要的也是最简单的经验早已回答了曲兰的疑问——教育的有效
方法就是体验成功!


本文摘自《成功智力-比智商更重要的潜能》。
**
\subsection{\href{http://www.jianshu.com/p/acb7a747ae39}{《你的知识需要管理》读书笔记·知识管理入门}}
\label{sec:org1bd191a}
第一次阅读《你的知识需要管理》这本书是在大二的暑假,第一次接触知识管理这个概念,
觉得打开了一个崭新的世界——以前虽然有类似的想法,但是如此系统地了解这是第一次。从
那时候起就开始有意识地开始实践起来,觉得这实在是一件对自我提升很有帮助的事。如今
重读了这本著作,并做了思维导图,分享出来。

由于书是写于2010年,许多知识管理的相关知识已经有些滞后,做导图时已经更新相关部分,
并根据自己的实践补充了一些内容。


《你的知识需要管理》思维导图
\subsection{1 为什么要进行知识管理}
\label{sec:orgafee285}

为什么要进行知识管理
一个人的安全感应该来自哪里?父母,单位,朋友还是伴侣?作者认为,一个成熟的人的安
全感应当来自自己而并非外在,因为父母可能会老去,单位可能会破产,朋友会离开,夫妻
也有可能化作分飞燕。想要拥有快乐的一生,想要成为一个有成就的人,就必须依靠自己。

人要靠自己,但靠自己并不是依靠自己的体力,而是要自己的脑力,靠知识。个人可以依靠
的知识,是指在一定的知识基础上,能够随着社会环境变化,不断地确定自己的专业方向并
快速学习知识、分享知识、使用知识、使用和创新知识并创造知识价值的过程。这个过程是
对你的知识进行有效管理的一个过程,也就是提升你的知识力的过程。

\subsection{2 知识管理的现状}
\label{sec:orgd487da6}

知识管理的现状
处在一个信息爆炸的时代,我们的身边从不缺乏信息,很多人其实是处在信息过载的状态下
的。不是所有的信息都是产生价值的,也不是所有信息都能最终转化成为知识。

所以在最开始,需要对于数据、信息和知识这三个概念进行区分。试举一例:

数据:37.5

单纯的数据并不能传递任何信息。

信息:孩子的体温是37.5℃

信息赋予了数据环境,使之变得有意义。

知识:对于孩子来说,37.5℃已经略微发烧,需要及时就医

也就是说知识是经过实践证明的,可以用来决策或者行动的信息。

其次,知识可以分为显性知识和隐性知识两类。所谓显性知识就是能够利用语言、文字、肢
体等方式表达清楚的知识,而隐性知识就是虽然知道如何做,但却很难告诉别人或写明白、
说明白的知识。从掌握知识角度讲,大量知识以隐性知识形式存在,而能显性化的部分较少。

隐性知识有以下特点:

对你是隐性知识,但对别人可能是显性

隐性知识需要环境限定,并非永远隐性

先将隐性知识显性化的就是知识创新开拓者

知识显性化能力决定了是人与人之间的能力差异
将隐性知识显性化有很多方法,如:讨论、回答提问、需求的压力、工作分解、流程分析等。

在作者看来,知识管理的问题可以细分为五部分内容,每一部分都是极重要。

\subsubsection{学习知识--你会学习吗}
\label{sec:orge5e4166}

\subsubsection{保存知识--用时能找到}
\label{sec:org1f9d507}

\subsubsection{共享知识--让人知道你知道}
\label{sec:org353f4e1}

\subsubsection{使用知识--让知识带来价值}
\label{sec:org8cc8fbb}

\subsubsection{创造知识--用创新超越竞争}
\label{sec:orgd34afff}
但是对于一个人的不同人生阶段,所要面对的知识管理的问题的侧重点是不同的。

作者的建议是:

在读书和工作的第一个十年,要更多关注第一个问题(知识的学习)、第二个问题(知识的
保存)和第三个问题(知识的共享和传播)。因为在30岁以前,你可以更多地去探索自己的
有事、特长来确定自己的学习目标和方向性,同时形成自己的良好的知识保存习惯,乐于并
善于分享,树立自己的专业形象;在这个阶段,因为相对来说知识创新的要求不高,尽可以
做简单的知识使用的工作,在不同的企业进行体验。

在工作的第二个十年,则需要更多的关注第三个问题(知识的共享和传播)、第四个问题
(知识的使用)和第五个问题(知识的创新),通过知识的共享和传播建立自己在某个细分
领域的专业形象,通过知识转化为专利或知识产权、转化为产品或服务直接提供给市场,通
过知识创新引领一个领域的发展与进步。

\subsection{3 学习知识}
\label{sec:org782f3f5}

学习知识
知识管理的第一个问题是学习知识的问题,再往下细分,可以分为学习什么知识和怎样学习。

\subsubsection{3.1 学习什么知识}
\label{sec:org0ab1fd9}

学习什么知识
对于知识的学习,有意识的去学习的效果要远远好于漫无目的的学习,在一个领域的精通钻
研好过于事事都浅尝辄止。所以在一开始,需要确定自己的学习方向。对于学习方向的确定,
有如下的方法:

\subsubsection{1)价值观是什么?}
\label{sec:orgdd4c372}

人生的每个阶段都面临选择,究竟该如何选择?这个时候就是价值观在起作用。


常见的价值观
\subsubsection{2)个人目标是什么}
\label{sec:org2271f72}

国家有国家的战略,企业有企业的战略,人也应当有自己的人生战略,也就是你究竟想要成
为怎样的人。

\subsubsection{3)个人性格是怎样}
\label{sec:org6cdeeca}

为了更深入了解自己,可以参与一些职业性格测试,如MBTI。

\subsubsection{4)通过别人了解自己}
\label{sec:orgdff3b8c}

选择是个最熟悉你的人,征求他们对你的认识。

在一个领域的学习至少要起码三年以上,才能在这个领域稍有发言权。


\subsubsection{知识结构金字塔}
\label{sec:org3a0a3cb}
知识结构可分为三个层次:

\begin{enumerate}
\item 第一,基础层次,是知识工作者必备的基本文化素养和修养;
\label{sec:org737dd7e}

\item 第二,中间基础层次,即较为系统的本专业知识,是知识结构的特色,体现不同知识工作者的特点和特色;
\label{sec:org3419390}

\textbf{\textbf{*}}

\item 第三,最高层次,这一层次是知识工作者优势所在,包括从事领域的最新成就、技术发展、方向和动态知识。
\label{sec:orgf766a28}
\end{enumerate}
\subsubsection{3.2 怎样学习}
\label{sec:orgf8a594a}

怎样学习
首先要确定适合自己的学习方法:有些人喜欢做笔记,通过做的过程,增加对知识的理解,
达到掌握知识的目的;有些人主要是靠听和看来进行学习,通过阅读资料、聆听课程以及与
别人交流达到学习的目的;还有一些人则习惯在对别人诉说的过程中,整理思路和知识。

作者推荐了在一个领域学习的学习方法模型:

\subsubsection{1)通过学习教材与阅读论文掌握领域基础知识}
\label{sec:org28dfb6a}

\subsubsection{2)通过利用网络搜索以及与同行交流了解领域全貌}
\label{sec:org37dd36c}

\subsubsection{3)通过阅读书刊、杂志、期刊,订阅博客和RSS参与行业研讨会的方式来追踪行业的最新知识。}
\label{sec:org3945fa6}

\subsubsection{4)在工作中有意识地利用知识或做一些知识显性化的工作来进行实践创新。}
\label{sec:org5e5e4f5}
\subsubsection{3.3 学习工具}
\label{sec:orgc46d9c7}

学习工具推荐
根据书中的推荐以及自己的实际状况对这部分内容进行了更新:

搜索引擎:

谷歌搜索www.google.cn

百度搜索www.baidu.com

必应搜索cn.bing.com
百科类网站:

维基百科www.wikipedia.org

百度百科baike.baidu.com

互动百科www.hudong.com
问答型网站

百度知道zhidao.baidu.com

知乎       www.zhihu.com
在线论文库

中国知网www.ckni.com

维普中文lib.cqvip.com
信息评价工具

豆瓣网(电影读书)

大众点评(美食)
网络课堂

网易云课堂study.163.com

多贝公开课www.duobei.com
信息订阅工具

RSS订阅

新闻订阅

微博订阅

微信公众账号
\subsection{4 保存知识}
\label{sec:org3835d53}

保存知识
网络上免费资源很多,很多人习惯于在电脑里云盘里囤积大量的资料但却不去咀嚼消化。事
实上单纯地下载和保存不学习,那些没有经过处理的知识对你并没有多少用处。聪明的知识
工作者只保存对他们当前和以后发展方向有关的知识,并对知识进行处理并建立索引,方便
在需要的时候查阅。

电脑作为我们知识管理的重要工具,电脑的文件管理很重要。

\subsubsection{电脑的管理}
\label{sec:orga165867}
作者提出了6步电脑管理法:

\subsubsection{1)删除:及时的删除电脑临时文件以及以后不用的文件/网络上可以方便找到的资源}
\label{sec:org7a18579}

\subsubsection{2)建立分类文件夹并及时命名文件和划类(分类深度不超过3层)}
\label{sec:orgfe1a3cf}

\subsubsection{3)整理电脑的快捷方式:只保留常用 ,放在任务栏/桌面建立文件夹分类}
\label{sec:org40fd922}

\subsubsection{4)在互联网上保存(利用云盘以及云笔记)}
\label{sec:org3eb8790}

\subsubsection{5)在电脑使用本地搜索工具提高寻找文件效率(推荐everything)}
\label{sec:orgfd7722d}

\subsubsection{6)知识保存工具(Total Commander、为知笔记)}
\label{sec:org2bb424a}
人脉管理作为知识管理的一个分支的意义在于朋友很多时候是我们知识获取的一个重要来源,
因此也是需要注重维护和管理的。


人脉管理
\subsection{5 共享知识}
\label{sec:orgad1a7d4}

共享知识
之所以共享知识,是因为共享实在可以给自己带来许多的好处,另外说开去,分享也是互联
网的四大精神(开放、平等、协作、分享)之一:

共享出来的知识才是被你真正掌握的知识,通过共享这一行为还有可能为你带来更多的合作,
可能是兴趣上的,也有可能是事业上的,并且可以结交志趣相同的高质量的朋友,这也是建
立个人品牌的重要方式。

共享知识可以通过专门的知识分享网站,比如简书、知乎,比较传统方式的还有论坛、博客
微博;此外也可以通过QQ群、微信群进行在线的知识交流,如果在一个领域学有所成可以将
自己的知识体系化出版实体书或者电子书。
\subsection{6 使用知识}
\label{sec:org8190f2f}



使用知识
中国自古以来有“书中自有黄金屋”的说法,而所谓使用知识就是将知识的价值发挥出来的
过程。通常来说,使用个人知识有以下三种方式:

\subsubsection{1)把个人知识与任务和项目结合,向某个固定的机构提供知识服务}
\label{sec:orgb0e841a}

\subsubsection{2)把个人知识表现为专利形式,通过销售专利向社会提供知识服务}
\label{sec:orgea2b928}

\subsubsection{3)把个人知识产品化,直接向社会提供产品或服务。}
\label{sec:org67a952e}
决定你的知识的价值的因素有两个:一个是你所拥有知识的独特性,独特性又可以分成两种:
一种是深,一种是博;第二个因素就是社会对你的知识的需求。

\subsubsection{因此想让自己升值的话,不妨可以试试这三招:}
\label{sec:orgaab63bb}

\subsubsection{1)向前看三年:}
\label{sec:org54eb206}

找到两三年后社会需求会变大的知识领域,并在这个领域学习和积累。

\subsubsection{2)持续提高知识的独特性}
\label{sec:org50b58bc}

如上所述,一个方法是某方向上的专,另一个是博(即资源整合的能力)。

\subsubsection{3)通过知识共享提高个人品牌}
\label{sec:org7be2c28}

在知识独特性差别不大的情况下,谁拥有品牌,谁就拥有更多机会。
\subsection{7 创新知识}
\label{sec:org5998023}

创新知识
创新是当前中国的众多热词之一,而正如同所有的房子的创造都必须先经过图纸设计在进行
施工,因此所有的创新的前提都是知识的创新和创造。

对于知识创新,学习是其前提。只有经过广泛地学习,尽量多地掌握背景知识、行业进展和
领域前沿以及更有效的知识管理才可能达到这一阶段。而知识创新的动力应当是社会的需求,
所以我们也必须学会找到需求、发现需求和挖倔需求。

\subsubsection{两个知识创新的工具:}
\label{sec:org37dee46}

\subsubsection{1)奥斯本检核表法}
\label{sec:orgc0a08b2}


奥斯本检核表法
奥斯本检核表法以该方法的创始人奥斯本命名,引导个人在创新过程中对照9个维度问题进
行思考,从而启迪思路,开拓思维想象的空间,促进产生新设想、新方案的方法。

\subsubsection{2)思维导图}
\label{sec:org6171f8e}


思维导图
思维导图也叫心智图或者脑图,由托尼·巴赞创造,本质是充分利用人类左右脑机能,以放
射性思考为基础,协助人们平衡科学与艺术、逻辑与想象,进而促进人们的学习和创新。
\subsection{8 写在最后}
\label{sec:org7cf9c41}
在我看来知识管理其实是一个私人的行为,需要个人根据自身的情况来自行制定适合自己的
知识管理方案。在这一过程中,是有一些通行的原则可以遵循的,《你的知识需要管理》中
提到的学习知识、保存知识、分享知识、使用知识和创新知识就不妨大胆拿来借鉴。在结合
自己在实践中逐步地积累相关的经验,最终形成适合自己的知识管理方案,达到自我提升的
目的。

\noindent\rule{\textwidth}{0.5pt}

思维导图下载:

《你的知识需要管理》思维导图 Novamind软件制作

作者:高小强LittleStrong
链接:\url{http://www.jianshu.com/p/acb7a747ae39}
來源:简书
著作权归作者所有。商业转载请联系作者获得授权,非商业转载请注明出处。
\subsection{\href{http://www.sohu.com/a/139847822\_335896?loc=4\&tag\_id=65788}{泰伯GIO以色列游学笔记 | 王昊: 泰伯带我“意会”以色列}}
\label{sec:org876f7d9}
泰伯网
有看点的空间地理信息资讯都在这
你还在等什么?

2017泰伯GIO以色列游学首期圆满结束,GIO俱乐部参访企业家的以色列游学笔记第二篇,带
来极海纵横CEO王昊的游学笔记。
很早就听过关于以色列的小故事,一听说泰伯要组织GIO的同仁们去以色列考察学习,立刻
就想起三个:
非利士人欺负犹太人,将军歌利亚是古代版的大鲨鱼奥尼尔,力大无穷,伟岸无比,所向披
靡,以至于无人应战。犹太人正待投降,年轻的大卫出手迎战。歌利亚说:咱们单挑,你走
近点,离我那么远我看不清。大卫轻灵助跑,发射石块暗器,正中歌利亚头部。巨人应声倒
地那一刻,大卫王恰奔至,手起刀落。大卫所以成为王。
这个故事浅层次的道理是:近视眼就得戴眼镜,否则就是吃亏;深层次的道理是:从来就没
有绝对的优势,或者绝对的劣势。身高力量是歌利亚的优势,就会给他带来移动缓慢的劣势。
优势就是一种隐形的负债,你不自知,还沾沾自喜,债务就会让你死的很难看。

大卫成为王以后,有一晚看到湖边有美女在月下洗澡,自己妻妾有四,却每每想念那夜那景,
欲罢而不能。于是设计害死还在前线为自己征战的勇士——美女的老公。美女唤作拔士巴,拔
士巴为大卫王生下犹太人智慧之大成者——所罗门王。
这个故事浅层次的道理是:切不可让自家好东西外露,财物损失是小,丢了性命也不是不可
能;深层次的道理是:最伟大的人也有很不光彩甚至是很恶的那一面,何况芸芸众生。世无
完人,用人之时,切忌追求完美。
所罗门王靠母后拔士巴的精心策划执掌犹太王国,有一天,两个妓女争夺一个小婴儿,都说
是自己的孩子,官司打到所罗门王那里。所罗门听罢双方陈词,言道:此事简单,刀斧手把
小孩劈成两半,一人一半。假妈妈立刻说:好极好极;真妈妈却说,我不争了,给她给她。
这个故事浅层次的道理是:人要是笨就少说话,察言观色再表态,女人更是得有点矜持;深
层次的道理是:用“分离博弈”让花园自除草,用好的制度,好的管理方式,让能干的人愈
发能干,使不合适的人自动淘汰。

犹太人用这些小故事一代代的武装自己的大脑,更不要说还有给予犹太人大智慧的《塔木
德》。可是我还是在想现代以色列是怎样的?创业企业有哪些发展经?欣欣然的跟家人里说要
去以色列考察学习了,我像看到花花绿绿糖果罐子的孩子,满脸的期盼,而家人却是以为我
要到战争前线。虽然巴以现在闹得不那么凶了,但叙利亚不还打着呢吗?特朗普不也还在添
乱吗?大家想象中的场景是呼啸而过的战斗机,背着炸弹冲向人群的大胡子。几乎所有没去
过以色列的人,对以色列及犹太人都是标签化的认识:动乱的国度,那里的人生活在兵荒马
乱之中;聪明的国民,犹太人的大脑等同于高智商。

大数据的时代,就是要粗暴的给人给物打标签,而对于那些还计划与人工智能同场竞技的人
来说,却要忌讳标签化的思维。说是如此,我也未能免俗。我热切期待出发的另一个大俗原
因是:作为创业者,没去过以色列约等于没有见识。近年来我们国家搞双创,创业轰轰烈烈,
纠集几个小伙伴,注册一个公司,写上几页BP(商业计划书),就算是创业了。可是创新,哪
儿有那么容易啊,尤其在物质丰饶的时代,各行各业的道路几乎早已堵满了前仆后继的死尸
活人。所以去传说中最拥有创业经验,最会创新的人群中看看,似乎就能找到难得的灵感。
一路行来,听了大学里面创业的课程,了解了以色列政府在整个国家发展中的举措,参观了
若干家创业企业,走访了拉比家中并倾听拉比的训诫,收获满满。每日见闻,泰伯小编早已
具文发布。我所切身体会的是,以色列人融入骨髓的创业愿望和精神,就像我们中国人与生
俱来的的重礼尊师,讲究人情一样,都是和这个民族传统的文化,独特的经历有直接的关系。
要说以色列从建国以来的成功,也只是其传统在现代社会自然而形成的结果。
在拉比家里的晚宴上,我们品尝着犹太安息日的以色列风格晚餐,感受着弥漫在空气中的宗
教味道,不自觉的增添了肃穆和尊重,不敢高声阔谈。听拉比娓娓道来他的故事,他的认识
和他的打算,虽然他本人没有总结一二三,不过我却感受到了很强的犹太人宿命:

\subsection{刻蚀在骨头上的危机意识}
\label{sec:orgbfd0f9a}

拉比说,他生活并不富裕,大老远的从精彩的纽约回到相对落后的耶路撒冷,曾不止一次的
想逃回去,他的父母和家人也是劝解,一个从小在美国长大的人,为什么一定要拖家带口的
在这个火药桶里生活?他的想法是,当下对以色列是个关键时期,作为犹太人的一份子,就
该在祖国的心脏去支持,去贡献。我对地缘政治关注不多,没法体会这个关键时期与以往有
什么不同,相反我觉得除了特朗普对全世界是一个不确定的危险因素外,其它都还好。对于
危机的敏感,使得犹太人已将“防患于未然,为风险做好准备”练就成为习惯。而犹太人对
待危机的态度,又从来不保守,不绥靖,相反往往是主动出击,用进攻作为最好的防守。我
们也从走访的创业公司的创业者身上看到了“大胆假设,小心求证”的实践精神。

\subsection{永不停息的奋斗精神}
\label{sec:orgdcb5039}

拉比50岁的年龄,生了11个孩子,最小的一个还在襁褓中。按照我们一群生活在北京的老爸
观感,您是不是也太累了点?要不要考虑学区房?要不要准备孩子的留学经费?要不要培养孩
子独到的特长?要不要和学校老师搞好关系?拉比的解答,人生就是一场奋斗,50岁离奋斗尾
声还早,生儿养女只是奋斗中的一个小小环节,培养子女也是应当应分的事情。况且延续人
类的基因,是上帝赋予的使命,是大爱。我们走进几家企业,看到的年轻人不少,步入中年
以后的创业者更是不少。skyline,mobileye,moovit的创始人可真是都不年轻了,关键是
他们并没有收成果,守疆土的大公司心态,还是想不断的折腾,创造出些新鲜玩意,与那些
行业大佬们去竞争,前往全球那些不熟悉的市场去探索。

\subsection{对知识的追求和敬畏}
\label{sec:org4651305}

拉比说,世界是被三根柱子支撑的:学习,崇拜和血统。真正让犹太人具有全球竞争力的,
就是其中之一:学习。拉比从小在美国生活,地道的纽约口音,对着远方的中国朋友说话,
语速倒是不快。他感觉美国的非犹太裔其实并不爱看书,相比较,他来到以色列后,感受到
浓浓的学习气氛。全家人都要有集体读书时间。放下一切电子设备,一起学习《塔木德》这
样的经典,也一起讨论新的思维和新的科学知识。他说,这是犹太人的传统。全世界的犹太
人,如果按照古训,一定都是要活到老,学到老。我想起罗胖在得到上倡导的终身学习和认
知迭代。对于创业者,这是特别重要的素质。

从拉比家走出,看着天边薄薄的月色,街角吹过耶路撒冷傍晚的凉风。心中盘算着这三点心
得,想想这其实也并不新鲜呀,对于犹太人的特征,这三点可能也是一个标签化的共识。道
理早已知道,自己践行却是那样的不易。思忖着自己创业做大数据,每日团队在训练机器,
用标签化的数据样本去发现地理的规律和区域的深度特征。大数据的这套思路已经都融入我
的意识习惯了。那我来以色列,走了半天,还是要用刻板的一二三去归纳一群人?

到一个陌生的环境中,考察学习最大的意义是什么?我感觉并非是一下子就从某个商学院中
学到了天机秘籍,从某个明星企业中领悟了真知灼见,而是在这个国家这群人中,感受人文
环境。

我突然想到最近的一本新书《意会:人文科学在算法时代的力量》(Sensemaking:The
Power of the Humanities in the Age of the Algorithm)。意会什么?意会人和人之间有
意义的区别,敏感的理解不同人在不同环境中的取舍。大数据的时代是理论终结的时代。如
果用机器去对犹太人进行判断,机器按照样本把犹太人的统计指标发现出来,提取特征即可,
符合这个特征的就是犹太人精神,给个概率值,机器不会关心这个精神到底是什么,以及产
生这种精神的原因,而我们人类就是要关心这些。我们要对一个概念有清晰的定义,要解释
其中的因果联系。对犹太人标签化是一种薄数据,薄数据就是可以量化的客观数据,我来到
以色列还要获取厚数据,这其中包括:

客观的知识,跟视角无关,不论谁都得承认犹太人的平均智商全世界第一;
主观的知识,我个人的各种感受,拉比的气场,哭墙的灵魂,死海的苦涩;
共享的知识,团里成员的感受,我们每日在车上分享的对以色列文化的感知和情绪,感受虽
然无法量化,但确实是在每个人心中的真实存在;
感觉的知识,那些创业者长期在所处环境中的直观感受,遇到问题几乎是潜意识中的反应。
在那个耶路撒冷的傍晚,我似乎是“意会”到了犹太人的选择。意会的高层次是洞见,就是
在一大堆复杂的事物中,清楚的看到其中的关键所在。再想到我们极海的地理大数据创业征
途,是用机器去训练薄数据,团队用行业知识的厚数据去获得意会的本领。算法+意会,希
望我们能早日幸福的体验创业的洞见。

从入境到出境,我们每日都在体会以色列人民对中国的友好,这其中有大家都知道的原因:
二战中中国曾经救助过那些流离失所,担惊受怕的犹太人,另一方面这几年去以色列参观访
问游玩的中国人有点如过江之鲫了。我们在特拉维夫,在耶路撒冷碰到了中欧和长江商学院
的考察团。在著名的景点,在特拉维夫机场,也与多个中国旅行团擦肩而过。世界上两个最
聪明的民族,以后在一带一路上会有更多的握手。

在拉比家告别的时候,拉比的朋友——一个在希伯来大学学习商业管理并操着流利中文的年轻
人跟我们说,以后再来以色列看高科技企业要找他,防止被骗,他会帮忙判断哪些是真正的
好企业,哪些公司就是包装一下准备卖给中国老板的。我唏嘘,去以色列考察这个“生意”,
已经成为一个充分竞争的市场了。我们在街头遇到的小商户多半是会说“你好”,“我爱中
国”。有一天晚上,我们在市中心的小酒屋外复盘一天的心得,有对年轻的恋人在隔桌,主
动跟我说:我能说几句中文。我调侃他:你挑一句你认为说的最地道的,他说:马马虎虎。
我哈哈大笑,心中却也一沉,难道我们中国人就是这个形象?好吧,向以色列创业者学习。
也祝愿中国所有的创业者和我一样,创业不马虎!

王昊
极海纵横CEO
极海纵横是专注大数据的新型地理信息公司。基于SaaS的Geohey地理平台,不仅可以使得传
统的地理数据处理、挖掘做到易用、移动互联化,还可以灵活的私有部署在用户企业系统上。
平台汇集了极海生产的中国最齐全、厚度最深的地理数据,帮助用户实现先进的商业智能,
大数据位置分析。在Facebook、Uber等巨头大力投资空间信息技术的今天,极海已经做到国
内最先行。
核心竞争优势包括:
1、全球级别地理数据,包括人口,自然,商业等16大类,1000多小类别,总条目超过8亿,
每天更新的数量超过300万条;
2、熟练掌握决策树,高效线性回归,深层卷积神经网络算法,全球领先的用于地理分析,
遥感影像识别等机器学习成果,帮助用户实现智能化的决策;
3、团队大部分成员多年服务于世界最大的地理信息软件公司esri,对地理信息,遥感,定
位等技术的理解和业务实践具有丰富的经验。

\section{大数据}
\label{sec:orgccf22fa}
\subsection{{\bfseries\sffamily DONE} 大数据时代的思维变革}
\label{sec:org7c4a47b}
\url{https://wenku.baidu.com/view/376226a276eeaeaad0f33014.html?re=view}
大数据如今成了一个炙手可热的词汇,成了各行各业的人们热烈谈论的话题。种种迹象表明,
大数据正向我们扑面而来,世界正急速地被推入大数据时代。因此,许多有识之士都急速呼
吁要热情拥抱“大数据时代” 。随着大数据时代的来临,我们的生产、生活、工作和思维
方式诸多方面都将进行大变革,我们将一改往日的小数据思 维和眼光,迅速以大数据思维
和视角来看待世界,看待社会和生活。 一、大数据时代的来临 20世纪80年代,以预测未来
而著称的美国未来学家阿尔文·托夫勒在其《第三次浪潮》中就曾经预测, 21世纪前后,人
类将进入信息时代,信息将成为物质、能量之后的第三个世界构成要素,并用极其煽动性的
语言描绘了信息时代的生产、生活、工作和学习等各方面的变革

[1] 。当时大多数 人都认为这是一个十分遥远的乌托邦。然而,仅仅几年功夫,随着计
算机的快速更新换代,世界就被托夫勒所说的“第三次浪潮”所席卷,被急速推入了信息时
代。 20世纪80年代以来,计算机的硬件和软件都按摩尔定律迅速发展 [2]39 。硬件体积
越来越小,但 功能越来越强大;软件迅速升级,并被模块化、智能化,计算机被迅速普及
到各行各业,渗透到生活的方方面面。由于计算机以处理离散数据见长,因此凡需计算机处
理的东西都必须用离散数据来表示,所涉对象也必须被编码成结构化数据。由于计算机及其
他智能设备的普及,由其采集的各类数据以铺天盖地之势爆发出来,在国际互联网的推波助
澜下,这些爆炸性增长的数据又成了公共数据。这些海量、杂乱的数据以前被看作无用而又
占据存储空间的“垃圾”,随着数据挖掘和处理技术的发展,这些“数据垃圾”迅速变废为
宝,成了炙手可热的资源。那些先知先觉的吃螃蟹者靠这些资源一夜暴富,成了时代的新宠
和标杆。在这些“数据富豪”的示范和引领下,“数据”变成了一种继物质、能源之后的宝
贵资源,占有数据就等于占有了财富。于是,各种数据都被收集和存储,数据规模爆炸式增
长,形成了数据的海洋。这些海量数据与小数据时代的寥寥数据相比简直不可
同日而语,因此被称为 “大数据”。大数据一词来源于英文Bigdata,用来指称“那些大小
已经超出了传统意义上的尺度,一般的软件工具难于捕捉、存储、管理和分析的数据” [2]
57 。据百度百科,“大数据”这个术语最早期 的使用可追溯到apacheorg的开源项目Nutch。
当时,大数据用来描述为更新网络搜索索引需要同时进行批量处理或分析的大量数据集。随
着GoogleMapReduce和GoogleFileSystem(GFS)的发布,大数据不再仅用来描述大量的数
据,还涵盖了处理数据的速度。不过,大数据被广泛传播,主要归因于美国麦肯锡公司。
2012年初,全球知名的咨询公司麦肯锡最早使用今天被大家理解的“大数据”概念,用来指
称数据量特别巨大,超过PE级别(1015~1018字节)并包括结构性、半结构性和非结构性的
数据
从某种程度上说, 大数据主要是数据分析的前沿技术。简言之,从各种各样类型的数据中,
快速获得有价值信息的能力,就是大数据技术。这也是大数据的概念一提出来就一呼百应的
原因,因为它属于技术,具有巨大的商业价值,具有促使该技术走向众多商业应用的潜力。
大数据是一个总称性的概念,它还可以细分为大数据科学、大数据技术、大数据工程和大数
据应用等领域。目前我们所说的大数据更多局限于大数据技术和大数据应用,而对大数据科
学和工程则还未重视。大数据科学关注大数据网络发展和运营过程中,发现和验证大数据的
规律及其与自然和社会活动之间的关系,而大数据工程指大数据的规划建设、运营管理的系
统工程。 有人把大数据的特点归纳为4个层面,并被 简称为4“V”[3]7 :第一,Volume
(大量),即数据数量巨大。从TB级别,跃升到PB级别(1TB=10 12 bt,1PB=1015bt)。第
二,Variety(多样),即数据类型繁多。除了标准化的结构化编码数据之外,还包括网络
日志、视频、图片、地理位置信息等等非结构化或无结构数据。第三,Velocity(高速),
即处理速度快,实时在线。各种数据基本上可以做到实时、在线,并能够进行快速的处理、
传送和存储,以便全面反映对象的当下状况。第四,Value(价值),即商业价值高,但价
值密度低。以视频为例,在连续不间断的监控过程中,可能有用的数据
仅仅有一两秒。 二、大数据引发的思维方式变革 “大数据开启了一次重大的时代转型。就
像望远镜让我们能够感受宇宙,显微镜让我们能够观测微生物一样,大数据正在改变我们的
生活以及理解世界的方式,成为新发明新服务的源泉,而更多的改变正蓄势待发。 ”[4]
1 大数据正在改变我们的一切,其中最重要的是从改变我们的思维方式开始,引发思维大变
革,并带来所谓的“大数据思维” 。所谓思维方式,就是我们大脑活动的内在程序,是一
种习惯性的思考问题和处理问题的模式,它涉及我们看待事物的角度、方式和方法,并由此
对我们的行为方式产生直接的影响。任何人都生活在一定的时代和环境中,其思考问题和解
决问题的习惯和模式都会受到时代和环境的影响,并
由此决定他怎样观察和理解这个世界。例如,文艺复兴以来,由于牛顿力学的巨大成功,人
们就用牛顿力学来看待一切,似乎世界就像一台巨大的机器,完全可以用牛顿力学的三大定
律和万有引力定律来认识和解释一切现象,以至于活生生的人类自身也变成了“机器”,这
就是著名的机械论思维方式。 随着Google、百度、腾讯、淘宝等网络公司的迅速崛起以及
他们的迅速致富,数据致富成了新的致富神话。山西的煤老板、王石等房地产商、拥有数百
万一线工人的富士康公司等,费了九牛二虎之力才取得亿万财富,而这些网络数据商则在短
短的几年时间就迅速超越了这些实体公司的财富,并且所费人力、物力和财力甚少。人们现
在才如梦方醒,知道了数据在我们这个时代成了最重要的资源之一。数据就是资源,数据就
是财富成了迅速深入人心的理念。一切皆用数据来观察,一切都用数据来刻画,一切数据也
被当作财富来采集、存储和交易,这就是所谓的“数字化生存”。“大数据是人们获得新的
认知、创造新的价值的源泉;大数据还是改变市场、组织机构,以及政府与公民关系的方法。
”[4]9 人们迅速地以数据的眼光来观察世界和理解、解释这个纷繁复杂的世界,这就是
所谓的大数据思维。按照舍恩伯格的说法:“所谓大数据思维,是指一种意识,认为公开的
数据一旦处理得当就能为千百万人急需解决的问题提供答案。 ”[4]167
曾几何时, 数据只是刻画世界的一种方便符号,而如今却成了财富,甚至有人提出世界的
本质就是数据。因此,随着大数据时代的来临,人类的思维方式必然会产生革命性的变革。
这些变革主要表现在如下几个方面: 第一,整体性,即用整体的眼光看待一切,由原来时
时处处强调部分到如今强调“一个都不能少”,不能只有精英,而其他只能“被代表”。西
方科学从古希腊开始就有寻找“始基”的传统,以牛顿力学为代表的近代科学家们更是擅长
分割整体,不断还原,通过研究作为基本构件的部分来把握整体行为,由此形成了西方科学
的还原论传统。在还原论眼中,万事万物都可以分解为部分,部分比整体更加重要,只要把
握了部分,整体就尽在掌握之中。这些部分也被称为要素,而整体则被称为系统。之所以重
视部分,原因当然无非有两个:一是当时的科学还处于刚刚开始的阶段,通过简
单的分解就可以取得丰硕的成果;二是当时的处理能力还不足以把握复杂的整体,于是采取
迂回的办法,通过分解为更简单的部分来把握复杂的整体。当整体只由简单的几个部分组成
时,当然其所有部分都会被详细研究。但当整体由众多的部分构成时,由于处理能力所限,
不可能对所有部分进行研究,于是只能选取其中的一些部分,试图通过这些部分来代表全部,
这就是统计学中十分著名的样本研究法。为了让这些部分能够代表整体,就有了如何科学抽
样的研究。但是,无论如何科学抽样,都有可能走样,部分都未必能够代表整体。于是就有
了以系统科学和复杂性研究为代表的整体论兴起以及中国古代整体论的复兴。但无论是西方
现代整体论还是中国古代的整体论,其整体都是抽象的整体,无法进行技术操作,只停留在
抽象的概念层面。随着大数据的兴起,整体和部分终于走向了统一。大数据理论承认整体是
由部分组成的,但面对大数据,我们不能用抽样的方法只研究少量的部分,而让其他众多的
部分“被代表”。在大数据研究中,我们不再进行随机抽样,而是对全体数据进行研究。正
如维克托所说:“要分析与某事物相关的所有数据,而不是依靠分析 少量的数据样本。 ”
[4]29 “当数据处理技术已经发生了翻天覆地的变化时,在大数据时代进行抽样分析就像
在汽车时代骑马一样。一切都改变了,我 们需要的是所有的数据,‘样本=总体’。”[4]
27 大数据技术将整体论的“整体”落到了实处,整体不再是抽象的整体,而是可以进行具
体操作的整体,而且能够真正体现整体的行为。在大数据时代,不再有“被代表”,整体真
正体现了全部,反映了所有的细节。 第二,多样性,即承认世界的多样性和差异性,由原
来的典型性和标准化到如今的“怎样都行”,一切都有存在的理由,真正做到了“存在的就
是合理的”。在小数据时代,人们获取数据和处理数据都不是那么容易,因此要求每个数据
都必须精确和符合要求,或者说按照某个格式或标准来采集统一结构标准的数据。例如我们
的手机号码、身份证号码都是统一格式的,在人口普查、经济普查等各种普查中,都严格要
求按照标准化的格式登记和填写。一旦产生非标准的数据就会当做无用数据而被排除。在计
算机的数据结构中,这些标准化的数据叫做结构化数据。然而,在大数据时代,随时随地都
在产生各类数据,而且这些数据没有统一要求或标准, 五花八门。按大数据的视野看来,
这些数据虽然没有标准化,但依然是宝贵的资源,无论是标准的还是非标准的数据都有其存
在的理由 。“我们乐于接受数据的纷繁复杂,而不再追求精确性。 ”[4]29 科学哲学家
费耶尔阿 本德认为,在科学方法上应该提倡无政府主义,没有标准 ,“怎么都行”。大数
据真正体现了这种科学方法论,也体现了德国哲学家的思想:凡存在的都是合理的,这些数
据既然产生并已经存在,就有其存在的理由,就有其合理性。大数据时代真正体现了百花齐
放的多样性,而不再是小数据时代的单调乏味的统一性。 第三,平等性,即各种数据具有
同等的重要性,由原来的金字塔式结构变成了平起平坐的平等结构,强调了民主和平等。任
何系统都有其组成结构,组成系统的各种要素按照某种结构组织起来而形成系统。在还原论
的影响下,小数据时代的科学技术特别强调系统的层次结构,钟情于金字塔式的、不平等的
等级结构,由此来强调系统要素之间的不平等性。在等级结构中,我们可以像剥洋葱一样层
层剥离,通过层层还原来不断揭示出要素之间的关系,并强调金字塔顶的基础作用以及上下
级的领导关系。在大数据的海量数据中,所有的数据更多地是处于平等关系,因此不会特别
突出某些数据的关键作用。在大数据时代,群众成了真正的英雄,而不再过分强调精英和英
雄的突出地位。 第四,开放性,即一切数据都对外开放,没有数据特权,从原来的单位利
益、个人利益变为全民共享。封闭导致混沌和腐败,开放则带来有序和生机。由于处理能力
的限制,以往的科学在对研究对象进行研究时,都要把对象与环境隔离开来,就像牛顿力学
在做力学分析时那样,这种分离、封闭的方法也深深地影响了我们的思维方式。在社会生活
中,我们也是把社会划分为不同的部门或利益共同体,整个社会就由大大小小诸多的部门或
利益共同体构成。为了自身的利益,各利益共同体都各自为政,不愿意把信息对外公布和分
享。当然,在以往的社会,即使想跟大众分享,也没有实现分享的技术途径。在大数据时代,
互联网、云技术等信息技术为我们提供了便捷的共享手段。遍地可见的电脑、智能手机、摄
像头以及其他诸多的信息采集设备和存储设备将海量数据置于公共空间,为公众共享信息提
供了基础。因此,大数据时代是一个开放的时代,一切都被置于“第三只眼”中,太阳底下
无隐私,分享、共享成了共识,传统的小集团利益被打破,社会成了一个透明、公开的社会。
这也符合大众的期望,因为大众就希望通过公开透明来消除因封闭、封锁而导致的腐败,开
放、共享带来社会经济的勃勃生机。 第五,相关性,即关注数据间的关联关系,从原来凡
事皆要追问“为什么”到现在只关注“是什么”,相关比因果更重要,因果性不再被摆在首
位。西方科学传统中,因果性是各门学科关注的核心,古希腊哲学家所谓的本源问题其实就
是因果关系问题,物理、化学、生物等学科所得到的所谓规律无非就是各种因果关系而已。
在传统科学中,由于科学工具和处理能力所限,只能寻找和处理简单的几个量之间的线性关
系。因为每个数据得来不易,所以几乎没有冗余数据,每个量总能找到其前因后果,因而形
成一个长长的因果关系链。但是,在大数据时代,由于数据量特别巨大,几乎都是海量,要
找出所有量与量之间的因果关系几乎是不可能的,因此只好把它们封装起来作为一个黑箱,
我们只关注这个黑箱的宏观行为,不甚关注其内部机制。我们通过比对来发现数据之间的相
关关系,找到宏观行为中具有显著相关的数据之间的变化关系。由于这些相关数据之间在黑
箱内经过了十分复杂的相互作用,不再是小数据时代的简单、直接的线性因果关系,而是复
杂、间接的非线性因果关系,因此大数据时代的相关关系比因果关系更重要。正如维克托所
说 :“我们的思想发生了转变,不再探求难于捉摸的因果关系,转而 关注事物的相关关
系。”[4]29 因此,大数据时代打破 了小数据时代的因果思维模式,带来了新的关联思
维模式。 第六,生长性,即数据随时间不断动态变化,从原来的固化在某一时间点的静态
数据到现在的随时随地采集的动态数据,在线地反映当下的动态和行为,随着时间的演进,
系统也走向动态、适应。在小数据时代,采集的数据都是某个时间点的静态数据,比如传统
的人口普查,必须规定在某时点开始普查,经历一段时间到某个时点结束,然后用几年的时
间来处理得来的静态数据。这些静态的人口数据不能及时反映出每时每刻人口生生死死的动
态变化,而是具有很长的时滞性,因此不能反映人口的实际状况。在大数据时代,由于基本
上可以做到在线采集数据,并能够迅速处理和反映当下的状态, 因此能够反映出实际的状
态。大数据时代的最大特点就是采用各种智能数据采集设备,随时随地采集到各种即时数据,
并通过网络及时传输,通过云存储或云计算进行即时处理,基本上不会滞后。此外,由于大
数据时代采集、存储、传输、处理、使用数据的便捷性,因此我们可以做到不断更新数据。
这些随时间流不断更新的数据正好反映了数据随时间的动态演化过程,这个过程构成了一幅
动态演化全景图。这种动态演化图景正好反映了数据的生长性。此外,系统可以根据即时的
动态信息来随时调整系统的行为,从而体现出系统的适应性。 三、大数据思维是一种复杂
性思维 大数据思维从诸多方面都体现了思维方式的重大变革,代表着思维发展的新方向
[5] 。不过,顺 着时间的脉络和思维的逻辑,我们很快就会发现大数据思维与世纪之交
兴起的复杂性科学和复杂性思维具有极大的相似性,更极端一点来说,大数据思维从本质上
来说就是复杂性思维。 复杂性思想古已有之,古希腊的亚里士多德以及整个古代哲学都具
有复杂性思想。黑格尔和马克思、恩格斯更是以辩证法的哲学形式加以表达,但复杂性科学
却一直等到20世纪90年代才兴起。美国三位诺贝尔奖获得者因为不满现代科学的学科分裂,
在新墨西哥州发起成立圣菲研究所(SFI),以便弥合学科裂缝,整合科学资源,特别是试
图从思维方式和科学方法论上超越长期以来占统治地位的机械思维和还原论方法。所谓复杂
性科学,并不属于某一门新学科,而是一种科学新思维和新方法论 [6] 。复杂性科学认
为,自然界和 人类社会都纷繁复杂,并不像牛顿力学等近现代科学所认为的那样简单。大
自然和人类的思维、行为并不完全严格按照线性因果关系来组织和行动,更多情况是随机、
自由或非线性、多样性的。传统的机械自然观和还原方法论把一切对象都当作一架静止的机
器,可以随意拆卸和组装,而且最终可以还原成某个基本原件。复杂性科学则持一种有机自
然观,把一切对象都看作是有生命的、会生成演化的系统。即使是最简单的几个要素经过非
线性相互作用,都有可能涌现出复杂的行为。正因如此,我们不能根据简单因果关系来推导
系统的行为。这也就是说,因为非线性相互作用,简单要素经过分岔、突变,会涌现出复杂
多样的斑斓世界。 牛顿力学、爱因斯坦相对论等传统的学科都基本上基于机械思维和还原
方法论,因此全部被称为简单性科学。简单性科学与复杂性科学在世界观、本体论、认识论
与方法论等诸多方面都有着革命性的差别。用美国科学哲学家托马斯·库恩的话来说,它们
属于不同的科学范式,而且相互的通约性比较小。也就是说,从简单性科学到复杂性科学,
是科学范式的不同转换,是典型的科学革命,其本体信念、认识趣向、共有价值、方法特性
和符号通式诸多方面都发生了根本的变化(见表1) [7] 。


表1所描述的从简单性科学到复杂性科学的5个维度的转变几乎也都适合用来描述从小数据时
代到大数据时代的转变。我们已经知道,大数据思维具有整体性、多样性、平等性、开放性、
相关性和生长性等特征,这些特性其实正好就是复杂性科学的典型特征。因此,可以得出结
论说,简单性科学与复杂性科学、小数据时代与大数据时代
具有某种平行性和对应性, 小数据属于简单性科学,而大数据属于复杂性科学。由此不难
看出,大数据的思维变革是科学范式从简单性科学走向复杂性科学的反映,而大数据思维从
本质上来说就是一种复杂性思维 [8] 。 可以说,小数据时代属于简单性科学时代,而大
数据时代属于复杂性科学时代,它们之间有时重叠交叉,有时各自发展。数据观的变革主要
与信息科学、信息论、计算科学以及人工智能相关。随着计算机技术、网络技术的发展,数
据处理的技术和能力有了翻天覆地的变化,从而引起了从小数据到大数据的革命性变革。可
以说,数据观的革命主要是因为技术革命引起的,因而大数据最突出的表现是数据处理技术
的革命性突破。正因为如此,大数据技术对百姓的生活、工作与思维产生了巨大的影响。从
简单性科学到复杂性科学的科学观变革主要与系统科学、系统论以及其他科学相关,它更多
属于科学思想观念和哲学思维等理念层次的变革,因而更多表现在各门学科的科学观念的革
命转变上。因此,科学观从简单性到复杂性的变革虽然也是一场革命,但它对生产、经济,
以及百姓的日常生活影响没有那么巨大,主要局限于科学和哲学等学术领域。 由此,我们
可以说,从简单性科学到复杂性科学的革命,与从小数据时代到大数据时代在本质上是相通
的,不过前者更多地表现在科学层面,而后者主要表现在技术层面;前者更多局限在思想领
域,后者则直接对我们的生产、生活和思维产生了全方位的影响。因此,大数据技术革命与
复杂性科学革命既有区别又有联系,但它们在思维变革方面是基本一致的。 四、结束语 当
前正在轰轰烈烈兴起的大数据革命是一场
影响巨大的科学技术革命,它必将改变世界,影响深远,必将使我们的学习、工作与生活彻
底改观,使我们的思维方式产生彻底的变革。大数据思维体现了复杂性科学的思维方式,并
且用最先进的数据采集、存储、传递和使用的技术让这种新思维得到全方位的落实,并带来
大机遇、大挑战、大变革,最终“从大数据走向大社会” [2]308 。在呼啸而 来的大数
据时代,一切坚固的东西正在烟消云散。大数据正在不断重塑我们的社会以及我们看待世界
的方式。因此,不管愿意与否,我们都必将被大数据时代的滚滚洪流席卷,要么成为一个弄
潮儿,要么彻底被时代淘汰。

参考文献:
[1]阿尔文·托夫勒.第三次浪潮[ M].北京:中信出版社,2006:83-85.
[2] 涂子沛.大数据[M].桂林:广西师范大学出版社, 2013.[3]李德伟.大数据改
变世界[M].北京:电子工业出版 社, 2013.[4]维克托·舍恩伯格,肯尼斯·库克耶.
大数据时代 [M].杭州:浙江人民出版社,2013. [5]LucianoFloridi.
Bigdataandtheirepistemologicalchal-lenge[J].PhilosTechnol,2012(25):435-437.
[6]黄欣荣.复杂性科学的方法论研究[M].重庆:重庆大 学出版社, 2011.[7]黄欣
荣.复杂性科学与中医[J].中医杂志,2013(19): 1621-1626. [8]艾伯特·巴拉巴西.
爆发:大数据时代预见未来的新 思维[ M].北京:中国人民大学出版社,2012:245






\subsection{{\bfseries\sffamily DONE} 大数据思维的十大原理}
\label{sec:org79bb11c}
\url{https://wenku.baidu.com/view/672dd09514791711cc7917c7.html}

\subsection{一、数据核心原理}
\label{sec:orgfcab892}

\subsubsection{从“流程”核心转变为“数据”核心}
\label{sec:org6bb7e2e}

大数据时代,计算模式也发生了转变,从“流程”核心转变为“数据”核心。Hadoop体系的
分布式计算框架已经是“数据”为核心的范式。非结构化数据及分析需求,将改变IT系统的
升级方式:从简单增量到架构变化。大数据下的新思维——计算模式的转变。

例如:IBM将使用以数据为中心的设计,目的是降低在超级计算机之间进行大量数据交换的
必要性。大数据下,云计算找到了破茧重生的机会,在存 储和计算上都体现了数据为核心
的理念。大数据和云计算的关系:云计算为大数据提供了有力的工具和途径,大数据为云
计算提供了很有价值的用武之地。而大数据比云计算更为 落地,可有效利用已大量建设的
云计算资源,最后加以利用。


科学进步越来越多地由数据来推动,海量数据给数据分析既带来了机遇,也构成了新的
挑战。大数据往往是利用众多技术和方法,综合源自多个渠道、不同时间的信息而获得的。
为了应对大数据带来的挑战,我们需要新的统计思路和计算方法。

\subsubsection{说明:}
\label{sec:org493455c}
用数据核心思维方式思考问题,解决问题。以数据为核心,反映了当下IT产业的变革,
数据成为人工智能的基础,也成为智能化的基础,数据比流程更重要,数据库、记录数据库,
都可开发出深层次信息。云计算机可以从数据库、记录数据库中搜索出你是谁,你需要什么,
从而推荐给你需要的信息。

\subsection{二、数据价值原理}
\label{sec:org466447e}

\subsubsection{由功能是价值转变为数据是价值}
\label{sec:orgbc30561}

大数据真正有意思的是数据变得在线了,这个恰恰是互联网的特点。非互联网时期的产品,
功能一定是它的价值,今天互联网的产品,数据一定是它的价值。

例如:大数据的真正价值在于创造,在于填补无数个还未实现过的空白。有人把数据比喻为
蕴藏能量的煤矿,煤炭按照性质有焦煤、无烟煤、肥煤、贫煤等分类,而露天煤矿、深山煤
矿的挖掘成本又不一样。与此类似,大数据并不在“大”,

而在于“有用”,价值含量、挖掘成本比数量更为重要。不管大数据的核心价值是不是预测,
但是基于大数据形成决策的模式已经为不少的企业带来了盈利和声誉。   数据能告诉我们,
每一个客户的消费倾向,他们想要什么,喜欢什么,每个人的需求有哪些区别,哪些又可以
被集合到一起来进行分类。大数据是数据数量上的增加,以至于我们能够实现从量变到质变
的过程。举例来说,这里有一张照片,照片里的人在骑马,这张照片每一分钟,每一秒都要
拍一张,但随着处理速度越来越快,从1分钟一张到1秒钟1张,突然到1秒钟10张后,就产生
了电影。当数量的增长实现质变时,就从照片变成了一部电影。   美国有一家创新企业
Decide.com   它可以帮助人们做购买决策,告诉消费者什么时候买什么产品,什么时候买
最便宜,预测产品的价格趋势,这家公司背后的驱动力就是大数据。他们在全球各大网站上
搜集数以十亿计的数据,然后帮助数以十万计的用户省钱,为他们的采购找到最好的时间,
降低交易成本,为终端的消费者带去更多价值。

在这类模式下,尽管一些零售商的利润会进一步受挤压,但从商业本质上来讲,可以把钱更
多地放回到消费者的口袋里,让购物变得更理性,这是依靠大数据催生出的一项全新产业。
这家为数以十万计的客户省钱的公司,在几个星期前,被eBay以高价收购。

再举一个例子,SWIFT是全球最大的支付平台,在该平台上的每一笔交易都可以进行大数据
的分析,他们可以预测一个经济体的健康性和增长性。比如,该公司现在为全球性客户提供
经济指数,这又是一个大数据服务。,定制化服务的关键是数据。《大数据时代》的作者维
克托·迈尔·舍恩伯格认为,大量的数据能够让传统行业更好地了解客户需求,提供个性化的
服务。

\subsubsection{说明:}
\label{sec:org978108b}
用数据价值思维方式思考问题,解决问题。信息总量的变化导致了信息形态的变化,
量变引发了质变,最先经历信息爆炸的学科,如天文学和基因学,创造出了“大数据”这个
概念。如今,这个概念几乎应用到了所有人类致力于发展的领域中。从功能为价值转变为数
据为价值,说明数据和大数据的价值在扩大,数据为“王”的时代出现了。数据被解释是信
息,信息常识化是知识,所以说数据解释、数据分析能产生价值。

\subsection{三、全样本原理}
\label{sec:orgd1d0c4e}

\subsubsection{从抽样转变为需要全部数据样本}
\label{sec:org8e224f0}

需要全部数据样本而不是抽样,你不知道的事情比你知道的事情更重要,但如果现在数据足
够多,它会让人能够看得见、摸得着规律。数据这么大、这么多,所以人们觉得有足够的能
力把握未来,对不确定状态的一种判断,从而做出自己的决定。这些东西我们听起来都是非
常原始的,但是实际上背后的思维方式,和我们今天所讲的大数据是非常像的。

举例:在大数据时代,无论是商家还是信息的搜集者,会比我们自己更知道你可能会想干什
么。现在的数据还没有被真正挖掘,如果真正挖掘的话,通过信用卡消费的记录,可以成功
预测未来5年内的情况。统计学里头最基本的一个概念就是,全部样本才能找出规律。为什
么能够找出行为规律?一个更深层的概念是人和人是一样的,如果是一个人特例出来,可能
很有个性,但当人口样本数量足够大时,就会发现其实每个人都是一模一样的。

\subsubsection{说明:}
\label{sec:org83f0769}
用全数据样本思维方式思考问题,解决问题。从抽样中得到的结论总是有水分的,而
全部样本中得到的结论水分就很少,大数据越大,真实性也就越大,因为大数据包含了全部
的信息。

\subsection{四、关注效率原理}
\label{sec:orgea8f562}

\subsubsection{由关注精确度转变为关注效率}
\label{sec:orgc37e383}

关注效率而不是精确度,大数据标志着人类在寻求量化和认识世界的道路上前进了一大步,
过去不可计量、存储、分析和共享的很多东西都被数据化了,拥有大量的数据和更多不那么
精确的数据为我们理解世界打开了一扇新的大门。大数据能提高生产效率和销售效率,原因
是大数据能够让我们知道市场的需要,人的消费需要。大数据让企业的决策更科学,由关注
精确度转变为关注效率的提高,大数据分析能提高企业的效率。

例如:在互联网大数据时代,企业产品迭代的速度在加快。三星、小米手机制造商半年就推
出一代新智能手机。利用互联网、大数据提高企业效率的趋势下,快速就是效率、预测就是
效率、预见就是效率、变革就是效率、创新就是效率、应用就是效率。
竞争是企业的动力,而效率是企业的生命,效率低与效率高是衡量企来成败的关键。一般来
讲,投入与产出比是效率,追求高效率也就是追求高价值。手工、机器、自动机器、智能机
器之间效率是不同的,智能机器效率更高,已能代替人的思维劳动。智能机器核心是大数据
制动,而大数据制动的速度更快。在快速变化的市场,快速预测、快速决策、快速创新、快
速定制、快速生产、快速上市成为企业行动的准则,也就是说,速度就是价值,效率就是价
值,而这一切离不开大数据思维。

\subsubsection{说明:}
\label{sec:orgd949e51}
用关注效率思维方式思考问题,解决问题。大数据思维有点像混沌思维,确定与不确
定交织在一起,过去那种一元思维结果,已被二元思维结果取代。过去寻求精确度,现在寻
求高效率;过去寻求因果性,现在寻求相关性;过去寻找确定性,现在寻找概率性,对不精
确的数据结果已能容忍。只要大数据分析指出可能性,就会有相应的结果,从而为企业快速
决策、快速动作、创占先机提高了效率。

\subsection{五、关注相关性原理}
\label{sec:org50bc63e}

\subsubsection{由因果关系转变为关注相关性}
\label{sec:org90179de}

竞争是企业的动力,而效率是企业的生命,效率低与效率高是衡量企来成败的关键。一般来
讲,投入与产出比是效率,追求高效率也就是追求高价值。手工、机器、自动机器、智能机
器之间效率是不同的,智能机器效率更高,已能代替人的思维劳动。智能机器核心是大数据
制动,而大数据制动的速度更快。在快速变化的市场,快速预测、快速决策、快速创新、快
速定制、快速生产、快速上市成为企业行动的准则,也就是说,速度就是价值,效率就是价
值,而这一切离不开大数据思维。

说明:用关注效率思维方式思考问题,解决问题。大数据思维有点像混沌思维,确定与不确
定交织在一起,过去那种一元思维结果,已被二元思维结果取代。过去寻求精确度,现在寻
求高效率;过去寻求因果性,现在寻求相关性;过去寻找确定性,现在寻找概率性,对不精
确的数据结果已能容忍。只要大数据分析指出可能性,就会有相应的结果,从而为企业快速
决策、快速动作、创占先机提高了效率。

关注相关性而不是因果关系,社会需要放弃它对因果关系的渴求,而仅需关注相关关系,也
就是说只需要知道是什么,而不需要知道为什么。这就推翻了自古以来的惯例,而我们做决
定和理解现实的最基本方式也将受到挑战。

例如:大数据思维一个最突出的特点,就是从传统的因果思维转向相关思维,传统的因果思
维是说我一定要找到一个原因,推出一个结果来。而大数据没有必要找到原因,不需要科学
的手段来证明这个事件和那个事件之间有一个必然,先后关联发生的一个因果规律。它只需
要知道,出现这种迹象的时候,我就按照一般的情况,这个数据统计的高概率显示它会有相
应的结果,那么我只要发现这种迹象的时候,我就可以去做一个决策,我该怎么做。这是和
以前的思维方式很不一样,老实说,它是一种有点反科学的思维,科学要求实证,要求找到
准确的因果关系。

在这个不确定的时代里面,等我们去找到准确的因果关系,再去办事的时候,这个事情早已
经不值得办了。所以“大数据”时代的思维有点像回归了工业社会的这种机械思维——机械思
维就是说我按那个按钮,一定会出现相应的结果,是这样状态。而农业社会往前推,不需要
找到中间非常紧密的、明确的因果关系,而只需要找到相关关系,只需要找到迹象就可以了。
社会因此放弃了寻找因果关系的传统偏好,开始挖掘相关关系的好处。
例如:美国人开发一款“个性化分析报告自动可视化程序”软件从网上挖掘数据信息,这款
数据挖掘软件将自动从各种数据中提取重要信息,然后进行分析,并把此信息与以前的数据
关联起来,分析出有用的信息。

非法在屋内打隔断的建筑物着火的可能性比其他建筑物高很多。纽约市每年接到2.5万宗有
关房屋住得过于拥挤的投诉,但市里只有200名处理投诉的巡视员,市长办公室一个分析专
家小组觉得大数据可以帮助解决这一需求与资源的落差。该小组建立了一个市内全部90万座
建筑物的数据库,并在其中加入市里19个部门所收集到的数据:欠税扣押记录、水电使用异
常、缴费拖欠、服务切断、救护车使用、当地犯罪率、鼠患投诉,诸如此类。

接下来,他们将这一数据库与过去5年中按严重程度排列的建筑物着火记录进行比较,希望
找出相关性。果然,建筑物类型和建造年份是与火灾相关的因素。不过,一个没怎么预料到
的结果是,获得外砖墙施工许可的建筑物与较低的严重火灾发生率之间存在相关性。利用所
有这些数据,该小组建立了一个可以帮助他们确定哪些住房拥挤投诉需要紧急处理的系统。
他们所记录的建筑物的各种特征数据都不是导致火灾的原因,但这些数据与火灾隐患的增加
或降低存在相关性。这种知识被证明是极具价值的:过去房屋巡视员出现场时签发房屋腾空
令的比例只有13\%,在采用新办法之后,这个比例上升到了70\%——效率大大提高了。
全世界的商界人士都在高呼大数据时代来临的优势:一家超市如何从一个17岁女孩的购物清
单中,发现了她已怀孕的事实;或者将啤酒与尿不湿放在一起销售,神奇地提高了双方的销
售额。大数据透露出来的信息有时确实会起颠覆。比如,腾讯一项针对社交网络的统计显示,
爱看家庭剧的男人是女性的两倍还多;最关心金价的是中国大妈,但紧随其后的却是90后。
而在过去一年,支付宝中无线支付比例排名前十的竟然全部在青海、西藏和内蒙古地区。

\subsubsection{说明:}
\label{sec:orge1c2f2d}

用关注相关性思维方式来思考问题,解决问题。寻找原因是一种现代社会的一神论,
大数据推翻了这个论断。过去寻找原因的信念正在被“更好”的相关性所取代。当世界由探
求因果关系变成挖掘相关关系,我们怎样才能既不损坏建立在因果推理基础之上的社会繁荣
和人类进步的基石,又取得实际的进步呢?这是值得思考的问题。

解释:转向相关性,不是不要因果关系,因果关系还是基础,科学的基石还是要的。只是在
高速信息化的时代,为了得到即时信息,实时预测,在快速的大数据分析技术下,寻找到相
关性信息,就可预测用户的行为,为企业快速决策提供提前量。

比如预警技术,只有提前几十秒察觉,防御系统才能起作用。比如,雷达显示有个提前量,
如果没有这个预知的提前量,雷达的作用也就没有了,相关性也是这个原理。比如,相对论
与量子论的争论也能说明问题,一个说上帝不掷骰子,一个说上帝掷骰子,争论几十年,最
后承认两个都存在,而且量子论取得更大的发展——一个适用于宇宙尺度,一个适用于原子尺
度。

\subsection{六、预测原理}
\label{sec:org83c7944}

\subsubsection{从不能预测转变为可以预测}
\label{sec:org65d458f}

大数据的核心就是预测,大数据能够预测体现在很多方面。大数据不是要教机器像人一样思
考,相反,它是把数学算法运用到海量的数据上来预测事情发生的可能性。正因为在大数据
规律面前,每个人的行为都跟别人一样,没有本质变化,所以商家会比消费者更了消费者的
行为。

例如:大数据助微软准确预测世界怀。微软大数据团队在2014年巴西世界足球赛前设计了世
界怀模型,该预测模型正确预测了赛事最后几轮每场比赛的结果,包括预测德国队将最终获
胜。预测成功归功于微软在世界怀进行过程中获取的大
量数据,到淘汰赛阶段,数据如滚雪球般增多,常握了有关球员和球队的足够信息,以适当
校准模型并调整对接下来比赛的预测。

世界杯预测模型的方法与设计其它事件的模型相同,诀窍就是在预测中去除主观性,让数据
说话。预测性数学模型几乎不算新事物,但它们正变得越来越准确。在这个时代,数据分析
能力终于开始赶上数据收集能力,分析师不仅有比以往更多的信息可用于构建模型,也拥有
在很短时间内通过计算机将信息转化为相关数据的技术。

几年前,得等每场比赛结束以后才能获取所有数据,现在,数据是自动实时发送的,这让预
测模型能获得更好的调整且更准确。微软世界怀模型的成绩说明了其模型的实力,它的成功
为大数据的力量提供了强有力的证明,利用同样的方法还可预测选举或关注股票。类似的大
数据分析正用于商业、政府、经济学和社会科学,它们都关于原始数据进行分析。

我们进入了一个用数据进行预测的时代,虽然我们可能无法解释其背后的原因。如果一个医
生只要求病人遵从医嘱,却没法说明医学干预的合理性的话,情况会怎么样呢?实际上,这
是依靠大数据取得病理分析的医生们一定会做的事情。
从一个人乱穿马路时行进的轨迹和速度来看他能及时穿过马路的可能性,都是大数据可以预
测的范围。当然,如果一个人能及时穿过马路,那么他乱穿马路时,车子就只需要稍稍减速
就好。但是这些预测系统之所以能够成功,关键在于它们是建立在海量数据的基础之上的。


此外,随着系统接收到的数据越来越多,通过记录找到的最好的预测与模式,可以对系统进
行改进。它通常被视为人工智能的一部分,或者更确切地说,被视为一种机器学习。真正的
革命并不在于分析数据的机器,而在于数据本身和我们如何运用数据。一旦把统计学和现在
大规模的数据融合在一起,将会颠覆很多我们原来的思维。所以现在能够变成数据的东西越
来越多,计算和处理数据的能力越来越强,所以大家突然发现这个东西很有意思。所以,大
数据能干啥?能干很多很有意思的事情。

例如,预测当年葡萄酒的品质
很多品酒师品的不是葡萄酒,那时候葡萄酒还没有真正的做成,他们品的是发烂的葡萄。因
此在那个时间点就预测当年葡萄酒的品质是比较冒险的。而且人的心理的因素是会影响他做
的这个预测,比如说地位越高的品酒师,在做预测时会越保守,因为他一旦预测错了,要损
失的名誉代价是很大的。所以的品酒大师一般都不敢贸然说今年的酒特别好,或者是特别差;
而刚出道的品酒师往往会“语不惊人死不休的”。

普林斯顿大学有一个英语学教授,他也很喜欢喝酒,喜欢储藏葡萄酒,所以他就想是否可以
分析到底哪年酒的品质好。然后他就找了很多数据,比如说降雨量、平均气温、土壤成分等
等,然后他做回归,最后他说把参数都找出来,做了个网站,告诉大家今年葡萄酒的品质好
坏以及秘诀是什么。


当他的研究公布的时候,引起了业界的轩然大波,因为他做预测做的很提前,因为今年的葡
萄收获后要经过一段的时间发酵,酒的味道才会好,但这个教授突然预测说今年的酒是世纪
最好的酒。大家说怎么敢这么说,太疯狂了。更疯狂的是到了第二年,他预测今年的酒比去
年的酒更好,连续两次预测说是百年最好的酒,但他真的预测对了。现在品酒师在做评判之
前,要先到他的网站上看看他的预测,然后再做出自己的判断。有很多的规律我们不知道,
但是它潜伏在这些大数据里头。

例如,大数据描绘“伤害图谱”

广州市伤害监测信息系统通过广州市红十字会医院、番禺区中心医院、越秀区儿童医院3个
伤害监测哨点医院,持续收集市内发生的伤害信息,分析伤害发生的原因及危险因素,系统
共收集伤害患者14681例,接近九成半都是意外事故。整体上,伤害多发生于男性,占
61.76\%,5岁以下儿童伤害比例高达14.36\%,家长和社会应高度重视,45.19\%的伤害都是发
生在家中,其次才是公路和街道。

收集到监测数据后,关键是通过分析处理,把数据“深加工”以利用。比如,监测数据显示,
老人跌倒多数不是发生在雨天屋外,而是发生在家里,尤其是旱上刚起床时和浴室里,这就
提示,防控老人跌倒的对策应该着重在家居,起床要注意不要动作过猛,浴室要防滑,加扶
手等等。

\subsubsection{说明:}
\label{sec:org6174965}

用大数据预测思维方式来思考问题,解决问题。数据预测、数据记录预测、数据统计
预测、数据模型预测,数据分析预测、数据模式预测、数据深层次信息预测等等,已转变为
大数据预测、大数据记录预测、大数据统计预测、大数据模型预测,大数据分析预测、大数
据模式预测、大数据深层次信息预测。

互联网、移动互联网和云计算机保证了大数据实时预测的可能性,也为企业和用户提供了实
时预测的信息,相关性预测的信息,让企业和用户抢占先机。由于大


数据的全样本性,人和人都是一样的,所以云计算机软件预测的效率和准确性大大提高,有
这种迹象,就有这种结果。

\subsection{七、信息找人原理}
\label{sec:org9f92909}

\subsubsection{从人找信息,转变为信息找人}
\label{sec:org1657c1f}

互联网和大数据的发展,是一个从人找信息,到信息找人的过程。先是人找信息,人找人,
信息找信息,现在是信息找人的这样一个时代。信息找人的时代,就是说一方面我们回到了
一种最初的,广播模式是信息找人,我们听收音机,我们看电视,它是信息推给我们的,但
是有一个缺陷,不知道我们是谁,后来互联网反其道而行,提供搜索引擎技术,让我知道如
何找到我所需要的信息,所以搜索引擎是一个很关键的技术。

例如:从搜索引擎——向推荐引擎转变。今天,后搜索引擎时代已经正式来到,什么叫做后搜
索引擎时代呢?使用搜索引擎的频率会大大降低,使用的时长也会大大的缩短,为什么使用
搜索引擎的频率在下降?时长在下降?原因是推荐引擎的诞生。就是说从人找信息到信息找
人越来越成为了一个趋势,推荐引擎就是说

它很懂我,知道我要知道,所以是最好的技术。乔布斯说,让人感受不到技术的技术是最好
的技术。

大数据还改变了信息优势。按照循证医学,现在治病的第一件事情不是去研究病理学,而是
拿过去的数据去研究,相同情况下是如何治疗的。这导致专家和普通人之间的信息优势没有
了。原来我相信医生,因为医生知道的多,但现在我可以到谷歌上查一下,知道自己得了什
么病。

谷歌有一个机器翻译的团队,最开始的时候翻译之后的文字根本看不懂,但是现在60\%的内
容都能读得懂。谷歌机器翻译团队里头有一个笑话,说从团队每离开一个语言学家,翻译质
量就会提高。越是专家越搞不明白,但打破常规让数据说话,得到真理的速度反而更快。

\subsubsection{说明:}
\label{sec:org18c6313}

用信息找人的思维方式思考问题,解决问题。从人找信息到信息找人,是交互时代一
个转变,也是智能时代的要求。智能机器已不是冷冰冰的机器,而是具有一定智能的机器。
信息找人这四个字,预示着大数据时代可以让信息找人,原因是企业懂用户,机器懂用户,
你需要什么信息,企业和机器提前知道,而且主动提供你需要的信息。

\subsection{八、机器懂人原理}
\label{sec:orgff53726}

\subsubsection{由人懂机器转变为机器更懂人}
\label{sec:org1d7c53f}

不是让人更懂机器,而是让机器更懂人,或者说是能够在使用者很笨的情况下,仍然可以使
用机器。甚至不是让人懂环境,而是让我们的环境来懂我们,环境来适应人,某种程度上自
然环境不能这样讲,但是在数字化环境中已经是这样的一个趋势,就是我们所在的生活世界,
越来越趋向于它更适应于我们,更懂我们。哪个企业能够真正做到让机器更懂人,让环境更
懂人,让我们随身携带的整个的生活世界更懂得我们的话,那他一定是具有竞争力的了,而
“大数据”技术能够助我们一臂之力。

例如:亚马逊网站,只要买书,就会提供一个今天司空见惯的推荐,买了这本书的人还买了
什么书,后来发现相关推荐的书比我想买的书还要好,时间久之后就会对它产生一种信任。
这种信任就像在北京的那么多书店里面,以前买书的时候就在几家,原因在于我买书比较多,
他都已经认识我了,都是我一去之后,我不说我要买什么书,他会推荐最近上来的几本书,
可能是我感兴趣的。这样我就不会到别的很近的书店,因为这家书店更懂我。
例如,解题机器人挑战大型预科学校高考模拟试题的结果,解题机器人的学历水平应该比肩
普通高三学生。计算机不擅长对语言和知识进行综合解析,但通过借助大规模数据库对普通
文章做出判断的方法,在对话填空和语句重排等题型上成绩有所提高。

让机器懂人,是让机器具有学习的功能。人工智能已转变为研究机器学习。大数据分析要求
机器更智能,具有分析能力,机器即时学习变得更重要。机器学习是指:计算机利用经验改
善自身性能的行为。机器学习主要研究如何使用计算机模拟和实现人类获取知识(学习)过
程、创新、重构已有的知识,从而提升自身处理问题的能力,机器学习的最终目的是从数据
中获取知识。

大数据技术的其中一个核心目标是要从体量巨大、结构繁多的数据中挖掘出隐蔽在背后的规
律,从而使数据发挥最大化的价值。由计算机代替人去挖掘信息,获取知识。从各种各样的
数据(包括结构化、半结构化和非结构化数据)中快速获取有价值信息的能力,就是大数据
技术。大数据机器分析中,半监督学习、集成学习、 概率模型等技术尤为重要。

\subsubsection{说明:}
\label{sec:orgb5f29a6}

用机器更懂人的思维方式思考问题,解决问题。机器从没有常识到逐步有点常识,这
是很大的变化。去年,美国人把一台云计算机送到大学里去进修,增
加知识和常识。最近俄罗斯人开发一台计算机软件通过图林测试,表明计算机已初步具有智能。
让机器懂人,这是人工智能的成功,同时,也是人的大数据思维转变。你的机器、你的软件、
你的服务是否更懂人?将是衡量一个机器、一件软件、一项服务好坏的标准。人机关系已发
生很大变化,由人机分离,转化为人机沟通,人机互补,机器懂人,现在年青人已离不开智
能手机是一个很好的例证。在互联网大数据时代,有问题—问机器—问百度,成为生活的一部
分。机器什么都知道,原因是有大数据库,机器可搜索到相关数据,从而使机器懂人。是人
让机器更懂人,如果机器更懂人,那么机器的价值更高。

\subsection{九、电子商务智能原理}
\label{sec:orge0f4459}

\subsubsection{大数据改变了电子商务模式,让电子商务更智能。}
\label{sec:org1eb4ae5}

商务智能,在今天大数据时代它获得的重新的定义。

例如:传统企业进入互联网,在掌握了“大数据”技术应用途径之后,会发现有一种豁然开
朗的感觉,我整天就像在黑屋子里面找东西,找不着,突然碰到了一个开关,发现那么费力
的找东西,原来很容易找得到。大数据思维,事实上它不是一个全称的判断,只是对我们所
处的时代某一个纬度的描述。

大数据时代不是说我们这个时代除了大数据什么都没有,哪怕是在互联网和IT领域,它也不
是一切,只是说在我们的时代特征里面加上这么一道很明显的光,从而导致我们对以前的生
存状态,以及我们个人的生活状态的一个差异化的一种表达。

例如:大数据让软件更智能。尽管我们仍处于大数据时代来临的前夕,但我们的日常生活已
经离不开它了。交友网站根据个人的性格与之前成功配对的情侣之间的关联来进行新的配对。
例如,具有“自动改正”功能的智能手机通过分析我们以前的输入,将个性化的新单词添加
到手机词典里。在不久的将来,世界许多现在单纯依靠人类判断力的领域都会被计算机系统
所改变甚至取代。计算机系统可以发挥作用的领域远远不止驾驶和交友,还有更多更复杂的
任务。别忘了,亚马逊可以帮我们推荐想要的书,谷歌可以为关联网站排序,Facebook知道
我们的喜好,而linkedIn可以猜出我们认识谁。
当然,同样的技术也可以运用到疾病诊断、推荐治疗措施,甚至是识别潜在犯罪分子上。或
者说,在你还不知道的情况下,体检公司、医院提醒你赶紧去做检查,可能会得某些病,商
家比你更了解你自己,以及你这样的人在某种情况下会出现的可能变化。就像互联网通过给
计算机添加通信功能而改变了世界,大数据也将改变我们生活中最重要的方面,因为它为我
们的生活创造了前所未有的可量化的维度。

\subsubsection{说明:}
\label{sec:orgc165d75}

用电子商务更智能的思维方式思考问题,解决问题。人脑思维与机器思维有很大差别,
但机器思维在速度上是取胜的,而且智能软件在很多领域已能代替人脑思维的操作工作。例
如美国一家媒体公司已用电脑智能软件写稿,可用率已达70\%。云计算机已能处理超字节的
大数据量,人们需要的所有信息都可得到显现,而且每个人互联网行为都可记录,这些记录
的大数据经过云计算处理能产生深层次信息,经过大数据软件挖掘,企业需要的商务信息都
能实时提供,为企业决策和营销、定制产品等提供了大数据支持。

\subsection{十、定制产品原理}
\label{sec:org07e95f5}

\subsubsection{由企业生产产品转变为由客户定制产品}
\label{sec:orge7ed542}

下一波的改革是大规模定制,为大量客户定制产品和服务,成本低、又兼具个性化。比如消
费者希望他买的车有红色、绿色,厂商有能力满足要求,但价格又不至于像手工制作那般让
人无法承担。因此,在厂家可以负担得起大规模定制带去的高成本的前提下,要真正做到个
性化产品和服务,就必须对客户需求有很好的了解,这背后就需要依靠大数据技术。

例如:大数据改变了企业的竞争力。定制产品这是一个很好的技术,但是能不能够形成企业
的竞争力呢?在产业经济学里面有一个很重要的区别,就是生产力和竞争力的区别,就是说
一个东西是具有生产力的,那这种生产力变成一种通用生产力的时候,就不能形成竞争力,
因为每一个人,每一个企业都有这个生产力的时候,只能提高自己的生产力,过去没有车的
时候和有车的时候,你的活动半径、运行速度大大提高了,但是当每一个人都没有车的时候,
你有车,就会形成竞争力。大数据也一样,你有大数据定制产品,别人没有,就会形成竞争
力。

在互联网大数据的时代,商家最后很可能可以针对每一个顾客进行精准的价格歧视。我们现
在很多的行为都是比较粗放的,航空公司会给我们里程卡,根据飞行公里数来累计里程,但
其实不同顾客所飞行的不同里程对航空公司的利润贡献是不一样的。所以有一天某位顾客可
能会收到一封信,“恭喜先生,您已经被我们选为幸运顾客,我们提前把您升级到白金
卡。”这说明这个顾客对航空公司的贡
献已经够多了。有一天银行说“恭喜您,您的额度又被提高了,”就说明钱花得已经太多了。
正因为在大数据规律面前,每个人的行为都跟别人一样,没有本质变化。所以商家会比消费
者更了消费者的行为。也许你正在想,工作了一年很辛苦,要不要去哪里度假?打开e-Mail,
就有航空公司、旅行社的邮件。

\subsubsection{说明:}
\label{sec:org9f280a6}
用定制产品思维方式思考问题,解决问题。大数据时代让企业找到了定制产品、订单
生产、用户销售的新路子。用户在家购买商品已成为趋势,快递的快速,让用户体验到实时
购物的快感,进而成为网购迷,个人消费不是减少了,反而是增加了。为什么企业要互联网
化大数据化,也许有这个原因。2000万家互联网网店的出现,说明数据广告、数据传媒的重
要性。

企业产品直接销售给用户,省去了中间商流通环节,使产品的价格可以以出厂价销售,让销
费者获得了好处,网上产品便宜成为用户的信念,网购市场形成了。要让用户成为你的产品
粉丝,就必须了解用户需要,定制产品成为用户的心愿,也就成为企业发展的新方向。
大数据思维是客观存在,大数据思维是新的思维观。用大数据思维方式思考问题,解决问题
是当下企业潮流。大数据思维开启了一次重大的时代转型。



\section{人工智能}
\label{sec:org99cc734}
\subsection{\href{http://blog.sina.com.cn/s/blog\_475b3d560102wizo.html}{为什么今天是人工智能的黄金时代?}}
\label{sec:org4b38b90}
\subsection{6月8日,}
\label{sec:org9065bb4}
应清华大学交叉信息研究院院长、世界著名计算机科学家姚期智院士邀请,向清华大学“姚
班”的同学们做了名为《人工智能的黄金时代》的演讲。姚教授是计算机界最负盛名的图灵
奖得主,2005年他与微软亚洲研究院合作在清华大学创办计算机科学实验班(简称“姚
班”),十多年来培养了一批批拔尖的创新人才。
\subsection{以下为演讲全文:}
\label{sec:org6c98db0}

谢谢大家!非常高兴有这个机会又一次来到清华,尤其是在我最尊敬的姚期智教授的邀请和
介绍之下。姚教授的姚班在全球已经享有盛名,我从Google到创新工场,看到有非常多成功
的工程师,都是在姚老师的培养之下成为了计算机界的顶尖人才。

在讲人工智能之前,我想向大家介绍一下我的一些可能不太为人熟知的背景:其实在进入几
个国际大公司任职之前,也就是在30多年前,我就进入了人工智能领域。我是在1980年首先
做的自然语言处理,1982年做的计算机视觉,1983做的语音识别,1985年做的人机对弈,
1996年做的VR/AR……但我们现在知道,那时候我的这些选择基本上都是非常“糟糕错
误”的职业选择,因为每一件事情,我都是在它的黄金时代之前、白银时代之前,甚至破铜
烂铁都不是的时代就涉足了。从这个事情上,其实我也想说,做计算机研究这个领域,本身
的素质能力当然都非常重要,但是还要在正确的时候选择正确的事情。我在错误的时候太过
狂热的跳进了人工智能领域,与此同时,过去的三四十年人工智能也是起起伏伏,一下很火,
一下又跌入谷底。

但现在是人工智能的黄金时代。可能各位也会问,凭什么这次说是人工智能的黄金时代?为
了说明这个问题,这次我肯定不只用一些理论来说服大家,毕竟我过去也做了这么多“错误
的选择”——我今天还带一些实际的数据来跟大家分享为什么我对今天的人工智能充满信心。
人工智能有很多分支,其中之一是机器学习,机器学习里面还有一个分支是深度学习,今天
我更多的会用深度学习作为案例。
\subsection{一、人工智能是一种工具}
\label{sec:org691d505}
最近人工智能成为全球热门新闻话题,很多是因为大家看到AlphaGo在几个月前击败了李世
石,最近在网上还传出年底之前它要挑战柯杰的消息。但在这个新闻的热度之下,有一点让
我觉得很可惜:大家对这个话题讨论的重心都放在了人工智能是不是在模仿人脑,“奇点”
是否即将来临这样的问题上,却没有真正关注人工智能对我们的现实影响。

“奇点”认为未来机器将有各种的智能、人类必须做一些事情来保护自己。我们在座的没有
任何一个人能够证明或否定“奇点”,但就我个人而言,我认为人工智能要取代人还是一个
非常遥远的事情。我觉得我们需要更关注的事情是人工智能是今天能够拿来用的工具,它能
帮助人类解决问题,能取代重复性的工作,能创造商业价值。正因为这个理由,我认为我们
今天进入了人工智能的黄金时代。

随便举几个例子:今天很多的工作以后大部分都会消失,比如说翻译,虽然现在还不是做的
那么完美,但是每年进步的都很快,再过几年人工的翻译可能就会非常难找到工作了。记者
也同样如此,如今90\%美联社的文章都是用机器来写的。几乎所有思考模式可以被理性推算
的工作岗位,在有足够数据支撑的时候,都会被取代。有人说十年之内一半的工作会消失,
有人说十五年之内一半的工作会消失,我觉得这些都是合理的揣测。

我想在座大部分都会相信这个理论,而如果你对此还有怀疑,你可以想想,为什么AlphaGo
这么厉害?就是因为它可以动用到几千台机器每天和自己对弈上万盘的围棋,而这人是做不
到的;以后为什么自动驾驶会这么厉害呢?因为它可以用它的各种的sensor在路上搜集数据,
这不是任何一个司机可以匹敌的。所以这些都是一些必然的过程。

\subsection{二、人工智能是什么?}
\label{sec:org0908268}

到底什么是人工智能呢?我觉得大概来说可能是有几个部分。
首先是感知,感知就是包括视觉、语音、语言;然后是决策,刚刚讲的做一些预测,做一些
判断,这些是决策层面的;那当然如果你要做一套完整的系统,就像机器人或是自动驾驶,
它会需要一个反馈。

在这些例子上可以看到,感知可能更多的是帮助识别图里面一个婴儿在沙发上抱着泰迪熊这
种。在推荐上面,我举的例子是一个用Google now通过你过去做的一些事情推测你下面要做
什么,在最下面的例子你会看到有一个无人驾驶的汽车,它有各种的sensor,它捕捉的信息
可以用来做最后的决策,比如怎么去操作方向盘、油门、刹车等等的。其实这三件事情的总
和就是今天所被归纳为的人工智能。

再从博弈、感知决策以及反馈四个方面回顾一下人工智能的发展历程。博弈今天就不讲太多
了,但是基本上我可以看到从我在大学做的Othello到Checkers

再到DeepBlue chess,经过很长的一段时间,终于有了今天AlphaGo打败了围棋世界冠军。
我们从中可以看到,这是一条长达三十多年的路程。

在感知方面,从我的博士论文发表到Nuance成为一个顶尖的公司,从中国诞生了科大讯飞到
美国的Deep Face、中国的Face 等等做得越来越好的企业,这些年也有很多的进步。还有一
些很特殊的例子,比如最近看到一些搞笑的比较Microsoft Tay在Twitter上开始跟人家交流
一下子就讲了一堆不堪的话,就被Microsoft撤回了,所以这里有很多的成功例子,也有很
多有趣的事件。

决策方面,从早期Microsoft Office里的工具到Google广告的推荐,然后到金融行业的很多
智能决策公司的出现,进步迅速。Google auto mail可能大家还没有看过,但是如果你现在
还在用gamil的话,会发现你有时候收到email,Google会跳出来问要不要发回复,有时候它
连回复都帮你写好了,而且写的很精确。这也是人工智能的体现。可能以后我们讲话都不用,
助理能帮我们搞定,人工智能的助理肯定也是一个方向。

最后是反馈,从CMU Boss早期的无人驾驶到Amazon用Kiva推动物流,再到最近的Pepper、
Google car,我们可以看到这个领域过去三四年特别的热,有很多看起来商业化已经做的非
常好。
\subsection{三、什么是深度学习?}
\label{sec:orgb1f1b25}

在这里,我要稍微深度讲一下深度学习。
深度学习是一种神经网络,但与之前的相比,它的特点是使用了多层网络,能够学习抽象概
念,同时融入自我学习,而且收敛相对快速。收敛快速可能是一种技巧,不见得是一个理论,
但是有一批人通过它解决了很多重要的问题。

简单的来说,如果我们有很多笑脸,然后我们把笑脸的像素输入到一个神经网络里面去,最
后你那儿希望让机器能识别这是姚明,那是马云,但是因为你这个深度学习的网络很深,要
一次性学会这么多也会比较困难,所以就需要用到一个比较快速收敛的技巧——自我学习。通
过自我学习,机器会逐步从大量的样本中逐层抽象出相关的概念,然后做出理解,最终做出
判断和决策。
比如它可以有好几层的nodes和connection,经过这些nodes和connection,它在每一个层次
会感知到不同的抽象特征,且一层比一层更为高级。这些都是通过自我学习实现的,而不是
人教的。经过自我学习,从一个脸输进去再从同样的一个脸输出来,它就从里面抽象的学习
到了一个人的脸重要特征。

经过这个学习之后,我再去做监督训练,看机器是否能够识别他们,如果不能,就在训练之
后做微调。例如,如果我输入了马云的脸,出来的却是王宝强,那训练系统就会告诉你的网
络说这个是错误的:这不是王宝强,这是马云。那接下来就是要进行微调,以便于下一次机
器看到这个脸时,能识别出是马云的概率高一些,出来王宝强的概率低一些。

但是这么一调也不能调的太过火了,要不然就会有overtraining的问题,我们就对整个数学
公式做一点微调,用大量的数据,不断重复的去教它,经过不断微调,那么它就很可能在多
次之后降低识别错误。

其实这一整套理论在二三十年前就已经有了,我在做我博士论文的时候,很多我的同事就在
做训练神经网络的工作。

深度学习在最初的时候训练速度特别特别慢,所以比较难进入工业级别或者是应用级别,比
如,你的手机是做不来这个的,因为它的速度实在太慢了。但经过这么多年,我们的计算机
变的越来越快,另外也有了更多取巧的训练和识别做法,深度学习的应用可能性也发生了变
化,它能被应用的领域越来越宽。多年前,我过早的进入了这一领域,但是现在,人工智能
大规模应用的时机已经到了。

凭什么这么说?一个很简单的评估标准就是,我们的深度学习或者是任何的机器学习,它是
不是超越人类的能力表现,如果超越的话,可能很多应用就会产生。比如在机场,如果机器
识别人脸的准确度超过人,那么我们那些边防的人就可能不需要那么多。这并不是说机器不
会犯错,而是说既然人不能比机器做的更好,那我不妨就用机器取代。

\subsection{四、深度学习的应用领域}
\label{sec:orgd64fde7}

在过去的五年,深度学习的准确度从75\%多提升到了97\%左右,而人的表现准确率大概是95\%。
从95\%到97\%听起来只进步了2\%,但实际上是把错误率降低了40\%,这是很大的进步。如果这
种进步持续,未来人工智能必然会超过人类的表现,同时也将可以进入一些可应用的领域。
这就是今天我讲人工智能进入黄金时代的证据:在很多领域,也包括我们在face 做的人脸
识别,包括了Apple、Google,科大讯飞的语音识别,它们的认知水平将在未来几年的时间
内超过人类,而一旦超过人类,应用就会快速的增加。


 深度学习首先可以应用于识别,包括人脸识别和语音识别等,这些可以用于安防,安检等。
人脸语音的数据来之不易,但是BI,商业的流程、互联网的数据却非常丰富。Google、百度
很早就已经在搜索,在广告以及推荐系统里面充分使用了类机器学习技术,解决该推荐什么
商品,一个商品怎么定价,在什么位置会卖的最多,应该把这样的产品卖给谁等问题。这一
类的推销可以直接产生经济价值,而社交媒体营销,整个互联网广告,这每一个领域都是几
十亿,几百亿甚至更大的市场。

将智能用于炒股其实也是一个不错的选择。在国内在国外,很多人都在做这方面创业的工作。
利用智能,我可以随时来算一篮子股票和期货应该如何对冲,以寻求最大的利润。顶尖金融
分析师也会做这个,但是他不可能把所有的股票

的排列组合都考虑一遍,但是机器可以二十四小时不睡觉,每天都在算怎么能赚最多的钱。
除此之外,deep learning深度学习的技术可以把各种的因素都融合进来,比如这个公司的
高管有没有变动,今天出了什么新闻,行业里还有没有什么变动……甚至你可以对一个智能
系统说如果明天巴西发生了地震,什么股票该被购买,甚至你可以说发生了地震不要问我,
你直接去买它就可以了。

银行保险方面,比如说贷款该不该审批,则无论是银行的贷款,还是P2P的贷款,都可以通
过机器来判断,而且数据未必要来自银行内部。

医学方面,因为我自己生过病,也深深的受过这方面的痛苦,我也感觉到在今天的医生的判
断真的不是最完善的。一方面医生有好有坏,顶尖的医生是非常少的;第二方面比如在癌症
方面,它每一年都有新的药出来,那每个医生每天忙着看病人,就不见得有时间去研究这些
药物,那些药物也不是每个国家都可以使用的。还有就是每一个人,他的各种特质,不见得
就适合用这个药。这些其实都是可以用机器学习来做出来的。

前一阵我在美国碰到了一些科学家,他们正在用机器学习的方法来发明新药。我们的科学研
究方面当然要有聪明的头脑和很好的实验,但是其中有一个很关键的部分,就要是一定的程
度去排列组合:试很多东西,对小白鼠先试试这个有没有用,再试试看那个有没有用,然后
再在猿猴身上实验,再进行人体实验。在以前,这整个过程都是由人脑完成,但是这个交给
机器来做也许会更精准。甚至有一家公司它养了非常多的白老鼠,他里面所有的实验都是通
过机器学习精准进行:每天白老鼠活了几只,死了几只,什么药可以进到下一步……这些都
是靠机器学习加上非常精密的系统来做。

我们发明的很多新的材料,都不是靠纯粹的科学方法推出来的,也是去试一试,把这个碰到
那个,就产生了有很特殊效应的材料。这些知识都可以输入我们的信息学习系统,通过它我
们可以帮助发明新的事物。

在教育方面也有应用。在学习的过程中,如果基础没有打好,下一个层次根本学不下去。智
能化的教育系统会识别你的学习水平,然后根据你的水平确定学习内容。比如,你的乘法没
有学好,机器就不可能让你去学除法。

当然学习外语也是很好的例子,我们今天的语音识别做的这么好,为什么我们学外语还是一
定要找外教,为什么语音识别不能再上一层楼呢?所以,当你的技术一提高了,语音识别应
用就不会只是我的讲话进去然后文字出来,它还有可能用在教育领域。

在这么多机会之下,这个人工智能会重塑亿万级别的领域。当然这个不是明天就会发生,因
为我觉得人工智能在很多方面还是相当大的欠缺……

\subsection{五、人工智能将重塑亿万级别的领域}
\label{sec:org8600f57}

人工智能会重塑很多亿万级别的领域。当然这个不是明天就会发生,因为今天我们在很多相
关方面仍存在相当大的欠缺。

比如,在我们的计算架构上面,现在还是需要时间去做算法的改进提升,需要去研究如何部
署云端架构,另外深度学习用时仍太长,这些还都是需要探索的内容,而且并没有一个标准
化的答案。

另外,算法框架也非常重要。我们可以看到有一些重要技术的推进,实际上是因为有了开源
或者API或者标准的出现,但现在仍有很多方面还没有出现相关标准。当然我们知道Google
的TensorFlow等提供了一些开源的方法,但是其实他们还没有真正的平台化,比如你把
TensorFlow丢给一个没学过机器学习的人,哪怕是清华大学顶尖的计算机系学生,他也很难
用其创造价值。如果清华的学生都不能,那它的普及性就有问题了。

为什么iOS、安卓能够做的很好,就是因为它产生了平台化效应,使得很多人能够比较容易
的介入。然后我们可以看到像Hadoop这样七八年前很多人觉得很高深的东西现在也慢慢变得
平台化了。今天,如何使得整个机器学习的体系平台化,以便于让更多的非专业人士能够使
用,这个是目前面临的一个很大的瓶颈,需要一定的发展时间才能得以突破。

在一些领域中,很多技术性问题可以在两三年内得到解决,但是还有很多问题并非如此简单,
比如说语义。我们说语音识别是相对简单的:音进来,字出去,这个非常明确,一个API就
可以调动。但是音进来,确定是何种情境的语义出去就很难。这些我觉得两三年远远还不够,
还需要更多的时间去理解。

传感器一定程度来说是价格的问题、如何普及的问题。现在我们看到Google Car虽然做的很
牛,但是正如驭势科技的吴甘沙说的,Google Car实际商业化的一个巨大瓶颈就是价位的问
题:传感器实在太贵了。因此要把这件事做下来就是一个鸡和蛋的问题——降低价格就需要量,
但量怎么起来?价格不下去量也起不来。要解决这个问题也需要一定的时间。

最后还有很多机械方面的问题。控制机械运动的算法,硬件运动后给出的回馈等等在机械部
门也还需要一些开发。

整体来说,虽然我认为机器学习、深度学习在突破人类的精确度方面已经做的非常好,但是
以上几个领域还是需要一些时间才能取得突破。但是这一天肯定是会来临的,我们怎么知道
会来临呢?

\subsection{六、Google的野心}
\label{sec:org1fc405d}

我们知道,不久前Google重新组织了公司架构,将搜索业务和其他前沿项目子公司都放进了
控股公司Alphabet公司。很多分析师说,Google把搜索和其他的业务分开来做Alphabet,是
为了优化它的股价,其实这种说法太表面了,他们没有了解一个真正有野心的公司在做什么。

一定程度上,Google之所以成立了Alphabet,是因为Google经过搜索和广告业务的积累,逐
步发展了一套我们可以简称为GoogleBrain的模式。Google Brain其实就是机器学习的大脑,
这个机器包括了平台也包括了专家,如果它用在搜索领域就是一个搜索引擎,如果它能够用
在医学领域,那它可能就是一个癌症诊断系统,它也可以用于人类寿命的延续以及智能家电
等各种不同领域。

所以Google的野心就是把机器学习作为一个核心,然后用它去解决非计算机非互联网领域的
各种问题。

当然它现在还不是一个整体平台,但Google 就会找一些极聪明的人来进军这些领域,有平
台的用平台,平台未成形的就用聪明才智来想办法。现在看来,Google这种模式也做成了很
多有意义的事情。所以,对于Google,我们千万不要低估了它的能力,因为这家公司可能是
未来推动人工智能平台化的最大力量。

怎么证明这是真的呢?从最近Jeff Dean演讲的一张图我们就可以看到Google内部有多少项
目在用深度学习。


我们可以看到,从2012年到今天,Google对深度学习的利用在快速增长,应用领域也极为广
泛。从这张图我们就可以看到Google,也就是现在的 Alphabet在人工智能方面是多么的有
野心。

再回到我原来的问题,我们现在是不是生逢其时,可以在正确的时候选择进入人工智能这个
领域呢?如果我们相信Google这帮人很聪明,如果我们相信Google对深度学习的使用逻辑,
我们也要相信人工智能的应用期即将来临。

\subsection{七、深度学习的挑战}
\label{sec:org03d8de3}

但是深度学习以及机器学习还面临很多挑战,这里有几个问题。

\subsubsection{第一个问题,就是我刚刚提到的:}
\label{sec:orgc2d64e5}
目前仍然没有一个统一的平台。在深度学习方面,现在人懂就是懂,不懂就是不懂。这就是
为什么Google最近花了重金不断在挖业界顶尖的人才,给年轻人开出的年薪甚至超过200万
美元。这些人也就是二十来岁,博士刚毕业不久,怎么
会这么值钱呢?


其实就是因为两个理由,第一,这些人进入了公司之后,会被投入到健康、医疗、预防等等
各个领域的研究。他们虽然每年拿走公司的两百万美金年薪,但是也许两年后他们就能在相
关领域创造出两亿美金的价值,所以对Google公司而言,这些人才实际上不贵,是非常划算
的。

第二个理由就是Google多雇一个,Facebook就得少雇一个。这不是开玩笑。因为在美国有三
个大公司在疯狂挖人工智能的人才——Google、Facebook和Microsoft,他们之间竞争激烈,
对人才的吸引力也不相上下。

\subsubsection{第二就是深度学习的网络太大,需要海量的数据。}
\label{sec:org762085d}

\subsubsection{第三,因为数据太多,所以计算特别的慢,所以需要非常大的计算量。}
\label{sec:org22b5aeb}

\subsubsection{第四点有点奇怪但也合理:}
\label{sec:orgb2f4c6e}
机器无法用人的语言告知做事的动机和理由。即便机器训练做了
很棒的深度学习,人脸识别、语音识别做的非常棒,但它不能和人一样,它讲不出来这是怎
么做到的。虽然有人也在做这方面的研究,但是在今天,如果一个领域是不断需要告诉别人
该怎么做,需要向别人去解释为什么

的,那这个领域对于深度学习来讲还是比较困难的。比如Alpha Go打败李世石,你要问
Alpha Go是为什么走这步棋,它是答不上来的。

即便有如此多的局限,我们还是认为人工智能在很多领域可以迅速应用,并且可以帮助企业
打造竞争壁垒。

人工智能如何帮企业打造竞争壁垒?可以从如下四个方面思考:

第一,如果你有垄断性的大数据,你就会有很大的优势。关于数据需要注意的几点是,首先
垄断性大数据不是公开的数据,不是剽来的数据,也不是买来的数据,因为这样的事情你能
做竞争对手也能做。其次,无标签的数据也不会给你带来优势。再次,如果是人工标签的数
据也不行,因为人工标签太慢了。最好的数据是闭环的数据,所谓闭环的数据就是在你应用
的时候可以捕捉到数据并且知道最终你根据数据做出的抉择对或不对。我们投资的face ,
它有和美图、阿里的合作,就一定程度形成了特别大的数据的优势。

第二,拥有庞大的机群。机群是很重要的,包括需要什么处理系统的支持,怎么去部署,用
什么样的计算架构等等。

第三,你要有一批特别懂的人。没有平台的时候,你就只能把一批人丢进去,让他们去解决
特别大的问题。

第四,当你没有平台的时候怎么办?我们就可以找一批特别聪明的人,让他们不断的调节算
法——当然这构成一个短期的竞争优势,从长期看,一旦大的人工智能平台出来,这种优势就
不存在了。所以现在来做人工智能,抓到这个先机是特别特别重要的。

\subsection{七、人工智能如何快速商业化?}
\label{sec:orgba691f5}

人工智能怎么能特别快的商业化呢?这里我要提供几个建议:


\subsubsection{第一,不要用人工智能去取代人。}
\label{sec:org1c5b4d7}

机器不一定要取代人,很多情况之下他只要能辅助人就可以了。我谈到了很多工作会消失,
但医生会全部失业吗?一定不会,应该是最高明的医生创造很多机器人给他人使用。记者就
不再需要了吗?写深度文章还是需要的,但简单拼拼凑凑的文字就不需要了。所以这些工具
一定程度上是在辅助人而不是取代人。

\subsubsection{第二,要聪明的找到容错的用户界面。}
\label{sec:org8cfabc2}

想想搜索引擎,搜索引擎的精确度其实是很低的,你想一想,当你去百度,Google搜索的时
候,它们给出的第一条就是你要的答案的情况有多少?我估计不会超过50\%,但是为什么我
们都说搜索引擎聪明,不说他笨呢?第一个理由当然是因为它博学,第二个则是因为它的界
面做的非常的聪明:它给用户提供很多结果,而用户只要能找到他满意的那个,就会认为搜
索引擎很棒,因为没有它的话,用户可能什么也找不到。这一类的容错的界面,即便它的识
别率很低,给你很多结果,让你在一定时间里得到满足,其实还是达到了一定的可用度。

\subsubsection{第三,让用户提供自然的大数据。}
\label{sec:orgd1d7e46}

当Siri推出的时候很多人都说“这就是个玩具而已”,认为它没有真实的用处,但是苹果靠
Siri收集了很多人的真实语音,收集了大量数据。

很多人把Siri当成一个搞笑工具,会问它诸如“你是男是女”这种无聊问题,苹果就把这些
无聊的问题深度分析了一下,去了解人们最常问的都是什么问题,然后他们就考虑能不能优
化Siri,让它对正常问题的解答能让人们在一定程度上得到满足。人们满足了以后,就会继
续的问,如此问题越问越多,苹果也就可以得到更多的数据。

苹果的这种数据收集方法非常聪明,值得借鉴。我们以前在学语音对话的时候,问的都是非
常正经的问题,到最后分析来分析去,不过是那固定的几万句,一直没有跳出这个框框,得
到的结果也就不会让人满意。但用一种有趣的方式,你就可以像草船借箭一样,去“借”到
几亿个数据。这些数据哪怕不精确也无妨,因为整体来说深度学习非常聪明,能把那些不精
确不精准的东西忽略掉。

\subsubsection{第四,关注局限领域。}
\label{sec:org75a45a1}

Google很伟大,它要做全天候全路况的无人驾驶,它想把全部竞争对手都击败,最后就剩一
个Google。这个计划很宏伟,但是是不是一定要这么做呢?我觉得不见得。其实我们完全可
以先做一个用于局限领域的无人车,把这样的一个产品先做起来,然后我们通过它获取数据,
学习教训,不断改进。

想想无人驾驶叉车。这个叉车是产生价值的,因为它取代了一个叉车工人去开叉车;它技术
难度相对低,因为它只要知道从A走到B;它不上路,不用担心政府的法律法规,不需要考虑
撞到人怎么办,是不是要停下。

Google Car能在高速公路上比99\%以上的人都开的更好,但是它碰到一些极端的情况,比如
大风大雨的漆黑天,它就没辙了,因为它不知道该怎么办,从来没看到过这种情况。这种情
况下只有把车子停下来,但那一停会发生什么呢?当然就追尾了。

 既然这种情况连Google也避免不了,为什么我们不先考虑做一些可控环境下的商业驾驶项
目?这也是一个值得思考的问题,不是说Google的路线不对,而是说有两种路线可以走。

\subsection{八、人工智能的未来蓝图}
\label{sec:org97a2fbe}

上图是我认为的人工智能的未来蓝图,这是我们创新工场现在对这一领域的理解,以及可能
会发生的顺序。

大数据应用方面,现阶段我们已经看到很多互联网应用,BI、商业自动化马上也会使用相关
的技术,未来几年,离钱最近、产生用户最多、产生价值最大的领域可能就是金融、医疗、
教育,当然也包括任何有大数据的行业。

在感知方面,今天的人脸识别、语音识别已经做的蛮好。对于VR/AR,我们在短期还不是太
乐观,但是随着它三五年以后慢慢得到普及,一定需要非常多的新的自然语言的界面。此外,
我们大胆预测三到五年之内会有一个人工智能平台出现。

我们并不认可家庭机器人会很快出现,理由是消费者的期望值是最高的,今天机器人的技术
还不行,犯错也太多,而且有时候会看起来太傻,另外价格也太贵,感应器不够灵敏。基于
这些理由,我们对家用机器人的投资还只限于一些给小朋友的玩具,或者小鱼在家这种用于
沟通的工具,这一类的家庭应用我觉得还是合理的,但要一个能够在家里帮你扫地做菜的机
器人出现,恐怕还是一个非常长期的事情。任何行业都要有经济理由来投资这个领域,不断
迭代优化它的技术,再进入下一个阶段,所以机器人简单来说应该是工业、商业,最后普及
到家庭,所以今天很多对家庭机器人过火的观点和做法我们是不认可的。

关于无人驾驶,我们的观点是虽然Google Car很伟大,但是因为它要去适应各种路况,所以
要到应用阶段也还需要很长的时间。我们认为可以先在局限环境中慢慢推进无人驾驶。

从长期看,未来人工智能会在所有的领域彻底改变人类,产生更多的价值,取代更多人的工
作,也会让很多现在重复性的工作被取代,然后让人去做人真正应该去做的事情。短期来说,
人工智能商业价值也很大,短期在很多领域都能产生价值。

\subsection{问答环节:人工智能的时代来了,人的时代结束了吗?}
\label{sec:org7444b6c}

\subsubsection{问题一:}
\label{sec:org4718ea7}
刚才听到了人工智能的介绍,真的认为人工智能给我们带来了很多的便捷,的确是
快要步入到一个黄金时代,很多人可以从劳动密集型的工作中解放出来。我的问题是,如果
很多事儿都可以交给机器来完成,那么剩余的劳动力是否会催生另一个黄金产业,比如服务
业等等?

\subsubsection{李开复:}
\label{sec:orgb5557f7}
这个问题很好,我曾经写过一篇文章《如果十年之后有一半的人失业了,下面该发
生什么》,讲过这个问题。悲观的预测是,因为现在的95后,00后大部分成长在虚拟世界里,
现实世界只是他们一个小小的补充,所以他们可能深陷其中而不能自拔了。反正人工智能让
他们失业,也找不到工作,同时政府也可以养着他们,他们不需要工作,那么干脆每天带着
虚拟眼镜起来就玩游戏,不思进取就好了。这个是我特别担心的,也是可能发生的。

乐观的是上天让我们来到这个世界,本来就不是让我们来做中介、助理这类劳力的工作,我
们被生为有感知,有大脑,应该是有更高的目标。那么,是不是上天让人工智能来到我们面
前,就是让一批有思考能力的人帮人类找到一条新的出路,让我们每个人能够找到自己的更
高目标,让一半失业的人找到他们的归宿,有自我实现的机会呢?

我对年轻人的建议是做最有热情的事情,要不然机器肯定取代你,要做就要做顶尖最棒最有
创意的事情,要不然你肯定会被取代。

\subsubsection{问题二:}
\label{sec:org259d84f}
您讲了很多前景中的事情,我们都觉得美好,但是在实现前景的道路上,最大的几
个技术瓶颈在哪里?在解决瓶颈的过程中,您看到的大公司解决这些瓶颈的方案是什么?

\subsubsection{李开复:}
\label{sec:org97880cf}
最大的瓶颈就是人工智能的平台。平台化会带来井喷的效应,但这个平台是什么我
今天也无法描述。你可以看到iOS的平台,安卓的平台等等,每一次都造成了井喷式的效应,
所以平台化是必然的也是必须的。对于无人驾驶和机器人方面,主要是感应器的价格等问题,
一旦问题解决,这些领域也会发生井喷效应。这些都需要聪明的大脑试着去解决。

\subsubsection{问题三:}
\label{sec:orga7e13ad}
第一个问题,您刚才提到人工智能的黄金时代到来了,那么您能举一个到三个导致
这种黄金时代到来的创新明星吗?第二个问题,当人工智能呼啸而来的时候,大家很快意识
到人工智能第一个急需解决的问题是人,如果人类把持不住,可能人的时代就过去了。

\subsubsection{李开复:}
\label{sec:orgb00135e}
今天有四位CEO,分别代表了芯片领域、金融领域、视觉识别领域和无人驾驶领域,
这四个公司虽然目前领域不同,但在到达下一个阶段后,一定会做平台,一定要扩张。任何
一个领域都是先做一个切入的应用,做好了之后再扩大,变成一个平台,最终获利最多的就
是那些有上下平台连接的公司。

关于你的第二个问题,即人类将面临的下一个挑战问题,不是机器人占领了世界,我们被机
器统治这样的问题。虽然我不排除这些事情发生的可能性,但是我呼吁大家先要应对一个现
实问题——下一阶段50\%的人可能失业的问题。这可能是一个最大的问题,毕竟在过去的工业
时代也好,信息时代也好,失业率从来没有这么高过,今天的政治经济体系也不能承担这么
高的失业率。所以我觉得聪明的人应该赶快解决这个问题,这个问题可能十年后就要到来了,
当这个问题解决完了之后,我们再去想下一个问题。

\subsubsection{问题四:}
\label{sec:org684bcbf}
您讲到的人工智能在各个领域都会以极快的速度超过人的领域,但是它是不是只在
某些领域做的更好,在某些人类没有开拓的领域,它有没有自我开拓的能力,根据已有的领
域和问题,开拓新的领域?“奇点”真的出现,是不是就说明人工智能有跨领域思考学习的
能力?

\subsubsection{李开复:}
\label{sec:orgd31afc8}
从过去已经发生的事情来看,还不存在人工智能的自我开创,未来有没有可能,这
其实是一个蛮有意思的话题。我想即便是一些人类未曾进入的领域,只要你能把它量化,变
成一个input/output,我觉得机器还是可以学成的。但是我也相信大部分人类的伟大创意都
不是那么容易量化的。

 第二个关于奇点的问题,回到刚才讲的,机器学习并不知道它自己在干什么,它只是一个
非常聪明的input/output而已。另外它没有自我意识。科幻小说和现实生活很大的差别是因
为科幻小说里面的机器都有自我意识,所以我觉得,由于机器的自我意识和自我创造力,还
有自我解释的行动能力,这三者都不存在,奇点来临也就不是那么容易的事情。我相信在未
来的一百年,这三点都有可能被克服,也会给人类带来机会和灾难,但是我们现在真正面临
的问题还是50\%的人可能失业的问题。

\subsection{2014年度辛 星Python教程第一版}
\label{sec:org4a1e184}
\url{https://wenku.baidu.com/view/d60bbd9ada38376baf1faec7.html}
\section{管理}
\label{sec:orgd80a399}
\subsection{信息技术工程系在线课程}
\label{sec:orgc028081}
\url{http://172.16.36.13/moodle/}
\subsection{信息技术工程系综合管理平台}
\label{sec:org8f14eb9}
\url{http://172.16.36.3/redmine/}
\subsection{信息技术工程系ERP}
\label{sec:orgfa9f136}
\url{http://172.16.36.11:8069/zh\_CN/}
\subsection{学院办公系统}
\label{sec:orged62c80}
\url{http://172.16.22.43:9082/ehome/oaIndex.do}\#
\subsection{高效能人士的七个习惯}
\label{sec:org098885b}
\url{http://wenku.baidu.com/link?url=CmXvHhwlRW4w5VMA3J8OOq7BsjmV4aBLounpOvi6-c60h5RhJADEkMuQYk3sgYnXiUnLFFkt9ld52AtMQffYmY1J\_tuU-qco4L0h4899GYq}

\subsection{{\bfseries\sffamily DONE} 大数据时代,我们如何做教师\textit{<2017-09-26 Tue>}}
\label{sec:org96fb139}
\url{http://blog.caijing.com.cn/expert\_article-151688-83377.shtml}

大数据时代,我们如何做教师

2015-05-28 18:40:04分类:未分类

2015年5月27日,在美国的波士顿,全球最大的国际教育工作者协会大会(NAFSA,National
Association of Foreign Student Affairs)上,美国厚仁教育首席发展官陈航先生发布了
首份中国留学生开除报告,报告称去年超过8000名中国留学生被美国大学开除,引起舆论哗
然。大家不知道的是,这几年与陈航先生交往,陈航所在的公司帮助开除学生重新回到校园
已经成为陈航所在的厚仁教育的一项主要任务。陈航公司也由于有一批专业指导不具备自理
能力的中国留学生的辅导员队伍,而声名鹊起,公司快速成长。100多名公司员工活跃于各
个美国的大学,用网络方式长期指导和支持在美中国留学生的学习、转学与困惑。陈航和他
的同伴们能够这样做,当然不是仅仅基于道德情操而是基于数据,陈航发布的《2015留美中
国学生现状白皮书》用翔实的数据,说明了在美留学生的状态。与此同时,一个叫做ZINCH
的网站迅速在留美家长中火爆,注册用户只要把学生的各科学业成绩和SAT分数、托福分数
以及特长输入这个网站,基本上可以八九不离十地知道会被什么大学录取、会被什么大学拒
绝。注册用户还会经常收到网站后面的各个大学的招生官的信,个性化地提供招生对口服务。
我注意到这几件事情是发现,这些背后依靠大数据资料提供服务的培训机构的教师,往往比
中国的高三老师和美国的高中的辅导员更加专业和准确,从某种意义上来讲,校园中的一部
分老师,已经被大数据的市场服务替代了。

这就引起我进一步思索:大数据时代,我们如何做教师?

\subsection{1、 信息过载时代的搜索、阅读与辨别真伪}
\label{sec:orga3b2930}

这10年,我分析过上海的数十个大学不同等级的学生,也在大专、本科、研究生课程中担任
教师教授同一门课:网络工程管理。我发现,从智力水平上,当然大专、本科、研究生是有
差距的,但是并不十分大,而学习习惯的差距的细微区别,反映在他们在创造性任务作业上
的差别却是天差地别的。举个例子,我发现大专学生在寻找资料的时候非常喜欢使用HAO123,
而本科学生第一习惯是百度,研究生一般使用学术知网。于是我稍微采取了一点措施:要求
大专学生不允许使用HAO123,而必须使用百度文库;要求本科学生不允许使用百度而必须使
用学术知网;要求我的研究生不允许使用中文文献而必须使用EI和SCI 检索文献。采用这些
措施后,我发现所有的学生的成果质量都明显上升一个档次。

在搜索解决后,大量的阅读是挡在学生学习中的一个障碍。我对比过中美学生的教材和阅读
文献的速度,在这两方面中美差距非常大。不仅仅如此,中国教师也是大面积失去了阅读能
力的一批人,近年来所谓教育家批发的心灵鸡汤被广泛转载让我不忍卒读,很大一部分原因
是我们绝大多数教师失去了阅读原著的能力。于是,我在我的两门SAKAI在线课程《网络工
程管理》和《电子商务安全》课程中放置了大量经过我筛选的资料,每门课达到10G 左右。
在我课程结业的要求中,要求学生针对问题写小组作业,而小组作业寻找的资料大量网络上
是找不到的,只能在我的课程平台上找到,而学生写作业过程中按照我的诱导的“抄袭过
程”,就是大于教科书100倍的阅读能力训练过程。

不仅仅是搜索与阅读,有时候在阅读材料中给学生故意设置大量的是死而非的结论让训练学
生的辨别能力比阅读更为重要。在一门医学课程中,我为医学院教师设置了针对肚子疼完全
不同的诊断案例,让学生针对病人症状寻找各种可能性和养成批判性思维。

大数据的普遍采用,相关性代替了因果性,需要教师做的东西更多了,而不是更少了,只是
原先大量的重复劳动可以交给计算机和网络,教师本身集中精力于创造性的教育设计中。

\subsection{2、 自带社交网络授课;}
\label{sec:org0044ce4}
在卡内基梅隆大学CYLAB,有一个非常好的传统,就是每周一中午,总有来自全世界的教师
和产业界人士会发表一个小时的演讲,在演讲之前,教师会先吃点东西,而演讲的过程中听
众都有免费的午餐,通过这种形式,在垃圾时间段充分进行了学术交流。让我感到意外的是,
非常多的来演讲的嘉宾,除了在中午演讲外,更多的是短期和长期地与CYLAB保持学术交流
甚至会被教授邀请到课程中表达自己的观点。我经常看到这所学校的教授上课的时候安排少
数课程甚至1/3的课程请自己的狐朋狗友来表达观点,我称这种形式为自带社交网络授课。
卡内基梅隆大学有非常丰富的课程平台、视频和文献检索系统,然而他们似乎更看重这种社
交的学习和授课,因为这种授课在数据资源丰富的时代,更能将最新的甚至是关于未来的设
想与师生及时沟通,而一个好教师不仅仅要求对本学科的信息和数据充分掌握,能否动用自
己的朋友圈为教育服务,更能体现教师的能力。在CYLAB 的因特尔实验室,会经常举办一些
开放的课程让师生参与,而课程的过程中,会调动因特尔全球的4个会场进行互动,这些更
加体现了这个一流高校的资源能力。而事实上,一门编号为15237的品牌课程(嵌入式设计),
其教师本身就是原因特尔的一个部门经理,每学期他不仅会带来因特尔最新的同事和设想,
还会带来五万美金的捐款,让每组学生用一些钱进行作品设计和考核,引起这门课的疯抢。

\subsection{3、筛选30年不变的知识与技能;}
\label{sec:orgdc3d944}

有些东西要面向未来,有些东西却是回归传统。由于计算机类的知识更新非常快,让教师往
往无所适从。常常出现的情况是教学计划刚刚根据技术的发展制定,等到学生毕业时就已经
落后,而另外一种情况又会出现:那种被认为很过时的东西,却常常具有非常持久的生命力。
比如,在物联网高速发展的今天,学生学习的东西往往会很快过时,而焊接技能、串口协议、
通信原理等,却不会过时。大数据时代,知识更迭和资源汇集会非常容易和迅速,作为教师,
能够筛选出对于学生30年不变的知识与技能坚持下去,永远是教师的基本功。

\subsection{4、为孩子设计适应性的阶梯与任务;}
\label{sec:org95b4c4c}

即使在同样一门课程中,针对不同基础和程度的学生,有了数据资源和在线爱课程,学生往
往会因为东西太多而无所适从,这里,就需要教师为孩子设计适应性的阶梯和任务,既有难
度和挑战,又不至于失去了乐趣。例如物联网和开源硬件这门课,从10岁的孩子到25岁的硕
士生都可以开,然而同样的资源的情况下,为孩子们布置的任务和提供的支持是完全不同的,
甚至界面也是不一样的。例如一个很有趣的事情,对于大学层次在一本以下的学生我发现大
家更喜欢使用MOODLE 课程,这种按照周次排列的课程和方式非常符合创新能力不是那么强
的学生的好感。而对于一本以上层次的学生研究生,SAKAI更适合他们的口味,因为这种资
源模式更加方便。

在网路教学资源充分丰富的今天,教学任务的设计成为教师的首要任务和核心能力。即使同
一个班,不同理解力的学生要考虑使用不同的任务策略。而对于不同类型的任务,其教学内
容的布置也需要教师充分从人性和教育学角度进行设计和实施。

\subsection{5、 为每一届学生项目制筹资与管理;}
\label{sec:org5f6e1c0}

苹果公司的联合创始人沃茨尼亚克在离开苹果后的一些年,致力于在中小学提供电脑教室:
电脑从娃娃抓起,这种风潮客观上促进了信息技术的发展。信息技术的学习由于技术进步非
常快,每一届学生毕业后,就需要重新根据技术的变化设计出新的学习内容和配套出新的课
程体系,沃茨尼亚克离开苹果的一些年,因此并没有武功荒废,反而发明出了万能遥控器这
种东西。

在中国有一个形容教师的词汇:教书匠,这个词汇肯定不适合大数据时代的教育。每个大学
老师培养一批学生4年,一批研究生2到三年,一批大专学生3年;一个中学老师培养一批中
学生需要3到四年。如果下一个四年还讲同样一套东西,不但是误人子弟,教师的光荣感也
会受到伤害。在大数据时代,好的教师更是一个不断更新的项目经理和销售员,他不仅仅提
出设想争取投资,还应该根据新的社会经济技术发展将社会具有前瞻性的技术开发出来原型,
体现在课堂,甚至输出到社会。

\subsection{6、围绕创造与天分,搭建合适的创造空间}
\label{sec:org1b9d9a0}

医学教育有一个非常好的传统,那就是实习医生制度。在西方传统的医学教育中,大学本科
必须修医学预科或者生物学相关课程。到了高年级的时候,必须到医学院附属医院从事临床
和门诊的医疗服务,在那里,导师不仅仅是教师,还是医生。稳定的医学制度产生了附属医
院这种通行的教育模式。IT技术其实和医药界非常象,但却没有机会采用这种制度,因此IT
界的教育培养效率远远低于医药界。

近两年风行的创客空间,给了IT和制造业、艺术教育一个新的思路。麻省理工大学的新媒体
实验室率先采用这种方式进行研究性学习。每年数以亿计的美金投入到这个实验室,这个实
验室的知名教授们会在这个象加工厂的创客空间搭建出学习和科研所需要的仪器设备和工作
场所,与学生一起进行研究和科研。博士生、硕士生、本科生甚至小孩子都可以来这里一显
身手。在卡内基梅隆大学的汉斯管理学院地下室,甚至还配备了住宿和淋浴设备,支持学生
24小时、48小时、72小时不间断地学习和科研。这样往往一个科研项目结束,也意味着一个
公司的诞生,董事长教授更受人尊敬,就像医生教授是必须的一样。

\subsection{7、捍卫教师的讲台,教会学生捍卫学习的习惯}
\label{sec:org77d79a0}

与中国“要给学生一壶水,自己要有一缸水”不同,大数据时代,处于激素水平较低年龄段
的教师,要求比学生强不应也不能是对教师的要求。教师与学生虽然都学艺术课,教师的水
平未必比学生高很多,因为教师与学生虽然都在艺术实验室,他们却分属不同的专业。
KATHY就是匹兹堡这样的一个教师。在VINCENTIAN高中校,KATHY已经将她的这门课GREASE讲
授了超过10年,也已经培养出了超过10多位顶级艺术学院的学生。每年10月开始,KATHY就
开始在全校的200多名学生中招聘30名选修课学生,进行对外的收费戏剧:GREASE的演出排
练:分角色、筹资、服装、舞台、化妆,KATHY是名校哥伦比亚大学的博士,学的是历史,
在学校还担任学生的心理咨询老师,然而她还有一个身份是匹兹堡著名的舞蹈教练。每次课
程,KATHY总是动用大量的社会关系为自己服务,自己家也有超过100亩地用于学生集中排练
当停车场。KATHY 非常热爱自己的这份工作,其实她这门课完全是亏本的,但是她得到了满
足。在她这门复杂专业的课程指导中,她也非常坦然接受和扶持在某种专业上比自己强的学
生和外聘教师。我女儿连续两年在这门戏剧表演中担任不同的角色:合唱和服装,即使在美
国高考中最艰苦的11月也不停止,不仅仅是因为女儿喜欢戏剧,更是因为女儿喜欢这个老师。
我非常乐于高兴地看到女儿从老师身上学到的专业、敬业、捍卫职业、工作习惯,从某种角
度上讲,学生从老师身上学到知识和技能是浅层次的学习,学习到习惯和工作方式才是最根
本的学习。

\subsection{8、从老师到导师:更简单的界面、更复杂和专业的支持。}
\label{sec:orgf034fb1}

文章写到这里,不得不暂做停顿,让我们回到本文的开头。在传统的教育中,教师年复一年,
是重复的教学,而学生是新鲜地学习。然而,信息化改变了这一切,学生从网络获得了最新
的一切,教师的工作显得无趣和乏味。然而,无趣和乏味不是因为教师就应该是无趣和乏味
的,而是因为教师没有面临竞争和淘汰。新东方出现了,让全国大学机构的英语老师面临考
证英语失去了市场,那是因为新东方更会使用大数据;新东方拥有4万老师而面临沪江英语
的网上动则3000学生一个班束手无策,不是因为新东方提供的服务不好,而是新东方的竞争
对手已经不是新航道而变成网络英语。更见简单的ZINCH 留学让网上的学生可以几乎免费地
直接知道传统留学机构非常资深收费很高的顾问收费很高的服务,背后是各个大学的招生官
为此付费,而这背后,是类似厚仁教育这样的机构不是老师的老师,提供更像老师的服务:
为学生提供更加专业、准确、需要和个性化的服务,而这所有的背后,是大数据这个看不见
的手,在掌控和支持。

该交给信息的就交给信息吧,教师不是不需要了,而是不一样了。

本文为《《教师教育论坛》约稿文章,转载请注明出处
魏忠
\subsection{高效能的个人工作方法}
\label{sec:org6c6b2c3}
\url{https://wenku.baidu.com/view/e0cd4514f18583d048645905.html}
\subsection{Facebook}
\label{sec:org142f573}
\url{http://kb.cnblogs.com/page/545228/}

\subsection{\href{https://wenku.baidu.com/view/c974011c227916888486d7b6.html}{德鲁克管理学精髓-jennyzeng}}
\label{sec:org57792d2}
\subsection{\href{https://wenku.baidu.com/view/c1b258d54afe04a1b171de3d.html}{养成良好的工作习惯}}
\label{sec:orga7700ee}
\subsection{\href{http://news.mbalib.com/story/237365}{世界上成长最快公司的17个共同点}}
\label{sec:org25c5ee9}
\subsection{1、都以为广泛人群谋求价值为最大动力,贯彻始终}
\label{sec:orgdfa17bc}

领导人不厌其烦宣导企业文化。创始之初,Google官方的公司使命为“集成全球范围的信息,
使人人皆可访问并从中受益”。不作恶(Don't beevil)是谷歌公司的一项非正式的公司口号。

2004年,Google创建了非营利的慈善机构Google.org,起始基金为10亿美金。组织的使命是
创建公众对气候变化、全球公共卫生、全球贫困等问题的意识。

阿里巴巴的3个理想:为1000万企业生存,为全世界1亿人创造就业机会,为10亿人提供网上
消费平台。

腾讯的使命:通过互联网服务提升人类生活品质。

扎克伯格在2010年受“连接”杂志访谈时表示,他还在为相同的目标在努力:“我最关心的
就是,如何让世界更开放。”

Facebook当初并不是作为一家公司来创立,而是为了实现一个社交使命:令全世界更加开放,
保持连接。

我们每天早上醒来想到的第一目的不是赚钱,我们知道实现使命的最佳方式是建立一家强大、
有价值的公司。

我们希望Facebook所有员工每天都能够在做每一件事情时专注于如何为世界带来真正的价值。

那些价值在10亿美元以上的创业企业有一个明显的共同点,它们专注于为终端消费者创造价
值。Facebook帮助用户相互联系,LinkedIn 为用户提供职场网络,Zynga 为用户提供游戏,
Pandora 为用户提供音乐服务。它们不仅有能力不断吸引新用户入驻,还能够在注册完成之
后将用户粘在自己的平台上。这一切的要诀就是,专注地为用户创造价值。

\subsection{2、都在创业早期获得大量投资}
\label{sec:org46def19}

1999年10月,阿里巴巴引入了包括高盛、富达投资(Fidelity Capital)和新加坡政府科技发
展基金等在内的首期500万美元天使基金。

2000年1月,阿里巴巴获得日本软银(SOFTBANK)的注资2000万美元。

1999年底,李彦宏携120万美元风险投资回国创建百度;2000年09月 DFJ、IDG等国际著名风
险投资公司为百度投入巨额资金。

2005年5月,Facebook获得AccelPartners的1270万美元风险投资。

06年4月,Peter Thiel、Greylock Partners和Meritech CapitalPartners额外投资
Facebook 2500万美元。

1999年6月7日,包括Kleiner Perkins公司和红杉资本在内的投资者为Google注资两千五百万美元。

1976苹果成立当年,乔布斯想将公司扩充并向银行贷款,但韦恩因为冒险投资失败导致的心
理阴影而退出了。当时的苹果电脑缺乏资金来源。乔布斯最后遇到麦克·马库拉(Mike
Markkula),麦克·马库拉注资9.2万美元并和乔布斯联合签署了25万美元的银行贷款。

\subsection{3、都强调团队合作的力量}
\label{sec:org69a0bc0}

阿里巴巴核心价值观:团队合作——共享共担,平凡人做非凡事。

团队合作是几乎所有伟大公司的核心价值观。

最初,我自己动手写Facebook代码,因为这只是我的一个心愿。后来,Facebook的大多数理
念和代码都来自于我们吸引到团队里的优秀人才。

\subsection{4、CEO都花至少五分之一时间吸引顶尖人才(早期更甚)}
\label{sec:org8905c29}

自 2001 年加入 Google 以来,埃里克·施密特帮助 Google 从硅谷的一家初创公司成长为
全球技术领头羊。

对于腾讯来说,业务和资金都不是最重要的。业务可以拓展,可以更换,资金可以吸收,可
以调整,而人才却是最不可轻易替代的,是我们最宝贵的财富。----- 马化腾

互联网公司,最有价值的就是人。我们的办公室、服务器会折旧,但一个公司,始终在增值
的就是公司的每一位员工。——李彦宏

小米人主要由来自微软、谷歌、金山、MOTO等国内外IT公司的资深员工所组成。

7月17日,百度CEO李彦宏在第三届百度“程序之星”决赛现场时表示,自己有三分之一时间
用于招揽人才。“我每天至少要把三分之一时间花在人才培养和管理上。”李彦宏表示。

李开复:任何一个好的老板至少把20\%的时间放在招聘上,我从来没有少过20\%。要尽量找一
些非常优秀的,最好是在某些方面比你优秀的人;每雇一个新人,就试着把团队的平均水平
提高,而不是下降;要让你的中、高层管理者不再有任何机会隐藏他们团队的人才,公司最
优秀的人才是属于公司的。

\subsection{5、CEO都是最后一轮的面试官(至少在早期)}
\label{sec:org94b38d2}

小米加入的所有员工都需要经过雷军的面试。

挽救雅虎的现任CEO梅耶尔亲自面试每一个员工。

谷歌CEO拉里·佩奇(Larry Page)会亲自批准每一次招聘。

\subsection{6、都保持专注(早期更甚)}
\label{sec:org7ac9514}

百度“论语”:专注如一。

百度“论语”:认准了,就去做,不跟风,不动摇。

谷歌信条:专心将一件事做到极致。

布林说,早在网络工作的最初阶段,他们就决定专做“搜索”,“搜索”关乎于信息,只有
它才能为人们的生活带来真正的变化。

雷军:七字诀第一是专注。比如现在像很多微博站每天没有很多更新,但是文章质量好很多,
在少就是多的时代里面,我们是信息过多,怎么样把东西做得精致,有价值,才是问题关键。

\subsection{7、都以广告、增值服务或电子商务为主要盈利模式}
\label{sec:orga3775b6}

腾讯选择了增值服务盈利模式。

选择广告为Google的主要盈利模式。

百度2001年08月在中国首创了竞价排名商业模式。

\subsection{8、都以全球用户为服务对象(几乎一开始就是)}
\label{sec:org29ce4a7}

1999年阿里巴巴创立B2B网上贸易平台,选择服务于全球。

亚马逊一开始的目标就是成为全球最大的网络书店。

谷歌一开始的目标就是帮助全世界使用各种语言的人获取信息。

Facebook开启国际化进程,到2005年底已扩展到加拿大、英国、墨西哥、澳大利亚、新西兰。

\subsection{9、都有导师和智囊团}
\label{sec:org1777c5e}

扎克伯格有包括乔布斯在内的多位导师。

孙正义在很长时间里都是阿里巴巴的顾问。

埃里克森斯密特是谷歌两位领导人的导师。

\subsection{10、CEO都首抓用户体验,把产品做到“极致”}
\label{sec:org8e56292}

亚马逊使命:成为地球上最以客户为中心的公司。

马化腾、周鸿祎和雷军都把自己当成公司最大的产品经理。

李彦宏:我花三分之一时间抓产品和技术。

乔布斯毫无疑问是公司最大产品经理。

百度“论语”:用户需求决定一切。

百度“论语”:把事情做到极致。

雷军:第二条极致,极致就是做到你能做的最好,极致就是做到别人达不到的高度。大家经
常“恭维”我,说小米山寨了iPhone,我真的不知道怎么表达。好的东西是不可能被抄袭的。
每次想山寨iPhone的时候,我看到人家那个图标做得那么极致,就想我们能做到吗?不行。
我只希望通过每天脚踏实地,一步一步努力,能离偶像再近一点,再近一点。 在过去的 5
年间,美国共产生了 38 家市值突破 1 亿美元的科技创业企业。这些企业的创始人都有相
似的特性——对产品狂热。他们往往亲身参与产品开发的方方面面,以保证消费者手中的产品
就是他们想象中的那样。正因如此,一些创业者在企业上市后依旧担任 CEO 一职,比如
Facebook 的 Zuckerberg 与 Zynga 的 Pincus。

雷军:第三个要讲的是口碑。我经常问大家一个问题,去过海底捞吗?海底捞就真的比五星
级餐馆好吗?为什么咱们比起会议中心就没口碑,这个海底捞就有口碑呢?其实,口碑的的本
质是超越用户的希望值。海底捞在一个很破的地方,当我们走进去的时候它超越了我们所有
的期望值,我们就觉得好。当我们去五星级餐馆的时候我们期望值很高,怎么可能超越呢?

“为发烧而生”是小米的产品理念。小米公司首创了用互联网模式开发手机操作系统、60万
发烧友参与开发改进的模式。

\subsection{11、都把握了当时的大趋势,在正确的时间做了正确的事}
\label{sec:org85d9dd1}

阿里巴巴:互联网趋势+电子商务。

谷歌:互联网趋势+搜索引擎。

百度:互联网趋势+中文搜索。

腾讯:互联网趋势+即时通讯。

Facebook:互联网趋势+社交网络。

小米:移动互联网+手机换代潮。

苹果和微软:微型计算机时代兴起。

亚马逊:互联网趋势+电子商务。

\subsection{12、都非常重视速度和执行力}
\label{sec:org573c354}

谷歌信条:越快越好。

Facebook:Done is better than perfect.比完美更重要的是完成;Move fast and
breakthings.快速行动,破除陈规。

1999年底,李彦宏和他的团队仅在短短6个月的时间内就完成目前中国最大、最好的中文搜
索引擎的开发工作。

百度“论语”:高效率执行。

一流的创意+三流的执行VS三流的创意+一流的执行,我宁愿选择后者——马云。

雷军:最后一个要诀就是快,我坚信“天下武功唯快不破”。有时候,快就是一种力量,你
快了以后能掩盖很多问题,企业在快速发展的时候往往风险是最小的,当你速度一慢下来,
所有的问题都暴露出来了。所以,怎么在确保安全的情况下提速是所有互联网企业最关键的
问题。

\subsection{13、都利用了顶尖的环境}
\label{sec:org44386b8}

1995年马云访问美国接触互联网;1999年阿里巴巴成立,总部设在杭州市,恰逢互联网在中
国刚刚起步;在经济环境上,中国正好处于改革开放的成熟阶段。

马化腾是潮汕人,10来岁随父母从汕头到深圳;腾讯成立于1998年11月,总部设在深圳,恰
逢互联网在中国刚刚起步;在经济环境上,中国正好处于改革开放的成熟阶段。

2000年1月由李彦宏、徐勇两人于北京中关村创立百度;在经济环境上,中国正好处于改革开
放的成熟阶段,又逢互联网在中国刚刚起步。

扎克伯格出生于纽约的一犹太人家庭;作为牙医和心理医生的儿子,扎克伯格从小就受到了
良好的教育,从小就是个电脑神童。

04年3月,Zuckerberg,McCollum和Moskovitz搬到加利福尼亚州的Palo Alto市(译者:斯坦
福大学所在地,硅谷的发源地)。

1999年3月,Google将办公场所搬至加州的帕罗奥多,这里是众多知名的硅谷初创公司所在的地方。

布林出生在苏联,大约在6岁时与父母移居至美国,父母两人皆毕业于莫斯科国立大学。

拉里·佩奇是密歇根大学计算机科学教授 Carl Victor Page 博士的儿子,他从六岁就开始
热衷于计算机。

\subsection{14、都有良好的人际关系}
\label{sec:org9c956d6}

马云是公认的高EQ,强大的个人魅力。

扎克伯格和盖茨、乔布斯很早就是亦师亦友的关系。

扎克伯格在哈佛他修习心理学与电脑并加入犹太学生兄弟会Alpha Epsilon Pi。

\subsection{15、运气都不是一般的好}
\label{sec:orgf60fccc}

最悬的是当时与深圳电信数据局的谈判,对方准备出60万元,马化腾坚持要卖100万元,始
终谈不拢,只好告吹。

众所周知,两位创始人差一点就把谷歌的前身卖掉。

一名男子在一个加油站拿枪指着扎克伯格进行抢劫。扎克伯格跳上自己的汽车逃跑,没有受
伤,否则后果难以想象。

\subsection{16、都是创新高手}
\label{sec:org902fa3f}

百度“论语”:问题驱动;创新求变;允许试错。

阿里巴巴和腾讯的核心价值观:创新。

Facebook每隔几个月就会举办一次黑客创新大赛。

苹果的创新指数长期居全球第一。

在Google,要求工程师们每周都花一天时间在个人感兴趣的项目上。这种近乎强制性的要求
造成Google News之类的新服务品种出现。

\subsection{17、都有足够的耐性}
\label{sec:org9df2e61}

在过去的 5 年间,美国共产生了 38 家市值突破 1 亿美元的科技创业企业。价值破 10 亿
美元的九家创业企业平均发展时间为 7 年。发展时间最短的是 Instagam,仅用时 2 年;最
长的是 Pandora,用时 11 年,名单上所有 38 家企业在上市 / 被收购前的平均发展时间
为 6.9 年。由此可见,即便科技行业虽然有着传统产业无可比拟的发展速度与机会,但罗
马也不是一天建成的。如果创业者想要取得耀眼的成就,需要有长期奋斗的心理准备,同时
发展有长期奋斗意愿的团队。

来源|微信公号:汇说FTsay 微信号:seovsem
\section{英语单词学习}
\label{sec:org171164e}
\subsection{Dangerous Trainers}
\label{sec:org5afd786}
Susan Gates
Thud! Thud! Thud!
My big brother's got some new trainers.
He wears them all the time.
Thud, thud, thud1 When he runs upstairs in them the whole house shakes.
Mum shouts, 'Stop that noise!'
My brother's new trainers are big and puffy and purple. They've got soles as
thick as tractor tyres.
\subsection{Test your vocab}
\label{sec:orgc3cffcc}
How many words do you know?
\url{http://testyourvocab.com/result?user=9002732}
\subsection{名词}
\label{sec:org65fab04}
\subsection{{\bfseries\sffamily DONE} computer}
\label{sec:orgc603ba9}
computer | BrE kəmˈpjuːtə, AmE kəmˈpjudər |
noun 计算机 jìsuànjī

to do [something] by computer or on a computer
用计算机做某事

to have/put [something] on computer
将某资料存入计算机

the computer is up/down
这台计算机在运行/无法运行

a personal/home/laptop computer
个人/家用/笔记本电脑

to be or work in computers
在计算机行业工作
\subsection{{\bfseries\sffamily DONE} laptop}
\label{sec:orgfb7df53}
laptop | BrE ˈlaptɒp, AmE ˈlæpˌtɑp |

noun 手提电脑 shǒutí diànnǎo

a laptop computer 笔记本电脑

\subsection{phone}
\label{sec:org9ef8211}

phone

1 | BrE fəʊn, AmE foʊn |
A. noun (telephone) 电话 diànhuà to be on the phone (to [somebody]); 正在(与某人)通电话 to talk to [somebody] over the phone; 与某人通电话 to tell [somebody] [something] by phone; 打电话告诉某人某事 to hear [something] over the phone; 在电话里听到某事
B. transitive verb 给…打电话 gěi… dǎ diànhuà ‹person, organization›; 打电话告知 dǎ diànhuà gàozhī ‹information, news›to phone France 往法国打电话 try phoning his home number 试试打他家的电话
C. intransitive verb 打电话 dǎ diànhuà to phone for a doctor/taxi 打电话叫医生/出租车 he phoned for the clerk to bring in the report 他打电话让文书把报告带来 PHRASAL VERBS phone in
A. intransitive verb 打电话 dǎ diànhuà to phone in sick 打电话请病假
B. transitive verb[phone in something, phone something in] 打进电话告知 dǎjìn diànhuà gàozhī ‹information, report›phone up
A. intransitive verb 打电话 dǎ diànhuà
B. transitive verb[phone up somebody, phone somebody up] 给…打电话 gěi… dǎ diànhuà
\subsection{rush}
\label{sec:orgcaf6b99}
rush
1 | BrE rʌʃ, AmE rəʃ |
A. intransitive verb
① (move with urgency) «person, vehicle» 急促移动 jícù yídòng to rush at [somebody]/[something]; 向某人/某物冲去 to rush down/up the stairs 冲下/冲上楼梯 to rush round the house 在屋子里四下忙活 to rush along the street 沿街冲去 to rush out of the room/up to [somebody] 冲出房间/冲向某人 he rushed off before I could tell him 我还没来得及告诉他,他就急着走了
② (flow strongly) «water, river» 奔腾 bēnténg ; «wind» 猛烈地刮 měngliè de guā ; «air» 急速流动 jísù liúdòng the sound of rushing water 湍急的水声 a rushing stream 湍急的小溪 the stream rushed down the mountainside 小溪从山坡上急流而下 the blood rushed to his face 血一下子涌到了他的脸上
③ (act quickly) 匆忙行事 cōngmáng xíngshì don't rush 别急 to rush at [something]; 匆忙做某事 to rush to do [something]; 匆忙做某事 to rush to explain 急忙解释
B. transitive verb
① (send urgently) 紧急运送 jǐnjí yùnsòng to rush [somebody]/[something] to [somebody]/[something]; 将…紧急运送至… ‹supplies, troops›to be rushed to hospital BrE or the hospital AmE 被紧急送到医院 police reinforcements were rushed to the scene 增援的警力被火速派往现场
② (send quickly) 迅速发送 xùnsù fāsòng to rush [somebody] [something]; 将…迅速发送给某人 ‹copy, book›please rush me my copy 请把我的那本赶紧送来
③ (do hastily) 仓促做 cāngcù zuò don't try to rush things 不要草率行事
④ (pressurize, hurry) 催促 cuīcù ‹person›I don't want to rush you, but … 我不想催你,但是… to be rushed off one's feet 忙得不可开交
⑤ (charge at) 冲向 chōng xiàng ‹building, platform, position›
⑥ AmE University «fraternity, sorority» 特别关注 tèbié guānzhù ‹student, freshman›
C. noun
① Countablesingular (fast movement) 快速的移动 kuàisù de yídòng a rush of photographers/volunteers 一拥而上的摄影师/志愿者 a rush for [something]; 朝…冲去 ‹exit, train, toilet›a rush for the door/towards the buffet 朝门口/自助餐台涌去 a rush to do [something]; 冲去做某事 to make a rush at or for [somebody]/[something] 朝某人/某处冲去
② Countablesingular (fast flow of water, blood, air) 涌动 yǒngdòng to have a rush of blood to one's cheeks 血涌上了双颊 a rush of blood to the head 头脑一热
③ Uncountable and countable(hurry) 匆忙 cōngmáng to be in a rush (to do [something]); 匆忙(做某事) to do [something] in a rush; 急匆匆地做某事 it all happened in such a rush 一切都发生得如此匆忙 we had a rush to make the deadline 我们匆匆赶在最后期限前完成任务 is there any rush? 着急吗? there's no rush 不必着急 what's the rush? 干吗这么急匆匆的? why (all) the rush? 急什么?
④ Countable(busy time) 繁忙 fánmáng the morning/evening rush (交通的)早上/晚上高峰 the Christmas/summer rush 圣诞节购物热潮/夏季旅游热潮 beat the rush! 避开交通高峰!
⑤ Countable(sudden demand) 抢购 qiǎnggòu a rush on or for [something]; 抢购某物
⑥ Countable(of emotion) 冲动 chōngdòng (of energy) 涌动 yǒngdòng to experience a sudden rush of adrenalin 突然热血沸腾
⑦ Countablecolloquial (thrill) [吸毒后的] 快感 kuàigǎn it gives you a rush 它会让你感到亢奋
⑧ Countable AmE University 学生联谊会纳新活动 xuésheng liányìhuì nàxīn huódòng rush party/week 学生联谊会纳新晚会/周
D. rushes noun plural Cinema 样片 yàngpiàn PHRASAL VERBS rush into transitive verb
① [rush into something] (undertake hastily) «person» 仓促进行 cāngcù jìnxíng ‹purchase, sale›to rush into marriage/a decision/a commitment 仓促结婚/作出决定/作出承诺
② (make do hastily) 使仓促进行 shǐ cāngcù jìnxíng to rush [somebody] into [something]/doing [something]; 催促某人做某事 don't be rushed into it 不要因为有人催就草率行事 rush out transitive verb[rush something out, rush out something] 匆匆印制 cōngcōng yìnzhì ‹edition, pamphlet›; 匆匆作出 cōngcōng zuòchū ‹announcement, statement›rush through transitive verb[rush through something, rush something through] 快速通过 kuàisù tōngguò ‹bill, amendment›; 迅速处理 xùnsù chǔlǐ ‹order, application›to rush a bill through parliament 使议案在议会匆匆通过

\subsection{University Spirit}
\label{sec:orgb11d42c}
大学精神
spirit

spirit | BrE ˈspɪrɪt, AmE ˈspɪrɪt |
A. noun
① Uncountable and countable(mind, will) 精神 jīngshén the power of the human spirit 人的精神力量 I'll be with you in spirit 我的心将会和你在一起 the spirit is willing but the flesh is weak 心有余而力不足
② Uncountable and countable(soul) 灵魂 línghún body and spirit 形与神 [somebody's] spirit is troubled, [somebody] is troubled in spirit 某人内心苦恼 brothers/sisters in spirit 精神上的兄弟/姐妹
③ Countable(supernatural being) 神灵 shénlíng nature spirits 自然神灵 the (Holy) Spirit 圣灵 an evil spirit 恶魔
④ Countable(person) 一类人 yī lèi rén a great/bold spirit 伟人/大胆的人 a leading spirit in the movement 运动的领袖
⑤ Countablesingular (essence, character) 实质 shízhì the spirit of the declaration/agreement 宣言/协议的精神 in the spirit not the letter of the law 根据法律的精神实质而不是字面意思 to be faithful to the spirit of the original «translation, film» 忠实于原作的精神 the spirit of the age or times 时代精神
⑥ Uncountable and countablesingular (mood) 心境 xīnjìng (attitude) 态度 tàidu in a friendly/forgiving spirit, in a spirit of friendship/forgiveness 以友好/宽容的态度 the party/holiday spirit 聚会/度假的心情 community/team spirit 集体/团队精神 a spirit of resistance/optimism 反抗/乐观情绪 to take [something] in the wrong spirit 误解 ‹remark, words›that's the spirit! 那才是好样的!
⑦ Uncountable(will) 意志 yìzhì (courage) 勇气 yǒngqì (energy) 活力 huólì fighting spirit 斗志 to break [somebody's] spirit 摧垮某人的意志 to be full of spirit 充满活力 to play with great spirit «team, player» 表现得极为勇猛
⑧ UncountableChemistry (distilled liquid) 精 jīng (distilled alcohol) 酒精 jiǔjīng aviation spirit 航空汽油 a spirit lamp/stove 酒精灯/炉
B. spirits noun plural
① (mood) 情绪 qíngxù to be in good/poor/high/low spirits 情绪好/不好/高昂/低落 to keep one's spirits up 保持高昂的情绪 to raise [somebody's] spirits 使某人精神振奋 my spirits rose/sank 我的情绪振奋/低落了
② especially BrE (alcohol) 烈酒 lièjiǔ
③ Pharmacology (essence) 精 jīng spirits of turpentine 松节油
C. transitive verb 偷偷带走 tōutōu dàizǒu to spirit [somebody]/[something]
away 把某人/某物偷偷带走

\subsection{thud}
\label{sec:org1751207}
\subsection{trainer}
\label{sec:org623f2ed}
\subsection{动词}
\label{sec:orgb6e3829}

\subsection{shave}
\label{sec:orgb63680f}
shave | BrE ʃeɪv, AmE ʃeɪv |
A. transitive verb
① (remove hair from) 剃…上的毛发 tì… shang de máofà ‹person, face›(remove with razor) 刮 guā ‹beard›; 剃 tì ‹hair›to shave one's legs/head 剃腿上的汗毛/剃头
② (graze) «plane, ball» 擦过 cāguo ‹arm, goalpost, treetop›
③ (trim) 刨掉 bàodiào ‹surface›to shave [something] off …; 从…上刨去 ‹wood, fraction›
④ (reduce) «person, company» 削减 xuējiǎn ‹profits, prices›
B. intransitive verb 刮脸 guā liǎn
C. noun 刮脸 guā liǎn to have a shave 刮脸 guā liǎn to give [somebody] a shave 给某人刮脸 a narrow or close shave 侥幸脱险 PHRASAL VERB shave off transitive verb[shave something off], [shave off something]
① (remove with razor) 剃掉 tìdiào ‹beard, moustache, hair›
② (trim) 刨掉 bàodiào ‹wood›
③ (take off) 减少 jiǎnshǎo ‹amount, time, distance›
\subsection{shake}
\label{sec:org6af0fb4}

shake | BrE ʃeɪk, AmE ʃeɪk |
A. transitive verb
① (past tense shook, past participle shaken) (move vigorously) 抖动 dǒudòng ‹mat, bag›; 甩动 shuǎidòng ‹prey›; 摇动 yáodòng ‹branch›shake the bottle 摇瓶子 it was a rough road, and we were shaken around quite a bit inside the car 道路崎岖不平,我们在车里颠得很厉害 to shake [something] at [somebody]/[something] 对某人/某物挥动某物 she shook the snow off or from her coat 她抖落了外衣上的雪 to shake hands 握手 wòshǒu to shake hands with [somebody], to shake [somebody's] hand, to shake [somebody] by the hand 和某人握手 to shake one's head 摇头 there'll be a few heads shaken over this scheme when it's made public 这一计划公布后会有人反对的 to shake a leg colloquial 快点儿 more than you can shake a stick at BrE colloquial 多得不得了
② (shock) 使震惊 shǐ zhènjīng his death/the news had clearly shaken them 他的死讯/这一消息使他们大为震惊
③ (weaken, impair) 动摇 dòngyáo ‹faith, confidence, theory›to shake [something] to its foundations 彻底动摇 ‹belief, system›
④ colloquial = shake off
②
B. intransitive verb(past tense shook, past participle shaken)
① (vibrate, tremble) «hand, person, voice» 颤抖 chàndǒu ; «ground, building» 颤动 chàndòng ; «grass, leaves» 摇动 yáodòng to shake with laughter/fear/fright/cold 笑得/吓得/害怕得/冻得浑身发抖
② colloquial (shake hands) 握手 wòshǒu to shake on [something]; 为某事握手祝贺 can we shake on it? 我们可以握手祝贺达成协议了吗?
C. reflexive verb(past tense shook, past participle shaken) to shake oneself 抖动身体 dǒudòng shēntǐ he shook himself to try and get the spiders off him 他试图抖落身上的蜘蛛
D. noun
① (act of shaking) 摇动 yáodòng with a shake of the or one's head 摇了摇头 give the bottle a shake before you pour 倒东西之前先摇一摇瓶子 to have a shake in one's voice 声音有些颤抖 to have the shakes colloquial 发抖 in a shake colloquial 马上 in two shakes (of a lamb's tail) colloquial 马上 no great shakes colloquial 不出色 to be no great shakes (at [something]) (在某方面)很一般 to get/give [somebody] a fair shake AmE colloquial 得到公平对待/公平对待某人
② AmE (milkshake) 奶昔 nǎixī
③ AmE (earthquake) 地震 dìzhèn
④ (amount sprinkled) [从容器里] 摇出的东西 yáochū de dōngxi add a few shakes of sea salt and black pepper 撒点儿海盐和黑胡椒 PHRASAL VERBS shake down
A. intransitive verb
① colloquial (settle) «machine» 运转正常 yùnzhuǎn zhèngcháng ; «person» 适应新环境 shìyìng xīn huánjìng
② (in container) «contents, powder, granules» 变得密实 biàn de mìshi
B. [shake somebody/something down, shake down somebody/something] transitive verb
① (cause to fall) 摇落 yáoluò ‹fruit, object›; (cause to settle) 摇密实 yáo mìshi ‹contents, powder›
② AmE colloquial (search) 彻底搜查 chèdǐ sōuchá ‹person, building, club›the store detective shook him down 那名商场保安搜了他的身
C. [shake somebody down, shake down somebody] transitive verb AmE colloquial (extort money from) 敲诈 qiāozhà shake off transitive verb[shake somebody/something off, shake off somebody/something]
① (let go by shaking) 抖落 dǒuluò the boy was clinging to Peter's neck, and Peter was trying to shake him off 男孩抱住彼得的脖子不放,彼得正想法把他甩下来
② (get rid of, escape from) 摆脱 bǎituō ‹bad mood, habit, tiresome person›they shook off the car that was tailing them 他们甩掉了尾随他们的汽车 I can't seem to shake off this flu 我这次感冒好像怎么也好不了 shake out transitive verb[shake something out, shake out something] (empty by shaking) 摇出 yáochū ‹coins, contents›to shake the bag out over the table 把袋子里的东西抖在桌子上 shake up
A. [shake something up, shake up something] transitive verb
① (mix) 摇松 yáosōng ‹cushion, pillow›; 摇匀 yáoyún ‹medicine, mixture›
② (reorganize) 重组 chóngzǔ ‹company, organization›
B. [shake somebody up, shake up somebody] transitive verb
① (make uncomfortable by jolting) 使…受颠簸 shǐ… shòu diānbǒ ‹passengers›
② colloquial (distress, shock) 震动 zhèndòng
③ colloquial (rouse to activity) 使振作 shǐ zhènzuò they need shaking up! 得让他们振作起来!

\subsection{形容词}
\label{sec:orgf46d360}
\subsection{puffly}
\label{sec:org90c45f5}
\subsection{purple}
\label{sec:org3c97cb7}
\subsection{副词}
\label{sec:org8d27473}
\subsection{代词}
\label{sec:org0363e93}
\section{学习笔记}
\label{sec:org6b87b0c}
\subsection{\textit{<2017-09-27 Wed 09:54>}}
\label{sec:orge86dc81}
早晨日记

\begin{center}
\includegraphics[width=.9\linewidth]{/Users/Mac/Pictures/IMG_2143.PNG}
\end{center}

学习笔记

\begin{center}
\includegraphics[width=.9\linewidth]{/Users/Mac/Pictures/IMG_2144.PNG}
\end{center}
\section{书法及练习}
\label{sec:orgfeddffd}
\subsection{{\bfseries\sffamily DONE} practice}
\label{sec:org94b34d5}
\subsection{灵飞经小楷}
\label{sec:org58596c3}
\url{http://www.360doc.com/content/16/0222/21/26707738\_536518887.shtml}
\subsection{兰亭序}
\label{sec:orgafdcc07}
\begin{center}
\includegraphics[width=.9\linewidth]{/Users/Mac/desktop/电子图书/ (晋)王羲之兰亭序大字贴.pdf}
\end{center}
\section{杂学}
\label{sec:org1da6287}
\subsection{\href{http://baijiahao.baidu.com/s?id=1582633309001573756\&wfr=spider\&for=pc}{一个大开脑洞的理论-递弱代偿理论}}
\label{sec:org338faa8}
今天我来介绍一下学术上非常有意思、发人深省的哲学理论-递弱代偿理论,这个理论认为:
宇宙万物演化遵循一个规律:越是原始简单的物种其存在度越高,越是后来衍生的复杂物种
其存在度越低,并且存在度呈递减趋势,此为“递弱”后衍的物种为了保证自身能够稳定长
期存在,就会相应增加和发展自身存续的能力及结构属性,此为“代偿”随着物种属性和复
杂度的提高,他的存在度趋近于零。

\subsection{递弱代偿的核心}
\label{sec:org18725cd}

我们从宇宙大爆炸说起:认为宇宙的产生起源于一次大爆炸,在爆炸中产生了时间、各种粒
子,这就是说在爆炸的一瞬间“凭空”产生了宇宙。而极小一部分粒子聚合为氢原子,其余
90\%的粒子构成了暗物质、暗能量。

\subsection{各种基本粒子}
\label{sec:orgba24657}

氢原子就需要粒子的结合。极大量的氢原子形成了恒星,在恒星内部氢聚变为氦,氦继续聚
变形成后面的碳、氧元素,再继续聚变形成更大原子量的元素,直到形成铁元素。铁元素以
后的元素要靠超新星爆发来形成。原子量越大的元素其存在度越弱,而且需要上一级的元素
聚变来形成。到了原子量更大的元素:铀等放射性元素,在自然界就不能稳定存在了,它会
自发衰变成其他元素,原子量比铀大的元素人类只有在实验室才能制造出来,况且制造出来
的元素存在度极低,往往需要极大能量才能制造几个元素分子,这些分子只能存在几万分之
一秒甚至几亿分之一秒,真是方生方死。

\subsection{有机物}
\label{sec:org7b64061}

说完了元素,再来说无机物、有机物。无机物需要有机物在适宜的条件下才能合成,虽然他
的特性丰富,但存在度明显不如元素单质。再说有机物,虽然有机物的性质、用途远比无机
物丰富,但是一块石头,你在自然界放1000年还是一块石头,用水煮还是石头。但是有机物
你放在自然界没多长时间就风化了,蛋白质等你拿水煮一下性质就改变了。

宇宙演化图景

虽然无机物、有机物存在度减弱了,但是你注意到没有,他们如果产生出来,完全不需要外
界输入能量和物质就能相对稳定的存在。
根据自然界由有序向无序的方向来说:生命的产生是违背自然法则的。生命体,无论是原始
的单细胞生物,还是植物,以及动物包括人,它们的存在完全依赖于外界环境,需要不断地
和外界进行物质和能量交换,比如一棵大树,它需要不断地吸收阳光、水、二氧化碳和其他
物质生成碳水化合物和氧气。人时时刻刻都在和外界进行能量物质交换,把人放在真空环境
中,存活不了1个小时。虽然动物的功能越来越丰富,这是它为了保持自己的稳定度,不得
不代偿出这些功能。
一个原始人,仅凭自己的双手就能在广袤的森林中生存下去,而现代的城市人你试着没有电、
没有网络、没有手机估计会疯掉的。


\subsection{现代化的战斗机-歼-20}
\label{sec:orge06290d}
\subsection{古代冷兵器的战争}
\label{sec:org8cf4dc8}
拿战争中的武器举例子,古代的武器,斩木为兵,刀枪等武器几个人,一把火炉就生产出来
了,生产出来后,基本上士兵不需要经过训练就能使用。而现代化的战斗机的生产需要科研
人员、熟练工人、先进的工厂和强大的工业基础才能生产出来。生产出来还没完,还需要仔
细维护,随时准备更换零部件,作战时需要加满油,需要极其精密的指挥系统、导航系统和
保障系统。使用战斗机的人需要长达10年的学习训练,耗费数以亿计的资金才能培养出来,
而一架飞机可能需要数十人为它服务。虽然战斗机的作战能力比刀枪强多了,但是他需要借
助的外部条件也太多了,以至于全世界能够生产先进战斗机的国家没几个。

\subsection{人类科技发展的结晶-高能粒子加速器}
\label{sec:org35abfa5}

根据递弱代偿理论,人类的存在有一个自然和科技极限,不是因为科技进步使人的生存能力
越来越强,而是人类的生存能力越来越弱,从迫使人类不断追求科技进步弥补自身存在度越
来越差的生存能力,最后走向宇宙演化的终点。
\subsection{\href{http://www.jianshu.com/p/f84f4f9f9313}{《如何成为专家》思维导图}}
\label{sec:orgb202acc}

初步整理了一下思路,将关于成就专家的过程进行了简单梳理:

1包括几个部分:知识工作者为什么需要成为你领域的专家、真正的专家是什么样的,区别伪
专家和骗子、如何成为专家(成就专家的修炼过程)、赢取认可(让别人认为你是专家)、
超越专家(避免”越专业越无能“)、专家能力迁移;

2个人认为成就专家的核心部分是知识、实践和思维、品牌四块,成为专家最重要的修炼是:
知识的广度和深度及其关联、参与重大项目和工作的实践、已经解决过大量困难的问题并能
够解决困难、复杂的新问题、思维的层次性和抽象性(概念、判断、推理等),这几块我没
有展开。

还有关键是:你得让别人认为你是专家!自我认为是专家、官方指定的专家都没有用,让你
的用户认可你!

3这个是我的思路整理,所以我个人感觉特别熟悉和想的特别清楚地就没有详细列出来。

在思考的过程中,跟许多朋友交流过,他们有的人已经是某个领域的专家,有的在成就专家
的路上,有的对专家根本不感兴趣。通过跟他们的交流,让我受益颇多。对于认可专家价值
的人,大家认为阻碍成为专家的核心因素是(不全):

没有明确的目标,不知道该选择什么?

工作太忙了,没有时间去学习与反思。

岗位工作枯燥,没有机会接触大项目;

不知道如何整合知识,不会总结和归纳等等。

以下是简单的思维导图,点击可以放大查看


图片可点击放大观看

您有好的建议,不妨写到留言里面,先谢谢!(本文作者为知名知识管理专家、《你的知识
需要管理》、《卓越密码:如何成为专家》作者,可以通过微信号:17331899联系它)
\begin{center}
\includegraphics[width=.9\linewidth]{/Users/mac/xiafile/xiafile/专家.jpg}
\end{center}
(微信公众号:【KMCenter】,关注个人成长与知识管理、知识库,欢迎勾搭!)

作者:田志刚
链接:\url{http://www.jianshu.com/p/f84f4f9f9313}
來源:简书
著作权归作者所有。商业转载请联系作者获得授权,非商业转载请注明出处。
\subsection{\href{http://www.hao123.com/mid/15574105736349851150?key=\&from=tuijian\&pn=2}{人和人的差距,就是这样拉开的!}}
\label{sec:orgd9e7a96}
在大家实践机会差不多的时候,这个差距其实在于完成每个任务或者项目后是否有成长和进
步:可能水平高的人还不如水平低的那个人做的事情多、勤奋,最后水平高的人也不一定是
读书最多的人,但由于他注重了每次做完事情后的总结、提炼,从而同样的做事情,甚至他
做事情更少但成长却更大,这样日积月累,就显示出不同。

\subsection{在知识工作者的学习实践过程中:}
\label{sec:org4b50cec}
在工作中学习,做完事情后的总结、提炼、提升是最大、最多的学习机会,而且从工作中学
习得到的经验和提高要比读书效率高很多。有许多人因为忽视了这一块的学习和长进,就会
出现虽然也努力、也“学习”,但很难有大的、快速的提高。

\subsection{总结提炼,这成了人与人水平和能力差距拉大的主要原因。}
\label{sec:orgb4a1996}

\subsection{那在工作中如何总结提炼?}
\label{sec:org2acd046}

许多时候即便你想总结也不知道总结什么?这就需要你不断地思考,在做不同事情的过程中
寻找相似性,找到需要学习的“点”,制定出自己事后学习的模版,这样有意识的按照模版
去总结、提炼,进而提高。
\subsection{具体可以参照这篇文章:不会总结提炼,你干多少活都没用(本文摘自《卓越密码:如何成}
\label{sec:orgbc6b737}
为专家》一书)

(本文作者为知名知识管理专家KMCenter主任、《卓越密码:如何成为专家》、《你的知识
需要管理》作者田志刚。)

\subsection{\href{http://baijiahao.baidu.com/s?id=1577752836880338007\&wfr=spider\&for=pc}{成功来自精准的勤奋,而不是平庸忙碌!不会总结提炼,你干多少活都没用}}
\label{sec:orgd516f5e}

“不期待成功,但愿通过努力不留遗憾。”是劣质勤奋者普遍的心态。而这背后,本质是,
为逃避真正的思考,愿意做任何事情。而最终,因为缺乏合理、系统的思考,缺乏对整个人
生的全局审判,越忙越穷。

\subsection{1为何你那么忙,还那么穷?}
\label{sec:org33f89ea}

我曾经读过很多同类的文章,但因为此问题过于庞杂,暂时没看到系统的解决方式。而今天
的文,恰恰是希望有条理地解码这个问题。

\subsection{说到穷与忙的关系,或许更本质的是:勤奋与能力的关系。}
\label{sec:org286e63b}
毕竟:

\subsubsection{穷的本质是技能提升速度过慢,而技能不足,直接制约能力变现;}
\label{sec:org725245d}

\subsubsection{至于忙,在于你看起来的忙碌与勤奋,是真勤奋?还是只是看起来很忙的假勤奋。}
\label{sec:orgbb30c92}

对穷忙族而言,往往有一个普遍认识:

为什么我这么努力,但生活还是如此费力?

为什么我这么勤奋,却依然和社会上绝大多数同类相差不大?
最终,导致穷忙族越忙越累,越累越焦虑,越焦虑越忙。
由此,进入了一个死循环。
而在这认知背后,其实是高维人群所不曾告诉低维人士的真相:
\subsubsection{一个勤奋学习各种碎片式知识理论的人,未必能成功。}
\label{sec:org007a4bc}
勤奋就能成功一向以来都是我们的认知误区,于是,太多人把短暂而珍贵的一生,演绎成碌
碌无为的一生。这一点,和诺基亚CEO约玛·奥利拉说的,同样悲伤。在诺基亚陷入困境时,

CEO约玛·奥利拉曾感慨地说:我们并没有做错什么,但不知为什么,我们输了。
而此文,正是要将勤奋与能力之间的问题,清晰化。

\subsection{2为什么很多人会越忙越平庸?}
\label{sec:orge50aab2}

\subsubsection{这点,恰恰来自物种的同频性。简单说,我们很难不成为绝大多数。}
\label{sec:org0506178}

\subsubsection{人天生就离平庸很近,离出色很远。}
\label{sec:orgc5b6006}

从生物天性而言,迁徙的鸟、群体性生物,包括人,一切动物行为都有同频性的。这里的同
频性是指:

\subsubsection{群体中的个体,往往会与群体中大多数保持相同的行动轨迹,而个体却从来不思考,这样的}
\label{sec:orgab145d0}
行动轨迹,是对是错,是否会产生系统性风险。

这话很拗口,从精神分析和社会传播角度来说,又称为“集体无意识”,简单说就是,人们
更擅长于做和别人一模一样的动作和生活选择,以此确保自己是足够“安全的”。

关于这点,我们不去从生物性的群体保护、个体融入群体以获得庇佑、为了通过群体行动力
而武装个体去展开了。
在过去,这样的天性(别人做什么,我做什么的思路),让不同物种得以存活,而在今天,
一个鼓励多元化、碎片化、移动互联化的时代里,则适得其反。人越随大流,越平庸。

这也就是人越忙越穷的开始,同频化的步调,让人们的勤奋毫无价值。因为,社会上成功的
是绝少数,你模仿大众,你注定黯然无光。
用最简单的一句话击穿勤奋者的玻璃心,那就是:
\subsubsection{你看起来很努力,努力到抛进人群中没人认出你。}
\label{sec:org2b85958}

\subsection{3劣质勤奋者}
\label{sec:org7aa444c}

什么叫劣质勤奋者?

\subsubsection{说的就是, 你的努力毫无个性可言,也毫无价值可言。}
\label{sec:org9fa4c91}

你只是和社会上那些看起来很努力很忙的人一样,每天坚持着工作,但你碌碌无为的一生,
压根没为自己和社会创造出任何价值,你只是为了更快地度过时间,只是为了对抗个体懦弱
无能的现实压力,然后,用尽一生力量,为自己营造出一个自我感觉良好的幻觉。
你试图说服自己:“人生也曾努力过,也曾付出过,也曾刻苦过”。但你只是沉溺于假象的
自我催眠者。

\subsubsection{低质量的勤奋其实是伪装起来的懒惰。}
\label{sec:org4f89af5}
劣质勤奋者以没有任何科学依据可言的瞎忙,来自我暗示,弥补看到别人成功后产生的心里
不平衡。对劣质勤奋者而言,他们的内心活动,幼稚、可怜、并可笑。
“不期待成功,但愿通过努力不留遗憾。”是劣质勤奋者普遍的心态。而这背后,本质是,
\subsubsection{为逃避真正的思考,愿意做任何事情。}
\label{sec:org36bc26f}
\subsubsection{而最终,因为缺乏合理、系统的思考,缺乏对整个人生的全局审判,越忙越穷。}
\label{sec:org0562e98}

对劣质勤奋者而言,不仅收入少,还是去了最珍贵的东西。因为,最宝贵的时间,在这看上
去很努力的自我催眠中,悄然逝去,一分一秒最终都变得一文不值。

\subsection{4如何避免成为一个劣质勤奋者?}
\label{sec:orgca304b5}

意识定义行为,一切从观点开始。以下,是避免成为“劣质勤奋者”的高维心智法则。

\subsubsection{无需全天学习和勤奋,只需要抓住两个时}
\label{sec:org5d0f52c}

1990年,三位心理学家为了对小提琴名家进行研究专门前往西柏林中心的艺术大学。他们试
图通过大量数据,解答一个基础问题:是什么因素让精英演奏家比中等演奏者更加优秀?
他们将研究对象分成两组,一组是有可能成为大师的精英演奏者,一种是普通的演奏爱好者。
通过研究,他们发现:

普通演奏者通常将工作分散到一天完成。一份将平均工作时间和每日活动小时对比的图表显
示,普通演奏者图表上的曲线是平滑的,工作时间与活动时间相近。
而精英演奏者则不同,他们将工作集中在两个明显时段完成。如果将他们工作时间与每日活
动时间相对比的表格描绘出来,会发现两个显著峰值:一个早上,一个下午。
越是顶尖演奏者,峰值越明显。而同样,越顶尖的,在时间峰值之外,他们休息和放松的时
间也要比所有人都多。

由此,最后结论是:

\subsubsection{一整天持续的勤奋,并不能决定成功,有节奏的努力,才能成功。}
\label{sec:org59717b2}

\subsubsection{费力工作不等于刻苦工作}
\label{sec:orgeda1fb5}

上面的演奏家对比研究,反映了一个事实:懂得把控时间的人,技能更好,也更轻松。这理
论也被称为:

\subsubsection{轻松的罗德奖学者悖论。}
\label{sec:org890cf20}

\subsubsection{此理论倡导的核心是:刻苦工作和费力工作是不同的。}
\label{sec:orga33e987}

刻苦工作是一种经过深思熟虑的训练,在你刻苦训练时,会感到痛苦,但你每天并不需花太
多时间去进行这项训练,刻苦训练给你的技能带来可衡量的增长,它能为你带来强烈的满足
感和动力。
因此,尽管艰苦训练是艰难的,它并不会耗尽你所有的能量,而且它能和放松的日子相完美
搭配。

而费力工作则恰恰相反,它令人耗尽所有能量的。你所度过的每一天都处于不正确的忙碌状
态,就像柏林学院的普通演奏者,感到疲劳且富有压力。

而这种忙碌并不能为你带来真正的成长,并且,你反倒会越来越焦虑。
而疲惫与焦虑,是成功路上的敌人。所以,你既要避免无意义的忙碌,又要避免耗费精力的
疲惫感,因此,轻松的罗德奖学者悖论直指一个本质:

\subsubsection{少做事,要么不做,要做就报以绝对专注。}
\label{sec:org4e295d4}

\subsubsection{既要确保效果,又要保存体能,避免疲惫。}
\label{sec:orgeb12df4}

\subsubsection{不是每一种变化都叫“量子跃迁”}
\label{sec:org2ab3e1f}

想要在有效时间和精力中,抓住成功的钥匙,就要持续不断保持标准动作,直到引发裂变。
说到标准动作,一如健身房里的胖子与瘦子,胖子经常边运动边聊天,而身材好的训练者,
从不闲聊,除了锻炼,就是安静休息,每个动作都不晃悠,做到最标准。

一切没把动作做到足够标准的努力,都是不及格的努力,因为,只有标准动作可以量化,只
有通过重复大量的标准动作,才能让你所从属的系统,发生改变。

而说到关于改变,又有两个维度。

第一序改变:发生在某一系统之内的改变,系统本身维持不变。只改变了系统里的元素;以
梦为例,你梦见任何,任何梦中情节改变,都属于第一序改变。

而第二序改变是指:发生在系统之外的改变,控制系统整体的前提改变,使系统转换到完全
不同的状态,也即改变之改变,简直就像量子跃迁。例如,从梦中醒来。
只有完成第二序改变,才是质的改变,从而带来根本上的跃迁。而只有一次次跃迁的,才能
让你靠近成功。

\subsubsection{在输入端努力不如在输出端努力}
\label{sec:org6eee9a4}

\subsubsection{不带着问题去努力,就是瞎忙。}
\label{sec:org7ee5ba5}

人们看起来劳劳碌碌,但他们压根不知道自己为什么要这么做。说到这,有一个词,叫:

\subsubsection{“用以致学”。}
\label{sec:orgebda6d6}

“用以致学”是欧美企业领导力倡导的高维心智管理学,在Joseph.A.Raelin的《WORK:
BASED LEARNING》和弗雷德蒙德·马利克的《管理成就生活》均有说明。

“用以致学”是指明确任务后的学习,是“行动学习”、“实践社区”和“实践与反思”等具体学习的基础。
在“用以致学”思维中,用“任务模型”取代“能力模型”是避免成为劣质勤奋者的主要思考准则。

“用以致学”揭示了一个真相:

\subsubsection{绝大多数人的勤奋是没有方向和目的的;}
\label{sec:org4da8123}

绝大多数人都只在输入端用力,而没在输出端使劲;
很多人的能力差,并不差在输入的能力和知识储备,而在于输出能力太弱。解决方案就是:
\subsubsection{强化输出端,形成 “输入 ——输出 —— 结果” 的良性闭环,其关键在于:用以致学,而非}
\label{sec:orgf78da56}
学以致用。

\subsubsection{成功需要基于项目、任务、问题的学习,而非,基于能力提升、知识储备的学习。}
\label{sec:org8910371}

《穷忙》一书的作者,美国作家戴维·希普勒就曾明确地指出:
失去自我时间控制的人会陷入真正贫穷,因为,这会引发更恐慌的心理暗示,自暴自弃。而
在当今美国,越是忙碌的人反而越穷,他们中的大多数都有一种自暴自弃的倾向。
最后,想说的是,成功来自“精准的勤奋”,而不是“平庸的忙碌”。

两个天赋差不多、努力程度差不多的人,都能够持续的坚持学习和思考,都有适合的实践机
会。但他们经过十年二十年后,仍然可能也会有巨大的差异,一个人已经成为了行业的专家
和高手,而另一个人虽然活干的不少,却没有成为专家,甚至在高手眼里还是新手或者入门
级?

我们在前面提到过,对于职场工作的成年人而言,其学习和能力提升的核心不在于多看或者
少看了几本书、多听或者少听了几节课,而在于他是否有实践的机会,并是否能够在干完活、
解决问题后能够实现长进和提升。

这个长进和提升来自于对于所学、所做和所思考内容的总结,在实践后会发现所学是否适合
新的环境、所思是否符合客观现实,所做的结果是否与自己的预期吻合。这个过程其实是一
种对自己亲身经历(学习、思考和实践)的检验、对照和发现,是需要提炼、概括、判断的
抽象、深入反思的过程。

\subsubsection{你可以发现那些真正的专家和高手做的事情可能并不是最多的,但他一定是长进最快的:因}
\label{sec:org443913b}
为它不仅仅是干完活、完成任务,而是在这个基础上又进行了反思总结,能够比大部分人往
前走了一步:抽取、抽象任务和项目的规律性,不仅会做还知道为什么要做、为什么要这样
做、还可以如何做、这些做事的方法可以提炼成模板、模型、框架,甚至可以做成相应的工
具和产品。

前面也提到过,随着人工智能能力的增强,未来许多做重复性工作的岗位会被替代,这就对
个体的能力提出更高的要求,总揽全局、抽象思考、更深入的理解用户需求等方面的能力成
为有竞争优势的人的核心能力。但不幸的是,在我们的教育下培养出来的人们,大部分不擅
长处理抽象的事物,不擅长形而上的推理与思辨。

\subsection{总结提炼的能力是每个知识工作者必须训练的基本能力,这种能力的获得也没什么捷径:}
\label{sec:org67a0252}
首先是意识 ,要有完成任务才是学习和个人长进开始的认识,不要认为事情做完了就结束
了;其次是不断的去尝试 ,刚开始试图去总结提炼时甚至不知道做什么或者不会去做,即
便提炼出来的东西也是浮于表面,无法深入到事物的本质,不要着急这都没关系,可以跟其
他人的总结提炼去比对,分析自己差在哪里:是问题界定、是概念能力还是思维的全面性?
找到问题再去改正问题即可。

随着总结能力的提升,总结的结果就会成为你下次做类似事情的指导,也会成为你带团队、
教会别人的原料,你的个人能力也随着提升!不干活不成,但仅会干活的人会沦落为“机
器”,没有人会认为一个机器需要升职和涨工资。所以,要干活,更要通过总结提炼超越干
活!

问题时,我们要总结提炼时该总结什么、提炼什么,得到什么样的结果才算真正有效的反思
并能够真正提升呢?

\subsection{基于我们的研究和实践看,每个人在实践后需要总结的无外乎三个方面:}
\label{sec:orgeec7bff}

\subsubsection{第一个方面:人的方面;}
\label{sec:orgd4028ee}

在老子的《道德经》第三十三章中说到“知人者智,自知者明。”相传刻在古希腊德尔斐的
阿波罗神庙的三句箴言之一,也是其中最有名的一句就是“ 认识你自己 ”。在古老文明中,
无论是欧洲还是中国,都认为了解、认识自己是个人修行的核心。

认识自己的方法有很多种,甚至许多宗教里面都提供了很多方法。但本质上对于现代人而言,
对自我的认识、了解不能仅仅来自于个人的反思和冥想,一定是来自于实践的验证:你认为
的自己跟在实践中验证过的自己是否是一致的,你对世界的认知、见解和判断是否正确一定
需要实践来检验,这也是辩证唯物主义认识论的观点。

德鲁克在自己的著作里面曾经介绍过欧洲的两个宗教组织受天主教统治的南欧的耶稣会和受
耶稣教统治的北欧的加尔文会都会采用一个很好的方法:“每当耶稣会的神甫或加尔文会的
牧师要做任何重要的事时,譬如说进行一项关键的决策,他们被要求 把预期的结果以书面
形式记录下来 。9个月以后,他们必须按照预期结果对实际结果进行 反馈分析 。这样,他
们很快就能知道自己在哪些方面做得很好,自己的优势在哪里, 并且也能知道自己必须在
哪些方面抓紧学习以及必须改变哪些习惯。最后,他们还能知道哪些方面自己缺乏天赋并无
法胜任。我自己采用这种方法至今已有50年了。这种方法能够揭示一个人的长处(一个人能
够了解自我, 这可是最重要的事),并且指出哪些方面需要改进,需要哪种性质的改进,
以及没有能力做的事和甚至不应尝试的事。一个人能了解自己的长处,知道如何发挥自己的
长处和自己无法胜任的事,这可谓是继续学习的关键。”

除了对于自我进行反思分析外,现代人的大部分任务都需要跟人合作才能完成,所以对于人
的方面的总结还涉及到对于他人(同事、外部的协作者等)进行分析。对于管理者,对于管
理对象的深刻理解也是提升个人领导力的基础手段,只有深刻的理解同事的优劣势、性格甚
至包括思维方式,才能够进行有效的分工、授权和合作,便于充分发挥每个人的优劣势,最
终实现目标。

\subsubsection{第二个方面:事的方面;}
\label{sec:orgcdd598c}

完成任务、负责或参与项目都是解决具体的问题,在这种实践中除了人的因素外,是否掌握
了完成事情的方法、是否注意了客观环境的差异性、是否能够随着要求、需求的变化而变化
都是影响事情完成的核心因素。

对于个体而言,总结和提炼首先要去分析事情的结果是否符合预期目标,大致分几种情况:
符合目标、比目标做的还好、比目标做的差、任务或者项目发生了变化没有执行完毕。无论
是哪一种情况,都有可以总结和提炼的地方,也有可以提升的地方:

\subsubsection{符合目标 :如果完成了目标可以考虑是否能够提升、是否可以把完成任务过程里面的认知、}
\label{sec:orga3425ec}
方法和工具标准化、结构化,做成模板、框架、模型,便于下次重用和复用。如果不去做这
些总结提炼,下次再做类似事情的时候仍然是凭运气,很难保证每次都达到目标。

\subsubsection{比目标做的好 :需要去分析是否目标定的合理(有许多故意将目标定的很低,这样每次都}
\label{sec:org0a5b125}
能超额完成)、超额完成目标的经验(认识上的、决策上的、方法上的还是因为外部因素),
然后优势的、创新的地方(流程、方法等)争取能够做到套路化,便于下次重用复用。认识
和决策上的经验能够记录、内化,下次可以重复。

\subsubsection{比目标做的差 :分析是否目标有问题,如果有问题定这样目标后面的原因是什么(个人认}
\label{sec:orge24745d}
识问题、外部问题等);执行过程存在的问题(认识、决策、方法、外部因素等),需要汲
取的教训是什么,未来如何规避。即便是没有完成目标的任务和项目,也可能获得经验,这
些经验亦需要总结提炼出来。

\subsubsection{任务或者项目发生变化 :这其实也是经常存在的问题,随着形式的变化当时的任务或者项}
\label{sec:org27e5b9b}
目已经没有必要了被取消、属于其他的任务或者项目被合并、资源短缺被终止等等,都有可
能。对于这样的任务或项目,通常不是操作者能够决定的,但参与的过程仍然有价值,在这
个过程中的思考、尝试和资源也应该被总结提炼,便于未来可用。

你的能力提升和展现都需要在完成任务和项目中进行,所以总结提炼最核心的部分在于能够
有机会做更多有价值的事情,并能够在做的过程中和结束后每次都能够让自己跃升一步。

\subsubsection{第三个方面:机制的方面。}
\label{sec:org17361d4}

每个知识工作者其实都是在既定的环境中生存和发展,所以对你身边的环境的认识和适应是
个人能力的重要方面。我们每个人都是“带着镣铐跳舞”,对于我们有能力去改变的地方我
们可以勇敢的去做,但对于时机不成熟或者不可能改变的我们只能去认识、理解和利用它。

在实践后的反思和提炼环节,除了人的方面、事的方面,还需要包括你所在环境下机制的方
面:对这个机制进行分析,找到可以利用的地方,规避它们的缺点和问题。譬如在一种环境
里面可能会支撑你做一个伟大的研究,允许你三年什么成果也没有,期待你最后能够一鸣惊
人。但在大部分环境下,如果你想要做一个伟大的产品和服务时,除了心里要有这个伟大的
目标外,你还要在这个过程中去产出让你的领导们能看到并认可的成果,否则你可能连半年
也没有走到项目就被终止了。在实践中你要去总结提炼你所在环境的优劣势,为你自己争取
发挥才能的更大空间。

人的长进需要读书和思考,更需要的是干活和完成任务、解决问题创造价值。在干活和完成
任务的过程中除了验证你所学所思的正确度和客观度,也会促进你学习到更多情景知识、了
解实践的多样性和复杂性,从而真正提升你的能力。但这个过程并不是自发完成的,总结和
提炼是其中极为重要的一环:

这个过程并不容易,因为真正的总结和提炼要求你能够客观的评价个人、他人、项目和任务,
能够真正对自己进行剖析、深挖自己的认识、发现自我的优劣势,这其实对于个人来说并不
舒服。但正是这样的不舒服,才能够让你去实现提升。

---------END---------
汇说FTsay泛金融革命发起与引领者!

普及金融常识,扫除金融文盲!让每个人更平等的参与金融市场是我们的使命!即使你生于
贫困,也绝不能死于贫困!

我们坚信励精修行,方可得道:投资是生活的艺术,交易是一生的修行!愿与你以投资交易
为生,共同奔向财富自由之路!

欢迎关注公号:汇说FTsay 加我微信:huishuoftsay 拉你入伙 找到组织!你不是一个人在
战斗!


\subsection{一直在说“上善若水”,原来这个才是真正的意思!}
\label{sec:org615c883}
\url{http://www.sohu.com/a/157716108\_488223?loc=4\&tag\_id=60056}
\subsection{千字文全文带拼音(附译文)}
\label{sec:org613e8a3}
\url{http://www.bangnishouji.com/guoxue/201305/1336.html}
\subsection{王羲之《兰亭序》神龙本高清晰单字版!}
\label{sec:org14fa5a4}
\url{http://blog.sina.com.cn/s/blog\_60d27f6d0102dv8z.html}
\subsection{孙过庭书谱 \url{http://www.360doc.com/content/14/0709/10/12074893\_393131239.shtml}}
\label{sec:org4d15beb}
\section{Blog Ideas}
\label{sec:org69e51c5}

\subsection{{\bfseries\sffamily TODO} practice}
\label{sec:org50c6c5c}

\textit{[2017-11-07 Tue 14:17]}

\section{文件学习}
\label{sec:org5a474e5}
\subsection{{\bfseries\sffamily DONE} 国家十三五规划}
\label{sec:orgea20531}
\subsection{第六篇 拓展网络经济空间 }
\label{sec:orgeb3188d}
  牢牢把握信息技术变革趋势,实施网络强国战略,加快建设数字中国,推动信息技术与经
济社会发展深度融合,加快推动信息经济发展壮大。


\subsubsection{第二十五章 构建泛在高效的信息网络 }
\label{sec:org7add1f7}

  加快构建高速、移动、安全、泛在的新一代信息基础设施,推进信息网络技术广泛运用,
形成万物互联、人机交互、天地一体的网络空间。

\subsubsection{第一节 完善新一代高速光纤网络 }
\label{sec:org9f467b8}

  构建现代化通信骨干网络,提升高速传送、灵活调度和智能适配能力。推进宽带接入光纤
化进程,城镇地区实现光网覆盖,提供1000兆比特每秒以上接入服务能力,大中城市家庭用
户带宽实现100兆比特以上灵活选择;98\%的行政村实现光纤通达,有条件地区提供100兆比
特每秒以上接入服务能力,半数以上农村家庭用户带宽实现50兆比特以上灵活选择。建立畅
通的国际通信设施,优化国际通信网络布局,完善跨境陆海缆基础设施。建设中国-阿拉伯
国家等网上丝绸之路,加快建设中国-东盟信息港。

\subsubsection{第二节 构建先进泛在的无线宽带网 }
\label{sec:org27dcca3}

  深入普及高速无线宽带。加快第四代移动通信(4G)网络建设,实现乡镇及人口密集的行
政村全面深度覆盖,在城镇热点公共区域推广免费高速无线局域网(WLAN)接入。加快边远
山区、牧区及岛礁等网络覆盖。优化国家频谱资源配置,加强无线电频谱管理,维护安全有
序的电波秩序。合理规划利用卫星频率和轨道资源。加快空间互联网部署,实现空间与地面
设施互联互通。

\subsubsection{第三节 加快信息网络新技术开发应用 }
\label{sec:org8614083}

  积极推进第五代移动通信(5G)和超宽带关键技术研究,启动5G商用。超前布局下一代互
联网,全面向互联网协议第6版(IPv6)演进升级。布局未来网络架构、技术体系和安全保
障体系。重点突破大数据和云计算关键技术、自主可控操作系统、高端工业和大型管理软件、
新兴领域人工智能技术。

\subsubsection{第四节 推进宽带网络提速降费 }
\label{sec:orgd0b55b9}

  开放民间资本进入基础电信领域竞争性业务,形成基础设施共建共享、业务服务相互竞争
的市场格局。深入推进“三网融合”。强化普遍服务责任,完善普遍服务机制。开展网络提
速降费行动,简化电信资费结构,提高电信业务性价比。完善优化互联网架构及接入技术、
计费标准。加强网络资费行为监管。

\subsubsection{第二十六章 发展现代互联网产业体系 }
\label{sec:orgb3426b0}

  实施“互联网+”行动计划,促进互联网深度广泛应用,带动生产模式和组织方式变革,
形成网络化、智能化、服务化、协同化的产业发展新形态。

\subsubsection{第一节 夯实互联网应用基础 }
\label{sec:org68a93aa}

  积极推进云计算和物联网发展。鼓励互联网骨干企业开放平台资源,加强行业云服务平台
建设,支持行业信息系统向云平台迁移。推进物联网感知设施规划布局,发展物联网开环应
用。推进信息物理系统关键技术研发和应用。建立“互联网+”标准体系,加快互联网及其
融合应用的基础共性标准和关键技术标准研制推广,增强国际标准制定中的话语权。

\subsubsection{第二节 加快多领域互联网融合发展 }
\label{sec:org31cca2e}

  组织实施“互联网+”重大工程,加快推进基于互联网的商业模式、服务模式、管理模式
及供应链、物流链等各类创新,培育“互联网+”生态体系,形成网络化协同分工新格局。
引导大型互联网企业向小微企业和创业团队开放创新资源,鼓励建立基于互联网的开放式创
新联盟。促进“互联网+”新业态创新,鼓励搭建资源开放共享平台,探索建立国家信息经
济试点示范区,积极发展分享经济。推动互联网医疗、互联网教育、线上线下结合等新兴业
态快速发展。放宽融合性产品和服务的市场准入限制。 

\subsubsection{第二十七章 实施国家大数据战略}
\label{sec:org81c56a2}

把大数据作为基础性战略资源,全面实施促进大数据发展行动,加快推动数据资源共享开放
和开发应用,助力产业转型升级和社会治理创新。

\subsubsection{第一节 加快政府数据开放共享}
\label{sec:org866052b}

全面推进重点领域大数据高效采集、有效整合,深化政府数据和社会数据关联分析、融合利
用,提高宏观调控、市场监管、社会治理和公共服务精准性和有效性。依托政府数据统
一共享交换平台,加快推进跨部门数据资源共享共用。加快建设国家政府数据统一开放
平台,推动政府信息系统和公共数据互联开放共享。制定政府数据共享开放目录,依法
推进数据资源向社会开放。统筹布局建设国家大数据平台、数据中心等基础设施。研究
制定数据开放、保护等法律法规,制定政府信息资源管理办法。

\subsubsection{第二节 促进大数据产业健康发展}
\label{sec:orgf5403b3}

深化大数据在各行业的创新应用,探索与传统产业协同发展新业态新模式,加快完善大数据
产业链。加快海量数据采集、存储、清洗、分析发掘、可视化、安全与隐私保护等领域关键
技术攻关。促进大数据软硬件产品发展。完善大数据产业公共服务支撑体系和生态体系,加
强标准体系和质量技术基础建设。

\subsubsection{第二十八章 强化信息安全保障}
\label{sec:orgbe81cec}

统筹网络安全和信息化发展,完善国家网络安全保障体系,强化重要信息系统和数据资源保
护,提高网络治理能力,保障国家信息安全。
\subsubsection{第一节 加强数据资源安全保护}
\label{sec:org456daf0}
建立大数据安全管理制度,实行数据资源分类分级管理,保障安全高效可信应用。实施
大数据安全保障工程,加强数据资源在采集、存储、应用和开放等环节的安全保护,加
强各类公共数据资源在公开共享等环节的安全评估与保护,建立互联网企业数据资源资
产化和利用授信机制。加强个人数据保护,严厉打击非法泄露和出卖个人数据行为。
\subsubsection{第二节 科学实施网络空间治理}
\label{sec:orgb0dda9d}
完善网络空间治理,营造安全文明的网络环境。建立网络空间治理基础保障体系,完善
网络安全法律法规,完善网络信息有效登记和网络实名认证。建立网络安全审查制度和
标准体系,加强精细化网络空间管理,清理违法和不良信息,依法惩治网络违法犯罪行
为。健全网络与信息突发安全事件应急机制。推动建立多边、民主、透明的国际互联网
治理体系,积极参与国际网络空间安全规则制定、打击网络犯罪、网络安全技术和标准
等领域的国际合作。
\subsubsection{第三节 全面保障重要信息系统安全}
\label{sec:org71e8a00}

建立关键信息基础设施保护制度,完善涉及国家安全重要信息系统的设计、建设和运行监督
机制。集中力量突破信息管理、信息保护、安全审查和基础支撑关键技术,提高自主保障能
力。加强关键信息基础设施核心技术装备威胁感知和持续防御能力建设。完善重要信息系统
等级保护制度。健全重点行业、重点地区、重要信息系统条块融合的联动安全保障机制。积
极发展信息安全产业。
\subsection{信息化重大工程}
\label{sec:orgd206e9a}
\subsubsection{(一)宽带中国}
\label{sec:org42a0f64}
建设高速大容量光通信传输系统,实施宽带乡村和中西部地区中小城市基础网络完善工程,
扩容互联网国际出入口带宽,部署第四代移动通信(4G)及后续演进技术,在有需求的区域
全面深度覆盖。
\subsubsection{(二)物联网应用推广}
\label{sec:orga139aba}

建设物联网应用基础设施和服务平台,推进物联网重大应用示范工程建设,广泛开展物联网
技术集成应用和模式创新,丰富物联网应用服务。

\subsubsection{(三)云计算创新发展}
\label{sec:org5fd4582}

支持公有云服务平台建设,布局云计算和大数据中心,提升云计算解决方案提供能力。推动
制造、金融、民生、物流、医疗等重点行业云应用服务,不断完善云计算生态体系。

\subsubsection{(四)“互联网+”行动}
\label{sec:org216d26f}

推动“互联网+”创业创新、协同制造、智慧能源、普惠金融、益民服务、高效物流、电子
商务、便捷交通、绿色生态、人工智能以及电子税务、便民司法、教育培训、科普、地理信
息、信用、文化旅游等行动,不断拓展融合领域。

\subsubsection{(五)大数据应用}
\label{sec:org1efc905}

建设统一的开放平台,逐步实现公共数据集开放,鼓励企业和公众发掘利用。推动政府治理、
公共服务、产业发展,技术研发等领域大数据创新应用。推进贵州等大数据综合试验区建设。

\subsubsection{(六)国家政务信息化}
\label{sec:org3844f14}

加快国家统一电子政务网络建设应用,完善审批监管、信息信息、公共资源交易、价格举报
信息等平台。加快国家基础信息资源库建设应用。

\subsubsection{(七)电子商务}
\label{sec:org0dc64ba}

支持电子商务基础设施建设,促进重点领域电子商务创新和融合应用,推动杭州等跨境电子
商务综合试验区建设,打造电子商务国际大通道。

\subsubsection{(八)网络安全保障}
\label{sec:orgd1c63db}

实施国家信息安全专项,提高关键信息基础设施、重要信息系统和涉密信息系统安全保障能
力及产业化支撑水平。实施国家网络空间安全重大科技项目,突破核心芯片、基础软件、关
键元器件及重点整机系统等关键技术,构建国家网络空间安全和保密技术保障体系。

\subsection{\href{http://www.gov.cn/zhengce/content/2015-05/19/content\_9784.htm}{中国制造2025}}
\label{sec:org256469a}

(五)健全多层次人才培养体系。
  加强制造业人才发展统筹规划和分类指导,组织实施制造业人才培养计划,加大专业技
术人才、经营管理人才和技能人才的培养力度,完善从研发、转化、生产到管理的人才培养
体系。以提高现代经营管理水平和企业竞争力为核心,实施企业经营管理人才素质提升工程
和国家中小企业银河培训工程,培养造就一批优秀企业家和高水平经营管理人才。*以高层次、
急需紧缺专业技术人才和创新型人才为重点,实施专业技术人才知识更新工程和先进制造卓
越工程师培养计划,在高等学校建设一批工程创新训练中心,打造高素质专业技术人才队伍。
强化职业教育和技能培训,引导一批普通本科高等学校向应用技术类高等学校转型,建立一
批实训基地,开展现代学徒制试点示范,形成一支门类齐全、技艺精湛的技术技能人才队伍。*
鼓励企业与学校合作,培养制造业急需的科研人员、技术技能人才与复合型人才,深化相关
领域工程博士、硕士专业学位研究生招生和培养模式改革,积极推进产学研结合。加强产业
人才需求预测,完善各类人才信息库,构建产业人才水平评价制度和信息发布平台。建立人
才激励机制,加大对优秀人才的表彰和奖励力度。建立完善制造业人才服务机构,健全人才
流动和使用的体制机制。采取多种形式选拔各类优秀人才重点是专业技术人才到国外学习培
训,探索建立国际培训基地。加大制造业引智力度,引进领军人才和紧缺人才。

\subsection{\href{http://www.miit.gov.cn/n1146290/n1146392/c5168620/content.html}{国家信息化发展战略纲要}}
\label{sec:orgfb8c000}

\subsection{\href{http://www.gov.cn/zhengce/content/2015-07/04/content\_10002.htm}{国务院关于积极推进“互联网+”行动的指导意见}}
\label{sec:orga6892b0}

\subsection{\href{http://www.gov.cn/zhengce/content/2016-05/20/content\_5075099.htm}{国务院关于深化制造业与互联网融合发展的指导意见}}
\label{sec:orgb55cf84}

\subsection{信息产业发展指南}
\label{sec:orga4c2d7f}


“十二五”以来,我国信息产业发展势头良好,产业体系不断完善,产业链掌控能力显著提
高,正日益成为我国创新发展的先导力量、驱动经济持续增长的新引擎、引领产业转型和融
合创新的新动力、提升政府治理和公共服务能力的新手段。当前,以信息技术与制造业融合
创新为主要特征的新一轮科技革命和产业变革正在孕育兴起,必须紧紧抓住这一机遇,加快
发展具有国际竞争力、安全可控的现代信息产业体系,为建设制造强国和网络强国打下坚实
基础。为科学引导“十三五”时期信息产业持续健康发展,根据“十三五”规划纲要、《中
国制造2025》、《国家信息化发展战略纲要》、《国务院关于积极推进“互联网+”行动的
指导意见》(国发〔2015〕40号)、《国务院关于深化制造业与互联网融合发展的指导意见》
(国发〔2016〕28号)等的部署,经国务院同意,特制定本指南,实施期限为2016-2020年。


\subsection{一、发展回顾及面临形势}
\label{sec:org4446b50}
****(一)“十二五”发展回顾。
“十二五”时期我国信息产业发展取得显著成效,比较优势和竞争能力发生深刻变化。一是
产业规模平稳较快增长。2015年信息产业收入规模达到17.1万亿元。彩电、手机、微型计算
机、网络通信设备等主要电子信息产品的产量居全球第一,电话用户和互联网用户规模居世
界首位。二是结构优化升级取得实质进展。2015年,软件和信息技术服务业占信息产业收入
比重由2010年的16\%提高到25\%,移动数据及互联网业务收入占电信业收入比重提升至27.6\%。
电子信息产品竞争力明显提升,对外贸易顺差稳步扩大。三是技术创新能力大幅提升。国内
信息技术专利申请总量已超过304.8万件,其中发明专利申请总量和授权量分别超过193.7万
件和7.48万件。具有自主知识产权的时分同步码分多址长期演进技术(TD-LTE Advanced)
成为第四代移动通信(4G)国际主流标准之一,并实现大规模商用。集成电路设计水平达到
16/14纳米,制造业实现28纳米小批量生产。多条高世代平板显示生产线建成投产。安全可
靠软硬件实现重要突破,一批骨干企业创新能力和竞争力大幅提升。四是信息基础设施加速
升级。宽带接入实现从非对称数字用户线路(ADSL)向光纤入户(FTTH)的跨越,移动通信
实现从3G向4G的升级。新增七个国家级骨干直联点建成开通,网间互通质量和效率大幅提升。
中国铁塔公司成立,电信基础设施共建共享迈向新高度。五是信息产业支撑引领作用全面凸
显。信息产业快速发展带动两化融合水平稳步提升,互联网对经济社会促进作用逐步显现。
2015年网络零售交易额达3.88万亿元,一批互联网龙头企业建立开放平台,成为带动大众创
业、万众创新的新渠道、新推力。智慧城市、智慧交通、远程医疗、互联网金融等新业态不
断涌现,加速经济社会运行模式深度变革。

但与此同时,我国信息产业核心基础能力依然薄弱,核心芯片和基础软件对外依存度高,要
素成本增长较快,关键领域原始创新和协同创新能力急需提升,引领产业发展方向、把握产
业发展主导权的能力不强;产品供给效率与质量不高,与发达国家相比,呈现出“应用强、
技术弱、市场厚、利润薄”的倒三角式产业结构;信息技术融合应用深度不够,新产品、新
业态、新模式发展面临体制机制障碍;网络与信息安全形势依然严峻,安全保障能力亟待提
升。

\subsubsection{(二)“十三五”发展形势。}
\label{sec:org61885c7}
新一轮技术创新引领产业新变革。全球信息产业技术创新进入新一轮加速期,云计算、大数
据、物联网、移动互联网、人工智能、虚拟现实等新一代信息技术快速演进,单点技术和单
一产品的创新正加速向多技术融合互动的系统化、集成化创新转变,创新周期大幅缩短,硬
件、软件、服务等核心技术体系加速重构,新业态、新模式快速涌现,我国信息产业实现跨
越发展的战略机遇窗口正在打开。同时,信息技术与制造、材料、能源、生物等技术的交叉
渗透日益深化,我国已形成的局部技术优势将面临新的挑战。


全球信息产业竞争加剧分工格局调整。发达国家依然占据信息产业价值制高点,在大力构建
信息经济新优势的同时,积极以信息技术为手段推动再工业化进程,争取未来全球高端产业
发展主导权。跨国企业加快重组步伐,以期在工业互联网、人工智能、智能制造等领域形成
新布局。一些信息产业新兴国家(地区)加快谋篇,积极参与全球产业再分工,承接资本及
技术转移。我国已成为全球最大的信息产品消费市场和制造基地,在互联网、通信服务、设
备与终端产品等领域形成了一批龙头企业,在全球产业分工体系中呈跃升态势,具备了跨越
发展的条件。同时,也面临发达国家“高端回流”和发展中国家“中低端分流”的双向挤压,
以及国内要素禀赋深刻变化、新旧增长动力转换的严峻挑战,转型升级任务更加紧迫艰巨。

国家重大战略实施对信息产业发展提出新要求。从世界范围看,信息产业日益成为重塑经济
发展模式的主导力量,创新融合、智能绿色、开放共享成为全球经济发展新特征。在我国,
信息产业也日益成为实施创新驱动战略、推进供给侧结构性改革的关键力量。创新驱动、制
造强国、网络强国、“互联网+”、军民融合等一系列国家重大战略的实施和居民消费升级,
要求加快完善信息基础设施、强化信息核心技术能力、提升信息消费体验、加强信息安全保
障、优化网络空间治理、繁荣信息产业生态,发挥更强有力的引领和支撑作用。

\subsection{二、总体要求}
\label{sec:org24de86d}
\subsubsection{(一)指导思想。}
\label{sec:orgebe312b}
全面贯彻党的十八大、十八届二中、三中、四中、五中、六中全会精神和习近平总书记系列
重要讲话精神,认真落实党中央、国务院决策部署,按照“五位一体”总体布局和“四个全
面”战略布局,牢固树立创新、协调、绿色、开放、共享的发展理念,推进供给侧结构性改
革,以支撑制造强国和网络强国等重大战略实施为使命,以加快建立具有全球竞争优势、安
全可控的信息产业生态体系为主线,坚持追赶补齐与换道超车并举、技术突破与强化应用并
重、对外合作与体系创新结合、全面发展与重点推进统筹,着力强化科技创新能力、产业基
础能力和安全保障能力,突破关键瓶颈,优化产业结构,提升产品质量,完善基础设施,深
化普遍服务,促进深度融合应用,拓展网络经济空间,加快重点项目建设和关键环节发展,
带动全面提升信息产业发展质量效益和核心竞争力,推动经济社会持续健康发展,支撑全面
建成小康社会奋斗目标如期实现。

\subsubsection{(二)基本原则。}
\label{sec:org21443ba}
——创新引领。坚持把创新作为引领发展的第一动力。着力提升核心基础软硬件创新能力,强
化关键共性技术研发供给,推动产业链协同创新。强化企业创新主体地位和主导作用,培育
一批具有国际竞争力的创新型领军企业。

——融合发展。坚持软件与硬件、技术与产品、产业链上下游等融合协同发展,完善产业生态
体系。促进军民用信息技术和产品深度融合,推动信息产业与其他行业跨界融合、集成创新,
加快传统行业改造提升,大力发展新业态、新模式。推动数据开放,加强共建共享,提高资
源利用效率。

——市场主导。充分发挥市场在资源配置中的决定性作用,更好发挥政府作用,强化企业主体
地位和市场应用牵引,深入推进简政放权、放管结合、优化服务,加快转变政府职能,为信
息产业创新发展和提质增效营造更加良好的市场环境。

——开放合作。坚持走出去与引进来相结合。进一步提升双向开放合作水平,优化信息网络国
际布局,提升产业国际化布局和运营能力,积极推动建立国际互联网发展新秩序,加强国际
间信息产业技术、标准、人才及产能合作。

——安全可控。统筹发展和安全,以安全保发展、以发展促安全。强化法治建设、标准制定、
技术支撑和市场监管,壮大信息安全产业,推进行业自律和社会监督,健全关键信息基础设
施安全保障体系。

——绿色低碳。坚持绿色发展、循环发展和低碳发展。推进信息技术在生产各环节的应用,加
速传统产业绿色化转型。加快提升电子信息产品和设备能效,不断降低信息基础设施能耗水
平。提高电子信息产品回收再利用水平。

\subsubsection{(三)发展目标。}
\label{sec:orgf247367}

到2020年,具有国际竞争力、安全可控的信息产业生态体系基本建立,在全球价值链中的地
位进一步提升。突破一批制约产业发展的关键核心技术和标志性产品,我国主导的国际标准
领域不断扩大;产业发展的协调性和协同性明显增强,产业布局进一步优化,形成一批具有
全球品牌竞争优势的企业;电子产品能效不断提高,生产过程能源资源消耗进一步降低;信
息产业安全保障体系不断健全,关键信息基础设施安全保障能力满足需求,安全产业链条更
加完善;光网全面覆盖城乡,第五代移动通信(5G)启动商用服务,高速、移动、安全、泛
在的新一代信息基础设施基本建成。


2020年信息产业发展主要指标

注:1.[ ]内数值为年均增速;
2.信息产业企业进入世界500强企业数量指标,指中国大陆进入《财富》500强的企业数量。

\subsection{三、主要任务}
\label{sec:org55a7f97}


\subsubsection{(一)增强体系化创新能力。}
\label{sec:org89191ae}

构建先进的核心技术与产品体系。围绕产业链体系化部署创新链,针对创新链统筹配置资源
链,着力在云计算与大数据、新一代信息网络、智能硬件等三大领域,提升体系化创新能力。
瞄准重大战略需求和未来产业发展制高点,支持专业机构研究发布重点领域技术创新指南,
提出瓶颈短板清单及优先级,引导市场主体创新突破。加强产学研用研发力量协调,统筹利
用国家科技计划(专项、基金等),支持关键核心技术研发和重大技术试验验证,强化关键
共性技术研发供给。加快信息产业军民融合深度发展,在技术研发、产业布局中充分考虑军
用需求和国防布局,着力加强军民联合攻关,在优先满足军工需要的同时带动民口技术进步
和产业发展。加强前沿领域重大布局,重点在未来网络、量子计算、平流层通信、卫星通信、
可见光通信、车联网、地海空天一体化网络、人工智能、类脑计算等关键领域,集中优势资
源开展原始创新和集成创新,增强新供给创造能力,抢占产业技术发展主动权和制高点。

建设高水平创新载体和服务平台。充分利用已有创新资源,探索政产学研用联合的新机制新
模式,在集成电路、基础软件、大数据、云计算、物联网、工业互联网等战略性核心领域布
局建设若干创新中心,开展关键共性技术研发和产业化示范。强化企业技术创新主体地位和
主导作用,支持优势企业建设一批高水平技术中心和创新实验室,支持企业联合高校、科研
机构等建设重点领域产学研用联盟,积极参与和组建开源社区,支持企业牵头承担国家重大
科技研发和产业化项目。整合优化信息科技资源,积极发挥行业协会/联盟、标准化组织、
中介组织和智库在战略与政策研究、统计分析、公共服务等方面的作用,建设和提升一批技
术创新、成果转化、标准规范、计量测试、认证检测、市场推广等公共服务平台。

强化标准体系建设与知识产权运用。进一步优化国家标准、行业标准、军用标准体系结构,
支持发展团体标准,加快构建产业化导向、军民通用的新一代信息技术标准体系,研究制定
智能硬件、传感器、智慧家庭、虚拟现实、云计算、大数据、太阳能光伏、锂离子电池等领
域综合标准化技术体系。加快基础标准、通用标准、安全标准、测试方法以及重点产品标准
制修订工作,不断提升技术、能耗、环保、质量、安全等方面规范要求。积极参与国际标准
化战略规划、政策和规则的制定,以国际标准提案为核心,推动我国更多信息通信领域标准
成为国际标准;加快转化我国产业发展急需的国际先进标准,推动国际国内标准接轨。建立
专利导航产业发展工作机制,加强信息产业关键核心技术知识产权储备和战略布局,推动技
术创新成果的知识产权转移转化;鼓励市场主体组建产业知识产权联盟,建立知识产权联合
创造、协同运用、共同保护和风险分担的机制;研究制定重点领域知识产权运营策略,健全
运营服务体系,促进知识产权的收储、许可和转让;支持引导行业组织、产业联盟加强知识
产权分析评议,防控知识产权风险。

\subsubsection{(二)构建协同优化的产业结构。}
\label{sec:org09a7b7c}
打造协同发展产业链。依托优势骨干企业,建设和完善信息网络、云计算、大数据、物联网、
工业互联网、智能终端、电子制造关键装备等一批重要产业链,以“硬件+软件+内容+服务”为
架构建设形成若干具有国际竞争力的产业生态。支持有条件的企业通过兼并重组、股权投资
等方式开展产业链上下游垂直整合和跨领域价值链横向拓展,提升价值创造能力和核心竞争
力。以产业集群为中心,实施商标品牌发展战略,提升产业链整体质量水平,加强团体标准、
知识产权和公共服务平台建设,强化商标品牌宣传与营销,打造一批具有国际影响力的产业
集群区域品牌。

提升产业基础能力。围绕基础软硬件、关键制造工艺、关键电子基础材料和工艺装备等,制
定重点领域瓶颈清单,组织实施重点领域“一揽子”突破计划。依托制造业质量提升专项行
动,针对信息产业重点产品,组织攻克一批长期困扰产品质量提升的关键共性质量技术。加
强可靠性和可测性设计、试验验证,积累准确有效的工艺参数数据,推广采用先进质量管理
方法、先进成型和加工方法、在线计量检测装置等,提高电子信息装备、材料和工艺技术的
可靠性、一致性、稳定性和有效性。制定和提升一批急需的国家计量基标准,加强信息产业
相关国家计量测试中心建设,构建信息产业计量测试服务体系。推动基础软硬件、基础材料
和工艺装备企业与下游企业对接,组织开展首台(套)、首批次示范应用,加快安全可靠基
础软硬件产品的市场化应用和推广。

增强企业创新活力。在信息产业重点领域设立市场化运作的投资基金,支持企业开展兼并重
组和引技引智,提高企业利用全球资源和开拓国际市场的能力和水平,形成以大企业集团为
核心、集中度高、分工细化、协作高效的产业组织形态。进一步完善和落实支持中小企业发
展的财税、金融政策,推动小微企业创业创新基地建设,大力扶持初创期创业创新型企业发
展。引导中小企业专注细分市场,激发中小企业创新活力,发展一批专精特“隐形冠军”企
业。充分发挥各类平台作用,支持信息产业中小企业创新发展,引导大中小企业建立更紧密
协作关系。支持企业将具有核心竞争力的专利技术向标准转化,提高企业综合竞争力。引导
企业树立质量为先、信誉至上的经营理念,切实增强质量和品牌意识,培育和弘扬精益求精
的工匠精神。全面提升行业企业信息技术运用能力,加快个性化制造、网络化协同制造、智
能制造等生产方式变革,创新发展新模式,推动企业向价值链高端转型。

优化产业空间布局。贯彻落实国家区域发展总体战略和主体功能区规划,引导地方发挥比较
优势,形成集成电路、基础软件、平板显示、智能终端、信息技术服务、云计算、大数据等
重点领域生产力差异化发展格局。重点推动长江经济带、珠三角、京津冀等创新资源密集地
区率先突破,建设具有全球竞争力的信息产业创新高地。支持中西部地区立足自身优势承接
信息产业转移,重点支持若干基础和条件较好的中心城市提高研发能力和产业层次,在特色
领域形成差异化竞争优势。合理引导人才、技术、资金、政策等要素资源集聚,建设一批信
息产业领域国家新型工业化产业示范基地,不断提高软件名城建设水平。扎实推进数据中心
布局优化,促进数据中心合理利用。

推动产业绿色发展。支持促进企业升级生产技术及工艺,鼓励企业开发绿色产品,推行电子
信息产品绿色设计,降低电子信息产品生产和使用能耗,引导绿色生产,促进绿色消费。持
续提高电子信息产品中有毒有害物质的限量要求,严格检测环节,确保限用物质含量符合国
家标准。研发支撑数据中心能源使用效率(PUE)量值等效可溯源的计量测试技术、方法和
装置。鼓励企业研发应用节能型服务器设备,采用高压直流、自然风冷等新型节能技术发展
绿色云计算数据中心。加快现有数据中心、基站等信息网络设施的节能改造,鼓励老旧高耗
能设备淘汰退网和绿色节能新技术应用。推动废弃电器电子产品处理与资源化利用技术研发,
制定废弃电器电子产品及重点拆解产物资源综合利用相关标准,搭建和推广基于互联网的回
收服务信息平台,推动生产者履行废弃电器电子产品回收处理相关责任,鼓励专业化回收处
理企业发展,促进再制造产业规模化发展。推动统一绿色产品标准、认证、标识体系的建立
实施。

\subsubsection{(三)促进信息技术深度融合应用。}
\label{sec:org4201fe3}

推动信息技术与制造业融合创新。推动制造业、“互联网+”和“双创”紧密结合,加快新
一代信息技术更大范围、更深程度融合渗透和创新应用,推动制造业智能化、绿色化、服务
化发展。建立完善智能制造和两化融合管理标准体系,全面推进两化融合管理体系贯标。推
进“数控一代”示范工程,加快突破传感器、可编程逻辑控制器(PLC)、工业控制系统等
智能制造核心信息设备,提升安全可靠水平。开展智能制造试点示范。推进信息物理系统
(CPS)关键技术研发及产业化,构建综合验证平台,开展行业应用测试和试点示范。以工
业云、工业大数据、工业电子商务和系统解决方案等为重点,开展制造业与互联网融合发展
试点示范,培育一批面向重点工业行业智能制造的系统解决方案领军企业。实施工业云及工
业大数据创新应用试点,建设一批高质量的工业云服务和工业大数据平台,推广个性化定制、
网络协同制造、远程运维服务等智能制造新模式。建设大型制造企业“双创”平台和为中小
企业服务的第三方“双创”服务平台,营造大中小企业协同共进的“双创”新生态。依托强
基工程,面向智能制造关键环节应用需求,重点扶持发展一批应用效果好、技术创新强、市
场认可度高的工业软件,推动先进适用工业软件在重点行业应用普及。积极推动用信息技术
改造提升制造业,着力提高产品和服务附加值。

积极推进“互联网+”行动。依托互联网平台,大力发展众创、众包、众扶、众筹,促进互
联网和经济社会融合发展。建立“互联网+”标准体系,加快互联网及其融合应用的基础共
性标准和关键技术标准研制推广。整合政府部门、电信企业、互联网企业、行业机构等各类
资源,集成资源申请、能力开放、技术支撑、创业孵化、测试认证、实验环境、业务咨询等
创业创新服务,提升信息通信企业对“双创”服务平台的支撑能力。推进“互联网+”安全
生产,提升安全生产重点领域企业的全过程、全链条在线监测和预警预控能力,强化跨部门、
跨区域信息共享与业务协同。开展新型网络经济培育行动,支持互联网企业、信息技术服务
企业、制造企业联合打造服务产业转型的平台经济模式,加快人工智能、云计算、大数据等
在经济活动中的发展应用,强化对智慧交通、智慧能源、智慧环保、高效物流、益民服务、
普惠金融、智慧医疗、现代农业等的支撑,发展基于电网的通信设施和新型业务。培育信息
消费新业态,拓展网络经济新空间。

加快发展信息技术服务。围绕政务、金融、能源、交通、环保、安全生产、电子商务、数字
内容等关键领域,提升信息技术服务企业的咨询设计、软件开发、集成实施、运行维护和测
试验证能力。支持信息技术企业突破业务建模、远程智能检查、大规模资源调度管理、自动
化运维、数据治理等关键技术,发展互联网运维服务、网络众包服务、微服务、智能服务等
新模式、新业态,加强对区块链、人工智能、虚拟现实、增强现实等新兴技术在行业系统解
决方案中的应用推广,加快向高端价值服务提供商转型。选择信息技术服务业集聚发展的城
市或区域,开展面向制造业的信息技术服务应用示范。总结行业先进实践经验,制定完善信
息技术服务相关规范。加快综合集成和智能运维平台研发和产业化进程,提升信息技术服务
保障能力。实施信息技术服务标准化工作五年行动计划,完善和推广信息技术服务标准
(ITSS),鼓励企业加快服务标准化和产品化。

\subsubsection{(四)建设新一代信息基础设施。}
\label{sec:org0e13eaa}

加快高速光纤宽带网建设。引导建成一批光网城市,基本完成老旧小区宽带接入铜缆替换,
鼓励企业通过引入新技术、更新老旧光缆等,进一步提升光纤宽带网络高速传送、灵活调度
和智能适配能力,消除宽带网络接入“最后一公里”瓶颈。进一步优化互联网骨干网络架构,
推动网间互通扩容和质量提升。开展新型交换中心试点,完善全方位、多层次、立体化的网
络互联体系。推动地面数字电视覆盖网和超高清交互式电视网络设施建设。实施电信普遍服
务补偿机制,推动相关企业加快对农村地区宽带网络覆盖和能力提升,基本实现行政村光网
全覆盖,并逐步向有条件的自然村延伸。

推动宽带无线接入网络升级演进。继续推动长期演进(LTE)网络建设,实现深度和广度覆
盖,提升网络质量。加速低速率和低频谱利用率网络退网和频率重耕,发展认知无线电技术,
拓宽4G网络发展空间,实现频分双工长期演进LTE FDD和TD-LTE融合发展。加强无线局域网
(WLAN)新技术研究,鼓励在城镇热点公共区域推广WLAN接入,提升WLAN与移动通信网络的
协同融合能力。推动5G网络研发和应用。加快边远山区、牧区及岛礁等的网络覆盖。

提升应急通信保障能力。着力提升应急通信保障网络能力、可用性和覆盖范围。完善国家应
急通信保障、装备储备体系。支持应急体系相关单位加强应急指挥手段建设,推动与应急通
信指挥系统信息共享。加强国家应急通信设施建设和通信保障队伍建设。完善天空地一体的
应急通信保障网络,推广突发事件预警信息系统应用。加强应急通信技术支撑能力建设。

增强卫星通信网络及应用服务能力。统筹规划卫星通信发展,加快卫星通信标准制定和更新,
推进关键部件、卫星整机、通信终端和系统、地面信息基础设施协调建设和军民融合发展,
推进天地一体化信息网络建设。构建宽带卫星电子政务网、防灾和应急卫星通信网,建设多
种卫星端站,补充地面网络难以布设地区的通信需求。推动卫星通信发展,逐步拓展建立区
域化、商业化的卫星通信服务体系,持续完善北斗导航技术,加快推动基于北斗的高精度时
频设备研发及应用,实现产业安全可控。创新北斗导航应用模式,发展位置服务,开展应用
示范。

加强下一代互联网应用和未来网络技术创新。推动下一代互联网改造升级和大规模商用,实
现互联网协议第4版(IPv4)向第6版(IPv6)的平滑过渡和业务互通。加强未来网络顶层设
计,加强未来网络长期演进的战略布局和技术储备,开展网络体系架构、安全性和标准研究,
重点突破软件定义网络(SDN)/网络功能虚拟化(NFV)、网络操作系统、内容分发等关键
技术,推动关键技术试验验证,组织开展规模应用试验。


\subsubsection{(五)提升信息通信和无线电行业管理水平。}
\label{sec:org0ca9773}

创新互联网行业管理。坚持政策引导和依法管理并举、鼓励支持和规范发展并行,促进互联
网持续健康发展。创新监管体系,积极运用大数据等先进技术加强对市场主体监管,形成覆
盖资源、接入、网络、业务各层面的互联网行业全周期管理体系。完善互联网基础资源管理
体系,严格落实网站、域名、IP地址和电话实名制。加快推广使用IPv6地址,推动开放IPv6
国际连接。建立和完善多部门联动管理机制,建立新业务备案和发展指引制度,加强互联网
与实体经济融合新型业务联合管理。坚持放管结合,推进以信用体系为代表的全流程监管支
撑体系建设,强化事中事后监管。建立互联网市场主体信用评价体系,依托国家企业信息公
示系统建立企业信息归集共享机制,健全守信联合激励和失信联合惩戒制度,推进市场分级
预警,营造公平诚信的市场环境。加强服务质量监管,保护用户权益和个人信息。积极引导
社会力量参与互联网行业管理,完善行业规范与自律公约,引导行业协会和第三方机构开展
行业自律、社会监督、评估认证等活动,推进形成政府主导,多方参与的共同治理格局。

完善电信行业管理。着力夯实电信业基础性支撑地位,建设高品质信息基础设施,提升行业
服务能力和质量。加快开展电信普遍服务试点工作。深入推进网络提速降费,推动简化电信
资费结构,提高电信业务性价比,规范企业经营、服务和收费行为。进一步放开竞争性领域
市场准入,抓好自贸区电信领域开放试点,推动对港澳等地区开放合作。

优化无线电频率和卫星轨道资源管理。优化国家频谱资源配置,加强无线电频谱管理,维护
安全有序的电波秩序。科学规划和合理配置无线电频率资源,统筹重点业务部门以及战略性
新兴产业发展的中长期用频需求,促进宽带中国、信息消费、“中国制造2025”和“互联网
+”行动涉及的无线电业务发展。加强对无线电频率和卫星轨道资源使用的基础性、前瞻性、
战略性重大问题及相关技术研究,加强卫星频率和轨道资源的可用性论证,做好卫星网络资
料的国际申报、协调及登记工作。开展无线电频谱使用评估,促进频谱资源有效开发利用。
深化台站管理,加强事中事后监管。加大无线电管理基础和技术设施建设投入,加强无线电
监管能力建设,实现广域、泛在的城区无线电监测网络覆盖,增强电波秩序维护能力。

\subsubsection{(六)强化信息产业安全保障能力。}
\label{sec:orgfe49bd1}

完善网络与信息安全管理制度。加紧制定实施关键信息基础设施保护、数据安全、工业互联
网安全等领域的部门规章和规范性文件。健全网络与信息安全标准体系,推动出台5G、物联
网、云计算、大数据、智能制造等新兴领域安全标准。加强安全可靠电子签名应用推广,推
动电子签名法律效力认定。建立健全身份服务提供商管理制度。明确关键信息基础设施安全
保护责任,完善涉及国家安全重要信息系统的设计、建设和运行监督机制,进一步加强对互
联网企业所有或运营的重要网络基础设施和业务系统的网络安全监管。健全跨行业、跨部门
的应急协调机制,切实提升网络与信息安全事件的预警通报、监测发现和快速处置能力。加
强政府和企业之间的安全威胁信息共享。加快推动实施网络安全审查制度。

加强大数据场景下的网络数据保护。探索建立大数据时代的网络数据保护体系,推动对网络
数据的分级分类监管,强化网络数据全生命周期保护,制定网络数据保护管理政策。督促企
业不断完善用户信息泄露社会公告制度,建立健全大数据安全信用体系。加快推动数据加密、
防泄露、信息保密等专用技术的研发与应用,推动建立安全可信的大数据技术体系。

推动信息安全技术和产业发展。着力突破关键基础软硬件和信息安全核心技术,增强漏洞挖
掘修补、攻击监测溯源等能力,强化“互联网+”、5G、SDN等新技术、新业态的安全风险应
对。实施国家信息安全专项,开展关键信息基础设施运行安全保护和要害信息系统网络安全
试点示范。推动信息安全产品和服务的研发和产业化应用。充分发挥政府引导作用,加快培
育骨干企业,发展特色优势企业,打造结构完整、层次清晰、竞争有力的产业格局。

提升工业信息安全保障能力。建立健全工业信息安全政策和标准体系,针对重点行业制定安
全管理政策以及管理指南、测评能力要求等安全标准。建立工业信息安全管理体系,完善工
业信息安全检查评测和信息共享机制,推动开展安全检查、漏洞发布、信息通报等工作,营
造安全的工业互联网环境。建设工业信息安全仿真、测试和验证平台,开展测试评估、安全
验证等技术研发,推动安全新技术、新产品试点应用,提升工业信息安全技术保障能力。


\subsubsection{(七)增强国际化发展能力。}
\label{sec:org813af3a}

提升产业国际化发展水平。推动引资与引技、引智相结合,鼓励和支持信息产业企业与境外
优势企业在研发创新、新产品开发、标准制定、品牌建设等高端环节开展合资合作,提高引
进来层次。支持企业在境外设立研发中心,充分利用各种国际创新资源。结合国家重大战略
实施,以信息基础设施建设、终端产品产能合作、重大工程总集成总承包等为牵引,带动产
业链上下游企业、先进技术标准、信息网络设备、配套服务等体系化、集群化走出去。支持
有条件的企业建设境外信息产业合作园区。提供企业走出去国别目录、项目对接等服务,引
导金融机构开展金融服务,降低企业走出去风险。深入推动中文域名推广和使用。主动参与
国际互联网标准制定,提高参与制定国际规则的能力及影响力。

优化信息网络国际布局。依托“一带一路”战略,构建高效跨境信息通道,推动与周边国家
信息通信设施互联互通,创新国际通信设施建设和运营模式,重点打通经中亚到西亚、经南
亚到印度洋、经俄罗斯到欧洲等陆上通道,推进重点方向国际海缆建设。完善我国国际通信
出入口布局,以亚非欧拉为主要方向提升我国国际互联网能力,加快推进海外网络服务提供
点(PoP)和互联网数据中心(IDC)建设。推进电信企业设立海外分支机构,加强国际通信
的质量监测和服务提升,为“走出去”中资企业及海外用户提供更完善、更优质信息服务,
实现我国信息业务的海外运营和落地。

\subsection{四、发展重点}
\label{sec:orge373f35}


\subsubsection{(一)集成电路。}
\label{sec:org7d885b1}

以重点整机和重大应用需求为导向,增强芯片与整机和应用系统的协同。着力提升集成电路
设计水平,不断丰富知识产权(IP)核和设计工具,突破中央处理器(CPU)、现场可编程
门阵列(FPGA)、数字信号处理(DSP)、存储芯片(DRAM/NAND)等核心通用芯片,提升芯
片应用适配能力。加快推动先进逻辑工艺、存储器等生产线建设,持续增强特色工艺制造能
力。掌握高密度封装及三维(3D)微组装技术,探索新型材料产业化应用,提升封装测试产
业发展能力。加紧布局超越“摩尔定律”相关领域,推动特色工艺生产线建设和第三代化合
物半导体产品开发,加速新材料、新结构、新工艺创新。以生产线建设带动关键装备和材料
配套发展,基本建成技术先进、安全可靠的集成电路产业体系。实施“芯火”创新行动,充
分发挥集成电路对“双创”的支撑作用。


\subsubsection{(二)基础电子。}
\label{sec:orgc7fcdb2}

大力发展满足高端装备、应用电子、物联网、新能源汽车、新一代信息技术需求的核心基础
元器件,提升国内外市场竞争力。拓展新型显示器件规模应用领域,实现液晶显示器超高分
辨率产品规模化生产、有源矩阵有机发光二极管(AMOLED)产品量产;突破柔性制备和封装
等核心技术,完成量产技术储备,开发10英寸以上柔性显示器件。突破微机电系统(MEMS)
微结构加工、高密度封装等关键共性技术,加快传感器产品开发和产业化。提升发光二极管
(LED)器件性能,推动高端场控电力电子器件推广应用,开发下一代电力电子器件,支持
典型领域推广应用。加强电子级多晶硅、高效太阳能电池及组件封装工艺创新和技术储备,
提升光伏发电系统集成水平及储能设备配套水平。积极发展电子纸、锂离子电池、光伏等行
业关键电子材料,重点突破高端配套应用市场。提升电子专用设备配套供给能力,重点发展
12英寸集成电路成套生产线设备、新型薄膜太阳能电池生产设备、锂离子电池关键材料生产
设备、新型元器件生产设备和表面贴装设备。研发半导体和集成电路、通信与网络、物联网、
新型电子元器件、高性能通用电子等测试设备。


\subsubsection{(三)基础软件和工业软件。}
\label{sec:orgd79773c}
建立安全可靠的基础软件产品体系,支持开源、开放的开发模式,重点推进云操作系统、云
中间件、新型数据库管理系统、移动端和云端办公套件等基础软件产品的研发和应用。强化
技术产品和终端应用协同互动,提升基础软件成熟度,加快集成适配优化。推动工业软件和
工业控制系统核心技术和产品的研发及应用,重点突破军工、能源、化工等安全关键行业工
业应用软件核心关键技术,构建先进产品体系,形成评测标准与规范;突破高档数控系统、
现场总线、通信协议、高精度高速控制和伺服驱动等工业控制系统关键技术,推动中高端数
控系统、伺服系统和控制系统研发。构建国家工业软件安全测试平台。加快工业大数据软件
与平台布局,促进重要工业领域系统解决方案定制化深度应用,打造工业云应用服务体系。


\subsubsection{(四)关键应用软件和行业解决方案。}
\label{sec:org27288c2}

着力发展基于云计算、大数据、移动互联网、物联网等新型计算框架和应用场景的软件平台
和应用系统。针对政府应用、公共服务、行业发展等重点需求,集中突破一批重点应用软件
和行业解决方案,深化普及应用。支持软件和信息技术服务企业面向公共服务领域积极开展
应用解决方案研发和信息技术服务,推动软件企业与传统行业企业深入合作,加快支撑传统
行业转型升级的软件及解决方案发展和应用,培育一批综合性解决方案提供商。

\subsubsection{(五)智能硬件和应用电子。}
\label{sec:orgbd29585}

突破人工智能、低功耗轻量级系统、智能感知、新型人机交互等关键核心技术,重点发展面
向下一代移动互联网和信息消费的智能可穿戴、智慧家庭、智能车载终端、智慧医疗健康、
智能机器人、智能无人系统等产品,面向特定需求的定制化终端产品,以及面向特殊行业和
特殊网络应用的专用移动智能终端产品。发展高水平“互联网+”人工智能平台,提升消费
级和工业级智能硬件产品及服务供给能力。加快智能感知技术创新,重点推动毫米波与太赫
兹、蜂窝窄带物联网(NB-IOT)、智能语音等技术在公共安全、物联网等重点领域开展示范
应用。支持虚拟现实产品研发及产业化,探索开展在设计制造、健康医疗、文体娱乐等领域
的应用示范。丰富智慧家庭产品供给,重点加大智能电视、智能音响、智能服务机器人等新
型消费类电子产品供给力度,推动完善智慧家庭产业链,引导产业向“产品+服务”转型升
级。开展智慧健康养老服务应用,支持健康监测和管理、家庭养老看护等可穿戴设备发展。
推广智慧交通创新与应用示范,推动基于宽带移动互联网的智能汽车与智慧交通示范区建设。
积极推进工业电子、医疗电子、汽车电子、能源电子、金融电子等产品研发应用。


****(六)计算机与通信设备。

引导产业链上下游合作,突破高端服务器和存储设备核心处理器、内存芯片和输入/输出
(I/O)芯片等核心器件,构建完善高端服务器、存储设备等核心信息设备产业体系。研究
神经元计算、量子计算等新型计算技术应用。支持发展低功耗低成本绿色计算产品,强化芯
片、软件、系统与应用服务适配,开展绿色计算应用示范,丰富应用服务模式,推动绿色计
算生态良性发展。创新绿色计算产业合作机制,搭建绿色计算产品创新公共服务平台,开发
和完善绿色计算接口标准、应用规范与产品检测认证体系。加快高性能安全工业控制计算机
以及可信计算、数据安全、网络安全等信息安全产品的研发与产业化。支持安全可靠工业控
制计算机在电网、水利、能源、石化等国民经济重要领域的应用。开发高速光传输设备及大
容量组网调度光传输设备,发展智能光网络和高速率、大容量、长距离光传输、光纤接入
(FTTx)等技术和设备。积极推进5G、IPv6、SDN和NFV等下一代网络设备研发制造。

\subsubsection{(七)大数据。}
\label{sec:orgcf89dbc}

突破大数据关键技术和产品,培育壮大大数据服务业态,完善大数据产业体系。深化大数据
应用创新,发展面向工业领域的大数据服务和成套解决方案。鼓励工业企业整合各环节数据
资源,基于大数据应用开展个性化定制、众包设计、智能监测、全产业链追溯、在线监控诊
断及维护、工控系统安全监控、智能制造等新业务。引导企业加快商业和服务模式创新,构
建基于大数据的民生服务新体系,在公共安全、自然灾害防治、环境保护等城市管理领域,
拓展和丰富服务范围、形式和内容。开展大数据产业集聚发展和应用示范区创建工作。在重
点行业开展应用试点,推进政府、金融、能源等重要行业大数据系统安全可靠软硬件应用。
培育数据采集、数据分析、数据安全、数据交易等新型数据服务产业和企业。在依法合规、
安全可控前提下加快大数据交易产业发展,开展第三方数据交易平台建设试点示范。组织制
定数据交易流通的一般规则和信息披露制度,逐步完善数据交易流通中的个人信息保护、数
据安全、知识产权保护等制度,建立数据交易流通的行业自律和监督机制。

\subsubsection{(八)云计算。}
\label{sec:orgcc1743b}

积极发展基础设施即服务(IaaS)、平台即服务(PaaS)、软件即服务(SaaS)等云服务,
提升公有云服务能力,扩展专有云应用范畴,围绕工业、金融、电信、就业、社保、交通、
教育、环保、安监等重点领域应用需求,支持建设全国或区域混合云服务平台。大力发展云
服务应用软件,促进各类信息系统向云计算服务平台迁移。积极发展基于云计算的个人信息
存储、在线工具、学习娱乐等服务。鼓励大企业开放平台资源,加强行业云服务平台建设。
建立为中小企业提供办公、生产、财务、营销、人力资源等基本管理服务的云计算平台。大
力发展面向云计算的信息系统规划咨询、方案设计、系统集成和测试评估等服务。支持第三
方机构开展云计算服务质量、可信度和网络安全等评估评测。优化云计算基础设施布局,建
设完善云计算综合标准体系。完善云计算环境下网络信息安全管理体系,加强技术管理系统
建设,强化新技术新业务评估,防范网络信息安全风险。

\subsubsection{(九)物联网。}
\label{sec:org05ff64f}

实施物联网重大应用示范工程,发展物联网开环应用,加快物联网技术与产业发展、民生服
务、生活消费、城市管理以及能源、环保、安监等领域的深度融合,形成一批综合集成解决
方案。应用物联网技术推动大田耕种精准化、园艺种植智能化、畜禽养殖高效化,促进形成
现代农业经营方式和组织形态。以车联网、智慧医疗、智能家居、智能可穿戴设备等为重点,
通过与移动互联网融合加快消费领域物联网应用创新。推进物联网感知设施规划布局,深化
物联网在智慧城市基础设施管理方面的应用。建立城市级物联网接入管理与数据汇聚平台,
推动感知设备统一接入、集中管理和数据共享利用。大力发展工业互联网,成立工业互联网
产业联盟,加快制定工业互联网标准体系,推动产业协同创新。组织开展工业互联网试点示
范,建设公共服务平台和管理平台,强化基础设施建设,全面打造低时延、高可靠、广覆盖
的工业互联网。



\subsection{五、政策措施}
\label{sec:org293663a}


\subsubsection{(一)深化体制机制改革。}
\label{sec:orgd2db6dd}

落实行政审批制度改革要求,积极推动电信法、网络安全法等立法。实现跨部门、跨平台的
网上并联审批,取消不必要的审批目录和不合理收费,完善负面清单,积极应用大数据、云
计算等新技术创新行业服务和管理方式。加快完善招投标和政府采购机制。加强事中事后监
管,完善监测和惩戒机制。积极推动电信领域混合所有制改革,鼓励民间资本通过多种形式
参与信息通信业投融资,激发非公有制经济和小微企业的活力与创造力。深化国有电信企业
改革,增强企业活力。加大基础电信领域竞争性业务的开放力度,通过市场竞争促进企业提
升服务质量。推动制定用户权益和个人信息保护相关法规,以及网络数据和用户信息保护分
级分类标准及具体规则。健全产业安全审查机制和法规体系,加强业务开放情况下的网络与
信息安全风险控制。加快发展信息产业技术市场,健全知识产权创造、运用、管理、保护机
制,严打假冒伪劣。

\subsubsection{(二)完善财税扶持政策。}
\label{sec:orgc771a8a}

创新财政支持方式,优化财政资金投入,充分利用国家新兴产业创业投资引导基金、先进制
造产业投资基金、中小企业发展基金、国家集成电路产业投资基金等政策性基金引导社会资
金,支持重大产业化项目发展。加强资源协调,充分利用现有资金渠道支持重大生产力布局、
关键产品产业化和重点产品示范项目。完善和落实支持创新的政府采购政策,推动信息产业
创新产品和服务的研发应用。鼓励地方积极探索利用政府抵用券等方式支持信息企业引进创
新技术、购置或租赁设备、培养人才等。继续落实软件和集成电路税收支持政策,以及研发
费用加计扣除和固定资产加速折旧等政策,推动设备更新和新技术应用。推动符合条件的重
大信息技术装备列入《首台(套)重大技术装备推广应用指导目录》,通过有关保险补偿试
点支持推广应用。

\subsubsection{(三)加大金融支持力度。}
\label{sec:orgd54895b}

建立完善多层次资本市场,支持符合条件的信息产业创新创业企业充分利用创业板拓宽融资
渠道,推动在全国中小企业股份转让系统挂牌的符合条件的信息产业中小企业向创业板转板。
丰富信息产业直接融资工具,积极推动项目收益票据、项目收益债、可转债等的应用。加大
产融信息对接力度,建立完善跨部门工作协调机制,搭建服务平台。鼓励商业银行创新信贷
产品和金融服务,推动知识产权质押融资、股权质押融资、供应链融资、信用保险保单质押
贷款等金融产品创新,在风险可控和商业可持续前提下,加大对信息产业发展的金融支持力
度。鼓励开发性、政策性金融机构在业务范围内,为符合条件的信息产业相关项目提供信贷
支持。按照国家统一部署,引导和支持符合条件的金融机构在试点地区面向电子信息领域创
新企业探索开展投贷联动试点,引导银行业金融机构对“中国制造2025”、“互联网+”行
动等涉及的信息产业重点领域实施差别化信贷政策。支持符合条件的信息产业企业建立资金
管理平台。鼓励信息产业骨干企业通过并购票据、并购基金、并购债等开展海外并购。

\subsubsection{(四)大力培养产业人才。}
\label{sec:org4416bb9}

鼓励高校加强信息产业新兴领域学科专业建设,面向产业发展需求制定人才培养目标和质量
标准,鼓励校企合作,建立人才实训基地。培养产业急需的各类科研人员、技术人才和复合
型人才,联合开展在职人员培训。继续做好国家软件与集成电路人才国际培训基地工作,充
分发挥基地作用,缩短人才培养周期。鼓励企业依托行业协会加强人才协作,推动信息产业
与传统制造业人才交流。设立融合型就业人才综合信息平台。实施企业经营管理人才素质提
升工程和专业技术人才知识更新工程,以急需紧缺人才为重点,着力加强信息技术领域专业
技术人才和经营管理人才培养。做好职业培训和职业资格认证工作。加强人才需求调查和预
测,建立健全信息产业高层次人才信息库。建设计算机技术与软件专业技术资格(水平)考
试和通信专业技术水平考试合格人员数据库,为企业选人用人提供服务。促进人才双向交流,
鼓励专业技术人才到国外学习培训交流,重点实施软件和集成电路人才出国培训专项;继续
实施软件和集成电路产业外国专家引进计划,引进一批具有国际影响力的学术技术带头人和
关键技术项目负责人。完善适应信息产业发展要求的人才引进和激励政策,创新引进渠道,
积极引进新型显示、智能硬件、云计算、大数据等领域高端人才。

\subsubsection{(五)切实加强组织实施。}
\label{sec:orgab3fc8e}

在国家制造强国建设领导小组的领导下,工业和信息化部、发展改革委联合牵头,各成员单
位分工协作、加强配合,共同推动指南落实。加强上下联动,引导各地区结合实际合理布局、
有序推进重大应用示范和产业化项目,减少低水平重复建设和投资,促进差异化发展。充分
发挥国家制造强国建设战略咨询委员会和相关行业协会/联盟等的作用,加强对信息产业新
技术、新产品、新业态、新模式、新趋势的跟踪研究。建立指南任务落实情况督促检查和第
三方评价机制,扎实开展动态监测和中期评估工作。做好“十三五”时期信息产业各行业领
域相关规划、政策、专项和工程与本指南的衔接。


\subsection{{\bfseries\sffamily DONE} \href{https://wenku.baidu.com/view/82acb4f8be23482fb5da4c13.html}{德国工业4.0与中国制造2025的区别}}
\label{sec:org58a5613}
德国工业4.0在全球产生重大影响,标志着全球加快全面进入以智能制造为核心的智能经济
时代。德国工业4.0与美国比较流行的第三次工业革命提法和一些学者说的第五次工业革命
等,都是以信息技术革命性突破为基础,反映了工业经济数字化、信息化、智能化、网络
化的发展趋势。本文着重分析德国工业4.0战略的主要特点,比较中国制造2025,得出一些
重要启示。

\subsection{德国工业4.0主要特点}
\label{sec:orga040586}

德国工业4.0可以概括为:一个核心,两个重点,三大集成,四个特征和六项措施。一个核心:互联网+制造业,将信息物理融合系统(CPS)
广泛深入地应用于制造业,构建智能工厂、实现智能制造。两个重点:领先的供应商策略,
成为“智能生产”设备的主要供应者;主导的市场策略,设计并实施一套全面的知识和技术
转化方案,引领市场发展。三大集成:企业内部灵活且可重新组合的纵向集成,企业之间
价值链的横向集成,全社会价值链的端到端工程数字化集成。四个特征:生产可调节,可
自我调节以应对不同形势;产品可识别,可以在任何时候把产品分辨出来;需求可变通,可
以根据临时的需求变化而改变设计、构造、计划、生产和运作,并且仍有获利空间;四是过
程可监测,可以实时针对商业模式全过程进行监测。六项措施:实现技术标准化和开放标
准的参考体系;建立复杂模型管理系统;建立一套综合的工业宽带基础设施;建立安全保障机
制和规章制度;创新工作组织和设计方式;加强培训和持续职业教育。

把德国工业4.0放在全球范围和历史背景中比较发现,可以看出德国工业4.0有以下5个
重要特点。

\subsubsection{基础性}
\label{sec:org9d50824}

与德国整体发展战略相关,发挥优势,应对全局性挑战。德国是一个很重视发展
战略规划的国家,特别是进入21世纪以来,一直在努力建立部门间高技术战略协调机制,
制定德国的创新发展国家战略。2006年。德国政府制定了“高科技战略”,2010年7月,发
布了《德国2020高技术战略》报告,2011年11月,德国政府特别提出把德国工业4.0作为
《德国2020高技术战略》的重心,具有非同一般的意义,充分说明德国制造业在德国整体
发展战略中的基础性作用。工业4.0有助于促进工业-科研联盟瞄准中长期科学和技术发展
目标,制定具体的创新战略和实施路线图,确保德国制造业的领先和优势地位,是保持德
国可持续发展的基础性战略。

\subsubsection{策略性}
\label{sec:org4362fc6}

与国际社会竞争相关,在国际相近的战略思维中,寻求不同的策略。加快先进制造业的发展
不仅仅德国高度重视,其他国家也同样高度重视。事实上,早在2006年,美国国家科学基
金会(NSF)就把CPS系统确定为关键研究领域。2011年3月,欧盟公布了“欧洲2020战略”,
提出800亿欧元预算,使其成为世界上最大的研发经费计划。日本也提出了类似的“工业智
能化”战略,重点发展人工智能、服务机器人等产业。德国很清楚自身近些年在竞争中的
不利态势,面临很大的压力。面对竞争形势,德国采取了包容、开放、有策略的战略,在
竞争中合作,在合作中竞争。德国为工业4.0配套制定了领先的供应商策略和主导的市场策
略,重点考虑将产品与恰当的服务相衔接,着力开发新的商业模式。这是很值得借鉴的。

\subsubsection{创新性}
\label{sec:org2cf1a38}

与工业发展历史相关,把累积、继承、创新有机结合起来,形成高水平的创新。
一方面,充分考虑了从工业化早期阶段吸取经验,继承和发扬现有工业的核心价值。在德
国,一般意义的信息和通信技术(ICT)至今仍然支持90\%的工业制造过程。因此,工业4.0很
重视借助传统工业和研究领域的传统优势。另一方面,德国工业4.0将制造领域的所有因素
和资源通过CBS系统构成全新的社会服务和实时保障平台,体现了深度的创新性。德国“工
业4.0”就是一种“再工业化”战略。但是与美日的“再工业化”不同,德国的“工业
4.0”核心在于利用互联网、物联网以及模块化技术,实现工业生产方式的变革,从根本上
改变传统工业中由于地理位置导致的生产、研发脱节现象,使工业生产技术的研发和升级
不再依赖物理上的互相接触,使德国在既保持自己科技研发技术优势的同时,也可以继续
享受全球化生产的优势。德国通过新理念、新战略、新技术,把信息化推向质的变化阶段,
推动以智能制造、互联网、新能源、新材料、现代生物为特征的新的工业革命。这一点是
很高明的。

\subsubsection{前瞻性}
\label{sec:org3d30f7c}

与未来趋势相关,正确研判当前的形势和未来的趋势,体现事
物发展规律。工业4.0是一个长期的发展战略;智能经济、智能世界是一个中长期的发展目
标。在现实与中长期目标之间,德国一方面特别注重充分发挥自身优势,对智能工厂进行
多层次、多视角的透视和描述,既体现可操作性,又体现前瞻性,为正确选择战略重点、
采取超常规的相应举措,提供理论和实践基础。另一方面,德国把实现技术标准化放在最
为优先的地位,抢先制定行业标准,企图继续占领全球制造业的制高点,除了在经济上获
得现实利益外,更重要的是为长远确立德国制造优势创造条件。

\subsubsection{市场性}
\label{sec:org9ddf15c}

与市场发展需求相关,适应市场、构建市场、引领市场,占据市场制高点。德国
把智能工厂作为智能化世界的一部分,以“工业智能化”带动“社会智能化”,为“社会
智能化”提供新一代网络基础设施、先进通信技术、智能控制系统设计、“大数据”分析
方法等各类软硬件,将成熟的“工业解决方案”转化为“社会服务解决方案”,在不同的
地区、不同的领域推广应用,构建新的市场空间。德国在构建市场的过程中,特别把CBS技
术和产品确立为主导市场,根据不同消费者的需求设计系列化的智能产品,建设智能社会,
建立全新的商业机会和模式,必将产生难以估量的市场价值。

\subsection{中国制造2025与德国工业4.0的比较}
\label{sec:org8105534}

2015年,中国在分析国内外市场的基础上,遵循产业升级与转型的客观规
律,编制中长期十年规划,颁布了中国制造2025,明确了十个重点行业,包含战略性新兴
产业、先进制造业、其他关系到国计民生的传统行业,以及相应的供应链和销售网。其主
线是两化深度融合,主攻方向是推进智能制造,主要形式是互联网+。“中国制造2025”与
德国“工业4.0”都是在新一轮科技革命和产业变革背景下针对制造业发展提出的一个重要
战略举措。比较两个战略可以看出各有特点,除了技术基础和产业基础不同之外,还存在
战略思想等方面的明显差异。德国工业4.0为德国工业发展描绘了细致的发展蓝图,反映了
德意志民族特有的认真与严谨,在战略思想、基础研究、技术教育、政策机构和措施方面
有很多值得我们学习和参考。

\subsubsection{1战略思想的差异}
\label{sec:org83fe059}

比较德国工业4.0与中国制造2025, 一个重要的区别在于,德国工业4.0战略是一个革
命性的基础性的科技战略。其立足点并不是单纯提升某几个工业制造技术,而是从制
造方式最基础层面上进行变革,从而实现整个 工业发展的质的飞跃。因此,德国工业
4.0战略的核心内容并不拘泥于工业产值数据这个层面上“量的变化”,而更加关注工
业生产方式的“质的变化”。相对于德国工业4.0,《中国制造2025》,则强调的是在
现有的工业制造水平和技术上,通过“互联网+”这种工具的应用,实现结构的变化和
产量的增加。这种区别就好比《中国制造2025》是在工业现阶段水平和思维模式上寻
求阶段内的改进和发展,德国则是寻求从工业3.0阶段跨越到工业4.0阶段,实现“质
的变化”。这种战略思想上的差别应该说是客观条件的反映,符合现实基础,但也说
明中国制造2025缺少战略上的理论深度和技术高度,也缺少市场上的感召力和影响力。



\subsubsection{2战略基础的差异}
\label{sec:orgf54d8d4}

战略基础包括基础研究、技术教育、人才培养等,是战略
实施成功的基本条件。仔细研究《德国工业4.0》,我们不难发现这个战略最
重要的因素是基础科学研究,很多细节方面的任务目标,都以“高、精、尖”的理论知识作
为依凭。致力于改善德国科学基础研究的条件,提高科研创新能力。相比之下,中国基础
学科的研究比较薄弱,科研创新能力不强,很难有重大突破。其根本原因,除历史基础条
件因素之外,也有政策的因素。在政策支持上,中国横向研究比纵向研究无论在数目上,
还是支持力度上都要大很多,导致中国应用型的研究领域较强,理论基础研究较薄弱。中
国还在制定国际化行业标准方面缺乏经验和条件。因此,我们有必要下大力加强基础研究。
同时,我们还有必要采取开放式的合作方针,积极成为网络化先进理论和先进标准体系的
重要接入者,积极开展国际合作,与包括德国在内的发达国家一起分享理论、技术与市场。
3战略措施的差异 在配套政策方面,德国为了有效实施工业4.0,比较重视对技术、政策和
环境等进行评估调整。比如,德国系统评估新技术对相关法律可能造成的颠覆性影响,以
及创新周期缩短可能导致相关规则架构频繁更新等,及时对现行不利于发展的各项规章制
度进行了修改。德国比较重视构建支持工业4.0的法律环境,及时对与企业责任、数据保护、
贸易限制、密码系统等相关法规进行调整,培养全国国民的竞争意识,比较重视反思和自
我调适。这一点很值得我们借鉴。

\subsubsection{在协同机构方面,德国成立了政府统一协调机构,}
\label{sec:orgefc12f9}

建立了第四次工业革命平台。德国信息技术通讯新媒体协会、德国机械制造联合会以及德
国电子工业联合会三个专业协会共同建立了秘书处,负责为优先主体研发路线图。我国除
了在中央政府层面成立由国务院领导同志担任组长的领导机构和战略咨询委员会之外,还
应该大力发挥行业协会的作用,加强行业协同机制建设。

\subsection{德国工业4.0比较中国制造2025对中国的启示}
\label{sec:orge09a219}

\subsubsection{1积极迎接智能经济新时代 工业4.0将使人类-技术}
\label{sec:org5e4f390}
(human-technology)和人类-环境(human-environment)的相互作用发生全新转变。借助CPS
系统,特别是互联网+,可以巨大地提升人的智能。智能是把人的智慧和知识转化为一种行
动能力。基于人类智慧、电脑网络和物理世界有机融合的经济具有更高的效率,这种效率
是传统工业无法达到的,因而智能一旦出现将以新的结构和形态取代传统工业,形成“智
能经济”。在智能经济时代,智能环保、智能建筑、智能交通、智能医疗等,构成智能经
济的不同领域;智能家庭、智能企业、智能城市、智能地区、智能国家、智能世界,构成智
能社会的不同层面。在智能经济时代,全球经济一体化的整体性更加突出,市场主体相互
之间内在联系更加紧密,社会经济系统对外更加开放。以智能工厂为特征的智能经济也很
可能是工业经济发展的最高阶段。可以预料:世界的不平衡性将更加突出,竞争的形式将
会改变,全球治理方式将有重大变化。对此,我们要有一定的准备,在发展战略、科技创
新、人文道德上占据制高点,形成良好态势。

\subsubsection{2积极探索中国特色工业化道路}
\label{sec:org2f1f59e}

我国还是一个发展中的国家,仍处于工业化进程中,落后与先进并存、传统与现代共生,需要积
极探讨中国特色工业化道路:提升传统产业与培育新兴产业相结合;传统手工艺与先进制造
业相结合;第一次工业化与第二次工业化相结合;信息化与工业化相结合。我国相当一个时
期可能还需要同时推动“工业2.0”、“工业3.0”和“工业4.0”,既要实传统产业的转型
升级,还要实现在高端领域的跨越式发展,建立既符合中国实际情况、又体现世界发展潮
流的中国工业体系,为全面实现小康社会,实现现代化提供坚实和广宽的基础。既要考虑
提高劳动生产率,又要考虑解决就业问题。

\subsubsection{3正确认识发达国家再工业化中的中国制造业}
\label{sec:orgcc5ab2e}

在当前国际形势下,中国制造业面临三方面挑战:一方面来自高端挑战。发达国家通
过“再工业化”,将“再工业化”与新的工业革命相结合,必定使发达国家在科技、信息、
资本等方面长期积累的优势进一步强化,成为科技革新与产业革命红利的主要受益者,使
不利于发展中国家的“中心-外围”世界分工体系进一步固化,进一步拉大与我国的距离。
另一方面来自低端挤压,印度、越南、印尼等发展中国家可能以更低的劳动力成本承接劳
动密集型产业的转移,抢占制造业的中低端,我国制造业在中低端广大市场的优势面临失
去的危险。再一方面来自内部的困境。从整体来看,我国自主创新能力不强,核心技术对
外依存度较高;制造业仍处于产业中低端水平,缺乏世界一流大型企业与知名品牌,在全球
产业链中的高附加值环节份额相对较小;产业结构不尽合理,技术密集型产业和生产型服务
业比重偏低,产业集聚和集群发展水平不高,产品质量问题比较突出;资源利用效率偏低,
环保问题严重;管理水平不高,管理效率低,导致管理成本高,严重影响产品竞争力。

中国制造业也迎来了三大机遇。首先是新的契机。新一轮科技革命和产业变革与我国加快
转变经济发展方式形成历史性交汇,国际产业分工格局进入重塑阶段,新理念、新技术、
新方式启动期有很多空白点,在某种程度上为全球提供了新的起跑线,也为中国赶超发展
提供了契机。其次是新的供需。发达国家“再工业化”与正在兴起的新一轮工业革命有机
结合,向我们展现了不同于传统流水线、集中化机器大生产的全新生产方式、生产要素、
组织模式,必将创造新的市场和供需,这些都是我国可以大展身手之处。再次,发达国家
“去工业化”和“再工业化”为我们提供了经验教训。发达国家过度“去工业化”及发展
高风险、高杠杆的金融业务,导致实体经济与虚拟经济脱节,我国充分汲取其教训,借鉴
其“再工业化”发展战略中具有前瞻性、符合发展大势的政策措施,根据不同类型行业的
特点,有重点、有差别地推进结构优化升级,通过突破研发、设计、营销网络、品牌和供
应链管理等制约产业结构升级的关键环节,完全有可能加快改造提升制造业。化挑战
为机遇,可能要考虑“争两头,保中间”的战略规划格局,建立中国特色的现代产业体系。
一头是集中优秀力量,大力增强集成创新能力,培育原始创新能力,加快拥有一批核心关
键技术,在一些重要的高端领域,争取一席之地。这一点,我们过去做到了,今后也应该
做到,从而在国际核心俱乐部有一定的话语权。一头是继续争取在低端有一定份额,努力
创造更多的就业机会。我们应该在长期底端基础上有所升级,全部升到高端是不现实的,
升到中端应该是我们的主要选项。克服“中国制造”所面临的困境,成为国内外市场优良
(中端)产品和服务的重要提供者。

\subsubsection{4高度重视互联网+企业组织变革}
\label{sec:orgfec73b4}

“互联网+”是科技与经济的有机结合,在实施“互联网+”战略中,互联网+企业组织变革具有特别重要的
意义,企业作为市场的重要主体和经济的细胞,除了利用互联网加强与市场的互联和联系、
推动网络化协同制造和服务之外,还要下大功夫增强内生动力,焕发内部活力。如何利用
信息技术改善重构生产要素,深化企业组织变革,创新生产方式,提升资产质量和服务功
能,适应市场需求和变化,是一个影响中国制造2025战略全局性的问题。解答这个问题首
先需要正确理解技术与组织的关系。技术结构与企业组织结构的关系是相互促进和相互构
建的过程,特别是互联网技术将企业的消费者、供应商、合作者和企业员工等各种关系全
部组织在电脑网络里,使信息的获取、处理、传递和应用变得高速便捷,必然要求企业的
生产方式、管理模式和组织机构做相应的调整和变革。在这种情况下,只有深化企业组织
变革,将互联网技术和企业生产方式紧密联合起来,形成有效的信息沟通反馈机制,才能
实现技术与组织的良性互动,才能使互联网技术的发展为企业所需要,企业才能成为推动
企业技术进步的主要力量。

\subsubsection{5加强中国制造2025基础工作}
\label{sec:org9e61017}
我国对基础研究、基础培训、基础设施等方面的重要性有一定的认识和措施,但缺乏深度、缺乏核心、缺乏灵魂。一项
大的战略,特别是涉及到一个国家中长期发展的大战略,必须要有自己的系统、深厚的理
论基础,必须要有自己的核心关键的创新技术,必须要有创新理念、勇于担当、能够解决
问题的人才。

虽然我们可以通过向西方发达国家学习的方式缩小技术差距,但是如果
理论研究无法赶上去,那么将永远落后于别人,进而失去真正的竞争力。在基础研究和基
础培训两方面,德国都有很多宝贵经验,值得我们学习。

如何加强基础研究、基础培训呢?可以考虑从基础设施建设着手。基础设施建设也是中国制
造2025重要一部分内容。中国制造2025必须要有配套的基础设施和能够获得的相应材料。
比较深入地研究分析中国制造基础设施工程,可以发现问题、解决问题,体现以问题为导
向的创新研究思路,既有针对性地加强理论研究,又为中国制造提供基础条件。从目前情
况看,很有必要梳理出中国制造重要基础设施名目,比如,宽带互联网基础设施、高效大
容量数据基础设施、IT基础设施、统一的安全保障构架和独特的标识符等。在比较参考国
际相应的先进基础设施基础之上,很有必要逐项制定中国制造基础设施项目的理论研究方
案和工程建设方案,为中国制造2025夯实基础。

不可否认,新的工业革命浪潮正在兴 起,智能世界的前景正在展现,激烈竞争的号角已经吹响。我们急不得,也慢不得。关键
是要深入研判,长线布局,措施得当。“中国制造”需要从要素驱动转变为创新驱动;从低
成本竞争优势转变为质量效益竞争优势;从资源消耗大、污染物排放多的粗放制造转变为绿
色制造;从生产型制造转变为服务型制造。中国制造发展战略核心要义应该是:以加快新一
代信息技术与制造业融合为主线,以提质增效为中心,以满足经济社会发展和国防建设对
重大技术装备需求为目标,以开放合作为手段,强化工业基础能力,提高综合集成水平,
完善多层次人才体系,培育有中国特色的制造文化,促进产业转型升级,实现制造业由大
变强的历史跨越。中国完全有可能成为新的工业革命的重要受益者,也为新的工业革命做
出重要贡献。



\subsection{国务院关于进一步扩大和升级信息消费持续释放内需潜力的指导意见}
\label{sec:orgfc7fd75}

作者:来源:中国政府网2017-08-24 17:59




 
国务院部署扩大和升级信息消费 力争2020年启动5G商用
 
  国务院关于进一步扩大和升级信息消费持续释放内需潜力的指导意见
  国发〔2017〕40号
  各省、自治区、直辖市人民政府,国务院各部委、各直属机构:
  近年来,随着互联网技术与经济社会深度融合,我国信息消费快速发展,正从以线上为主加快向线上线下融合的新形态转变,网络提速降费深入推进,消费主体不断增加、边界逐渐拓展、模式深刻调整,带动其他领域消费快速增长,已成为当前创新最活跃、增长最迅猛、辐射最广泛的经济领域之一,对拉动内需、促进就业和引领产业升级发挥着重要作用。但与此同时,我国信息消费有效供给仍然创新不足,内需潜力仍未充分释放,消费环境亟待优化。为进一步扩大和升级信息消费、持续释放发展活力和内需潜力,现提出以下意见。
  一、总体要求
  (一)指导思想。
  全面贯彻党的十八大和十八届三中、四中、五中、六中全会精神,深入贯彻习近平总书记系列重要讲话精神和治国理政新理念新思想新战略,认真落实党中央、国务院决策部署,统筹推进“五位一体”总体布局和协调推进“四个全面”战略布局,坚持稳中求进工作总基调,牢固树立和贯彻落实创新、协调、绿色、开放、共享的发展理念,以推进供给侧结构性改革为主线,优化信息消费环境,进一步加大网络提速降费力度,加速激发市场活力,积极拓展信息消费新产品、新业态、新模式,扩大信息消费覆盖面,加强和改进监管,完善网络安全保障体系,打造信息消费升级版,不断释放人民群众日益增长的消费需求,促进经济社会持续健康发展。
  (二)基本原则。
  坚持创新驱动。推动信息消费与大众创业万众创新、“互联网+”深度融合,鼓励核心技术研发和服务模式创新,促进新一代信息技术向消费领域广泛渗透,创造更多适应消费升级的有效供给,带动多层次、个性化的信息消费发展。
  坚持需求拉动。以满足人民群众期待和经济社会发展需要为出发点和落脚点,加快拓展和升级信息消费,推动信息产品供给结构与需求结构有效匹配、消费升级与有效投资良性互动,用安全、便捷、丰富的信息消费助力经济升级和民生改善。
  坚持协同联动。以企业为主体,促进信息消费产业链协同发展,加强网络、平台、支付、物流等支撑能力建设,构建完善的信息消费生态体系。统筹促发展与保安全,持续优化信用安全、市场环境和权益保护,营造“能消费、敢消费、愿消费”的环境,形成政府、企业、消费者多方协同的良好发展格局。
  (三)发展目标。
  到2020年,信息消费规模预计达到6万亿元,年均增长11\%以上;信息技术在消费领域的带动作用显著增强,信息产品边界深度拓展,信息服务能力明显提升,拉动相关领域产出达到15万亿元,信息消费惠及广大人民群众。信息基础设施达到世界领先水平,“宽带中国”战略目标全面实现,建成高速、移动、安全、泛在的新一代信息基础设施,网络提速降费取得明显成效。基于网络平台的新型消费快速成长,线上线下协同互动的消费新生态发展壮大。公共数据资源开放共享体系基本建立,面向企业和公民的一体化公共服务体系基本建成。网络空间法律法规体系日趋完善,高效便捷、安全可信、公平有序的信息消费环境基本形成。
  (四)重点领域。
  生活类信息消费。创新发展满足人民群众生活需求的各类便民惠民服务新业态,重点发展面向社区生活的线上线下融合服务、面向文化娱乐的数字创意内容和服务、面向便捷出行的交通旅游服务。
  公共服务类信息消费。推广高效、均等的在线公共服务,重点发展面向居家护理的智慧健康服务、面向便捷就医的在线医疗服务、面向学习培训的在线教育服务、面向利企便民的“互联网+政务服务”。
  行业类信息消费。培育支撑行业信息化的新兴信息技术服务,重点发展面向垂直领域的电子商务平台服务,面向信息消费全过程的网络支付、现代物流、供应链管理等支撑服务,面向信息技术应用的综合系统集成服务。
  新型信息产品消费。升级智能化、高端化、融合化信息产品,重点发展面向消费升级的中高端移动通信终端、可穿戴设备、数字家庭产品等新型信息产品,以及虚拟现实、增强现实、智能网联汽车、智能服务机器人等前沿信息产品。
  二、提高信息消费供给水平
  (五)推广数字家庭产品。鼓励企业发展面向定制化应用场景的智能家居“产品+服务”模式,推广智能电视、智能音响、智能安防等新型数字家庭产品,积极推广通用的产品技术标准及应用规范。加强“互联网+”人工智能核心技术及平台开发,推动虚拟现实、增强现实产品研发及产业化,支持可穿戴设备、消费级无人机、智能服务机器人等产品创新和产业化升级。依托消费品工业“三品”专项行动,促进信息产品相关企业争创“中国质量奖”。
  (六)拓展电子产品应用。支持利用物联网、大数据、云计算、人工智能等技术推动各类应用电子产品智能化升级,在交通、能源、市政、环保等领域开展新型应用示范。推动智能网联汽车与智能交通示范区建设,发展辅助驾驶系统等车联网相关设备。推进农业物联网区域试验工程,推动信息技术与农业生产经营、市场流通、资源环境保护等相融合。
  (七)提升信息技术服务能力。支持大型企业建立基于互联网的“双创”平台,为全社会提供专业化信息服务。发挥好中小企业公共服务平台作用,引导小微企业创业创新示范基地平台化、生态化发展。鼓励信息技术服务企业积极发展位置服务、社交网络等新型支撑服务及智能应用。支持地方联合云计算、大数据骨干企业为当地信息技术服务企业提供咨询、研发、培训等技术支持,推动提升“互联网+”环境下的综合集成服务能力。鼓励利用开源代码开发个性化软件,开展基于区块链、人工智能等新技术的试点应用。
  (八)丰富数字创意内容和服务。实施数字内容创新发展工程,加快文化资源的数字化转换及开发利用。构建新型、优质的数字文化服务体系,推动传统媒体与新兴媒体深度融合、创新发展。支持原创网络作品创作,加强知识产权保护,推动优秀作品网络传播。扶持一批重点文艺网站,拓展数字影音、动漫游戏、网络文学等数字文化内容,丰富高清、互动等视频节目,培育形成一批拥有较强实力的数字创新企业。发展交互式网络电视(IPTV)、手机电视、有线电视网宽带服务等融合性业务。支持用市场化方式发展知识分享平台,打造集智创新、灵活就业的服务新业态。
  (九)壮大在线教育和健康医疗。建设课程教学与应用服务有机结合的优质在线开放课程和资源库。鼓励学校、企业和其他社会力量面向继续教育开发在线教育资源。推动在线开放教育资源平台建设和移动教育应用软件研发,支持大型开放式网络课程、在线辅导等线上线下融合的学习新模式,培育社会化的在线教育服务市场。加强家庭诊疗、健康监护、分析诊断等智能设备研发,进一步推广网上预约、网络支付、结果查询等在线就医服务,推动在线健康咨询、居家健康服务、个性化健康管理等应用。
  (十)扩大电子商务服务领域。鼓励电商、物流、商贸、邮政等社会资源合作构建农村购物网络平台。支持重点行业骨干企业建立在线采购、销售、服务平台,推动建设一批第三方工业电商服务平台。培育基于社交电子商务、移动电子商务及新技术驱动的新一代电子商务平台,建立完善新型平台生态体系。积极稳妥推进跨境电子商务发展。
  三、扩大信息消费覆盖面
  (十一)推动信息基础设施提速升级。加大信息基础设施建设投入力度,进一步拓展光纤宽带和第四代移动通信(4G)网络覆盖的深度和广度,促进网间互联互通。积极参与“一带一路”沿线重要国家、节点城市网络建设。加快第五代移动通信(5G)标准研究、技术试验和产业推进,力争2020年启动商用。加快推进物联网基础设施部署。统筹发展工业互联网,开展工业互联网产业推进试点示范。推进实施云计算工程,引导各类企业积极拓展应用云服务。积极研究推动数据中心和内容分发网络优化布局。
  (十二)推动信息消费全过程成本下降。重点在通信、物流、信贷、支付、售后服务等关键环节全面提升效率、降低成本。深入挖掘网络降费潜力,加快实现网络资费合理下降,充分释放提速降费的改革红利,支持信息消费发展。建立标准化、信息化的现代物流服务体系,推进物流业信息消费降本增效。鼓励金融机构开发更多适合信息消费的金融产品和服务,推广小额、快捷、便民的小微支付方式,降低信息消费金融服务成本。
  (十三)提高农村地区信息接入能力。深化电信普遍服务试点,助力网络扶贫攻坚、农村信息化等工作,组织实施“百兆乡村”等示范工程,引导社会资本加大投入力度,重点支持中西部省份、贫困地区、革命老区、民族地区等农村及偏远地区宽带建设,到2020年实现98\%的行政村通光纤。全面实施信息进村入户工程,开展整省推进示范,力争到2020年村级信息服务站覆盖率达到80\%。
  (十四)加快信息终端普及和升级。支持企业推广面向低收入人群的经济适用的智能手机、数字电视等信息终端设备,开发面向老年人的健康管理类智能可穿戴设备。推介适合农村及偏远地区的移动应用软件和移动智能终端。构建面向新型农业经营主体的生产和学习交流平台。推动民族语言软件研发,减少少数民族使用移动智能终端和获取信息服务的障碍。鼓励各地采用多种方式促进信息终端普及。
  (十五)提升消费者信息技能。实施消费者信息技能提升工程,选择部分地区开展100个以上信息技能培训项目,通过多种方式开展宣传引导活动,面向各类消费主体特别是信息知识相对薄弱的农牧民、老年人等群体,普及信息应用、网络支付、风险甄别等相关知识。组织开展信息类职业技能大赛,鼓励企业、行业协会等社会力量开展信息技能培训。
  (十六)增强信息消费体验。组织开展“信息消费城市行”活动。鼓励地方和行业开展信息消费体验周、优秀案例展示等各种体验活动,扩大信息消费影响力。鼓励企业利用互联网平台深化用户在产品设计、应用场景定制、内容提供等方面的协同参与,提高消费者满意度。支持企业加快线上线下体验中心建设,积极运用虚拟现实、增强现实、交互娱乐等技术丰富消费体验,培养消费者信息消费习惯。
  四、优化信息消费发展环境
  (十七)加强和改进监管。坚持包容审慎监管,加强分类指导,深入推进“放管服”改革,继续推进信息消费领域“证照分离”试点,进一步简化优化业务办理流程,推行清单管理制度,放宽新业态新模式市场准入。强化事中事后监管,积极应用大数据、云计算等新技术创新行业服务和管理方式,在信息消费领域推行“双随机、一公开”监管,完善守信联合激励和失信联合惩戒制度。严厉打击电信网络诈骗、制售假冒伪劣商品等违法违规行为,整顿和规范信息消费环境。深化电信体制改革,鼓励民间资本通过多种形式参与信息通信业投融资。做好自由贸易试验区电信领域开放试点,加大基础电信领域竞争性业务开放力度,适时在全国其他地区复制推广。
  (十八)加快信用体系建设。健全用户身份及网站认证服务等信任机制,提升网络支付安全水平。结合全面实施统一社会信用代码制度,构建面向信息消费的企业信用体系,加强信息消费全流程信用管理。规范平台企业市场行为,加大对信息消费领域不正当竞争行为的惩戒力度,推动建立健全企业“黑名单”制度,将相关行政许可、行政处罚等信息纳入全国信用信息共享平台和国家企业信用信息公示系统,并依法依规在“信用中国”网站公示,营造公平诚信的信息消费市场环境。
  (十九)加强个人信息和知识产权保护。贯彻落实网络安全法相关规定,加快建立健全个人信息保护法律法规体系和管理制度。严格落实企业加强个人信息保护的责任,全面规范个人信息采集、存储、使用等行为,防范个人信息泄露和滥用,加大对窃取、贩卖个人信息等行为的处罚力度。健全知识产权侵权查处机制,提升网络领域知识产权执法维权水平,加强网络文化知识产权保护。
  (二十)提高信息消费安全性。加强网络信息安全相关技术攻关,为构建安全可靠的信息消费环境提供支撑保障。落实网络安全等级保护制度,深入推进互联网管理和网络信息安全保障体系建设,加强移动应用程序和应用商店网络安全管理,规范移动互联网信息传播。完善网络安全标准体系,建设标准验证平台,支持第三方专业机构开展安全评估和认证工作。做好网络购物等领域消费者权益保护工作,依法受理和处理消费者投诉举报,切实降低信息消费风险。
  (二十一)加大财税支持力度。深入推进信息消费试点示范城市建设。鼓励各地依法依规采用政府购买服务、政府和社会资本合作(PPP)等方式,加大对信息消费领域技术研发、内容创作、平台建设、技术改造等方面的财政支持,支持新型信息消费示范项目建设。落实企业研发费用加计扣除等税收优惠政策,促进社会资本对信息消费领域的投入。经认定为高新技术企业的互联网企业,依法享受相应的所得税优惠政策。
  (二十二)加强统计监测和评价。完善信息消费统计监测制度,进一步明确统计范围,将智能产品、互联网业务、数字内容等纳入信息消费统计。加强中央、地方、行业、重点企业间的协调联动,强化信息消费数据采集、处理、发布和共享。建立健全信息消费评价机制,研究建立并定期发布信息消费发展指数,加强督查检查,指导和推动信息消费持续健康发展。
  各地区、各部门要进一步统一思想,充分认识新形势下扩大和升级信息消费对释放内需潜力、促进经济升级、支持民生改善的重要作用,按照本意见要求,根据职责分工,加强组织实施,抓紧制定出台配套政策措施,强化协调联动,形成工作合力。各地方要因地制宜制定具体实施方案,明确任务、落实责任,扎实做好相关工作,确保各项任务措施落实到位。
  附件:重点任务分工方案
  国务院        
  2017年8月13日     
  (此件公开发布)
  附件
  重点任务分工方案
  (有关任务目标截至2020年)
  

dkkdkkdkdkdkdkkkd
dkkdkkdkdkdkdkkkd
dkkdkkdkdkdkdkkkd
dkkdkkdkdkdkdkkkd
dkkdkkdkdkdkdkkkd
dkkdkkdkdkdkdkkkd
dkkdkkdkdkdkdkkkd
dkkdkkdkdkdkdkkkd
dkkdkkdkdkdkdkkkd
dkkdkkdkdkdkdkkkd
dkkdkkdkdkdkdkkkd
dkkdkkdkdkdkdkkkd
dkkdkkdkdkdkdkkkd
dkkdkkdkdkdkdkkkd
\section{文档}
\label{sec:org580cf77}
\subsection{{\bfseries\sffamily DONE} 大学习大调研大改进报告}
\label{sec:org7717b2a}
\subsection{信息系概况}
\label{sec:orgbc649e7}
现有在校生1505人,教职工22人,管理干部4人。主任夏乐斋,副主任陈振超、李倩,学生
科副科长王树兵。专职辅导员一人:李元生;体育课教师一人:翟朋;英语老师一人:谢清
强;语文一人:冯爱琳;实验室管理员一人:乔孟平;专业课老师13人:张青、朱元凯、李
霞、李长英、王霞、宋晓玲、黄志艳、张小童、吕岱松、彭云、苏海红、薛玲、陈亮。
现在师生比1:68.4,严重超标。今年春招报名约200人,加上夏招,今年预计500-600人,
在校生规模将达1700-1800人。
存在问题:
1、管理队伍不足,需加强。现有四人,不能满足管理需要。
2、专业教师严重不足。按1:50计尚需45人,缺额30以上。
3、新专业老师无,新上云计算、大数据、人工智能人才无。
4、辅导员队伍需建立,完善管理体系。
建议:校内调整、招聘。
\subsection{{\bfseries\sffamily DONE} 信息技术工程系简介}
\label{sec:org53107cc}
\subsection{基本情况}
\label{sec:orgab3e65b}
信息技术工程系经过名校建设和发展,建成了全新的实训实验平台,有实训终端近300台。
建有andride 及IOS开发机房,物联网开发实训室,拥有云技术中心,全体教师经过培训,
正从传统计算机教育转向大数据、云技术、人工智能。近年来,教学及管理围绕企业正在和
将要用到的技术进行,先后在学生管理及教学中应用了企业级丁丁管理手机软件,项目化管
理平台、在线教育平台、ERP平台等。在大数据建设、云技术应用及管理平台使用上积累了
宝贵的实践经验。随着新技术人才需求地激增,招生也成倍增长,15年300人,16年400人,
17年近600人。
\subsection{下一步发展思路}
\label{sec:orgaad9216}
重点建设和推广云技术、大数据管理与应用、人工智能的应用。以建设教学及示范的大数据、
云技术中心、人工智能的应用示范中心为目标,服务于泰安学校、社区、及中小企业。形式
上建成智能化物联网大楼,起引领示范作用。
\subsection{总结}
\label{sec:org94f0d9e}
\subsection{部门总结}
\label{sec:orga18810e}
\subsubsection{2017年}
\label{sec:org30b470b}
信息技术工程系2017年工作总结

2017年,我系深入学习贯彻落实习近平新时代中国特色社会主义思想和党的十九大精神,紧
紧围绕学院2017年工作重点,着力做好系部学生的思想引领和成长服务工作,全面提升教育
教学质量和学生的综合素质,努力完成学院交办的各项工作任务。现就本年度系部工作汇报
如下:

一.打造过硬管理队伍

1.加强班主任队伍建设和学生思想政治教育。

加强对专(兼)职班主任的教育和管理,逐步培养出一支素质较高、作风较好、能力较强、
经验较足、肯奉献、较稳定的专(兼)职班主任队伍。加强大学生思想政治教育,践行社会
主义核心价值观,加强德育教育,强化对学生世界观、人生观和价值观的引导。激励师生用
习近平新时代中国特色社会主义思想武装头脑,确保党的意识形态领域工作占领高校思想阵
地。

2.加强学风建设,加大对学生干部的培养力度。

学风建设是学校管理永恒的主题。系部在年初就制定了本年度学生管理工作要点及学生活动
计划,并要求各班主任结合班级实际情况拟定班级工作计划,开展丰富多彩的学生活动,在
活动中创建特色班集体。本学期系部先后开展活动40余次,参与学生1000余人次。在平时的
日常管理方面,进一步完善纪律检查制度,系部在考勤方面坚持每月底对各班考勤情况进行
检查,并不定期对各班出勤情况进行抽查。

三.筑牢过硬安全稳定工作体系。

系部牢固树立“安全重于泰山”的思想观念,加强对学生的安全教育和管理。通过“安全疏
散演练”、杜绝大功率电器使用、班主任专题会议、主题班会等形式让学生学习掌握安全常
识,树立自觉遵守国家法律法规和校规校纪的意识,确保生命财产不受损失,努力维护正常
的教学生活秩序。

四.教科研工作成效突出

成功承办2017年泰安市职业技术院校信息化教学大赛。积极参与课题申报,鼓励师生参加各
类赛事赛。2017年,李倩负责的山东省职业教育精品资源共享课程(《Java项目综合实训》)
建设立项、李倩负责的山东省职业教育教学改革研究重点资助项目—《基于混合式教学模式
下的创客孵化教育研究与实践》立项,获资助经费4万元、李倩/朱元凯指导的泰安市大学生
科技创新行动计划 —《智能气囊辅助轮椅研究设计与开发》立项,黄志艳老师在《电子技术
与软件工程》发表论文一篇。由张小童老师指导的王晨等同学在《第十三届山东省青年职业
技能大赛》中荣获动画制作员赛项一等奖一项、在《第六届全国高校廉政文化作品征集暨廉
洁教育系列活动》中获一等奖一项;黄志艳、张青老师指导的李建霖等同学荣获第九届山东
省大学生科技节《2017年第十五届山东省大学生软件设计》二等奖1项、张小童、朱元凯老
师指导的王晨同学获第十五届山东省大学生科技文化艺术节《国学微电影》二等奖1项;黄
志艳、张青、张小童三位老师获得优秀指导教师荣誉称号。2017年山东省信息化教学大赛中,
李倩、朱元凯老师荣获二等奖一项,李倩、宋晓玲老师荣获三等奖一项。

五.扎实推进优质专科高等职业院校建设和教学诊断改进工作

在建立Bitnami Moodle Stack 课程网络平台同时,部署试用ERP系统,通过平台建设,累计
资料,推进混合课堂教学模式与方法改革。为优质专科高等职业院校建设和教学诊断改进工
作打基础。

六.加大就业招生工作力度

加大就业工作力度,实施就业困难学生“一对一”帮扶工程,为毕业班学生提供就业创业指
导和帮助,我系2017届毕业生初次就业率达100\%。加大招生宣传力度,计算机应用技术专业
2017年单招学生数全校第三位。2017年新生报到560余人,比2016年多140余人。

七.存在问题

一是教职工服务意识还不够强,工作力度不够大,在一定程度上,被动工作还占有相当比例。
二是教学资源严重不足。2017级新生入校后,师资力量和教学用房严重失衡。班主任当做了
辅导员用,有的老师带了6个班的班主任,造成班级管理质量有所下降。

八.2018年工作计划

1.认真学习贯彻党的十九大精神和习近平新时代中国特色社会主义思想。持续推进大学生
思想政治教育工作和意识形态工作。加强党风廉政师德师风建设。
2.全面深化教学改革。完善专业建设规划,做好特色专业、重点建设专业的建设工作。
3.积极落实学校制定的学生工作制度,使广大教职工特别是学生工作者树立创新的理念,保
持创新的头脑,积极思考创新的点子,形成创新的方案。
\begin{enumerate}
\item 切实抓好安全稳定工作。
\end{enumerate}


二〇一七年十二月


\subsection{{\bfseries\sffamily DONE} 个人述职}
\label{sec:orgc17c091}
\subsubsection{2017年}
\label{sec:org8263c3c}
2017年度述职报告
泰山职业技术学院信息技术工程系主任 夏乐斋

2017年,深入学习贯彻落实习近平新时代中国特色社会主义思想和党的十九大精神,紧
紧围绕学院2017年工作重点,着力做好系部学生的思想引领和成长服务工作,全面提升教育
教学质量和学生的综合素质,努力完成学院交办的各项工作任务。

一.打造过硬管理队伍

1.加强班主任队伍建设和学生思想政治教育。

加强对专(兼)职班主任的教育和管理,加强大学生思想政治教育,践行社会主义核心价值
观,加强德育教育,强化对学生世界观、人生观和价值观的引导,确保党的意识形态领域工
作占领高校思想阵地。

2.加强学风建设,加大对学生干部的培养力度。

本学期系部先后开展活动40余次,参与学生1000余人次,步完善纪律检查制度,系部在考勤
方面坚持每月底对各班考勤情况进行检查,并不定期对各班出勤情况进行抽查。


二.筑牢过硬安全稳定工作体系。

系部牢固树立“安全重于泰山”的思想观念,加强对学生的安全教育和管理。通过“安全疏
散演练”、杜绝大功率电器使用、班主任专题会议、主题班会等形式让学生学习掌握安全常
识,树立自觉遵守国家法律法规和校规校纪的意识序。

三.教科研工作成效突出

成功承办2017年泰安市职业技术院校信息化教学大赛。省级立项两项,省级一项,
,发表论文多篇。省以上竞赛获奖项。

四.扎实推进优质专科高等职业院校建设和教学诊断改进工作

部署试用ERP系统,踏实做教学诊断改进基础工作。

五.加大就业招生工作力度

加大就业工作力度,2017届毕业生初次就业率达100\%。加大招生宣传力度,2017年单招学生
数全校第三位。2017年新生报到560余人。


六.存在问题

意识还不够强,工作力度不够大,被动工作还占有相当比例。
对上争取不够,教学资源严重不足。


二〇一七年十二月
\subsection{教代会材料}
\label{sec:org332d599}
泰山职业技术学院信息分会二届三次双代会代表讨论意见
一、对大会议题(草案)修改意见、建议
1.《院长工作报告》
第7页倒数第6行:原为:为此,今年着重抓好九个方面工作:
建议修改为:为此,今年着重抓好九个方面工作:
前后一致。
2.《财务工作报告》
在支出项目中,无法看出实际用于教学和学生管理部分的数目与占比。
二、建议
1.教代会在报预算前召开,否则提案项目不能报预算,没有预算所有提案项目不能落到实处。
2.学院应有计划推进老旧建筑的节能改造,彻底解决取暧办公条件。
3.提案1建设落实,各系部及学院都用。
4.各项提案、计划传达给全体职工,了解学院发展。
5.系部学术委员会详细细则需完善及制订。

代表组:信息代表组
2018年3月10日
\section{emacs下的git工具 magit 简介}
\label{sec:org230d937}
\section{www-mitpress.mit.edu/sicp/}
\label{sec:orgf6e2c86}
\end{document}