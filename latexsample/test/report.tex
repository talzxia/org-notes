\documentclass[12pt]{article}
\usepackage[BoldFont,SlantFont]{xeCJK}
\setCJKmainfont[BoldFont=SimHei]{SimSun}
\setCJKmonofont{SimSun}%设置缺省中文字体
\parindent 3em %段首缩进

\usepackage[a4paper,left=26mm,right=25mm,top=35mm,bottom=35mm,headheight=2.17cm,headsep=12mm]{geometry}
\setmainfont{Times New Roman}%缺省英文字体 Times New Roman
\setCJKmainfont[BoldFont={方正小标宋简体},ItalicFont={Adobe 楷体 Std R}]{华文仿宋}%方正书宋_GBK Adobe Song Std L
\setCJKsansfont[BoldFont={方正书宋简体}]{方正中等线简体}
\setCJKmonofont{方正中等线简体}
\XeTeXlinebreaklocale "zh"
\XeTeXlinebreakskip = 0pt plus 1pt

\setCJKfamilyfont{new}{方正苏新诗柳楷简体}
\setCJKfamilyfont{note}{方正启体简体}
\newfontfamily\gara{Adobe Garamond Pro}


\linespread{2}
\begin{document}
% \large{}
\fontsize{15.75pt}{30mm}
\section{举例}
\begin{verbatim}
标点。
\end{verbatim}
汉字Chinese数学$x=y$空格

\section{一张白纸折腾出一个模板}


    基于本模板追求视觉上的美观的角我以前从未写过类文我以前从未写过类文件,所以,写这个模板的过程必然是折腾的过程,在写模板的过程中,
最主要参考了《写给\LaTeXe 类与宏包的作者》、moderncv.cls 文件、
href{http://aff.whu.edu.cnhuangzh}{武汉大学黄正华老师的论文模板}、《\LaTeXe 完全
学习手册》The Not So Short Introduction to \LaTeXe 以及各大 疑问解答网站,

\textsf{\huge{}在此为
无私奉献的组织和个人表示感谢!}

\emph{基于本模板追求视觉上的美观的角}

基于本模板追求视觉上的美观的角度,强烈建议使用者安装./fonts/文件夹下的字体。出于版权的考虑,务必不能将此模板用于涉及盈利目的的商业行为,否则,后果自负,本模板带的字体仅供学习使用,如果您喜欢某种字体,请自行购买正版。本文主要使用的字体如下
基于本模板追求视觉上的美观的角度,强烈建议使用者安装./fonts/文件夹下的字体。出于版权的考虑,务必不能将此模板用于涉及盈利目的的商业行为,否则,后果自负,本模板带的字体仅供学习使用,如果您喜欢某种字体,请自行购买正版。本文主要使用的字体如下

基于本模板追求视觉上的美观的角度,强烈建议使用者安装./fonts/文件夹下的字体。出于版权的考虑,务必不能将此模板用于涉及盈利目的的商业行为,否则,后果自负,本模板带的字体仅供学习使用,如果您喜欢某种字体,请自行购买正版。本文主要使用的字体如下

基于本模板追求视觉上的美观的角度,强烈建议使用者安装./fonts/文件夹下的字体。出于版权的考虑,务必不能将此模板用于涉及盈利目的的商业行为,否则,后果自负,本模板带的字体仅供学习使用,如果您喜欢某种字体,请自行购买正版。本文主要使用的字体如下

基于本模板追求视觉上的美观的角度,强烈建议使用者安装./fonts/文件夹下的字体。出于版权的考虑,务必不能将此模板用于涉及盈利目的的商业行为,否则,后果自负,本模板带的字体仅供学习使用,如果您喜欢某种字体,请自行购买正版。本文主要使用的字体如下

基于本模板追求视觉上的美观的角度,强烈建议使用者安装./fonts/文件夹下的字体。出于版权的考虑,务必不能将此模板用于涉及盈利目的的商业行为,否则,后果自负,本模板带的字体仅供学习使用,如果您喜欢某种字体,请自行购买正版。本文主要使用的字体如下

基于本模板追求视觉上的美观的角度,强烈建议使用者安装./fonts/文件夹下的字体。出于版权的考虑,务必不能将此模板用于涉及盈利目的的商业行为,否则,后果自负,本模板带的字体仅供学习使用,如果您喜欢某种字体,请自行购买正版。本文主要使用的字体如下



\end{document}
