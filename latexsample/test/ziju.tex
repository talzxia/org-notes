\documentclass[12pt,a4paper]{ctexbook}
%\usepackage {fontspec}
%\usepackage {fontenc}
\setCJKmainfont[BoldFont={方正小标宋简体},ItalicFont={Adobe 楷体 Std R}]{Adobe Song Std L}%方
%\setCJKmainfont{华文仿宋}
\RequirePackage{geometry}
\geometry{
  a4paper,
  left=27mm,  %% or inner=23mm
  right=27mm, %% or outer=18mm
  top=25.4mm, bottom=25.4mm,
  headheight=2.17cm,
  headsep=4mm,
  footskip=12mm
}


\begin{document}
%\ziju{0.1}
\zihao{3}
\hfill {泰山职业技术学院}
\chapter{信息技术工程系}
\section{编辑技术}
\fangsong{}
%\CJKfamily{华文标宋}

\setlength{\baselineskip}{30pt}
中华人民共和国中华人民共和国中华人民共和国中华人民共和国1、《蜂》

  唐·罗隐


  欲求贤才栋梁,天空陆地海洋。
  半世东奔西忙,今又远航,
  路遥山高水长。

  7、《七绝·师恩难忘》

  相逢一见太匆匆,校内繁花几度红。
  厚谊常存魂梦里,深恩永志我心中。
写 ElegantNote 模板的初衷是为了简化我在写笔记中的工作,因为我不会写类文件和包文件,所以,最当初是想拜托小L做出一个华丽,清爽的 \LaTeX{} 模板,最好是类文件,而且因为这样可以简化导言区复杂的内容。后来,和小L一拍即合,遂开始一起做Elegant\LaTeX{}的设计。

在学校的时候,搞定了定理环境样式的代码。因为不想重复 China\TeX{} 那个经典的页眉页脚,我找到了计量书上的一个图案,小L拿 Ti\emph{k}Z 一点一点把那个画出来了,不过我最后还是用的截取的方式得到的图案。慢慢地,我们把初步的样子做出来了。

2013年的暑假开始后,我对那个初步的模板做了一点改动,然后用它写了Dynamic Programing 的笔记,并且,在写的过程中,对模板加了封面,也就是模板现在的封面(logo 在Version 2.00中已经改了)。至此,模板的大致样子终于出来了,不过当时也在写笔记的过程中知道了某些不足,比如
\end{document}