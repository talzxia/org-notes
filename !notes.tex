% Created 2016-08-11 四 05:43
\documentclass[11pt]{ctexart}
                                        \usepackage[utf8]{inputenc}
                                        \usepackage[T1]{fontenc}
                                        \usepackage{fixltx2e}
                                        \usepackage{graphicx}
                                        \usepackage{longtable}
                                        \usepackage{float}
                                        \usepackage{wrapfig}
                                        \usepackage{rotating}
                                        \usepackage[normalem]{ulem}
                                        \usepackage{amsmath}
                                        \usepackage{textcomp}
                                        \usepackage{marvosym}
                                        \usepackage{wasysym}
                                        \usepackage{amssymb}
                                        \usepackage{booktabs}
                                        \usepackage[colorlinks,linkcolor=black,anchorcolor=black,citecolor=black]{hyperref}
                                        \tolerance=1000
                                        \usepackage{listings}
                                        \usepackage{xcolor}
                                        \lstset{
                                        %行号
                                        numbers=left,
                                        %背景框
                                        framexleftmargin=10mm,
                                        frame=none,
                                        %背景色
                                        %backgroundcolor=\color[rgb]{1,1,0.76},
                                        backgroundcolor=\color[RGB]{245,245,244},
                                        %样式
                                        keywordstyle=\bf\color{blue},
                                        identifierstyle=\bf,
                                        numberstyle=\color[RGB]{0,192,192},
                                        commentstyle=\it\color[RGB]{0,96,96},
                                        stringstyle=\rmfamily\slshape\color[RGB]{128,0,0},
                                        %显示空格
                                        showstringspaces=false
                                        }
\author{mac}
\date{\today}
\title{}
\hypersetup{
 pdfauthor={mac},
 pdftitle={},
 pdfkeywords={},
 pdfsubject={},
 pdfcreator={Emacs 25.0.95.1 (Org mode 8.3.5)}, 
 pdflang={English}}
\begin{document}

\tableofcontents

\section{Quick notes}
\label{sec:orgheadline70}

\subsection{{\bfseries\sffamily DONE} listening music and relaxing}
\label{sec:orgheadline1}
\textit{[2016-08-02 二 09:36]}

\subsection{{\bfseries\sffamily DONE} 安装souce code pro 字体 成功}
\label{sec:orgheadline2}
\textit{[2016-08-02 二 10:53]}

\subsection{{\bfseries\sffamily DONE} relaxing and listening music}
\label{sec:orgheadline3}
\textit{[2016-08-02 二 16:12]}

\subsection{{\bfseries\sffamily DONE} relaxing}
\label{sec:orgheadline4}
\textit{[2016-08-03 三 05:52]}

\subsection{{\bfseries\sffamily DONE} }
\label{sec:orgheadline5}
Emacs是神的编辑器还是人的编辑器?
\url{http://blog.csdn.net/lishuo_os_ds/article/details/8069039}
  后记:其实用什么样的工具,还是能在一定程度上反映一个人的心态和能力。譬如说
Emacs/Vim这种编辑器,如果你没有足够的冒险精神,没有足够的探索意识和一定的信息收
集和学习能力,你不会去用它。为什么,因为一个IDE环境的编辑器可以不用多少时间就能
搞掂,而Emacs/Vim却不是你三两天就能搞定的。大多数人仅仅停留在这个Emacs/Vim怎么这
么操蛋,这么反人类这种心理状态,因而也就不会下工夫学习。而对于IT这个技术日新月异
的行业,没有足够的探索意识和学习能力,早晚被淘汰。



\textit{[2016-08-03 三 10:32]}

\subsection{{\bfseries\sffamily DONE} hack 拒绝传统,看 Facebook 如何以三大法宝化茧成蝶}
\label{sec:orgheadline6}
\url{http://kb.cnblogs.com/page/545228/}
Hack 的详细定义主要有三点:马上上手、快速搞定和持续迭代。而 Growth Hack 的本质,
就是增长方面用 Hack 的方法解决。在 Facebook 绝对拒绝传统(大家一起来开会),而是
技术驱动、数据为王。


\textit{[2016-08-03 三 16:19]}

\subsection{{\bfseries\sffamily DONE} brow and learn spacemacs}
\label{sec:orgheadline7}
swiper use

\textit{[2016-08-03 三 20:11]}

\subsection{8:00--9:00 learn spacemacs}
\label{sec:orgheadline8}
vim masks
org-mode tages


\textit{[2016-08-08 一 09:00]}

\subsection{{\bfseries\sffamily DONE} begin listen Engilsh songs.}
\label{sec:orgheadline9}
\textit{[2016-08-08 一 09:03]}

\subsection{{\bfseries\sffamily DONE} 二、Facebook 的工程师文化是怎样的?}
\label{sec:orgheadline32}
 

\subsubsection{ 特点一:Hack Culture}
\label{sec:orgheadline12}
  首先,Hack Culture,可以说是「黑客文化」。注意并不是字面意义上的「黑客」,在扎格伯格招股书上就说过,Hack Culture 并不是黑别人电脑,而是一种态度和做事的方法。
  现在去美国已经非常容易了,如果飞机降落在旧金山机场,着陆的时候会越过 Facebook 总部,建议大家到时候可以留意一下,这么一群建筑就是整个 Facebook 总部的样子。
特别有意思的是,这个广场的中间位置其实是有字的,在飞机上就会看得非常清楚,就是
「Hack」。当我们说「我们要做一些 Hack 之类的」,它所表达的意思是「如果你有什么想
法,有创新的、古灵精怪的各种想法都可以,马上去做!」

\begin{enumerate}
\item Hack 的详细定义主要有三点:
\label{sec:orgheadline11}

\begin{enumerate}
\item 马上上手、快速搞定和持续迭代。
\label{sec:orgheadline10}
而 Growth Hack 的本质,就是增长方面用 Hack 的方法解决。在 Facebook 绝对拒绝传统
(大家一起来开会),而是技术驱动、数据为王。

 做了一个东西之后先发给用户,看一下用户的反馈或者搞一批测试的用户,最后看多少
用户觉得你这个好,通过数据的采集方式来论证是否可行。在 Facebook 有很多牛的方法
可以搜集数据,所以可以很明显的看出这个版本用了 AB 测试之后到底是好还是坏。

  在 Facebook 并没有一个专门的部门做 Hack,而是鼓励所有的人去发挥自己的想象力,
想做什么就可以做。公司的角落有各种各样的「Hack」,营造出一种气氛,鼓励员工去创新,
鼓励那些牛逼的人待在这个地方工作,这样的公司文化是比较愉悦的。

  
\end{enumerate}
\end{enumerate}

\subsubsection{特点二:Design+Engineering}
\label{sec:orgheadline14}
  如果说苹果是 80\% 的精力很重视自己的设计,谷歌大概非常重视工程,在 Facebook
设计和工程各占一半,这可能和大家想的不一样。Facebook 的新办公室出自一位著名的设
计师,有各种奇妙的元素,开始我认为这些设计是浪费,后来我慢慢发现这种环境下工作心
情非常愉悦,很多人自发工作 10 个小时左右,而且在这种环境下心情很放松的状态,更能
激发你 Hack。

\begin{enumerate}
\item 非常迷人的工程师文化
\label{sec:orgheadline13}
    后来我看到国内的一些互联网公司也有这种感觉,比如最近的创业贵族今日头条。所以如果
想把互联网公司做大,就要有非常迷人的工程师文化,才能把牛逼的人招进来。


  
\end{enumerate}

\subsubsection{特点三:Open}
\label{sec:orgheadline16}
  我第一天入职 Facebook 的时候,最大的感受就是这三点:
首先是 Open,非常 Open 的 checking。新人第一天入职,领到一台电脑,连上网后,所有
的数据都可以看到:产品的月活、日活、每一个功能,甚至可以像股票交易系统看到这样的
数据,比如日本的用户 30 岁以下的人在用 news feed 的时候情况到底怎样,从今年年初
到现在变化是怎样的等。所有的东西都开放给工程师,没有任何权限。

  在 Facebook,codebase 是所有 software engineer 都有权限把它拷贝下来。我(作
为一个 facebook phone 和 iOS app 的工程师)可以把安卓的 codebase 拉下来,也可以
把 PHP、推荐算法、引擎各种代码库给 git clone 下来,所有的权限都是开放的。但开放
的前提,如果把公司内部信息泄漏出去就会被开除掉,历史上也发生过好几起这样的事情。
所以,在 Facebook 基本上不可能出现扎克发的一封邮件会被爆光。刚开始回国的时候,我
很震惊,各种 CEO 的邮件都曝光在公众眼中,这种事在硅谷基本不可能。

  所以 Hack 意味着内部信任和开放,同时对外也要遵守非常严格的规定。

\begin{enumerate}
\item 最后一点就是 Open Space,
\label{sec:orgheadline15}
很开放的环境,给人的感觉是「网吧式的工作环境」。美国公司的办公室给人的感
觉比较粗糙,比如说天花板上没有任何装饰,柱子上还有很多铁锈一样的东西,但
是它又有非常细节的地方。

  比如说,显示器都还不错,有各种饮料和零食,随便吃;椅子是著名的 Aaron chair。
最有趣的是:椅子右侧有两个按钮,一个是上升和下降,平时就是这么正常的工作,累了之
后,久座对身体不好,就把桌子升起来,变成站立式办公。站起来有一个好处是,当好几个
人在一起讨论问题的时候,大家可以站在一起看。很多时候写代码的时候,把设计师也带过
来,直接问「我做了一个原型你看有什么问题的吗?」有问题就直接改了。

  扎克和所有人一样,坐这么一个位置,开放式工位旁边有很大的玻璃房子,用于开会,
扎克没有太多的会议或者太多出去的时间,基本上每天在公司里面专注于自己的事务。

 
\end{enumerate}

\subsubsection{ 三、Facebook 团队组成:设计师、产品经理和工程师}
\label{sec:orgheadline17}
首页新闻博问闪存招聘园子[登录·注册]
 知识库
知识库 专题 .NET技术 Web前端 软件设计 手机开发 软件工程 程序人生 项目管理 数据库 最新文章
您的位置:知识库 » 创业
拒绝传统,看 Facebook 如何以三大法宝化茧成蝶
作者: 覃超  来源: 虎嗅网  发布时间: 2016-06-24 22:15  阅读: 3513 次  推荐: 18   原文链接   [收藏]
拒绝传统,看 Facebook 如何以三大法宝化茧成蝶:人才吸引、工程师文化和项目开发流程
  文/覃超 ,本文来自公众号 InfoQ(infoqchina)
  我将结合之前在 Facebook 的四年工作经验,介绍 Facebook 创新的管理方法以及整个工程文化形成的方法。
  从大学宿舍到完整生态链:Facebook 12 年发展历程
  首先我想解释下为什么我要说 Facebook,并不是因为我在那里工作过。先请看一下 Facebook 的简单发展史:
1.jpg
  2004 年,扎克伯格在大学宿舍里开发了 Facebook,很长一段时间它只是一个简单的网站。但是后来,扩展到硬件,开发了各种产品,占据了整个社交领域,至少是北美的霸主地位。Facebook 还推进了 Connectivity(全民联网计划),给发展中国家提供免费的无线网络,再到后面做 VR、AR、人工智能等现在非常火的新领域,形成了完整的生态圈。
  看 Facebook 12 年的发展,我一直在想一个问题,它是怎么从大学生宿舍里面一个简单的扎克伯格自己的个人项目,最终发展成为可以上市的公司?而且现在整个生态链布局的已经非常完善了,这个公司是怎么做出来的?
  我思考了很久之后,发现有一句话很有意思,21 世纪最重要的是什么? 就是人才。现在互联网创业很多技术基本上都是开源的,很多硬件随手可得。
人才流动 .png
  上图是在五年前硅谷非常流行的一张图,描述了大公司之间人才的流动情况。图中,每个圆点是一个公司,和圆点颜色相同的边表示这个公司的人才流入。可以看到 Facebook 基本上和其他点间连线都是蓝色的,所有公司都在向 Facebook 输入人才。而谷歌基本上在从微软和雅虎搜集人才,在其他地方是流失人才的。
  所以,从五年前吸纳了很多人才开始,Facebook 才有了今天的成就。有一群最牛的工程师、产品经理和设计师在这里,所以经过五年的布局和人才培养,才造就了现在全生态链都有一个非常好的格局。
  所以我在想,里面整个工程师文化做得比较牛的是什么?因为我自己在 Facebook 工作过,所以结合自身经验带来了这个分享:在公司里面如何塑造比较好的公司文化?怎样把牛人吸引过来?
5.png
  总共五个部分:
首先是简单的自我介绍;
接下来讲 Facebook 的工程师文化;
以及团队的组成,包括设计师、产品经理和工程师;
人员是如何管理的,怎样能有效控制工程师的工作积极性以及给予相应的报酬;
最后一点也是最重要的,以上这些对我们中国公司有怎样的启示。
 

\subsubsection{ 一、我是谁:从 Carnegie Mellon 到 Facebook}
\label{sec:orgheadline18}
Self Introduction.png
  这是我的简历,大学和企业在技术方面的差距还是很大的。从 Carnegie Mellon 大学毕业后,我加入了 Facebook,开始做的是 CTO 亲自主导的比较神秘的项目,后来去做了 Facebook 的 APP,主要是 iOS,也做过 Voice Message 等。

\subsubsection{  二、Facebook 的工程师文化是怎样的?}
\label{sec:orgheadline22}
 

\begin{enumerate}
\item  特点一:Hack Culture
\label{sec:orgheadline19}
  首先,Hack Culture,可以说是「黑客文化」。注意并不是字面意义上的「黑客」,在扎格伯格招股书上就说过,Hack Culture 并不是黑别人电脑,而是一种态度和做事的方法。
  现在去美国已经非常容易了,如果飞机降落在旧金山机场,着陆的时候会越过 Facebook 总部,建议大家到时候可以留意一下,这么一群建筑就是整个 Facebook 总部的样子。
Office.png
  特别有意思的是,这个广场的中间位置其实是有字的,在飞机上就会看得非常清楚,就是「Hack」。当我们说「我们要做一些 Hack 之类的」,它所表达的意思是「如果你有什么想法,有创新的、古灵精怪的各种想法都可以,马上去做!」
Hack.png
  Hack 的详细定义主要有三点:马上上手、快速搞定和持续迭代。而 Growth Hack 的本质,就是增长方面用 Hack 的方法解决。在 Facebook 绝对拒绝传统(大家一起来开会),而是技术驱动、数据为王。
hack culture.png
  做了一个东西之后先发给用户,看一下用户的反馈或者搞一批测试的用户,最后看多少用户觉得你这个好,通过数据的采集方式来论证是否可行。在 Facebook 有很多牛的方法可以搜集数据,所以可以很明显的看出这个版本用了 AB 测试之后到底是好还是坏。
  在 Facebook 并没有一个专门的部门做 Hack,而是鼓励所有的人去发挥自己的想象力,想做什么就可以做。公司的角落有各种各样的「Hack」,营造出一种气氛,鼓励员工去创新,鼓励那些牛逼的人待在这个地方工作,这样的公司文化是比较愉悦的。
 

\item  特点二:Design+Engineering
\label{sec:orgheadline20}
  如果说苹果是 80\% 的精力很重视自己的设计,谷歌大概非常重视工程,在 Facebook 设计和工程各占一半,这可能和大家想的不一样。Facebook 的新办公室出自一位著名的设计师,有各种奇妙的元素,开始我认为这些设计是浪费,后来我慢慢发现这种环境下工作心情非常愉悦,很多人自发工作 10 个小时左右,而且在这种环境下心情很放松的状态,更能激发你 Hack。
Hack 激发 .png
  后来我看到国内的一些互联网公司也有这种感觉,比如最近的创业贵族今日头条。所以如果想把互联网公司做大,就要有非常迷人的工程师文化,才能把牛逼的人招进来。
 

\item  特点三:Open
\label{sec:orgheadline21}
  我第一天入职 Facebook 的时候,最大的感受就是这三点:
Engineering cultrue.png
  首先是 Open,非常 Open 的 checking。新人第一天入职,领到一台电脑,连上网后,所有的数据都可以看到:产品的月活、日活、每一个功能,甚至可以像股票交易系统看到这样的数据,比如日本的用户 30 岁以下的人在用 news feed 的时候情况到底怎样,从今年年初到现在变化是怎样的等。所有的东西都开放给工程师,没有任何权限。
  在 Facebook,codebase 是所有 software engineer 都有权限把它拷贝下来。我(作为一个 facebook phone 和 iOS app 的工程师)可以把安卓的 codebase 拉下来,也可以把 PHP、推荐算法、引擎各种代码库给 git clone 下来,所有的权限都是开放的。但开放的前提,如果把公司内部信息泄漏出去就会被开除掉,历史上也发生过好几起这样的事情。所以,在 Facebook 基本上不可能出现扎克发的一封邮件会被爆光。刚开始回国的时候,我很震惊,各种 CEO 的邮件都曝光在公众眼中,这种事在硅谷基本不可能。
  所以 Hack 意味着内部信任和开放,同时对外也要遵守非常严格的规定。
Open space.png
  最后一点就是 Open Space,很开放的环境,给人的感觉是「网吧式的工作环境」。美国公司的办公室给人的感觉比较粗糙,比如说天花板上没有任何装饰,柱子上还有很多铁锈一样的东西,但是它又有非常细节的地方。
  比如说,显示器都还不错,有各种饮料和零食,随便吃;椅子是著名的 Aaron chair。最有趣的是:椅子右侧有两个按钮,一个是上升和下降,平时就是这么正常的工作,累了之后,久座对身体不好,就把桌子升起来,变成站立式办公。站起来有一个好处是,当好几个人在一起讨论问题的时候,大家可以站在一起看。很多时候写代码的时候,把设计师也带过来,直接问「我做了一个原型你看有什么问题的吗?」有问题就直接改了。
  扎克和所有人一样,坐这么一个位置,开放式工位旁边有很大的玻璃房子,用于开会,扎克没有太多的会议或者太多出去的时间,基本上每天在公司里面专注于自己的事务。
 
\end{enumerate}

\subsubsection{ 三、Facebook 团队组成:设计师、产品经理和工程师}
\label{sec:orgheadline23}
Team.png
  很多人问我,Facebook 的项目团队是怎样的?
  一般情况,如果是做一个简单的小功能,一般是一个设计师加两个工程师;比较大一点的项目,比如说改版、在新版当中开发两三个功能,基本上两三个工程师一起做,iOS messenger app 五到十位工程师和两到三位产品经理,和国内配比差不多。
  比较有意思,Facebook 没有测试,他们比较贵,很多时候都是我们自己测,我们 Unit Test 并不多,覆盖率 10\% 不到,但是我们有非常严格的 Code Review。所以如果你要学习一点,在工程上面、执行上面让 bug 减少的话就是代码审核,交到这个 master branch 里面的代码必须预先经过代码审核,直接看代码,没有什么问题就提交,如果提交进去后来发现 Bug 最后进行修复 Bug 花的时间和精力是之前的三倍十倍。
  整个流程一开始规划要做什么东西、要做什么功能、需求是什么,接下来设计师和工程师互相合作,比较有意思的是整个流程每个决策都要参与,而且每个决策之间互相是交互式的,工程师也可以说这个需求根本不能做或者说不用之类的。
office table.png
  这是我的桌子,当时无意中拍了一张照片,后面两个都是工程师,我们在讨论我们的消息收发的时候是怎样的,那个时候已经过了下班的时间。
  有人问,为什么你们的产品开发的比较快或者做的比较好,有没有什么秘诀?其实并没有太多的秘诀。
Design.png
  首先,人和人之间互相尊重,同时用 Scrum,大家都坐在一起有任何进展马上当面沟通,虽然我们远程会议系统特别强大,各种功能开个远程会议也行,但是我们鼓励在一起坐下来聊。团队最初期的时候就要开始协作,不同角色的人坐在一起讨论,不像国内分阶段分得特别明显。最后,设计师和开发者在工作的后期联系是非常紧密的。
Zuck Review.png
  最后,还有很重要的一点:Facebook 有 Zuck Review。也就是一些比较大的功能或产品,扎克会亲自安排看一下,也就是下面的人或者整个大的 PM 会亲自跟扎克说,这个地方你要过一遍,即使再忙他都会亲自来盯。
  他会决定这个功能到底是做还是不做,决定产品的 UI、功能、交互调整等,和网上风传的马化腾或张小龙其实风格差不多。我感觉 Facebook 和腾讯有些类似,都是一个产品型 CEO 主导的公司,扎克亲自来盯。
小 zuck.png
  图上有两位中国人,其中一位是做广告的葛爷,给 Facebook 赚了很多钱。Zuck 有时会用一种直觉性和你讲一些话,很多功能被他砍掉,大部分时间他都是做出了正确的决定。
  我认为 Zuck Review 给人最重要的感觉就是鼓舞,如果这个东西扎克亲自来看,优先级方面会给下面的工程师或者整个团队一个非常明确的交代,这个事到底重不重要,需不需要。
  关于优先级我想强调,大家都是技术人员,很多人在学校里面学习都不错,但在工作的时候发现有些不适应,需要注意的是在顶级公司或者特别牛逼的互联网公司工作,最重要的一点是分清优先级,这和学习的时候完全不一样。
  工作中的事情是做不完的,你在工作的时候是连续的;不像在学校的时候一个学期隔着一个学期,最后期末考试,你知道自己有什么反馈。但是在工作的时候活是干不完的,所有东西,周围很多人让你做这个做那个,最重要做的事情是分清优先级,任何一个任务发过来的时候,心里面把它积累起来,哪个任务比较重要的先做,而不是交给你一个任务马上去做。
  所以这里优先级,很多人我看到能力很强的人,最后遇到一个瓶颈,关键的问题是自己没有分清优先级,去做一些比较简单或者自己喜欢,或者是觉得自己能做的事情,而不是做最有影响力的事情。
  具体说来,和学生时代相比就是:学习上的课程是有限的,作业也是有限的,而且还有相对明确的截止日期(homework deadline)和最后一个期末考试;考试完结后,几乎学业清空一段时间。但是在工作上你会发现你没有一个类似暑假或者寒假的东西,另外最可怕的是你的活是干不完的,对的,是无穷无尽的。特别你是在一个上升期的互联网公司的话,给你任务的速度很多时候是超过你的处理速度的。
  所以这个时候,你在接到一个被分配的任务或者一个 email 要求你干什么的时候,你不是要马上可以做,而是要强迫自己停顿下,分清现在这个任务的优先级,然后分配好开始时间,之后再开始做。这点尤其重要!特别当看到一个简单或者重复性的任务被 email 或者 tower(或者 teambition)上分配来的时候,不要因为任务简单就马上跳上去干,不然这样极可能被简单重复劳动把自己的时间全部占光,最后没来得及干重要的事情,或者没有精力去思考更加长远更有影响力的事情。
  所以,重申一次,去做 impact 和 urgency 最高的事情(这种事情一般来说不是很愉悦,甚至是比较棘手或者说是无从下手的事情),而把简单重复的活尽量后排(或者 delegate 出去)。这时你才会发现你的忙碌是有意义的,而不是做“伪工作”(pseudo work)。
  我常常看到一些毕业不久的人每天都很忙,但却没有抓住重点,只是为了忙碌而忙碌,或者用更加贴切的话描述是:“为了感动自己而忙碌”。很多时候这样的忙碌,最后都是一个屁。之前在 Facebook 里,对于这样的同事有一个称号叫做“pseudo worker”,领导的职责是直接给他们透彻的反馈,让他们认清自己工作的 impact 最大化的地方到底在哪儿,同时告诫他们要忍住低 impact 的简单任务的诱惑。
  对的!那些垃圾任务有着一种诱惑;诱惑着没有定力的人一直去做,一直去做,感觉自己特别有成就感,特别“忙碌和充实”。所以要小心!在国内创业路上也有很多这样的创始人(或者 cofounders),他们自己的方向可能都没怎么想清楚,或者路线没有执行得当,却一天到晚在朋友圈晒自己和同事们的加班,觉得这样的“忙碌”很充实。其实这是一种很可怕而且对自己和团队既不负责的做法。
  一般工作时长的惊人但又没有 unicorn 估值的公司,我总觉得加班是一种羞耻,是自己团队不会分优先级或者战略不明确的表现;如果创始人还一直在那里秀加班来感动自己的话,我的建议是尽早离开。同时我还敢打赌,90\% 这样的公司在加班(和日常工作时间里)时的效率是偏低的。
  另外,Zuck Review 可以从用户的角度进行分析。有的时候我们做一个产品或做一个技术,一直做的时候会把很多东西想的过于简单,而用户很多时候比较傻或能够一秒钟变傻,会觉得这个东西并不好用。这一点感觉扎克做得比较好,扎克自己不是特别懂技术细节,如果他觉得这个地方为什么这么难用,会给你讲很多有意思的东西。
 

\subsubsection{ 四、Facebook 是怎样利用 OKR 进行人才管理的?}
\label{sec:orgheadline27}
OKR.png
  接下来是整个 Facebook 的管理是怎样,即 OKR。在 Facebook,OKR 意味着每六个月或每一年,制定一下你个人的目标、团队的目标以及公司的目标是什么,接下来行动就可以了。
  

\begin{enumerate}
\item 第一点,在目标制定的时候你要以结果为导向或者以影响力为导向,不要为了做而做、或者做一些伪工作。在工作的时候很多人会做一些伪工作或者简单的工作,也就是自己愿意做的工作。
\label{sec:orgheadline24}
 

\item  第二点,在 Facebook 会看每六个月、这半年的指标到底是什么东西。
\label{sec:orgheadline25}
  

\item 第三点,每年 6 月底、12  最后,一个月之后评估结果就会出来,将决定你的奖金多少、是否升职,年终绩效评估将决定你的现金奖金是多少,年底除了现金还有股票的追加。不管任何级别,只要是工程师都会给你相应的股票,每过一年年底绩效评估将决定给你追加多少股票,一般都会追加股票。
\label{sec:orgheadline26}
360 perf review.png
  具体的绩效考核怎么做?首先是国内常讲的 360 度评估,每 6 个月做一次,主要是四个部分:自评、同事评价、直属上司评价和老板评价。最后比较有意思的是,你可以决定这个东西是否开放、被谁看到。一般有 85\% 左右的人会选择开放,这是很恐怖的一个数据,基本上互相之间都是开放的。最后一点就是 HR 和整个 Team calibration,从上面再校准一次。
your bonus.png
  最后就是奖金,给你规定一个奖金,在 10\% 到 25\% 的区间。看你在哪个级别,新进来是 10\%,越到上面越高。然后要乘以你的个人绩效,0 表示没有奖金,一般在 1.25 左右,4.5 就很高了。最后再乘以一个公司的绩效,公司那几个高层对公司这半年来做得怎么样打一个分,如果公司做得很不错,所有人的薪水都会加。
 
\end{enumerate}

\subsubsection{ 五、师夷长技以制夷:对中国互联网公司有什么启示?}
\label{sec:orgheadline31}
  最后,我想说说 Facebook 的管理之道对中国互联网公司的启示是什么。虽然在 Facebook 工作很好,但我更喜欢加入中国的公司或者自己创业,和一帮国人在一起做一个公司,有一个牛逼的产品能够放到国际市场上和西方对打。
Start Up.png
  首先想强调一点,很多人说 Facebook 工程师文化特别好,但是它的文化并不是与生俱来的。前几天为了佐证这个观点我专门看了一下 2007 年大家对于扎克的想法,那个时候公司一团糟,偶尔有几个比较厉害的人,Facebook 现在比较牛的工程师文化是在 2008 年、由一个女孩 Molly Graham 逐步营造起来的。
  Molly 营造公司文化的过程,在这篇文章有阐述,当时,Facebook 从 400 人快速增长到 1000 人,公司已经管不过来了,一团糟,大家互相埋怨对方,干活非常没有效率,做了很多低效的事情。那么怎么把公司管好,同时让更优秀的人可以持续进来呢?Molly 建议 Facebook 建立工程师文化,她当时让扎克自己写了十条他觉得比较牛逼的人是怎样的,当然那十条大部分就是由扎克自己的气质决定的。
  这十条标准写出来之后,在公司里面反复强调,同时招人也招符合这些条件的人。所以可以得出一个结论:公司 80\% 的文化来自于创始人。
  最后一个结论,当一个公司变大,比如从A轮到B轮的时候,一定要营造出自己的文化。所以如果你是创始人,在公司还小的时候随便怎么弄,但是你自己要很明确有一杆秤;当公司到了 500 人以上的时候,这个时候一定要建立自己的公司文化。
  对技术人员来说,判断公司文化很多时候都是看创始人,看创始人是干什么出身的。如果他是做生意的,那么这个公司或许并不是你的最佳归属,即便讲得再牛。
  举今日头条的例子,他们做得比较好,老大本身也是技术出身,他们公司对技术人员的待遇非常好,还去硅谷挖了很多牛逼的人,把公司氛围营造得非常好。所以看创始人是可以看出来这个公司文化到底是怎样的。
  对任职管理者的工程师来说,在创业的过程当中有四点需要注意:
 

\begin{enumerate}
\item  第一,盯一线产品。下面的人不怕你 challenge 他,怕的是把这个东西做完之后上面的人不看,他就会觉得自己所有的辛苦努力全部都浪费了。
\label{sec:orgheadline28}
  第二,6 个月要做一次 Performance Review,这将决定员工的奖金和股票。

\textbf{**}
  

\item  第三,Code Review。在工程方面并不是用最好的技术最重要,而是把 Code Review 加进来,这并不是为了查出错,而是有时候要注意自己看一下,你提交的代码整个逻辑是不是清晰,同时会留下记录以便如果后来的人想学习你这个功能是怎么写的,他可以从这个上面看所有的记录,这是工程师之间互相切磋和交流的一个工具。
\label{sec:orgheadline29}

\item   第四,很多中国的创业公司有常会忽视的,入职培训和 Wiki 要写好。麦肯锡曾总结成功的公司的最大的经验是:Wiki 做得很好。可以把公司中每个人的知识归档,以及把每个人牛逼的知识在整个团队里扩散出去,是非常重要的一件事。
\label{sec:orgheadline30}
  (作者公众号:覃超帝国兴亡史(qc\(_{\text{empire}}\) ),谢谢大家的支持!转载请联系公众号:覃超帝国兴亡史(qc\(_{\text{empire}}\) )获取授权)
18 0

标签:工程师文化 Facebook
« 上一篇:如果你做的事情毫不费力,就是在浪费时间



推荐链接
50万行VC++源码:大型组态工控、电力仿真源码库
程序员找工作,就在博客园招聘频道
程序员问答平台,解决您的技术难题
创业热门文章
做正确的事情,等着被开除
一个程序员的创业历程
如果你做的事情毫不费力,就是在浪费时间
为什么我辞职去创办一个科技公司
创业的21条军规
创业最新文章
拒绝传统,看 Facebook 如何以三大法宝化茧成蝶
如果你做的事情毫不费力,就是在浪费时间
某种理想的团队
互联网组织的未来:剖析GitHub员工的任性之源
程序员如何参与创业
最新新闻
滴滴股权迷雾:反垄断审查或揭露庞大产业帝国
“负面”缠身 特斯拉赌局愈发凶险
NASA为太空发射系统准备装配大楼 世界最大
柳氏姐妹的“出行之战”:从相撕到相爱
滴滴优步合并后司机奖励全面缩水 乘客感觉价格上涨
热门新闻
媒体揭立体快巴疑理财骗局 总设计师仅小学文化
Microsoft .NET Framework 4.6.2 发布
中国宣布研制组合动力飞行器 比可回收火箭牛多了
贵州乡村来了一位新老师 马云雨中上课为教育也是拼了
中传博士凌晨猝死在教学楼 家属认为系过劳死

关于我们联系我们广告服务沪ICP备09004260号 © 2016博客园RSS


\textit{[2016-08-08 一 09:38]}
\end{enumerate}

\subsection{{\bfseries\sffamily DONE} 三步到位,下一个职场达人就是你了}
\label{sec:orgheadline51}
\url{http://news.mbalib.com/story/231918}

\subsubsection{一、足够自信,“做一只打不死的蟑螂”}
\label{sec:orgheadline37}

法国哲学家卢梭说:“自信对于事业简直是奇迹,有了它,你的才智可以取之不竭。一个没
有自信心的人,无论他有多大才能,也不会有成功的机会。”可见自信心对身处职场中的你
是相当重要,自信心就是要你对自己行为的正确性足够坚信,抱有充分的信心。自信不等于
自负,自负的目的在于赢得他人的赞许,根据别人的看法来评价自己,其结果往往是弄巧成
拙,贻笑大方。西施捧心是一种美,东施也去捧心,却惊跑了左右四邻。

  “职场如战场”,很多人在工作上难免会有不顺心的时候,这时你该端正心态,端正心
态就是要变被动工作为主动工作,变要我工作为我要工作。做事没有一个正确的心态是不会
取得成功的,态度懒散,做事拖拉,是任何企业所不能容忍的。

  再者你应该提高自己的心理素质:
\begin{enumerate}
\item  1.有乐观的工作态度。
\label{sec:orgheadline33}
以微笑的目光、平静的心态去看待一切,建立健康、愉快、丰富的工作模式。

\item  2.有多元思维方式。面对同一种境况要有多种考虑和选择。
\label{sec:orgheadline34}

\item   3.不断地充实自己。
\label{sec:orgheadline35}
把职场环境的变化看成是迎接挑战和再学习的机会。不要在瞬息万变的职场环境面前惊惶失措,愁眉不展。

\item  4.遇事不慌,致力于问题的解决。
\label{sec:orgheadline36}
通过回顾和全面分析,发现目前问题的症结所在,然后制定解决的对策。
\end{enumerate}

\subsubsection{二、管理好你的时间}
\label{sec:orgheadline49}

\begin{enumerate}
\item  1.设立明确目标成功,就是完成目标。
\label{sec:orgheadline38}
个人时间管理的目的是让你在最短时间内实现更多你想要实现的目标;你必须把今年度4到10
个目标写出来,找出一个核心目标,并依次排列重要性,然后依照你的目标设定一些详细的
计划,你的关键就是依照计划进行。


\item  2、要列一张总清单。把今年所要做的每一件事情都列出来,并进行目标切割。
\label{sec:orgheadline43}

\begin{enumerate}
\item 1、年度目标切割成季度目标,列出清单,每一季度要做哪一些事情;
\label{sec:orgheadline39}

\item 2、季度目标切割成月目标,并在每月初重新再列一遍,碰到有突发事件而更改目标的情形便及时调整过来;
\label{sec:orgheadline40}

\item 3、每一个星期天,把下周要完成的每件事列出来;
\label{sec:orgheadline41}

\item 4、每天晚上把第二天要做的事情列出来。
\label{sec:orgheadline42}
\end{enumerate}

\item  3、20:80定律。
\label{sec:orgheadline44}
用你80\%的时间来做20\%最重要的事情,因此你一定要了解,对你来说,哪些事情是最重要的,
是最有生产力的。谈到个人时间管理,有所谓紧急的事情、重要的事情,然而到底应做哪些
事情?当然第一个要做的一定是紧急又重要的事情,通常这些都是一些突发困扰,一些灾难,
一些迫不及待要解决的问题。当你天天处理这些事情时,表示你个人时间管理并不理想。成
功者花最多时间在做最重要,可是不紧急的事情,这些都是所谓的高生产力的事情。然而一
般人都是做紧急但不重要的事。你必须学会如何把重要的事情变得很紧急,这时你就会立刻
开始做高生产力的事情了。


\item  4、每天至少要有半小时到1小时的"不被干扰"时间。
\label{sec:orgheadline45}
假如你能有一个小时完全不受任何 人干扰,自己关在自己的房间里面,思考一些事情,或
是做一些你认为最重要的事情。这一 个小时可以抵过你一天的工作效率,甚至有时候这一
小时比你三天工作的效率还要好。


\item   5、要把每一分钟每一秒做最有效率的事情。
\label{sec:orgheadline46}
你必须思考一下要做好一份工作,到底哪几件事情是对你最有效率的,列下来,分配时间做它做好。

\item   6、要充分地授权。
\label{sec:orgheadline47}
列出你目前生活中所有觉得可以授权的事情,把它们写下来,然后 开始找人授权,找适当
的人来授权,这样效率会比较好。


\item   7、同一类的事情最好一次把它做完。
\label{sec:orgheadline48}
假如你在 做纸上作业,那段时间都做纸上作业; 假如你是在思考,用一段时间只作思考;打
电话的话,最好把电话累积到某一时间一次把它 打完。当你重复做一件事情时,你会熟能
生巧,效率一定会提高。
\end{enumerate}



\subsubsection{三、踏实做人做事,不急于获得成功}
\label{sec:orgheadline50}
  有些人在职场中过度自信,不了解自己的实际能力,工作时往往会自告奋勇,要求负责
超过自己能力的工作,在失败的时候,会希望用更高的功绩来弥补之前的承诺。在这种情况
下很容易出现连败的情况。心理学家发现,凡是那些有所建树的职场成功人士,办事踏实而
稳重,并且他们从来不急于求得成功。因为他们的自信适度,而且懂得追求成功需要自己一
步步努力,因此,他们相对来讲要踏实稳重很多。


  要想成为职场达人,将自己推上事业的巅峰,首先你得足够自信,其次管理好你的时间,
最后做到踏实做人最事。这就是所谓的“世间自有公道,付出定有回报”。



\textit{[2016-08-08 一 16:12]}

\subsection{{\bfseries\sffamily DONE} 扎克伯格:创业者,你不需要成为无所不能的超人!}
\label{sec:orgheadline55}
19岁的马克•扎克伯格在哈佛大学的宿舍中成立了thefacebook.com.。而如今,32岁的他已
经一跃成为身价3400亿美元的社交网络创始人,被人们冠以“第二盖茨”的美誉。


  扎克伯格称,2004年建立Facebook的最重要的原因是基于人的连接,当时互联网上可以
找到几乎所有的东西(新闻、音乐、书、电影、买东西),可是没有服务帮我们找到生活上
最重要的东西,那就是人。


  “世界这么大,咱们创业吧”,随着科技圈的发展,涌现出了越来越多的新兴创业家。
在创业之前,你不妨先看看马克•扎克伯格提出的关于创业的3点建议:


\subsubsection{建议一:你得喜欢而且“用心”}
\label{sec:orgheadline52}
  Facebook赋予社会中的每位个体一个公开的身份,改革了分享生活的方式,呼吁我们重
视自己的内心想法。扎克伯格表示,当你有了使命,它会让你更专注,更用心。


  关于“用心” 如果你有了使命,你不需要有完整的计划,往前走吧!你只需要更多用
心。世界上最好的创意就是你喜欢做、而且能比别人做得好的事情。扎克伯格举例说,
Internet.org就是一个例子,该组织致力于让全世界的人都可以上互联网。我们并没有什么
商业模式可言,他说,但是有钱的人都已在Facebook上了。


  伟大的东西都是由一群最用心的人创造的。这位Facebook CEO称,人们不该担心自己会
犯错误,因为你在一生中总会犯一些错误。你应该克服困难,勇往直前,而不是垂头丧气。


  总结:关乎创业,在你开始做之前,不要只问自己怎么做,要问自己:为什么做?你应
该相信你的使命。解决重要问题。要用心。不要放弃,要一直向前看。所以梦想还是得有,
万一见鬼了呢?


\subsubsection{建议二:不怕犯错的人生观}
\label{sec:orgheadline53}
  在创业的过程中,错误是不可避免的一个组成部分。曾经有人问扎克伯格要如何克服创
业过程中的各种挑战,比如说找到合适的投资人、研发受到所有用户喜爱的产品等等。他是
这样回答的:“在这个世界上,没有人是无所不能的。但是你可以成立一支由朋友和家人组
成的强大后援团,让自己拥有不断前行的动力。”


  许多想要创业的年轻人总是太过追求成功,害怕犯错。但是扎克伯格的想法是,只要拥
有一群强有力的后援军,并且寻找到一群志同道合的朋友,你就能够渐渐发现错误内在的价
值。所以,没必要强迫自己成为无所不能的超人,你所要做的只是坚持自己的初衷,敢于尝
试,不怕犯错。


  总结:创业道路长且艰难,不要因为频繁犯错,而选择放弃 。中国有一句古话说得好:
“只要功夫深,铁杵磨成针”。一直努力,你会改变世界。


\subsubsection{建议三:Done is better than perfect(脚踏实地做好自己的工作,不要追求完美)}
\label{sec:orgheadline54}
  在Facebook位于加州的总部,“Done is better than perfect”这句话作为公司的口
号刷在墙上。换句话说,只要你努力了,完成了自己的工作,提升了自己的能力,这就是一
种成就和荣誉。


  扎克伯格解释了这句话背后的意义:“其实,从某种程度上来说,这句话的核心就是指
我们要学习黑客做事的方式。与其花上好几天的时间来讨论某一想法是否可行,还不如像他
们那样直接研发出一个原型,通过真正效果来判断其成效。”


  Facebook人力资源副总裁罗莉•格勒尔(Lori Goler)表示:“公司的关注点在于确保所
有员工能够在一个包容和具有挑战性的环境里工作,使得他们可以在人生任何一个阶段出色
工作。对于能够创造一个适合所有人的企业文化,我们感到自豪。”


  总结:在创业阶段,脚踏实地做好本分工作是一项最基础也是最重要的事情。脚踏实地
就是不能这山望着那山高,你得有所“满足”,才能做得“完美”。


  最后用扎克伯格的话来做总结:“你们可以成为全球领导者,可以提高人们的生活,可
以用互联网影响全世界。” 所以创业者你总得去相信点什么。



你不敢跨界,就有人跨过来打劫!不是外行干掉内行!而是趋势干掉规模!跨界是难,不垮
是死!8月10日(周三)晚上8点。资深生涯咨询师昂sir与您分享!跨界or转行?如何从门
外汉成长为行业先锋?


\textit{[2016-08-08 一 16:45]}

\subsection{{\bfseries\sffamily DONE} 创业者你不得不学的}
\label{sec:orgheadline56}
独立者    2016/08/03 14:36
  生意上,有这样两种截然相反的人。

  有人生怕别人舒服,尽量让别人不舒服,而只要自己舒服就行。

  还有一类人生怕别人不舒服,尽量让别人舒服,哪怕委屈自己。

  因为我做猎头职业的原因,我们猎聘的老总有几十万年薪的,也有几百万的,甚至有过千万级年薪的老总。

  要问我对这些老总有什么本质感觉上的不同,我的回答是,越是高薪的老总在与其交往中他会越让你感觉到舒服。

  跟千万年薪的老总谈,谈上两到三个小时,无论我说的话是酸甜苦辣等味道,他们都能把每一句话平缓接起来回答,而从不让一句话落地或磕碰,让人感觉非常舒服。

  就像打太极,无论什么招式,全部是以柔克刚。

  这就是高手过招,化解问题于无形之处,于无声之中。

  他们之所以挣千万年薪,自有千万年薪的价值,让人舒服程度也许就是一个衡量指标。

  常常发现越是与年薪水平低的人交流越容易让人不舒服。

  回想平时大家之间的沟通交流,磕磕绊绊,到处充满着不舒服的感觉。

  你不让别人舒服,别人就会让你不舒服。

  想提高年薪吗,就从如何让别人舒服着手,提升这方面的品质和修养。

  我们平常说话两个人都容易伤害到对方,引起争执。

  我曾与一位级别很高的70多岁的老人交谈,他的每一句话都不会伤及到任何一个人,不会让周围的任何一个人感觉不舒服。

  在一起聚餐十多人,每一句话都能照顾到所有的人,无论男女老少都感觉舒服,这是何等的修养。

  战争年代就是千方百计把敌人消灭掉,想法让敌人不舒服。

  在和平建设年代,你让别人舒服的程度,决定着你成功的程度。

  你让别人不舒服,直接影响着你的成功。

  李嘉诚请马云吃饭的故事是一个很好的证明:

  长江CEO班有30几个同学,包括马云、郭广昌、牛根生等国内大家认为很了不起的人。有一次,班上组织我们去香港见一次李嘉诚,他可谓华人世界的超级大哥了。

  没见面之前,心里有个情景假定,比如约会衣服要穿整齐等,当时我就想:见老大哥相当于见领导,一般我们见这种人,可能第一见不到大哥先见到椅子、沙发;第二伟大的人来了,我们发名片人家不会发名片;第三人家跟你握手然后你站着听讲话,就像我们被接见,在人民大会堂听讲话我们鼓掌就完了;最后吃饭肯定有主桌,大哥在那坐一下,吃两筷子说忙先走了;然后我们很激动回来写感想……

  结果这次见面完全颠覆了之前的想法。

  首先电梯一开,长江顶楼,70多岁的大哥站着跟我们握手,这样的开场很不一样,我有点愣。其次,一见面大哥先发名片,这个也很诧异,而且发名片还给你递过来一个盘子,递盘子干吗?抓阄,盘子里有号,拿名片顺便抓个号,这个号决定你吃饭的时候坐哪桌,避免到时候我们这些同学为谁坐1号桌,谁坐2号桌心里有想法。后来才知道,照相也根据这个号,站哪就是哪。我觉得挺好,大家避免尴尬。

  站好之后我们小人物的能力出现了,我们就鼓掌希望大哥讲话,大哥说没准备讲话,但这时候大哥不讲我们小人物角色演不下去,所以必须让他讲,这个经历经常有,最后大哥说,我没有准备,我只讲八个字叫做“创造自我,追求无我”。

  这一听大哥读书很多,学历不高读书很多,讲的都是哲学,“创造自我,追求无我”,讲完了普通话又用广东话讲一遍,之后发现还有老外用英文再讲一遍,就讲这八个字讲完了我们体会这话里的深意。

  什么叫追求自我?你在芸芸众生中,把自己越做越强大,自我膨胀,超越别人,这个过程就容易给别人以压力。因为你强大了以后很强势,就像你老站着,别人蹲着,别人就不舒服。所以你要追求无我,让自己化解在芸芸众生中,不要让别人感觉到你的压力。一方面创造自我,一方面让自己回归于平淡,让自己舒服也不给大家制造压力。

  听完讲话我们开始鼓掌,然后开始吃饭。我运气不错,抽到了跟大哥一桌,我当时想,和大哥挺近的,这样吃饭可以多聊一会儿,所以开始没着急说话,没想到吃十几分钟的时候大哥站起来说抱歉要到那边坐一下。这时我们才发现,四张桌子,每个桌子都多放了一副碗筷,他每个桌子都坐。一个小时的吃饭时间,他四个桌子轮流坐,而且几乎都是15分钟,到这时,大家都被大哥周到和细致的安排感动了。

  大哥大概每个桌子转完基本也就结束了,结束之后他没先走,逐一跟大家握手,在场的每个人都要握到,墙角站着一服务员,大哥专门跑到那和他握手。这时候我想起看过他的一个演讲,问他们有没有关于这个的书,当时没准备,他交代下面一下,结果下车的时候那个书就送到我手里了。整个过程让我们每个人都很舒服。

  这就是大哥之所以成为大哥的原因,这就是他的软实力。他具有一种看不到的能力,这个能力是价值观,用他的话说就是追求无我,他让每个人都舒服。后来我跟我们班班长提到这事,他说老先生就是因为做人周到真诚,所以很多人到了香港都愿意和他做生意。

\textit{[2016-08-08 一 16:53]}

\subsection{{\bfseries\sffamily DONE} 在江湖上混需要养成的10个好习惯}
\label{sec:orgheadline67}
独立者    2016/08/08 11:27
  “人之患在好为人师”,我也特烦教导别人。一来是认为每个人的情况不一样,很难一
概论之。那些号称他的成功可以复制的,不是为了骗你钱买书的,就是教你抄袭造假骗人的。
二来我光讲、你光听,基本没用。我好好讲《易筋经》,你好好听,你还是不会少林武功。
所以,你们想听我讲,刚入职场应该注意什么,让我为难了,想来想去,还是说说好习惯。
在江湖上混,养成好习惯第一,其他就在你们各自的特质和造化了。

\subsubsection{ 第一个习惯是及时。}
\label{sec:orgheadline57}
收到的短信邮件,24小时内一定回复,信号不好不是借口。约好了会议,要及时赶到,北京
交通拥堵、闹钟没响、你妈忘了叫你起床不是借口。

\subsubsection{第二个习惯是近俗。}
\label{sec:orgheadline58}
尽管信息爆炸,要学会不走马观花。长期阅读两种以上财经期刊,知道最近什么是大奸大滑
大痴大傻。长期阅读两种以上专业期刊,知道最近什么是最新最酷最潮。

\subsubsection{第三个习惯是学习。}
\label{sec:orgheadline59}
一年至少要念四本严肃书籍。严肃书籍的定义是,不是通常在机场能买到的,不是近五年出的,不是你看了能不犯困的。

  

\subsubsection{第四个习惯是动笔。}
\label{sec:orgheadline60}
在现世,能想明白、写清楚的年轻人越来越少,眼高手低的年轻人越来越多。一年至少写四
篇文章,每篇至少两千字。写作的过程,也是沉静、思考和凝练的过程,仿佛躲开人群、屏
息敛气、抬头看到明月当头。

\subsubsection{第五个习惯是强身}
\label{sec:orgheadline61}
。每天至少慢运动半小时,比如肢体伸展、瑜伽、站桩、静坐。每周争取专门锻炼一次,每
次两个小时以上。保持身体健康、不经常请病假,也是职业管理者的基本素养。

\subsubsection{第六个习惯是爱好。}
\label{sec:orgheadline62}
争取培养一个你能长期享受的爱好,不见得很复杂,比如发呆、倒立,甚至不见得你能做得
比其他人好很多,比如自拍、养花。工作有时候会很烦,要学会扯脱。很多争吵,如果争吵
双方都闭嘴,回房间发呆、自拍、闭眼、睡觉,第二天基本会发现,完全没有争吵的必要。

\subsubsection{第七个习惯是常备。}
\label{sec:orgheadline63}
除了睡觉的时候,手机要开机,要让你的同事能找到你。如果和上级出差,你的手机几乎要
时刻攥在手上。手机没电了不是借口,即使你用的是iPhone,也可以配个外挂电池。

\subsubsection{第八个习惯是执行。}
\label{sec:orgheadline64}
万事开头难,所以见到事儿就着手立办,马上开头。不开头,对于这件事儿的思绪要占据你
的内存很多、很久。见了就做,做了就放下了。

\subsubsection{ 第九个习惯是服从。}
\label{sec:orgheadline65}
接到一项似乎很不合理的工作,忌马上拒绝或抱怨。第一,和上级充分沟通,从他的角度理
解任务。有时候,你心中对此项工作的要求远远高于上级要求。第二,降低对自己的要求,
有自信,不必做每件事都得一百分。第三,上述两条还不能解决心头不快,放下自己,服从。

\subsubsection{第十个习惯是收放。}
\label{sec:orgheadline66}
阳光之下,快跑者未必先达,力战者未必能胜。同学们啊,从学校毕业之后,不再是每件事
都是一门考试,不再是每门考试你都要拿满分和拿第一。收放是一种在学校里没人教你的技
巧,练习的第一步是有自信,不必事事胜人。

   当然,如果你们说,这些习惯太俗,想仰天大笑出门去,这些世俗习惯完全可以不理。
 内心之外,我祝福你们找到不世俗的山林、不用装修的岩洞、不搞政府关系的和尚和不爱
 财的姑娘。
CO\(_{\text{2}}\)
H\(^{\text{2}}\)

\emph{dkdk}
\uline{dkdkkd}
\texttt{dkdkkd}

\textit{[2016-08-08 一 16:57]}

\subsection{{\bfseries\sffamily DONE} eat breakfast and make purchases}
\label{sec:orgheadline68}
\textit{[2016-08-09 二 10:26]}

\subsection{{\bfseries\sffamily DONE} learn spacemacs}
\label{sec:orgheadline69}
\textit{[2016-08-10 三 04:00]}
\end{document}
