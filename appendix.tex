\chapter{关于金庸}
\label{sec:appendixname}

\section{金庸简介}

金庸\footnote{参见维基百科-\href{http://zh.wikipedia.org/wiki/\%E9\%87\%91\%E5\%BA\%B8}{金
      庸}},大紫荊勳賢,OBE,原名查良鏞,(Louis Cha Leung Yung,1924年2月6日-)浙江海寧人,東吳大
  学法學士、英國劍橋大學歷史碩士、博士。查良鏞在1948年移居香港,以筆名金庸著作多部膾炙人口的武俠小說,
  是华人界最知名的武俠小說作家之一。金庸亦是香港新闻、文艺界的杰出创业者及评论家,以及著名的社会活动
  家。金庸、古龍與梁羽生普遍被認為是新派武侠小说的代表作家,被誉为武侠小说作家的「泰斗」,更有「金迷」
  们尊称其为「金大侠」或「查大俠」,亦被喻為「香港四大才子」之一\Cite{jinyongcn}。
\begin{itemize}
\item 傳說金庸對古典樂十分喜好,判別能力超強。可以隨便聽過一段樂句,就告訴你這是哪位作曲家的哪首作品。
\item 金庸愛車,跑車为尤。
\item 接受訪談時,金庸認為自己像張無忌(《倚天屠龍記》),妻子像夏青青(《碧血劍》)。
\item 金庸在《鹿鼎記》后記中,表示自己最喜歡的作品是《倚天屠龍記》,《笑傲江湖》,《神鵰俠侶》和《飛狐外
  傳》。
\item 金庸在《倚天屠龍記》后記中,表示自己最愛的女性人物是小昭(《倚天屠龍記》)。
\item 曾开除金庸的國立政治大學(其前身即中央政治學校)于2007年5月19日,授予金庸榮譽博士學位。
\item 金庸於香港大學設立“查良鏞學術基金”,並擔任主席,主力邀講各國學者定期舉行學術講座和研討會。
\item 香港大學於2009年2月27日成立國際金庸研究會。研究會並非註冊社團,由金庸及前港大中文系主任單周堯教授擔任顧問,創會會長為李思齊教授及伍懷璞教授,現任會長為港大文學院黎活仁副教授\Cite{jinyongcn}。
\end{itemize}

%%% Local Variables: 
%%% mode: latex
%%% TeX-master: "./thesis2"
%%% End: 
