\documentclass{swfcthesis}

\begin{document}

\Title{金庸笔下武功详解}
\Author{周伯通}
\Advisor{王重阳}
\AdvisorTitle{祖师}
\AdvisorInfo{王重陽\footnote{参见维基百科-\href{http://zh.wikipedia.org/wiki/\%E7\%8E\%8B\%E9\%87\%8D\%E9\%99\%BD}{王重阳}}(1113年1月13日-1170年),原名中孚,字允卿,又名世雄,字德威,入道后改名喆,字知明,道号重阳子,故称王重阳。北宋末京兆咸阳(今陕西咸阳)大魏村人。中國道教分支全真道的始創人,后被尊为道教的北五祖之一。他有七位出名的弟子,在道教历史上称为北七真。}
\Month{六}
\Year{一二三七}
\Univ{中华武林大学}
\Docname{本科毕业(设计)论文}
\School{全真玄门正宗学院}
\Subject{打架斗殴专业}
\Abstract{中國武術\footnote{参见维基百科 -
    \href{http://zh.wikipedia.org/wiki/\%E4\%B8\%AD\%E5\%9B\%BD\%E6\%AD\%A6\%E6\%9C\%AF}{中国武术}}
  是中國传统文化的重要一環。兩廣人稱為功夫,民國初期簡稱為國術(後為中央國術館正式採用之名稱);被視
  為中國文化之精粹,故又稱國粹。由於歷史發展和地域分佈關係,衍生出不同門派。中國武術主要內容包括搏擊
  技巧、格鬥手法、攻防策略和武器使用等技術。當中又分為理論和實踐兩個範疇。從實踐中帶來了有關體育、健
  身、和中國武術獨有之氣功、及養生等重要功能。理論中帶來了不少前人之經驗和拳譜記錄。因此,它体现中华
  民族对攻防技击及策略上的理解。加上經驗上積累,以自立、自強、健體養生為目標的自我運作,練習套路时顯
  示出身體動作之優美姿態。中國武術往往帶有思想冶鍊的文化特徵及人文哲學的特色、意義,對現今中國的大眾
  文化有著深遠影響\Cite{wushucn}。}
\Keywords{金庸,武术,一陽指,双手互搏,空明拳,七傷拳,吸星大法,葵花宝典,九陰真經,九陽真
  經,天山六陽掌,天羅地網勢,蛤蟆功,倚天屠龍功,弹指神通,先天功,打狗棒法,全真剑法,摧心掌,降龍
  十八掌,六脈神劍,火焰刀,黯然銷魂掌,龍爪擒拿手,兰花拂穴手,龍象般若掌,劈空掌,玉女素心剑法,北
  冥神功,碧海潮生曲}
\Acknowledgments{感谢师兄王重阳传我全真玄门正宗功夫,感谢段皇爷不杀之恩,感谢刘贵妃不怨旧恶,感谢桃
  花岛主黄老邪助我练就空明拳和双手互搏之术,感谢郭靖兄弟让我看《九阴真经》,感谢小龙女教我驭蜂之术。}
\enTitle{Jin Yong's Chinese Martial Arts Illustrated}
\enAuthor{Zhou Botong}
\enUniv{Chinese Kungfu University}
\enSchool{School of Taoism}
\enAbstract{Chinese martial
  arts\footnote{Wikipedia - \href{http://en.wikipedia.org/wiki/Chinese\_martial\_arts}{Chinese martial arts}}, also referred to by the Mandarin Chinese term wushu and popularly as kung fu, are a number of fighting styles that have developed over the centuries in China. These fighting styles are often classified according to common traits, identified as "families", "sects" or "schools" of martial arts. Examples of such traits include physical exercises involving animal mimicry, or training methods inspired by Chinese philosophies, religions and legends. Styles which focus on qi manipulation are labeled as internal, while others concentrate on improving muscle and cardiovascular fitness and are labeled external. Geographical association, as in northern and southern, is another popular method of categorization\Cite{wushu}.}
\enKeywords{Jin Yong, Chinese martial arts, Kungfu}

\makepreliminarypages %生成封面、摘要、英文摘要……
\frontmatter          %目录部分由此开始
\tableofcontents      %生成目录
\listoffigures        %图片目录
\listoftables         %表格目录
\mainmatter           %正文部分由此开始

写啊写啊写啊写啊写啊写啊写啊写啊写啊写啊写啊

                      %正文部分到此结束
\Appendix{}           % 附录部分由此开始

\cleardoublepage
\phantomsection
\addcontentsline{toc}{chapter}{参考文献}

\begin{thebibliography}{99}
\bibitem{shediao} 金庸, 《射雕英雄传》, 广州出版社, 2003. 
\bibitem{shendiao} 金庸, 《神雕侠侣》, 陕西人民出版社, 1992. 
\bibitem{yitian} 金庸, 《倚天屠龙记》, 广州出版社, 2002. 
\end{thebibliography}

%%% Local Variables: 
%%% mode: latex
%%% TeX-master: "./thesis2"
%%% End:  % 把参考文献(bibliography.tex)包含进来
\advisorinfopage{}     % 生成【教师简介】
\acknowledgmentspage{} % 生成【鸣谢】
\chapter{关于金庸}
\label{sec:appendixname}

\section{金庸简介}

金庸\footnote{参见维基百科-\href{http://zh.wikipedia.org/wiki/\%E9\%87\%91\%E5\%BA\%B8}{金
      庸}},大紫荊勳賢,OBE,原名查良鏞,(Louis Cha Leung Yung,1924年2月6日-)浙江海寧人,東吳大
  学法學士、英國劍橋大學歷史碩士、博士。查良鏞在1948年移居香港,以筆名金庸著作多部膾炙人口的武俠小說,
  是华人界最知名的武俠小說作家之一。金庸亦是香港新闻、文艺界的杰出创业者及评论家,以及著名的社会活动
  家。金庸、古龍與梁羽生普遍被認為是新派武侠小说的代表作家,被誉为武侠小说作家的「泰斗」,更有「金迷」
  们尊称其为「金大侠」或「查大俠」,亦被喻為「香港四大才子」之一\Cite{jinyongcn}。
\begin{itemize}
\item 傳說金庸對古典樂十分喜好,判別能力超強。可以隨便聽過一段樂句,就告訴你這是哪位作曲家的哪首作品。
\item 金庸愛車,跑車为尤。
\item 接受訪談時,金庸認為自己像張無忌(《倚天屠龍記》),妻子像夏青青(《碧血劍》)。
\item 金庸在《鹿鼎記》后記中,表示自己最喜歡的作品是《倚天屠龍記》,《笑傲江湖》,《神鵰俠侶》和《飛狐外
  傳》。
\item 金庸在《倚天屠龍記》后記中,表示自己最愛的女性人物是小昭(《倚天屠龍記》)。
\item 曾开除金庸的國立政治大學(其前身即中央政治學校)于2007年5月19日,授予金庸榮譽博士學位。
\item 金庸於香港大學設立“查良鏞學術基金”,並擔任主席,主力邀講各國學者定期舉行學術講座和研討會。
\item 香港大學於2009年2月27日成立國際金庸研究會。研究會並非註冊社團,由金庸及前港大中文系主任單周堯教授擔任顧問,創會會長為李思齊教授及伍懷璞教授,現任會長為港大文學院黎活仁副教授\Cite{jinyongcn}。
\end{itemize}

%%% Local Variables: 
%%% mode: latex
%%% TeX-master: "./thesis2"
%%% End: 
     % 如果有附录(appendix.tex)的话,把它包含进来
\include{src}          % 把程序代码(source.tex)附在这里

\end{document}
%%% Local Variables: 
%%% mode: latex
%%% TeX-master: t
%%% End: 